% Options for packages loaded elsewhere
\PassOptionsToPackage{unicode}{hyperref}
\PassOptionsToPackage{hyphens}{url}
%
\documentclass[
]{article}
\usepackage{amsmath,amssymb}
\usepackage{iftex}
\ifPDFTeX
  \usepackage[T1]{fontenc}
  \usepackage[utf8]{inputenc}
  \usepackage{textcomp} % provide euro and other symbols
\else % if luatex or xetex
  \usepackage{unicode-math} % this also loads fontspec
  \defaultfontfeatures{Scale=MatchLowercase}
  \defaultfontfeatures[\rmfamily]{Ligatures=TeX,Scale=1}
\fi
\usepackage{lmodern}
\ifPDFTeX\else
  % xetex/luatex font selection
\fi
% Use upquote if available, for straight quotes in verbatim environments
\IfFileExists{upquote.sty}{\usepackage{upquote}}{}
\IfFileExists{microtype.sty}{% use microtype if available
  \usepackage[]{microtype}
  \UseMicrotypeSet[protrusion]{basicmath} % disable protrusion for tt fonts
}{}
\makeatletter
\@ifundefined{KOMAClassName}{% if non-KOMA class
  \IfFileExists{parskip.sty}{%
    \usepackage{parskip}
  }{% else
    \setlength{\parindent}{0pt}
    \setlength{\parskip}{6pt plus 2pt minus 1pt}}
}{% if KOMA class
  \KOMAoptions{parskip=half}}
\makeatother
\usepackage{xcolor}
\usepackage[margin=1in]{geometry}
\usepackage{graphicx}
\makeatletter
\def\maxwidth{\ifdim\Gin@nat@width>\linewidth\linewidth\else\Gin@nat@width\fi}
\def\maxheight{\ifdim\Gin@nat@height>\textheight\textheight\else\Gin@nat@height\fi}
\makeatother
% Scale images if necessary, so that they will not overflow the page
% margins by default, and it is still possible to overwrite the defaults
% using explicit options in \includegraphics[width, height, ...]{}
\setkeys{Gin}{width=\maxwidth,height=\maxheight,keepaspectratio}
% Set default figure placement to htbp
\makeatletter
\def\fps@figure{htbp}
\makeatother
\setlength{\emergencystretch}{3em} % prevent overfull lines
\providecommand{\tightlist}{%
  \setlength{\itemsep}{0pt}\setlength{\parskip}{0pt}}
\setcounter{secnumdepth}{-\maxdimen} % remove section numbering
\ifLuaTeX
  \usepackage{selnolig}  % disable illegal ligatures
\fi
\usepackage{bookmark}
\IfFileExists{xurl.sty}{\usepackage{xurl}}{} % add URL line breaks if available
\urlstyle{same}
\hypersetup{
  hidelinks,
  pdfcreator={LaTeX via pandoc}}

\author{}
\date{\vspace{-2.5em}}

\begin{document}

\subsection{Teil A: Der Anfang der
Philosophen}\label{teil-a-der-anfang-der-philosophen}

Dein Zeitpunkt ist dann jetzt angenommen. Bist du dir über deinen
Aufenthalt im klaren? Es gibt dich da wo du bist auch und nicht nur den
Inhalt deiner Vorstellung über irgendeinen Ort, an dem du jetzt gern
wärst? Gut, dann können wir mit der Lektüre beginnen. Zuerst habe ich
Parmenides herausgesucht, wohl, weil vielleicht dadurch großes Unheil
abgewendet werden kann: daß du gleich zu Anfang dich entscheiden
müßtest, wem du Glauben schenken sollst, mir oder dem Literaten. Deshalb
also eine erzwungene primäre Quelle, die du versuchen kannst zu
verstehen. Danach wirst du auch dem Literaten anders begegnen und
vielleicht ist es dann nicht mehr so wichtig wie deutlich jemand spricht
sondern nur was er eigentlich sagen wollte, \emph{nachdem} du ihn
unterbrochen hast. So kommen wir zu den nächsten Griechen, die vermerkt
wurden als Anfang des Denkens, dem Medium der Philosophie. Doch für
heute hättest du genug gelernt, \emph{von allen Fragen befreie deinen
Geist. }Geh schlafen, Thomas.\\
Ich stand aber wirklich dort: sah hervor aus einem Teeststrauch und
schaute in meinen Buchenwald, an den Bäumen war Regen heruntergelaufen
und die nassen Stämme voller Geschlechter blickten auf mich zurück als
wärs in mir gewesen, daß ich grad versinke. Solln wir solches zulassen?
Da waren Tränen glaub ich irgendwo n.~und bevor es sie nicht heraustrieb
würde ich einmal versagen wußte ich.\\
Man sagte mir: Gehn Sie in ihr Leben zurück als wärs fremdes Leben, das
sie da täglich führen und lassen das sie sonst herbeiträgt auf jeder
kleinen Schwinge eines Vogels vor dem Tor draußen enden, vor der
Hoffnung, vor der falschen Freiheit versprechenden Pforte und vor allen
Dingen der Zeit die dir n.~bleibt für den Schlaf. Denn der wird dir
gemessen, gezählt, gewogen und zerteilt in die kleinen sichtbaren
Einheiten zwischen Wachrausch und Wachtraum die dein Zyklus hier zeigt.
Darum das Du und die gemeinsame Erinnerung.\\
Man sagt außerdem: spiel nicht mit den Schlangen, die Schlange ist ein
verworfenes Tier, ein gefallenes Tier und wenn du Gefallen an Schlangen
hast ist was nicht inordnung mit dir. Ich sammelte aber Häute die man
mir brachte, trug sie und lange Zeit einen Giftzahn aus der Sammlung
eines der Väter Freunde bei mir sowie andere geheimere Sachen in einem
geschnitzten Stopfmopf in der Hosentasche. Wenn ich als ich den ersten
verlor kurz nach Griechenland flüchtete vielleicht in der unheimlichen
Ahnung mit ihm dort ein Stück Seele verloren zu haben das der Sommer in
Asprovalta aus mir erzeugt hatte; flüchtete wohin ich nie von selbst
gegangen wäre ohne von seinem Fetisch besessen zu sein: nach namenlosen
Gewässern also; so ist wann immer jetzt die Angst auftaucht ich könnte
sein Gegenstück, geschnitzt aus dem letzten Holz des Friedbergers von
eben dem Baume der daselbst stand im Garten des Griechen genauso
verlieren, sie ungleich schwächer und zwingt nicht zur Aufgabe der
Gewohnheiten - mit einem Stopfmopf sollte ja gestopft werden und nicht
geschrieben, oder? Siehst du, und deshalb wenn ich nach ihm suchen müßte
würde ich nach etwas anderem suchen als früher. Aber glücklicherweise
hat er mich ja w.gefunden und war schon als zum Stab erstarrte Schlange
längst zurückgekehrt bevor von dem endgültig in magischem Harz
gefertigten Werkzeug ich Gebrauch machen mußte.

\end{document}
