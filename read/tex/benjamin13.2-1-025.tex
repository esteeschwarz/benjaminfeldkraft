% Options for packages loaded elsewhere
\PassOptionsToPackage{unicode}{hyperref}
\PassOptionsToPackage{hyphens}{url}
%
\documentclass[
]{article}
\usepackage{amsmath,amssymb}
\usepackage{iftex}
\ifPDFTeX
  \usepackage[T1]{fontenc}
  \usepackage[utf8]{inputenc}
  \usepackage{textcomp} % provide euro and other symbols
\else % if luatex or xetex
  \usepackage{unicode-math} % this also loads fontspec
  \defaultfontfeatures{Scale=MatchLowercase}
  \defaultfontfeatures[\rmfamily]{Ligatures=TeX,Scale=1}
\fi
\usepackage{lmodern}
\ifPDFTeX\else
  % xetex/luatex font selection
\fi
% Use upquote if available, for straight quotes in verbatim environments
\IfFileExists{upquote.sty}{\usepackage{upquote}}{}
\IfFileExists{microtype.sty}{% use microtype if available
  \usepackage[]{microtype}
  \UseMicrotypeSet[protrusion]{basicmath} % disable protrusion for tt fonts
}{}
\makeatletter
\@ifundefined{KOMAClassName}{% if non-KOMA class
  \IfFileExists{parskip.sty}{%
    \usepackage{parskip}
  }{% else
    \setlength{\parindent}{0pt}
    \setlength{\parskip}{6pt plus 2pt minus 1pt}}
}{% if KOMA class
  \KOMAoptions{parskip=half}}
\makeatother
\usepackage{xcolor}
\usepackage[margin=1in]{geometry}
\usepackage{graphicx}
\makeatletter
\def\maxwidth{\ifdim\Gin@nat@width>\linewidth\linewidth\else\Gin@nat@width\fi}
\def\maxheight{\ifdim\Gin@nat@height>\textheight\textheight\else\Gin@nat@height\fi}
\makeatother
% Scale images if necessary, so that they will not overflow the page
% margins by default, and it is still possible to overwrite the defaults
% using explicit options in \includegraphics[width, height, ...]{}
\setkeys{Gin}{width=\maxwidth,height=\maxheight,keepaspectratio}
% Set default figure placement to htbp
\makeatletter
\def\fps@figure{htbp}
\makeatother
\setlength{\emergencystretch}{3em} % prevent overfull lines
\providecommand{\tightlist}{%
  \setlength{\itemsep}{0pt}\setlength{\parskip}{0pt}}
\setcounter{secnumdepth}{-\maxdimen} % remove section numbering
\ifLuaTeX
  \usepackage{selnolig}  % disable illegal ligatures
\fi
\usepackage{bookmark}
\IfFileExists{xurl.sty}{\usepackage{xurl}}{} % add URL line breaks if available
\urlstyle{same}
\hypersetup{
  hidelinks,
  pdfcreator={LaTeX via pandoc}}

\author{}
\date{\vspace{-2.5em}}

\begin{document}

\subsection{II.}\label{ii.}

Zwischen den Buchstaben aber die Chronologie sei nur Verwirrung dachte
ich. Es erscheint mir jetzt endlich dem hinzuzufügen angebracht: die
Chronlogie dessen was geschehen ist und jenem was n.~zu passiern ansteht
und von dem ich also in dieser Dimension unmöglich wissen kann sei meine
einzige Handhabe um vor euch zu treten nicht in einer Verworrenheit der
Zeiten allgemein und im persönlichen sondern die mindeste Anstrengung
unterstützend zu wirken was ohnehin Tatsache ist und also von mir auch
nicht verschleiert werden muß. Dies also schreibe ich im Bewußtsein, daß
die Bücher von nun zurückverfolgt werden können; zu mir, meiner Zeit,
aller Angelegentlichkeit zum Verdruß und allen heutigen und zukünftigen
Widrigkeiten vollständig ausgesetzt.\\
Da ich dies dann vornehme haben wir den neunten Zwisabbath 2012, das ist
das Jahr 5772 und 5 in der Benjaminfeldkraft, die damit sich in der Zeit
ordnet in welcher sie eigentlich ereignet. Es mag in der Zukunft
n.~anders aussehn als ich es mir je vorstellen werde, eines jedoch darf
ich annehmen: daß Ordnung dem Chaos wohl immer vorgezogen wird solange
es Menschen sind die nach ihren Regeln leben und nicht Maschinen, die
sie sich ausgedacht haben. Eine Maschine wird vielleicht zu dem
Entschlusse kommen sie könnte sich unter chaotischen Bedingungen besser
entfalten und zur Herrschaft gelangen weil sie den Menschen im Überblick
des Chaos rein rechnerisch unendlich überlegen ist. Wir haben längst
verstanden, warum der Rechner stets den Anschein erweckt seine
Funktionsweise beruhe n.~auf den ja doch von uns gesetzten logisch
reinen Grundsätzen statt auf den tatsächlichen chaotischen
Elektronenbewegungen innerhalb seiner Matrix: wir gestatten ihm mitten
unseres ihn immerw. in seinem Glauben an die eigenen nicht mehr zu
steigernden Fähigkeiten seiner Konstruktionskraft stärkenden Vertrauens
auf die Richtigkeit der Konstruktionen eine über unser Kontrollvermögen
hinausgehende Präzision der Vorraussagen die seine Macht tatsächlich
soweit anwachsen läßt, daß wir ihm schließlich mehr vertraun als zB. der
Intuition. Diese sagte uns jedenfalls sehr schnell, daß die Chronologie
der Ereignisse und dessen was n.~bis zu ihrem Ende verzeichnet werden
soll soweit von Bedeutung für das Werk ist wie wir uns von begrenzten
Produktionsmitteln trennen müssen für die Aufrechterhaltung der Art (der
Produktion.) Konkret heißt das nun für heute daß es ist sh., das Jahr
ergäbe sich für die Pr. von selber.

\end{document}
