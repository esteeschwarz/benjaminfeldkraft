% Options for packages loaded elsewhere
\PassOptionsToPackage{unicode}{hyperref}
\PassOptionsToPackage{hyphens}{url}
%
\documentclass[
]{article}
\usepackage{amsmath,amssymb}
\usepackage{iftex}
\ifPDFTeX
  \usepackage[T1]{fontenc}
  \usepackage[utf8]{inputenc}
  \usepackage{textcomp} % provide euro and other symbols
\else % if luatex or xetex
  \usepackage{unicode-math} % this also loads fontspec
  \defaultfontfeatures{Scale=MatchLowercase}
  \defaultfontfeatures[\rmfamily]{Ligatures=TeX,Scale=1}
\fi
\usepackage{lmodern}
\ifPDFTeX\else
  % xetex/luatex font selection
\fi
% Use upquote if available, for straight quotes in verbatim environments
\IfFileExists{upquote.sty}{\usepackage{upquote}}{}
\IfFileExists{microtype.sty}{% use microtype if available
  \usepackage[]{microtype}
  \UseMicrotypeSet[protrusion]{basicmath} % disable protrusion for tt fonts
}{}
\makeatletter
\@ifundefined{KOMAClassName}{% if non-KOMA class
  \IfFileExists{parskip.sty}{%
    \usepackage{parskip}
  }{% else
    \setlength{\parindent}{0pt}
    \setlength{\parskip}{6pt plus 2pt minus 1pt}}
}{% if KOMA class
  \KOMAoptions{parskip=half}}
\makeatother
\usepackage{xcolor}
\usepackage[margin=1in]{geometry}
\usepackage{graphicx}
\makeatletter
\def\maxwidth{\ifdim\Gin@nat@width>\linewidth\linewidth\else\Gin@nat@width\fi}
\def\maxheight{\ifdim\Gin@nat@height>\textheight\textheight\else\Gin@nat@height\fi}
\makeatother
% Scale images if necessary, so that they will not overflow the page
% margins by default, and it is still possible to overwrite the defaults
% using explicit options in \includegraphics[width, height, ...]{}
\setkeys{Gin}{width=\maxwidth,height=\maxheight,keepaspectratio}
% Set default figure placement to htbp
\makeatletter
\def\fps@figure{htbp}
\makeatother
\setlength{\emergencystretch}{3em} % prevent overfull lines
\providecommand{\tightlist}{%
  \setlength{\itemsep}{0pt}\setlength{\parskip}{0pt}}
\setcounter{secnumdepth}{-\maxdimen} % remove section numbering
\ifLuaTeX
  \usepackage{selnolig}  % disable illegal ligatures
\fi
\usepackage{bookmark}
\IfFileExists{xurl.sty}{\usepackage{xurl}}{} % add URL line breaks if available
\urlstyle{same}
\hypersetup{
  hidelinks,
  pdfcreator={LaTeX via pandoc}}

\author{}
\date{\vspace{-2.5em}}

\begin{document}

\subsection{Teil B: 1. Über allen Gipfeln ist
Ruh.}\label{teil-b-1.-uxfcber-allen-gipfeln-ist-ruh.}

Ja, es werden Namen ausgesprochen einzelner der I. Brüder Unserer Frau
von B\ldots{} mit denen ich zB meine Kindheit verbrachte irgendeines
Bauernumlands. Da war an einem einzigen Ort um ungestört zu beten ein
grober Feldweg, ins Nichts führend, also für die anderen auf eine
unbedeutende Wiese, mich aber zu einer großen Eiche und weil ich hoch
sprang konnte ich auch alleine einen untern Ast erklimmen wofür sie
sonst einen Kumpanen brauchten; mit dem zusammen wollte man jedoch nicht
beten - falls man n.~kein Bruder hatte. Dann saß ich im Baum und so
einsam wie selten. Man sah mich nicht, sah keinen rufen durch das Laub
und kümmerte mich nicht was sie überall taten. Ich habe später ähnliches
erlebt als ich mit dem Boot in den See hinein ruderte so weit bis jemand
(Ewa?) die am Ufer nach mir riefe nur ein Punkt war und nicht sah ob ich
schlief oder was anderes wie ins Gebet mich versenkte. Aber es sollten
ja Namen ausgesprochen werden\ldots{} zB. dessen der uns eben verließ
wie wir n.~zu jung waren darin mehr zu sehn als wenn es gerade mich
selbst erwischt hätte bei Rüstungsturnerein und besoffen vom Dach
pinkeln etc. Sein Name ausgesprochen gab der ersten Verbindung (V.) ihr
Recht. Doch habe ich Löwenstein nie gesehn und werde auch nicht von
seinem Material wissen bevor ich mich auf einen Weg gemacht habe wie den
Feldweg damals wo ich dann als einziger die Hohe Eiche richtig erkannte.
Sie stand nun auf einem bißchen vonnem alten Dolmenhügel und eigentlich
wenn ich es recht bedenke war da immer unser Spielplatz gewesen, jetzt
tauchen auch die andern endlich auf mit Micky, Donald osä. doch keiner
allein, so war es, weil wohl jemand meinten sie in den Höhlungen hauste
unheimlich. Und ich war so mutig?

\end{document}
