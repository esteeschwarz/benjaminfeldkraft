% Options for packages loaded elsewhere
\PassOptionsToPackage{unicode}{hyperref}
\PassOptionsToPackage{hyphens}{url}
%
\documentclass[
]{article}
\usepackage{amsmath,amssymb}
\usepackage{iftex}
\ifPDFTeX
  \usepackage[T1]{fontenc}
  \usepackage[utf8]{inputenc}
  \usepackage{textcomp} % provide euro and other symbols
\else % if luatex or xetex
  \usepackage{unicode-math} % this also loads fontspec
  \defaultfontfeatures{Scale=MatchLowercase}
  \defaultfontfeatures[\rmfamily]{Ligatures=TeX,Scale=1}
\fi
\usepackage{lmodern}
\ifPDFTeX\else
  % xetex/luatex font selection
\fi
% Use upquote if available, for straight quotes in verbatim environments
\IfFileExists{upquote.sty}{\usepackage{upquote}}{}
\IfFileExists{microtype.sty}{% use microtype if available
  \usepackage[]{microtype}
  \UseMicrotypeSet[protrusion]{basicmath} % disable protrusion for tt fonts
}{}
\makeatletter
\@ifundefined{KOMAClassName}{% if non-KOMA class
  \IfFileExists{parskip.sty}{%
    \usepackage{parskip}
  }{% else
    \setlength{\parindent}{0pt}
    \setlength{\parskip}{6pt plus 2pt minus 1pt}}
}{% if KOMA class
  \KOMAoptions{parskip=half}}
\makeatother
\usepackage{xcolor}
\usepackage[margin=1in]{geometry}
\usepackage{graphicx}
\makeatletter
\def\maxwidth{\ifdim\Gin@nat@width>\linewidth\linewidth\else\Gin@nat@width\fi}
\def\maxheight{\ifdim\Gin@nat@height>\textheight\textheight\else\Gin@nat@height\fi}
\makeatother
% Scale images if necessary, so that they will not overflow the page
% margins by default, and it is still possible to overwrite the defaults
% using explicit options in \includegraphics[width, height, ...]{}
\setkeys{Gin}{width=\maxwidth,height=\maxheight,keepaspectratio}
% Set default figure placement to htbp
\makeatletter
\def\fps@figure{htbp}
\makeatother
\setlength{\emergencystretch}{3em} % prevent overfull lines
\providecommand{\tightlist}{%
  \setlength{\itemsep}{0pt}\setlength{\parskip}{0pt}}
\setcounter{secnumdepth}{-\maxdimen} % remove section numbering
\ifLuaTeX
  \usepackage{selnolig}  % disable illegal ligatures
\fi
\usepackage{bookmark}
\IfFileExists{xurl.sty}{\usepackage{xurl}}{} % add URL line breaks if available
\urlstyle{same}
\hypersetup{
  hidelinks,
  pdfcreator={LaTeX via pandoc}}

\author{}
\date{\vspace{-2.5em}}

\begin{document}

\subsection{Außen 1}\label{auuxdfen-1}

Und die Geschichte: was hat sie mit Geoffrey zu tun? Er ist vielleicht
einmal aus Irland eingewandert in das schöne, neue und große Land
America, also nicht er, sondern seine Vorfahren: unsere Vorgänger. Das
bringt uns daselbst ins Spiel, wo wir es mit ihm zu tun haben werden im
angehenden 3. Jahrtausend. Er ist ein bloodbag. Seine Frau, die ihn
intuitiv mit diesem Ausdruck belegte, war sich nachher unsicher, wie sie
zu diesem Wort schließlich gekommen ist. Er selbst fand etwas von seiner
Vergangenheit darin aufgehoben, das ihn kennzeichete, aber zeichnete in
den Farben der Urahnen, deren Geschichte hier also n.~nicht beendet war:
er war der Nachkomme von Generationen kleiner, rothaariger
Irenkatholiken, so man in ihrer Konsequenz und Sturheit, dem
missionarischen Eifer und der armutsvollen Gläubigkeit in Irland und
seinen Kolonien n.~immer findet. Aber das soll nicht unser Problem sein.
Hier geht es um mehr. Die Geschichte, wie unser Kapitel anfing, sollte
handeln von seinem Ringen mit dem Schicksal wie Jakob am Jabbok, von
seiner Angst vor dem Meerkater wie vor einem flammenbewehrten Cherub,
von seiner Qual durch die Littaney (annalivia) wie durch die schwer
verständliche Zungenrede eines Süßenweintrunkenen, kurz, von all dem
soll gesprochen werden, dessen ihm Irlands beraubt war um ihn dem Land
zuzuführen, das die wahre Heimat schon immer gewesen ist, die ihm
versprochen war und die nun als Braut auf ihn wartete, של זהב, die
goldene Stadt a la fin du chemin des calices.

\end{document}
