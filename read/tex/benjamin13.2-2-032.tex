% Options for packages loaded elsewhere
\PassOptionsToPackage{unicode}{hyperref}
\PassOptionsToPackage{hyphens}{url}
%
\documentclass[
]{article}
\usepackage{amsmath,amssymb}
\usepackage{iftex}
\ifPDFTeX
  \usepackage[T1]{fontenc}
  \usepackage[utf8]{inputenc}
  \usepackage{textcomp} % provide euro and other symbols
\else % if luatex or xetex
  \usepackage{unicode-math} % this also loads fontspec
  \defaultfontfeatures{Scale=MatchLowercase}
  \defaultfontfeatures[\rmfamily]{Ligatures=TeX,Scale=1}
\fi
\usepackage{lmodern}
\ifPDFTeX\else
  % xetex/luatex font selection
\fi
% Use upquote if available, for straight quotes in verbatim environments
\IfFileExists{upquote.sty}{\usepackage{upquote}}{}
\IfFileExists{microtype.sty}{% use microtype if available
  \usepackage[]{microtype}
  \UseMicrotypeSet[protrusion]{basicmath} % disable protrusion for tt fonts
}{}
\makeatletter
\@ifundefined{KOMAClassName}{% if non-KOMA class
  \IfFileExists{parskip.sty}{%
    \usepackage{parskip}
  }{% else
    \setlength{\parindent}{0pt}
    \setlength{\parskip}{6pt plus 2pt minus 1pt}}
}{% if KOMA class
  \KOMAoptions{parskip=half}}
\makeatother
\usepackage{xcolor}
\usepackage[margin=1in]{geometry}
\usepackage{graphicx}
\makeatletter
\def\maxwidth{\ifdim\Gin@nat@width>\linewidth\linewidth\else\Gin@nat@width\fi}
\def\maxheight{\ifdim\Gin@nat@height>\textheight\textheight\else\Gin@nat@height\fi}
\makeatother
% Scale images if necessary, so that they will not overflow the page
% margins by default, and it is still possible to overwrite the defaults
% using explicit options in \includegraphics[width, height, ...]{}
\setkeys{Gin}{width=\maxwidth,height=\maxheight,keepaspectratio}
% Set default figure placement to htbp
\makeatletter
\def\fps@figure{htbp}
\makeatother
\setlength{\emergencystretch}{3em} % prevent overfull lines
\providecommand{\tightlist}{%
  \setlength{\itemsep}{0pt}\setlength{\parskip}{0pt}}
\setcounter{secnumdepth}{-\maxdimen} % remove section numbering
\ifLuaTeX
  \usepackage{selnolig}  % disable illegal ligatures
\fi
\usepackage{bookmark}
\IfFileExists{xurl.sty}{\usepackage{xurl}}{} % add URL line breaks if available
\urlstyle{same}
\hypersetup{
  hidelinks,
  pdfcreator={LaTeX via pandoc}}

\author{}
\date{\vspace{-2.5em}}

\begin{document}

\subsection{Die letzten Worte unseres H.S. am
Kreuz}\label{die-letzten-worte-unseres-h.s.-am-kreuz}

Und manche meinten etwas anderes gehört zu haben. Ich aber hörte den
Eliah rufen. Doch so weit wollten wir uns jetzt n.~nicht vorwagen, das
waren die nächsten Geschichten des nächsten Tages und die von morgen
weil morgen immer der nächste Tag gewesen war. Heute ist die Geschichte
eine andere geworden mit dem letzten Traum des Gehenkten, den ich
abgefangen hatte, aufgefangen in der liquid crystal blackbox mit den
Mitteln der Kantate, mit dem Tenor des Gustavsliedes und dem Pathos in
der Anbetung Manons. Aber dieser Gehenkte entpuppte sich als
Frau\ldots{} ist immer schon eine Frau gewesen - doch\ldots{} eine
Totgebärmaschine. Hier zu sehen sind ein paar ihrer Kinder, aufgereiht
an ihren Nabelschnüren hängend, die sich um die Hälse schlangen. Manon
aber: lebte, sollte leben und du sollst es auch! schrie sie mir ins
Gesicht, damit ich geboren werden kann.\\
Ich hingegen hielt mich an anderen Orten auf, wo es schwerfiel,
Leidenschaft wirklich zu entdecken. Institution, Bethäuser und
Pfarrstühle immer wo ich hinsah, keine Regungen in diesen verstockten
Herzen, soviele Wunder man verbringen wollte. Und Wunder waren
vollbracht. Es stand ein Werk plötzlich auf des Menschen, das mir
Eingebung wurde und Weisung. Nicht, daß ich dessen gerade bedurfte, aber
die damalige Leichtigkeit in jenem täglichen Hingang, nein: Weggang, die
ich bemerkte, schulte den Blick für das Angelegentliche. Das war eben
auch, daß man dankbar sein mußte und zwar nicht in der Demut verharrend,
aber in ihr aushaltend, bis eine Erhebung gewollt war. Also sank ich oft
auf die Knie, öfter, als ich es vorher jemals hätte ahnen können. Jeder
Tod\ldots{} der nicht dorthin ging, wo ich ihn wünschte, verlangte eine
erneute Buße, so als wären alle Gebote niemals erschienen und ich müßte
jeden Tag neu taumeln lernen und sich erheben. \emph{Erst jenseits der
Kastanien }könnten solche letzten poetischen Worte in eines Knienden
oder Hängenden Traum sein, der mich vielleicht schon viel früher
erreicht hätte, wenn ich nur die Ohren zu hören gehabt hätte damals, als
die Fontanelle n.~geöffnet war. Doch das schloß sich (bis zur
Benjaminfeldkraft) und ließ meine Hände achtlos am Körper abfallen,
statt dem Nächsten (Celan) in sein Herz zu greifen. Jetzt, im späten
Danach, gab es ja keine Urfehde, es gab staatlich verordneten Frieden
international und nur Schurkenstaaten hielten sich nicht daran. Die
durfte man aber bekämpfen, so einfach war das. Gott schütze mich vor
einem genauso wie damals, das man nicht einmal denken darf und bewahre
mir eine intuitive Schranke gegen das Gehörte, so daß ich es guten
Gewissens verurteilen will. Und das Wollen\ldots{} wir sehen jetzt, wie
sich der schwarze Hut anzieht der schwarze Rock mit Regeln zu lesen über
den Tag, die schon fast selbst Gebete sein könnten, so fest sind sie
innen in mir drin zu vernehmen. Augustinus wieder zu vergessen lerne
ich, was n.? Darauf nur Schweigen, aber das endlich dürfen, als wenn es
alles wäre, was je zu wollen möglich war, urgründliches, ein vollkommen
verstandenes Schweigen aus der Tiefe jener pia anima. Das hieß nur
ausatmen, einatmen, schlafen, essen und: toujours travailler. Wir halten
uns ja schon eine Weile auf in diesem Buch. Seamus. Ach ja. Daran muß
n.~gearbeitet werden, sagt es mir. Fange endlich an zu bauen. Es gab ein
Konstrukt, es gab die Idee der Vier Bücher, die immer ein erstes
geblieben sind, bis es mich in dieses (Winkel) hier verschlug. Wo
plötzlich der Irenkathole war. Vielleicht auch oder ganz sicher war die
erste Bewegung eine französische gewesen. In jener klein erwachsenden
Welt bildete sich eine Gesinnung aus, tout en francais, telle petit que
je ne peux pas vraiment dire si l\textquotesingle etait un esprit ou
jusqu\textquotesingle{} un truc d\textquotesingle imagination, une
danish trick dance des angehenden Existenzialisten. So weit hat mich die
Sprache schon gebracht, daß ich den Satz nicht auf Anhieb deuten konnte,
also: die Systeme wechselnd (1/9/8/9) - was niemand, der es nicht auch
so erlebt hat anders als obsolet bezeichnen würde. Er war doch der
einzige übriggeblieben aus dem französischen Ahnherd meiner Familie, dem
die Flucht nach dem Heute gelungen war. Ein Satzbau, der bereitwillig
aufgenommen wurde vom preußischen Kurfürsten und der unsere Denkweise
bis heute zu spalten vermag in eine deutsche und eine französische
Grammatik. Denn auf der anderen übermächtigen Seite fand ich mich ja dem
Iren gegenüberstehen und war mir nicht mehr sicher, wie lange ich mich
oder ob überhaupt ihnen verweigern wollte: Seamus hat seinen Anteil
längst gewonnen, aber ob ich mich wirklich eines Tages für Augustinus
entschied? Lieber zu beichten als in der peregrinatio die höhere Ehre
Gottes der eigenen Ehre zu sehen? Nej, nej, da sträubt sich etwas ganz
deutlich\ldots{} erst die Puritaner\ldots{} aber um welchen Preis! Wir
müssen zurück ins Hudson Valley, um diese Frage nicht nur iroise
beantworten zu können. Vielleicht, weil ich einmal jemanden um 20\$
betrog und einen andern um 220 Drachmen; aber das ist abgegolten oder?
Der Jäger kassierte sein Kopfgeld und der Kopf rollte - was er nicht
mehr mitansehen mußte (Demokrit) weil das alles Parallelebenen sind, wo
immer solches stattfindet. Nur in der harmlos primitiven Jetztzeit
gliedern sich die Den Satz strukturierenden Bauelemente eben so, und ich
möchte einmal darauf achten, was da so vor sich geht.\\
Haben wir aufgepaßt? Damals? Als wir n.~so klein waren? 1. Wir gingen
jeden Tag da raus, wo alle anderen auch hingingen mit den
aufgeschnallten Lederranzen, im Winter auch auf Schneegleitern über die
zugeschneiten Straßen, sommers oft mit dem Fahrrad später. Also laßt uns
zählen: 3,11,1,4,3,4. Um auf die 5.x zu kommen jener kleinen, kleinen
Kraft, die uns die Benjaminfeldkraftldkonstante bereitstellte, müssen
wir uns n.~etwas anstrengen, wir lagen leider nur bei 4.3333. Aber
ergibt sich vielleicht etwas anders sinnvolles daraus? Ebenmaß in allen
äußeren Angelegenheiten, so daß uns fast schon schwerfällt, uns zu
erinnern, wie wir den Satz begonnen haben, allein, wir hätten ihm, ihm
(und wenn ich sage er, so meine ich Ihn, Ihn\ldots) eine Dankbarkeit
gegenüber empfunden, die aus mehr bestanden hätte denn aus einer
schlichten Dezimalzahl: ich wüßte nicht mehr, wie ich nur einen dieser
Sätze überhaupt habe schreiben können ohne die Kraft. Aber es ist ja
irgendwie gegangen. Auch ohne Kontrapunkt. Den gibt es jetzt, die Fuge
erhält sich von selbst zwischen den gefürchteten Weißbuchstaben. Ich muß
weiter ausholen: Einmal ging ich einen Wald spazieren, das war ein
Urwald, und jemand hatte ihn gemacht, weil er es ein Geschenk sein
sollte so, daß sich niemand wirklich darin zurechtfinden würde außer, er
äße von einem Baum, der war genannt nach der Erkenntnis von gut und
böse. Ich irrte also herum darin ohne Weg und Ziel und hatte mir bald
recht schön einen Katalog angelegt von allem, was mir begegnete. Aber
ich konnte nichts benennen, das war mir zu schade. Also bat ich, daß mir
jemand einen dieser beschriftetet Äpfel da geben möge mit den Zeichen
zur Erklärung; so klein war ich aber, daß ich diese n.~nicht lesen
konnte. Und so aß ich einen der war giftig (gewesen) und vergiftete mich
an dem Wurm, der darin kroch. Denn der war unrein: aber das wußte ich
jetzt\ldots{} ich könnte immer so weiter erzählen, jedoch unterbreche
ich hier die Substanz für den Einschub: wir sind bei einer Konstante von
10.1666. Und das nur, weil ich mich nicht auf die deutsche Grammatik
verließ, sondern zuhilfe nahm, was aus fremden Sprachen in mir
schlummerte; ohne selbst erst zu wissen, welche das wohl wären. (Der
alte Noah Utnapischtim jener vollgreise Steineleser aus den Zeiten
von-vor-der-Flut hatte das bewirkt in den geretteten Geschöpfen: ein
Zusammengehörigkeitsgefühl, weil man gemeinsam der Katstrophe entronnen
war. Würde sich heute, wer weiß unser Nachbar sein die nächsten
Jahrtausende\ldots)\\
Ich komme jetzt zu den letzten fünf streitbaren Sätzen dieses
allgegenwärtigen Buches zweiten Bandes ersten Teils: ob man sie so nennt
oder anders: immer sind entweder wir gemeint oder die anderen und so
lange wir uns n.~nicht selber heiligsprechen können, sind es immer die
Anderen. Und also wird es immer weiter Streit geben: bis \emph{wir}
alles abgelegt haben:\\
Ich trage eine Kappe, die mir das Denken ermöglicht in diesen heiligen
Hallen.\\
Im Hof stürzen mit gespenstischem Lärm Eisschollen vom Dach. Der Wind
weht ein bißchen, oft wird er laut.\\
Aber man kann glauben, davon nichts wissen zu wollen. Ich bin einmal den
Elijahu suchen gegangen und jetzt erinnere ich mich an seine ersten
Worte, so wahr ich hier stehe unter den Lebenden. Kostbare Worte, eines
für das andere; ich hebe sie mir für später auf. Wenn ihr auf sie stoßt,
denkt daran, daß ich sie hörenkann wie aus der eigenen Stimme, die
riefe. Und das ist nur ein Appell. Der andere hört gleich auf.\\
\emph{Der wiedergefundene Ort }ist jetzt wichtig und darf nicht
überlesen werden. Es gibt sie irgendwo als Zeit. Zwischen den
Glockenschlägen habe ich die Viertelstunden, um den Gedanken zu
behalten. Aber das gelingt nicht sofort, meistens also wenn ich dann
hier saß habe ich etwas davon verbraucht, das ich hierher mitnahm. Wie
Tabak, den es nur in der Stadt gab und Roten Karkade, vielleicht Tinte,
mit Sicherheit Papier. Die Schreibmaschine ist befüllt und die Bücher
zählen nicht als Ration; aber das vergängliche Material -- davon darf
ich nur wenig mit zurück nehmen. Und so finden wir uns hier ein.

\end{document}
