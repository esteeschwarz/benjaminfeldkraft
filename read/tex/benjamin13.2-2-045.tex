% Options for packages loaded elsewhere
\PassOptionsToPackage{unicode}{hyperref}
\PassOptionsToPackage{hyphens}{url}
%
\documentclass[
]{article}
\usepackage{amsmath,amssymb}
\usepackage{iftex}
\ifPDFTeX
  \usepackage[T1]{fontenc}
  \usepackage[utf8]{inputenc}
  \usepackage{textcomp} % provide euro and other symbols
\else % if luatex or xetex
  \usepackage{unicode-math} % this also loads fontspec
  \defaultfontfeatures{Scale=MatchLowercase}
  \defaultfontfeatures[\rmfamily]{Ligatures=TeX,Scale=1}
\fi
\usepackage{lmodern}
\ifPDFTeX\else
  % xetex/luatex font selection
\fi
% Use upquote if available, for straight quotes in verbatim environments
\IfFileExists{upquote.sty}{\usepackage{upquote}}{}
\IfFileExists{microtype.sty}{% use microtype if available
  \usepackage[]{microtype}
  \UseMicrotypeSet[protrusion]{basicmath} % disable protrusion for tt fonts
}{}
\makeatletter
\@ifundefined{KOMAClassName}{% if non-KOMA class
  \IfFileExists{parskip.sty}{%
    \usepackage{parskip}
  }{% else
    \setlength{\parindent}{0pt}
    \setlength{\parskip}{6pt plus 2pt minus 1pt}}
}{% if KOMA class
  \KOMAoptions{parskip=half}}
\makeatother
\usepackage{xcolor}
\usepackage[margin=1in]{geometry}
\usepackage{graphicx}
\makeatletter
\def\maxwidth{\ifdim\Gin@nat@width>\linewidth\linewidth\else\Gin@nat@width\fi}
\def\maxheight{\ifdim\Gin@nat@height>\textheight\textheight\else\Gin@nat@height\fi}
\makeatother
% Scale images if necessary, so that they will not overflow the page
% margins by default, and it is still possible to overwrite the defaults
% using explicit options in \includegraphics[width, height, ...]{}
\setkeys{Gin}{width=\maxwidth,height=\maxheight,keepaspectratio}
% Set default figure placement to htbp
\makeatletter
\def\fps@figure{htbp}
\makeatother
\setlength{\emergencystretch}{3em} % prevent overfull lines
\providecommand{\tightlist}{%
  \setlength{\itemsep}{0pt}\setlength{\parskip}{0pt}}
\setcounter{secnumdepth}{-\maxdimen} % remove section numbering
\ifLuaTeX
  \usepackage{selnolig}  % disable illegal ligatures
\fi
\usepackage{bookmark}
\IfFileExists{xurl.sty}{\usepackage{xurl}}{} % add URL line breaks if available
\urlstyle{same}
\hypersetup{
  hidelinks,
  pdfcreator={LaTeX via pandoc}}

\author{}
\date{\vspace{-2.5em}}

\begin{document}

\subsection{guhl}\label{guhl}

jemand sprach von "mitmurmeln\ldots"? Daran kann ich mich nicht
aufrichten. Die Hölderlinverse, ja, wenn einer sie hersagte, die
murmelte ich mit; aus Freude, weil ich etwas gekonnt habe, das mehr war
als Blumen streuen. Aber Gebete habe ich nicht. Die Brücken sind schmal
und hölzerne, dünne Stege, die keiner gefahrlos betritt. Immer setzt man
sich aus, wenn man einen solchen Weg wählt; nur, niemand wählt ihn sich
ja wirklich, sondern wird hinzugezogen durch dieses Graben.\\
Da sehen sie wie die Vorangehenden dem Vorangegangenen etwas n.~zu geben
versuchen, eine Gebärde tun wollen. Aber schwach, so schwach sind sie da
und deshalb hölzerne dünne Pfade, auf denn sie gehen und spüren, daß es
sie gradn. hält für die Handlung, die keinen Willen duldet keine
Richtung kein Drängen und keine Zeit läßt, zu überlegen. Wir seen wie
sie es versuchen. Das Pfeifen hören, sie kennt die Worte. Vielleicht
trieb das sogar meine Arme, pater: die Liturgie, die nicht
verheimlichte, Liturgie. Dann doch, der redet und weiß, wasser sagt und
der es sagt und nicht nochmal. Hölderlin bei HB, das war gut gestottert,
oder? Er wußte das. Nur wer schon. Ich mit -- eingefallenen Händen und
unwegsamen Lippen. Ein weiteres, die Nonem. Aber mitmurmeln? Selten denk
ich, aber gerade, nur dann, zählts hinzu wie verweigertes Gelächter.
Gehen lernen und ernstsein. Es passierin Fehler und ich merke sie mir:\\
-- lernen sei "nur als Aufgabe der Kunst eine Kunst" verriet HB. Hier
muß ich mich erinnern, weil die Lebensdaten jener meiner beiden
Vordenker sich überschneiden, als ich aus dem Harz zurückkomme. Der
andere (die eine der sich ineinanderstarrenden Unendlichkeiten) gibt mir
weiter Lernaufgaben über was n.~zu erfahren war und HB konnte es nicht
mehr, weil er voranging. Aber Fragen hat er hinterlassen, genug, um mein
Leben lang mich ihnen widmen zu können und also setze ich Kraft darein
aus seinen Schriften. Doch er wurde irgendwann politisch, das greift
mich an, der ich es nicht bin. Muß man es aber? Man sagt ja und nichts
ist nicht dafür. Dann weigere ich mich, sage nein und habe ein reineres
Gewissen als eine falsche Moral und gezüchtete Ansichten. Doch das ist
schon viel zu viel.\\
-- sagen Sie n.~nein, anstatt zuzuschlagen? Das sind total überholte
Denkmuster. Dies wissen, haben sich längst darauf geeinigt, daß der
Erstschlag die einzige Möglichkeit ist, einer Vernichtung zu entgehen.
Deshalb soll die Politik draußen bleiben, die keiner verdient. Mach also
das Buch zu das du eben aufgeschlugst um wenigstens etwas davon zu
wissen, worüber sonFuror und Lossprechung erteilt und zurückgezogen und
Sprache ermächtigt und entmächtigt als Hetze. Dann doch Nein und muß mir
selbst diese Zungenrede verbieten, weil es in mir nach Fortgang schreit
und das somnambule Hindämmern durch ungewußte Bewußtseinszustände eine
Entwicklung unmöglich macht, deren Wurzel also befreit werden muß aus
meinem Hirn stattd. angelegentlich: ich leider doch anfange von
Heisenberg über Quandten zu lesen. Was ja immer meine Befürchtung war,
trat also ein als mache nun möglich weil ich es jetzt lese, daß ich in
der Vergangenheit davon weiß und den Zusammenhang in diesem Hirn, das
die Zeitlichkeit überhaupt nicht kennt und molekular gleich in was
hinein Vergangenheit oder Zukunft dächte man etwa. Und so lese ich in
den Quarten und mir selbst meine Zeit zurecht. Und laß mich also gehen
an die Punkte, wo ich mir damals den Zuspruch gewünscht habe eines
Zukünftigen, der mir sagt, daß alles gut w. und die Zeit es richte. Ist
es gerichtet jetzt? Ja, ist es. Siamesisch ist nur die Nacht zwischen
der Woche, die man also grad übersteht in der Vorfreude kommender
gerechter Freiheit. Bezahlte Arbeit ist hierabgezogen von dem, was man
gerne, alles andere wäre Berufung getan. Wenn ich hier sitzen studieren
und das eben aufschreiben erarbeiten will, ist die Nacht der Verbündete,
den ich dazu brauch, und was sie mir nimmt ist mein Feind. Also ist die
A. mein Feind so sehr ich sie zum Leben brauche; so sehr ich sie zum
Leben liebe anders zu denken ist Illusion.\\
-- aber was führt davon zu Seamus? Seine irische Anmut? Die Idee war ein
Geschenk, das ich nicht die Kraft haben würde abzulehnen. Deshalb mußte
ich ihr zuhören und alles an ihm hat sich nach mir umgewendet, weil er
mein Hören studierte, wie sie das auch anstellen mochte, aber er tat es,
die Geschichte.

\end{document}
