% Options for packages loaded elsewhere
\PassOptionsToPackage{unicode}{hyperref}
\PassOptionsToPackage{hyphens}{url}
%
\documentclass[
]{article}
\usepackage{amsmath,amssymb}
\usepackage{iftex}
\ifPDFTeX
  \usepackage[T1]{fontenc}
  \usepackage[utf8]{inputenc}
  \usepackage{textcomp} % provide euro and other symbols
\else % if luatex or xetex
  \usepackage{unicode-math} % this also loads fontspec
  \defaultfontfeatures{Scale=MatchLowercase}
  \defaultfontfeatures[\rmfamily]{Ligatures=TeX,Scale=1}
\fi
\usepackage{lmodern}
\ifPDFTeX\else
  % xetex/luatex font selection
\fi
% Use upquote if available, for straight quotes in verbatim environments
\IfFileExists{upquote.sty}{\usepackage{upquote}}{}
\IfFileExists{microtype.sty}{% use microtype if available
  \usepackage[]{microtype}
  \UseMicrotypeSet[protrusion]{basicmath} % disable protrusion for tt fonts
}{}
\makeatletter
\@ifundefined{KOMAClassName}{% if non-KOMA class
  \IfFileExists{parskip.sty}{%
    \usepackage{parskip}
  }{% else
    \setlength{\parindent}{0pt}
    \setlength{\parskip}{6pt plus 2pt minus 1pt}}
}{% if KOMA class
  \KOMAoptions{parskip=half}}
\makeatother
\usepackage{xcolor}
\usepackage[margin=1in]{geometry}
\usepackage{graphicx}
\makeatletter
\def\maxwidth{\ifdim\Gin@nat@width>\linewidth\linewidth\else\Gin@nat@width\fi}
\def\maxheight{\ifdim\Gin@nat@height>\textheight\textheight\else\Gin@nat@height\fi}
\makeatother
% Scale images if necessary, so that they will not overflow the page
% margins by default, and it is still possible to overwrite the defaults
% using explicit options in \includegraphics[width, height, ...]{}
\setkeys{Gin}{width=\maxwidth,height=\maxheight,keepaspectratio}
% Set default figure placement to htbp
\makeatletter
\def\fps@figure{htbp}
\makeatother
\setlength{\emergencystretch}{3em} % prevent overfull lines
\providecommand{\tightlist}{%
  \setlength{\itemsep}{0pt}\setlength{\parskip}{0pt}}
\setcounter{secnumdepth}{-\maxdimen} % remove section numbering
\ifLuaTeX
  \usepackage{selnolig}  % disable illegal ligatures
\fi
\usepackage{bookmark}
\IfFileExists{xurl.sty}{\usepackage{xurl}}{} % add URL line breaks if available
\urlstyle{same}
\hypersetup{
  hidelinks,
  pdfcreator={LaTeX via pandoc}}

\author{}
\date{\vspace{-2.5em}}

\begin{document}

\subsection{k}\label{k}

Die zwei Textformen, die zur Auswahl stehen, sind: ein gerader, aus zwei
Richtungen dem jeweiligen Außenliegenden zustrebender und selbstgefügter
Verlauf von einem links initiierten Anfang des "Wortes" hin zur rechts
liegenden ungewissen Begrenzung, welche erst dann evident wird, wenn das
Wort einen Sinn erfüllt; zweitens eine in der Deutung des Autoren
liegende willkürliche Ausführung einer nur ihm bekannten Initiale, deren
Niederschlag im Schriftbild ersichtlich wird durch die schwache
Veränderung im Duktus, d.h. nur in ihrer Abhebung von der vorläufig
beendeten Initiale. Für welche Variante sich zu entscheiden ist bei der
Erstellung des Textes richtet sich nach der angestrebten Wirkung. Soll
dieser dazu dienen, einen Inhalt zu \emph{vermitten,} der für sich
allein Gültigkeit beanspruchen kann und welcher vergleichbar ist mit
anderen Inhalten anderer Versuche und der in einem gewissen Sinne
austauschbar ist, heißt: der sich nicht an eine Form binden muß, so
führt sicherlich die erste Textform schneller und eindeutiger ans Ziel,
das heißt aber nicht, daß unbedingt von ihr Gebrauch zu machen ist.
Vielmehr wird im Umgang mit der fixierten Sprache gerade der langsame
Austausch zu dem, was schlechthin erlaubt, das Gesprochene weiterzugeben
oder besser: Wiederherzugeben. Langsam in diesem Fall deshalb, weil in
der Subjektsuche, auf die sich der Leser im Falle eines
Einverständnisses mit der erzählten Geschichte (des Inhaltes)
zwangsläufig begibt, ein Findemoment sich erst dann einstellt, wenn
d.~physische Auffassung nahezu alle Erklärungen anzunehmen bereit ist,
d.~ihr den Text aufzuschließen als in der Lage erscheinen. Daß im
negativen Anteil des Gewußten der umfassendere Sinn sich dadurch aber
herstellt - weil er nach außen hin neutral ist (ein neutraler Rest am
negativ reagierenden Körper) - ist Glück sowohl als auch Verurteilung
zum wirklichen Finden des Subjektes. Glück: vielleicht liegt im Ende
durch Einmaligkeit auch die Sicherheit ihrer Konsequenz; daß es also
keine weitere Möglichkeit gäbe, dieses auszusprechen und anzuhören,
dieses weiterzugeben oder erst: irgendwo wirklich werden zu lassen
außerhalb des Denkens. Verurteilt, weil gleichzeitig auch jene schwache
Autorität, die es durch die Konsequenz erfährt, den Druck ausübt, es aus
seiner Autonomie zu erheben und Gesetz werden zu lassen. Warum wehrt es
sich aber gegen die Erhebung? Gibt es doch Verstehen? Ließe man sich auf
eine bestimmte Sprache ein, die selbst zu wählen gewesen wäre: würde man
sich denn übernehmen, wenn es einmal nicht die eigene ist? Im fremden,
zwar ohne Widerstand gehandhabten Gerät das der Geist einem vorgibt sind
die Möglichkeiten zu einer wirklich erklärbaren Welt begrenzt. Im
eigenen jedoch: im verurteilten zur steten Selbstdiagnose und an die
Handlung geknüpften inhaltlichen Werdegang des zu äußernden Materials -
dort lassen sich immer dieselben Matrizen verwenden, um jegliche
Operation auszuführen; zwar sind auch diese in ihren Dimensionen n-fach
beschränkt, aber es ist eine Beschränkung, die der menschliche Geist
sehr langsam durchbrechen wird.\\
Der siamesischen, die unserer Epoche unmittelbar vorausging, waren
Maschinen, wie die Menschheit auch erfand und überall da einsetzte, wo
sie etwas zu sehr anstrengte ein Geschenk des Genius, das man nicht
ablehne. Daß man also davon Gebrauch zu machen pflegte und jedes als
Segen ansah, was einem die Arbeit erleichterte ohne den Zweifel zu
haben, der heute schon als normal gegenüber der Technik angesehen wird,
brachte Innovationen gleich naiv hervor, die uns hier ankommen ließen wo
wir uns jetzt Engelabendland befinden:

\end{document}
