% Options for packages loaded elsewhere
\PassOptionsToPackage{unicode}{hyperref}
\PassOptionsToPackage{hyphens}{url}
%
\documentclass[
]{article}
\usepackage{amsmath,amssymb}
\usepackage{iftex}
\ifPDFTeX
  \usepackage[T1]{fontenc}
  \usepackage[utf8]{inputenc}
  \usepackage{textcomp} % provide euro and other symbols
\else % if luatex or xetex
  \usepackage{unicode-math} % this also loads fontspec
  \defaultfontfeatures{Scale=MatchLowercase}
  \defaultfontfeatures[\rmfamily]{Ligatures=TeX,Scale=1}
\fi
\usepackage{lmodern}
\ifPDFTeX\else
  % xetex/luatex font selection
\fi
% Use upquote if available, for straight quotes in verbatim environments
\IfFileExists{upquote.sty}{\usepackage{upquote}}{}
\IfFileExists{microtype.sty}{% use microtype if available
  \usepackage[]{microtype}
  \UseMicrotypeSet[protrusion]{basicmath} % disable protrusion for tt fonts
}{}
\makeatletter
\@ifundefined{KOMAClassName}{% if non-KOMA class
  \IfFileExists{parskip.sty}{%
    \usepackage{parskip}
  }{% else
    \setlength{\parindent}{0pt}
    \setlength{\parskip}{6pt plus 2pt minus 1pt}}
}{% if KOMA class
  \KOMAoptions{parskip=half}}
\makeatother
\usepackage{xcolor}
\usepackage[margin=1in]{geometry}
\usepackage{graphicx}
\makeatletter
\def\maxwidth{\ifdim\Gin@nat@width>\linewidth\linewidth\else\Gin@nat@width\fi}
\def\maxheight{\ifdim\Gin@nat@height>\textheight\textheight\else\Gin@nat@height\fi}
\makeatother
% Scale images if necessary, so that they will not overflow the page
% margins by default, and it is still possible to overwrite the defaults
% using explicit options in \includegraphics[width, height, ...]{}
\setkeys{Gin}{width=\maxwidth,height=\maxheight,keepaspectratio}
% Set default figure placement to htbp
\makeatletter
\def\fps@figure{htbp}
\makeatother
\setlength{\emergencystretch}{3em} % prevent overfull lines
\providecommand{\tightlist}{%
  \setlength{\itemsep}{0pt}\setlength{\parskip}{0pt}}
\setcounter{secnumdepth}{-\maxdimen} % remove section numbering
\ifLuaTeX
  \usepackage{selnolig}  % disable illegal ligatures
\fi
\usepackage{bookmark}
\IfFileExists{xurl.sty}{\usepackage{xurl}}{} % add URL line breaks if available
\urlstyle{same}
\hypersetup{
  hidelinks,
  pdfcreator={LaTeX via pandoc}}

\author{}
\date{\vspace{-2.5em}}

\begin{document}

\subsection{guhl}\label{guhl}

Die Schrift hat ihrn Zenit erreicht - und wird abgelöst von dem, was
heute zwischen den Bildern (Ideen) und ihr steht: der Sprache. Damit
meine ich nicht Sprache als lautende Schrift. Es wäre schön, wenn es
damit bewendet sei. Aber hier ist Sprache als etwas anderes gemeint: Als
Mittler zwischen den Symbolen; nicht mehr als Mittel, dessen man sich
bedient, um Symbole auszudrücken. Es ist scheinbar so geworden, daß sie
sich zu einem im Austausch zwischen den die Wirklichkeit
symbolisierenden Zeichen untereinander in ihren konkreten Entsprechungen
entstehenden Feld entwickelt hat, das sie nur n.~transzendieren muß, um
neue Erkenntnisse aufzunehmen; aber eben nicht mehr wie ehedem, um sich
selbst fortzuentwickeln. Das meint: ihr Zenit ist erreicht. Und das
heißt, daß sie sich im Regress befindet und das Unbekannte nicht mehr
länger das solche ist, es ist nur n.~nicht ausgelotet. Aber ein Lot,
welches die ausreichende Tiefe oder Weite hätte, könnte ihr Ende
bestimmen. Und das bringt uns zur zweiten Textform: Initialen, die sich
selbst vollenden, wenn mit einer notwendigen Geschwindigkeit ausgeführt.
Aus dieser läßt sich in Beziehung zum bemessenen Gesamtumfang ein
statistischer Wert für einen folgenden dritten Wortbuchstaben errechnen.
Da dieser bereits bis zu einem genau zu ermittelnden Überhang in die
zweite Worthälfte hineinreicht, ist die Wahrscheinlichkeit für jeden
folgenden Buchstaben immer größer als 50 Prozent und damit also
unbedingt treffend. Die große Frage ist: können wir uns darauf
einlassen, daß nur die Geschwindigkeit, also die Sicherheit, mit der die
Initialen vergeben werden, ausschlaggebend ist für den weiteren
inhaltlichen Verlauf oder ist es eine beängstigende Vorstellung, daß wir
so weitgehend behoben sind unseres Zutuns, daß eine Maschine mit der
Fähigkeit, die Initialen zu vergeben, ebensolches schaffen könnte?

\end{document}
