% Options for packages loaded elsewhere
\PassOptionsToPackage{unicode}{hyperref}
\PassOptionsToPackage{hyphens}{url}
%
\documentclass[
]{article}
\usepackage{amsmath,amssymb}
\usepackage{iftex}
\ifPDFTeX
  \usepackage[T1]{fontenc}
  \usepackage[utf8]{inputenc}
  \usepackage{textcomp} % provide euro and other symbols
\else % if luatex or xetex
  \usepackage{unicode-math} % this also loads fontspec
  \defaultfontfeatures{Scale=MatchLowercase}
  \defaultfontfeatures[\rmfamily]{Ligatures=TeX,Scale=1}
\fi
\usepackage{lmodern}
\ifPDFTeX\else
  % xetex/luatex font selection
\fi
% Use upquote if available, for straight quotes in verbatim environments
\IfFileExists{upquote.sty}{\usepackage{upquote}}{}
\IfFileExists{microtype.sty}{% use microtype if available
  \usepackage[]{microtype}
  \UseMicrotypeSet[protrusion]{basicmath} % disable protrusion for tt fonts
}{}
\makeatletter
\@ifundefined{KOMAClassName}{% if non-KOMA class
  \IfFileExists{parskip.sty}{%
    \usepackage{parskip}
  }{% else
    \setlength{\parindent}{0pt}
    \setlength{\parskip}{6pt plus 2pt minus 1pt}}
}{% if KOMA class
  \KOMAoptions{parskip=half}}
\makeatother
\usepackage{xcolor}
\usepackage[margin=1in]{geometry}
\usepackage{graphicx}
\makeatletter
\def\maxwidth{\ifdim\Gin@nat@width>\linewidth\linewidth\else\Gin@nat@width\fi}
\def\maxheight{\ifdim\Gin@nat@height>\textheight\textheight\else\Gin@nat@height\fi}
\makeatother
% Scale images if necessary, so that they will not overflow the page
% margins by default, and it is still possible to overwrite the defaults
% using explicit options in \includegraphics[width, height, ...]{}
\setkeys{Gin}{width=\maxwidth,height=\maxheight,keepaspectratio}
% Set default figure placement to htbp
\makeatletter
\def\fps@figure{htbp}
\makeatother
\setlength{\emergencystretch}{3em} % prevent overfull lines
\providecommand{\tightlist}{%
  \setlength{\itemsep}{0pt}\setlength{\parskip}{0pt}}
\setcounter{secnumdepth}{-\maxdimen} % remove section numbering
\ifLuaTeX
  \usepackage{selnolig}  % disable illegal ligatures
\fi
\usepackage{bookmark}
\IfFileExists{xurl.sty}{\usepackage{xurl}}{} % add URL line breaks if available
\urlstyle{same}
\hypersetup{
  hidelinks,
  pdfcreator={LaTeX via pandoc}}

\author{}
\date{\vspace{-2.5em}}

\begin{document}

\subsection{HB}\label{hb}

Bleiben wir erst einmal bei der Angst: was wäre das Werk, an dem wir
dann also schrieben die Jahre über und wenn wir es irgendwann meinten,
beenden zu müssen und eine Verlautbarung bevorstünde bzw. notwendig
würde - was wären die Jahre uns wert, die wie an seiner Herstellung
gearbeitet hätten und nun Möglichkeiten, daß es über uns hinauswüchse,
gebraucht wären? Wie weit müßte sein Schatten reichen, damit jemand
glaubte, es veröffentlichen zu müssen? Nur bis zum nächsten Werk? Ja. In
jenem nur findet sich der Schlüssel, der dem Lektor das vorliegende zu
erklären vermag. Wort für Wort, Satz für Satz, Buch um Buch. Und genau
so liegen am Anfang die zwei autonomen Initialen, die wir uns zu
vergeben stets scheuten und darum dauerte es lange und länger immer, bis
man an ein neues Ende kam. Doch es ist ja diesmal keine Schreibmaschine
und also ist das Zögern unnötig, es ist auch kein Schönschrifteintrag
ins Poesiealbum der zuerst Angebeteten, wir können uns erlauben,
rücksichtslos die Buchstaben zu wählen. Dann lernt das System dazu und
"ist bei der weiteren Auswahl behilflich" klingt so spröde wie ein
Einkaufssatz\ldots{} Deshalb ist die Angst berechtigt, so einfach es
sich anhört: wenn wir uns auf der absteigenden Hangseite bewegen, kuckt
man nicht mehr nach oben, sondern nur n., wie schlimm man fallen könnte,
wenn man den sicheren Stand verläßt. Darum ist das nächste Werk der
einzige Preis, den wir zahlen können aus den bisher verdienten Mitteln.
Wir befinden uns damit sowohl immer am Ende und Anfang eines
hermeneutischen Zirkels, dessen Schluß nur dahin lauten kann, daß eben
die Aufgabe sich selbst zur Aufgabe hat. Und das ist merkwürdigerweise
keine Tautologie. Es wäre eine, wenn es heißen würde: Kunst ist nur bei
Aufgabe der Kunst eine Aufgabe. Aber ich sage:\\
Kunst ist nur als Aufgabe der Kunst eine Aufgabe. Und: es gibt hier
keine Kursiven mehr\ldots{} die immerhin Sicherheit waren.\\
Reicht uns unsere selbst S. mit den Initialen zu verfahren, um dem
Inhalt eine von nur von uns gewollte Richtung zu geben? Wenn wir diese
Frage laut und deutlich bejahen können, bevor sich ein anderer Gedanke
dazwischen einfindet, also ja sagen in all unseren Fehlern, die damit
einhergehen, wenn wir den Terminus selbst bestimmen wollen: dann bleibt
es unser. Also sagen wir diesmal, für heute: ja. Auch wenn er nicht
entgleitet. Auch wenn es ohne Poesie zu sein scheint - vielleicht stellt
sie sich ein, wenn wir zu sehr versuchten, uns ihrer zu bemächtigen. Das
mußte schwach bleiben, wenn es keine Endgültigkeit mehr gäbe. Die wir
hiermit erreichen und eigentlich sogar erstmals in einer echten
Konsequenz: daß ich nicht wissen kann, ob etwas von dem, was gesprochen,
hinübergerettet werden wird in die Schrift.

\end{document}
