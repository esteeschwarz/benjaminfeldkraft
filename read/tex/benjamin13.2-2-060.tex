% Options for packages loaded elsewhere
\PassOptionsToPackage{unicode}{hyperref}
\PassOptionsToPackage{hyphens}{url}
%
\documentclass[
]{article}
\usepackage{amsmath,amssymb}
\usepackage{iftex}
\ifPDFTeX
  \usepackage[T1]{fontenc}
  \usepackage[utf8]{inputenc}
  \usepackage{textcomp} % provide euro and other symbols
\else % if luatex or xetex
  \usepackage{unicode-math} % this also loads fontspec
  \defaultfontfeatures{Scale=MatchLowercase}
  \defaultfontfeatures[\rmfamily]{Ligatures=TeX,Scale=1}
\fi
\usepackage{lmodern}
\ifPDFTeX\else
  % xetex/luatex font selection
\fi
% Use upquote if available, for straight quotes in verbatim environments
\IfFileExists{upquote.sty}{\usepackage{upquote}}{}
\IfFileExists{microtype.sty}{% use microtype if available
  \usepackage[]{microtype}
  \UseMicrotypeSet[protrusion]{basicmath} % disable protrusion for tt fonts
}{}
\makeatletter
\@ifundefined{KOMAClassName}{% if non-KOMA class
  \IfFileExists{parskip.sty}{%
    \usepackage{parskip}
  }{% else
    \setlength{\parindent}{0pt}
    \setlength{\parskip}{6pt plus 2pt minus 1pt}}
}{% if KOMA class
  \KOMAoptions{parskip=half}}
\makeatother
\usepackage{xcolor}
\usepackage[margin=1in]{geometry}
\usepackage{graphicx}
\makeatletter
\def\maxwidth{\ifdim\Gin@nat@width>\linewidth\linewidth\else\Gin@nat@width\fi}
\def\maxheight{\ifdim\Gin@nat@height>\textheight\textheight\else\Gin@nat@height\fi}
\makeatother
% Scale images if necessary, so that they will not overflow the page
% margins by default, and it is still possible to overwrite the defaults
% using explicit options in \includegraphics[width, height, ...]{}
\setkeys{Gin}{width=\maxwidth,height=\maxheight,keepaspectratio}
% Set default figure placement to htbp
\makeatletter
\def\fps@figure{htbp}
\makeatother
\setlength{\emergencystretch}{3em} % prevent overfull lines
\providecommand{\tightlist}{%
  \setlength{\itemsep}{0pt}\setlength{\parskip}{0pt}}
\setcounter{secnumdepth}{-\maxdimen} % remove section numbering
\ifLuaTeX
  \usepackage{selnolig}  % disable illegal ligatures
\fi
\usepackage{bookmark}
\IfFileExists{xurl.sty}{\usepackage{xurl}}{} % add URL line breaks if available
\urlstyle{same}
\hypersetup{
  hidelinks,
  pdfcreator={LaTeX via pandoc}}

\author{}
\date{\vspace{-2.5em}}

\begin{document}

\subsection{M}\label{m}

Licht-wellen: ein Geström von Partikularansätzen versucht, die Richtung
zu ändern. Es werden Aufgaben verbreitet zur Spitze hin, dort delegiert
man, die untern abgeordneten Wesenheiten wissen nicht genau, was aus den
Vorstößen wurde, bis das Gesetz galt. Verbote treten dann plötzlich in
Kraft, wo man sie nicht erwartete, weil sie n.~nicht vorstellbar waren.
Doch die Strafen waren darum nicht leichter. Wir lernten (auf der
untersten Ebene) den Anweisungen zu gehorchen, die an die Basis
ergingen. Viel mehr blieb uns nicht übrig. Ich gebe etwas weiter davon
hier, was den Nachkommenden erleichtern soll, mit dem Bestrafungssystem
des Apparates zurechtzukommen, wie ich es bisher begriffen habe.\\
Immer ist von den übrigen Menschen die Rede gewesen. Wenn man es aber
einmal sich erlaubt, den Blick zu heben über den Horizont des von der
Demut geforderten hinaus wird schnell ersichtlich, daß es nicht einen
weiteren Menschen überhaupt gibt, der über einen selbst erweitert wäre.
Wie Fortsätze des eigenen Bewußtseins erscheinen mir die anderen. Warum
ist das so? Ich glaube, daß es einmal vor Zeiten eine Spaltung gab in
ein Innen und ein Außen des Menschen, die jetzt aufgehoben ist. Doch
auch wenn ich dafür dankbar sein sollte, daß diese Spaltung nicht mehr
existiert und wir eins sind mit unserem und in unserem Geist und der
Körper ein funktionierendes Gerät ist, das jenem seinen Ausdruck
erlaubt, so würde ich gern erfahren haben wie es war, als die Worte
Dualismus oder Plural n.~Bedeutungen hatten. Wir können hier es nicht
mehr verstehen, die Einsheit ist so umfassend, daß es nicht vorstellbar
ist, wie es damals war. Die Alten haben ein Wissen zusammengetragen und
es niedergelegt, ich sah das Papier, auf welchem alles aufgezeichnet
wurde. Aber selbst n.~einmal auf Papier etwas verzeichnen? Wie könnte
ich das tun, wo doch alles, was festgehalten wird wie ein Vorwand
erscheint, etwas von sich abzutrennen: und das haben wir doch erkannt
ist pathologisch. Das Archiv, in dem ich die Nächte verbringen darf,
wenn das Tagesgeschehen sich verliert, ist hinter mir zugesperrt worden
und man läßt niemanden ein. Also habe ich von Sonnenuntergang bis zum
Aufgang Zeit etwas zu finden, jede Nacht. Finden und nur meinem Blick
ausgesetzt sind die auf den Monitoren haftenden Elektronenstrahlen. Da
bilden sich Funken ab, bis sie überspringen und das bekannte Geräusch
erzeugen im Innenohr, das ich schon so lange geöffnet halte und endlich
ist etwas zu hören. Immer wollte ich es und setzte mich aus und wartete
auf die Funken aber nie war da was, das sich beschreiben ließ. Doch das
ist jetzt anders. Die Zeilenfunktion der Augen ist aufgehoben und sie
nehmen sie wahr: die Gitter um jedes einzelne gesprochene Wort. Das war,
was nun verlangt wurde, ausgedrückt zu werden. Also sollte jemand
sprechen kommen, am besten von selbst zu mir, so daß ich ihn auswählen
könnte und dabei wüßte: diese Sprache ist das eingeständige an ihm und
egal, wer er ist mit welchem Hintergrund und wie erfahren in der
Ausführung von Beschreibungen. Das wird alles ersteinmal unwichtig sein,
wichtig nur allein wäre, daß es eine Stimme gibt die etwas von dem sagt,
was jetzt gesagt werden muß. Was ist es? Und wer wird es sein?

\end{document}
