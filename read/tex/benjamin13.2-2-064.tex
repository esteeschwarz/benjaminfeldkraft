% Options for packages loaded elsewhere
\PassOptionsToPackage{unicode}{hyperref}
\PassOptionsToPackage{hyphens}{url}
%
\documentclass[
]{article}
\usepackage{amsmath,amssymb}
\usepackage{iftex}
\ifPDFTeX
  \usepackage[T1]{fontenc}
  \usepackage[utf8]{inputenc}
  \usepackage{textcomp} % provide euro and other symbols
\else % if luatex or xetex
  \usepackage{unicode-math} % this also loads fontspec
  \defaultfontfeatures{Scale=MatchLowercase}
  \defaultfontfeatures[\rmfamily]{Ligatures=TeX,Scale=1}
\fi
\usepackage{lmodern}
\ifPDFTeX\else
  % xetex/luatex font selection
\fi
% Use upquote if available, for straight quotes in verbatim environments
\IfFileExists{upquote.sty}{\usepackage{upquote}}{}
\IfFileExists{microtype.sty}{% use microtype if available
  \usepackage[]{microtype}
  \UseMicrotypeSet[protrusion]{basicmath} % disable protrusion for tt fonts
}{}
\makeatletter
\@ifundefined{KOMAClassName}{% if non-KOMA class
  \IfFileExists{parskip.sty}{%
    \usepackage{parskip}
  }{% else
    \setlength{\parindent}{0pt}
    \setlength{\parskip}{6pt plus 2pt minus 1pt}}
}{% if KOMA class
  \KOMAoptions{parskip=half}}
\makeatother
\usepackage{xcolor}
\usepackage[margin=1in]{geometry}
\usepackage{graphicx}
\makeatletter
\def\maxwidth{\ifdim\Gin@nat@width>\linewidth\linewidth\else\Gin@nat@width\fi}
\def\maxheight{\ifdim\Gin@nat@height>\textheight\textheight\else\Gin@nat@height\fi}
\makeatother
% Scale images if necessary, so that they will not overflow the page
% margins by default, and it is still possible to overwrite the defaults
% using explicit options in \includegraphics[width, height, ...]{}
\setkeys{Gin}{width=\maxwidth,height=\maxheight,keepaspectratio}
% Set default figure placement to htbp
\makeatletter
\def\fps@figure{htbp}
\makeatother
\setlength{\emergencystretch}{3em} % prevent overfull lines
\providecommand{\tightlist}{%
  \setlength{\itemsep}{0pt}\setlength{\parskip}{0pt}}
\setcounter{secnumdepth}{-\maxdimen} % remove section numbering
\ifLuaTeX
  \usepackage{selnolig}  % disable illegal ligatures
\fi
\usepackage{bookmark}
\IfFileExists{xurl.sty}{\usepackage{xurl}}{} % add URL line breaks if available
\urlstyle{same}
\hypersetup{
  hidelinks,
  pdfcreator={LaTeX via pandoc}}

\author{}
\date{\vspace{-2.5em}}

\begin{document}

\subsection{guhl}\label{guhl}

Rainer Maria Rilke; aber es ist wirklich egal sagt er, nur, daß ich es
jetzt\\
daß ich der Meister bin, der so dich malte,\\
bleibt nach dem Traum und macht den Mut mir jung."\\
hören kann, sei wichtig und daß meine eigene Erinnerung mich nicht
betrügen wird, wenn ich auf der Suche bin nach ihm, der nicht genannt
werden will. Also wird sein Namen verschwiegen und ich in Phrasen von
ihm reden, die bekannt sind und jedem, der jemals Papiere gelesen hat,
seine Vergangenheit auferstehen lassen. Es ist sinnlos, sich dagegen zu
wehren, ein Rest Schuld ist nicht abzutragen. Wir selbst sind es ja nie
gewesen und können deshalb nicht bereuen, auch nicht stellvertretend für
die in alle Grade hineinreichende Verbindung bis man das Glied fand, das
ihn erzeugte; auch ohne genetischen Hinweis auf seine Herkunft.\\
Ich kann nicht erzählen, wie sehr mich der Tod des Alten veränderte. Ich
werde diese Veränderung sein lassen in der Zeit der Erzählung, so daß
sie keine Macht erlangt über das, was von ihm aus der erzählten Zeit
hinübergerettet wird in deine eigene, die du dann von ihm liest.
Erschütterungen sind nicht gut für das Fundament, auf dem die Geschichte
ruht, immer bekommt es die Spaltungen der Basisbewegung mit und auch
wenn es n.~so aussieht, als trage es alle Zeiten sicher uns durch das
Geschehen, spürt man wie Risse im Innern von Steinen, daß etwas instabil
ist daran. Nur Gewißheit haben wir erst, wenn der Stein zerschlagen
wird, Simon. Deine Intuition läßt dich erfahren, wann es so weit ist.
Aber dann muß der erste Schlag treffen, einen zweiten gab es nie - oder
sie wissen davon nichts.\\
Leicht fällt es ein wenig zu leicht, das Gedenken anzuzuapfen und es
wachsen zu lassen, schon unwichtig in welche Richtung auch es sich
ausdehnt. Da passiert zu viel, als daß es wirkliche Einsichten gäbe. Für
die müßte ich mich schon längst geflüchtet haben und zurückgezogen aus
der Welt. So wie es der Alte tat, obwohl er lebenlang so deutlich zutage
trat und nie sein Gefühl für das verlor, was ihn uns ja jetzt verbindet:
Zeitlichkeit. Angemessenheit. Demütige Kenntnis und die Kraft zu lehren.

\end{document}
