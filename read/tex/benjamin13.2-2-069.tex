% Options for packages loaded elsewhere
\PassOptionsToPackage{unicode}{hyperref}
\PassOptionsToPackage{hyphens}{url}
%
\documentclass[
]{article}
\usepackage{amsmath,amssymb}
\usepackage{iftex}
\ifPDFTeX
  \usepackage[T1]{fontenc}
  \usepackage[utf8]{inputenc}
  \usepackage{textcomp} % provide euro and other symbols
\else % if luatex or xetex
  \usepackage{unicode-math} % this also loads fontspec
  \defaultfontfeatures{Scale=MatchLowercase}
  \defaultfontfeatures[\rmfamily]{Ligatures=TeX,Scale=1}
\fi
\usepackage{lmodern}
\ifPDFTeX\else
  % xetex/luatex font selection
\fi
% Use upquote if available, for straight quotes in verbatim environments
\IfFileExists{upquote.sty}{\usepackage{upquote}}{}
\IfFileExists{microtype.sty}{% use microtype if available
  \usepackage[]{microtype}
  \UseMicrotypeSet[protrusion]{basicmath} % disable protrusion for tt fonts
}{}
\makeatletter
\@ifundefined{KOMAClassName}{% if non-KOMA class
  \IfFileExists{parskip.sty}{%
    \usepackage{parskip}
  }{% else
    \setlength{\parindent}{0pt}
    \setlength{\parskip}{6pt plus 2pt minus 1pt}}
}{% if KOMA class
  \KOMAoptions{parskip=half}}
\makeatother
\usepackage{xcolor}
\usepackage[margin=1in]{geometry}
\usepackage{graphicx}
\makeatletter
\def\maxwidth{\ifdim\Gin@nat@width>\linewidth\linewidth\else\Gin@nat@width\fi}
\def\maxheight{\ifdim\Gin@nat@height>\textheight\textheight\else\Gin@nat@height\fi}
\makeatother
% Scale images if necessary, so that they will not overflow the page
% margins by default, and it is still possible to overwrite the defaults
% using explicit options in \includegraphics[width, height, ...]{}
\setkeys{Gin}{width=\maxwidth,height=\maxheight,keepaspectratio}
% Set default figure placement to htbp
\makeatletter
\def\fps@figure{htbp}
\makeatother
\setlength{\emergencystretch}{3em} % prevent overfull lines
\providecommand{\tightlist}{%
  \setlength{\itemsep}{0pt}\setlength{\parskip}{0pt}}
\setcounter{secnumdepth}{-\maxdimen} % remove section numbering
\ifLuaTeX
  \usepackage{selnolig}  % disable illegal ligatures
\fi
\usepackage{bookmark}
\IfFileExists{xurl.sty}{\usepackage{xurl}}{} % add URL line breaks if available
\urlstyle{same}
\hypersetup{
  hidelinks,
  pdfcreator={LaTeX via pandoc}}

\author{}
\date{\vspace{-2.5em}}

\begin{document}

\subsection{P}\label{p}

Die menschliche Vorstellungskraft: ein Feld, ein begehbares Sensorium
inmitten unwirklicher Fixpunkte, die nur scheinbar etwas festhielten -
Dauer ließ sich nicht ausmachen unter den gegebenen Umständen; daß also
etwas vorhanden wäre außer unseren wahrnehmenden Sinnen: außerhalb
dieser. Zum Beispiel ein anderer Mensch? Ein Gehirn? Wir sind uns nicht
einig darin, ob man aus den Vorstellungen, die wir vom anderen haben
schon seine Existenz ableiten soll. Ich bin ihre Bedingung. Ich mache,
daß er in meiner Welt erscheint (welche für mich erste Priorität hat -
ist ein Nachsatz, der wichtig ist in der fortschreitenden Untersuchung.)
Er taucht auf (selbständig und unberechenbar) und wird erkannt zu einem
Teil meiner Erkenntniswelt, von jetzt an läßt sich sein Auftreten
berechnen mit wachsender Sicherheit über die Wahrscheinlichkeit bei
zunehmenden Kenntnissen über das Verhalten usw\ldots{} wir nähern uns
dem theoretischen Konstrukt der Vokalerzeugung unter den Bedingungen der
Benjaminfeldkraft, das den Anfang des dazugehörigen Kapitels bildete.
Also stellen wir fest, daß die Wortvollendungen nach den ersten zwei
Buchstaben auf der Basis einer Geschwindigkeitsanalyse ihrer Abfolge
leicht erfolgen können. Der Gegensatz zum früher schon gebrauchten
Vervollständigungsassistenten in Telefonen, der zwar auch laut seinem
Wortschatz die fehlenden Buchstaben ergänzend anbot war: während jener
nur mit dem relativ dünnen Material arbeitet, das ihm im Laufe der Zeit
von seinem Besitzer oktroyiert wurde, ist unser kleiner Prozessor
\ldots-\ldots{} mit dem wir uns hier verständigen dazu verdammt, sich
nicht semantisch oder gar grammatisch unserer bisherigen Schöpfungen
bedienen zu können, sondern nur rein litterarisch. Das will heißen: auf
jeden ersten Buchstaben folgt eine zweiter mit einer statistischen
Geschwindigkeit, die gemessen wird. Im Mittel wird diese Geschwindigkeit
immer ein E hervorbringen, einfach, weil dieses der in unserer Sprache
am häufigsten verwendete Buchstabe ist. Gefolgt von einem konsonanten N,
darauf ein I, erste also sprechende Verbindung, die die Maschine von
selbst hervorbringt, wird ENI heißen, ähnliches Ich im hebräischen
Anfang, wie wir es längst wissen.\\
Das war das erste Wort. Und so rattert er die Möglichkeiten heraus, die
aufeinander folgen, bis - und jetzt kommen wir ins Spiel - uns eines
davon zusagt (konditioniert: positiv belegt war) der Elektronenpattern,
die das Hirn erreichen und einrastet und: ausfällt, Sediment, Struktur,
festes, geringster Widerstand weil selbst in sich ruhend. Anders im
Telephon, wo alles semantisch verankert ist: hier aber nur über die
Elektronenbahnen in der Brownschen Röhre sich ausdrückend. Und dann es
kommt es auch an statt in der liquid crystal blackbox nur für eine
weitere Endorphinausschüttung zu verschwinden. Gehen wir dem nach
ermahne ich mich, um nicht einem Schweif zu folgen, der sich anbietet.
Aber dafür muß ich ausholen. Immer war es die Sabbathnacht, wo so etwas
passierte. Und ich muß dankbar sein, daß es mich dann wenigstens
erreicht: das Partikelgestöber von dem Celan sprach? Vielleicht. Eine
Möglichkeit unter vielen erprobten und einigen auch von mir selbst. Das
Malen auf vorgegebener Unterlage ist ungefähr damit vergleichbar: man
nimmt etwas an - ein stark vergrößerter Ausschnitt aus einem Bild/ einer
Struktur/ einer zufällig entstandenen/ hergestellten/gefundenen/ einem
Material, das als Unterlage dienen kann und gleichzeitig Andockpunkte
bereitstellt für die eigene Kreativität. Ab und zu wird sich etwas in
sowas finden, das einen erinnert an Formen, meist Gesichter oder Figuren
und dann führt man es aus immer konkreter. Zwischenstufen werden
eingescannt und auf dem Ausdruck weitergearbeitet. Wie ein Wort also,
das zum nächsten führt, zufällig. Ich sehe es (es wird hier hinten
abgebildet auf den Zäpfchen) und dann ruft es schon nach dem Begleiter.
Nicht, daß ich mich jetzt aus der Verantwortung ziehen möchte - aber die
Funktion des Rauschens war erkannt, sobald die abgelegte Verkleidung der
zerstörten Röhre offenbarte, was ich ja nur geahnt habe: Da waren
wirklich die Doppelspalte auf der Mattscheibe eingebrannt, im optimalen
mittleren Abstand zur Netzhaut dienten sie als perfekte Blende, um nicht
hindurch auf die andere Seite sehen zu müssen/ es zu können! Aber wir
verlieren uns im Bedeutungsschwangeren\ldots{} die andere Seite hat es
ja nie wirklich gegeben, immer nur: den Anderen; als einziges Gegenüber,
das uns zählt. Es steht genau so wie ich am Pult seines mit unendlicher
Sicherheit vorhandenen Hörsaals/Lehrstuhls dieser ehrw. Universität und
ich glaube er sprach gerade über Mäeutik und die Mutter des Sokrates,
als ihn meine sehr persönliche Mitteilung erreichte. Wir würden uns
sehen und kennenlernen müssen und daß Heidegger und Marx durchaus in
einen Zusammenhang paßten, der n.~jenseits von uns ausgedachter Utopien
läge: im Labyrinth des Innenohrs hörte man ihre Theorien klingeln ohne
daß die Moleküle eine Chance hätten, sie zu interpretieren. Aber es
wurden Muster geprägt und Cluster sammelten sich um die vielzitierten
Stellen herum an, so daß ich alles fand, was ich jemals gelesen hatte;
wenn es Verbindungen gab, n.~so schwache zum Gehör\ldots{} und dann
wurde es jetzt Zeit, sie zu benutzen und dies Geschenk anzunehmen. P =
Pause.

\end{document}
