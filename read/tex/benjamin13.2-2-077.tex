% Options for packages loaded elsewhere
\PassOptionsToPackage{unicode}{hyperref}
\PassOptionsToPackage{hyphens}{url}
%
\documentclass[
]{article}
\usepackage{amsmath,amssymb}
\usepackage{iftex}
\ifPDFTeX
  \usepackage[T1]{fontenc}
  \usepackage[utf8]{inputenc}
  \usepackage{textcomp} % provide euro and other symbols
\else % if luatex or xetex
  \usepackage{unicode-math} % this also loads fontspec
  \defaultfontfeatures{Scale=MatchLowercase}
  \defaultfontfeatures[\rmfamily]{Ligatures=TeX,Scale=1}
\fi
\usepackage{lmodern}
\ifPDFTeX\else
  % xetex/luatex font selection
\fi
% Use upquote if available, for straight quotes in verbatim environments
\IfFileExists{upquote.sty}{\usepackage{upquote}}{}
\IfFileExists{microtype.sty}{% use microtype if available
  \usepackage[]{microtype}
  \UseMicrotypeSet[protrusion]{basicmath} % disable protrusion for tt fonts
}{}
\makeatletter
\@ifundefined{KOMAClassName}{% if non-KOMA class
  \IfFileExists{parskip.sty}{%
    \usepackage{parskip}
  }{% else
    \setlength{\parindent}{0pt}
    \setlength{\parskip}{6pt plus 2pt minus 1pt}}
}{% if KOMA class
  \KOMAoptions{parskip=half}}
\makeatother
\usepackage{xcolor}
\usepackage[margin=1in]{geometry}
\usepackage{graphicx}
\makeatletter
\def\maxwidth{\ifdim\Gin@nat@width>\linewidth\linewidth\else\Gin@nat@width\fi}
\def\maxheight{\ifdim\Gin@nat@height>\textheight\textheight\else\Gin@nat@height\fi}
\makeatother
% Scale images if necessary, so that they will not overflow the page
% margins by default, and it is still possible to overwrite the defaults
% using explicit options in \includegraphics[width, height, ...]{}
\setkeys{Gin}{width=\maxwidth,height=\maxheight,keepaspectratio}
% Set default figure placement to htbp
\makeatletter
\def\fps@figure{htbp}
\makeatother
\setlength{\emergencystretch}{3em} % prevent overfull lines
\providecommand{\tightlist}{%
  \setlength{\itemsep}{0pt}\setlength{\parskip}{0pt}}
\setcounter{secnumdepth}{-\maxdimen} % remove section numbering
\ifLuaTeX
  \usepackage{selnolig}  % disable illegal ligatures
\fi
\usepackage{bookmark}
\IfFileExists{xurl.sty}{\usepackage{xurl}}{} % add URL line breaks if available
\urlstyle{same}
\hypersetup{
  hidelinks,
  pdfcreator={LaTeX via pandoc}}

\author{}
\date{\vspace{-2.5em}}

\begin{document}

\subsection{T}\label{t}

Ob er den Fund abgeben soll, hat er mich gefragt. Aber wie soll ich
darauf eine ehrliche Antwort geben; die wertvollen Dinge wenn sie einmal
den Weg zu uns gekommen sind, gehören ja niemandem mehr außer der vagen
Allgemeinheit von Verlierern von Dingen, die es immer gibt wo etwas zu
verlieren ist. Und schon lange gehörte jener dazu, auch wenn man
glaubte, nichts mehr zu haben zum Verlieren. Da war es ja: Glauben; und
ist bis zuletzt n.~da, wenn alle Hüllen gegangen sein werden, also
n.~nachdem wir die Körper schon verabschiedet haben. Warum ich das weiß?
Seamus, was stellst du für Fragen\ldots{}\\
Ich wollte sie eigentlich beantworten, aber mir ging die Luft aus von
der Verwunderung über so viel Demut. Wir sagen daß es keine dummen
Fragen nur überflüssige Antworten gibt. Doch glaube ich, der richtige
Steller bringt auch eine extreme Frage zustande wie: kann sich der
Aufwand lohnen, einer Idee bis zu ihrem Auslauf im Gedanken zu folgen,
der \emph{allmählich verfertigt} wird während des Sprechens? Würde ich
sonst mit Ihnen sprechen? Ja vielleicht ist es so, wir haben eine
angemessene Geschwindigkeit entwickelt, den Gedanken fortfahren zu
lassen so lange wir uns in der Bahn befinden die er vorgibt. Rechts,
links scheinen Verbindungen auf im Zusammenhang mit der ursprünglichen
\emph{idea}, von der ich n.~nicht einmal sagen kann, sie überhaupt
gehabt zu haben; nur gegeben hat es sie in irgendeiner wahrscheinlichen
Reichweite meiner erfassenden Sinne und kam vor. Jetzt lassen sich die
Sätze nicht mehr zurückverfolgen, dafür liegen die initianten
Wortanfänge zu sehr verhangen hinter den Nebeln, die sich aufbauen
sobald mir die \emph{ahnung} greifbar erschien des Satzes. Überschritten
das klaffende Loch zum nächsten Wort gibt es keinen Rückgang mehr, der
von ihmnicht begründet wäre. Der erste hat einmal in den \emph{temps
perdu} versucht, den Lesern etwas vorzuspielen von seinen Ereignissen -
aber wir haben uns alle immer an etwas anderes erinnert als er wollte,
logisch.

\end{document}
