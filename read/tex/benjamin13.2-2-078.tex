% Options for packages loaded elsewhere
\PassOptionsToPackage{unicode}{hyperref}
\PassOptionsToPackage{hyphens}{url}
%
\documentclass[
]{article}
\usepackage{amsmath,amssymb}
\usepackage{iftex}
\ifPDFTeX
  \usepackage[T1]{fontenc}
  \usepackage[utf8]{inputenc}
  \usepackage{textcomp} % provide euro and other symbols
\else % if luatex or xetex
  \usepackage{unicode-math} % this also loads fontspec
  \defaultfontfeatures{Scale=MatchLowercase}
  \defaultfontfeatures[\rmfamily]{Ligatures=TeX,Scale=1}
\fi
\usepackage{lmodern}
\ifPDFTeX\else
  % xetex/luatex font selection
\fi
% Use upquote if available, for straight quotes in verbatim environments
\IfFileExists{upquote.sty}{\usepackage{upquote}}{}
\IfFileExists{microtype.sty}{% use microtype if available
  \usepackage[]{microtype}
  \UseMicrotypeSet[protrusion]{basicmath} % disable protrusion for tt fonts
}{}
\makeatletter
\@ifundefined{KOMAClassName}{% if non-KOMA class
  \IfFileExists{parskip.sty}{%
    \usepackage{parskip}
  }{% else
    \setlength{\parindent}{0pt}
    \setlength{\parskip}{6pt plus 2pt minus 1pt}}
}{% if KOMA class
  \KOMAoptions{parskip=half}}
\makeatother
\usepackage{xcolor}
\usepackage[margin=1in]{geometry}
\usepackage{graphicx}
\makeatletter
\def\maxwidth{\ifdim\Gin@nat@width>\linewidth\linewidth\else\Gin@nat@width\fi}
\def\maxheight{\ifdim\Gin@nat@height>\textheight\textheight\else\Gin@nat@height\fi}
\makeatother
% Scale images if necessary, so that they will not overflow the page
% margins by default, and it is still possible to overwrite the defaults
% using explicit options in \includegraphics[width, height, ...]{}
\setkeys{Gin}{width=\maxwidth,height=\maxheight,keepaspectratio}
% Set default figure placement to htbp
\makeatletter
\def\fps@figure{htbp}
\makeatother
\setlength{\emergencystretch}{3em} % prevent overfull lines
\providecommand{\tightlist}{%
  \setlength{\itemsep}{0pt}\setlength{\parskip}{0pt}}
\setcounter{secnumdepth}{-\maxdimen} % remove section numbering
\ifLuaTeX
  \usepackage{selnolig}  % disable illegal ligatures
\fi
\usepackage{bookmark}
\IfFileExists{xurl.sty}{\usepackage{xurl}}{} % add URL line breaks if available
\urlstyle{same}
\hypersetup{
  hidelinks,
  pdfcreator={LaTeX via pandoc}}

\author{}
\date{\vspace{-2.5em}}

\begin{document}

\subsection{U}\label{u}

Unter Fremdheit: wollen wir eigentlich nur das aus dem Unbekannten
fassen. Also Seamus, für mich der missionarisch wandelnde Irenkathole,
ist davon ein Begriff, obwohl ich nicht allzuweit von Orthodoxie
entfernt bin, seiner Vorstellung entsprechend. Irgendwo lehrt er und
etwas predigt, dabei immer im klaren darüber, wie sich alles umkehren
ließe, wenn er einmal nicht ganz überzeugt wäre - jeder Schritt möglich
einer der in den Abgrund führt. Er ist dort fremd, wo er auftaucht,
sonst wäre die Mission unnötig. Ich aber stelle ihn mir vor: das heißt,
er ist in seiner Mission unterwegs und sendet. Wir können das empfangen
und weitergeben, so lest ihr es. Es geht aus der Fremdheit in uns ein,
vertraut sich uns weiter, dann stellt es Bedingungen. Wir haben es
herausgefordert, haben wir? Warum lasen wir überhaupt in seinen
Schriften, die ihren häretischen Charakter gar nicht mehr bemüht waren
zu verbergen, so daß jeder, der sich uns anschloß unweigerlich ihn
akzeptierte, \emph{"und wenn ich sage Er, so meine ich Ihn, Ihn!!" - die
Natur, pah!!! "Ein Hundsfott hat Courage\ldots"} Das nur, daß wir selbst
die Subjekte sind, die sich der Wissenschaft (Woyzeck) zur Verfügung
stellen, wo sie uns auch aufzugreifen vermag, angreifen; an unserem
Stolz, an unserem Gewissen, anderswo an der Eitelkeit oder auch nur der
Notwendigkeit anerkannt zu werden in einem weiteren sich vollendenen
Geist.\\
Seamus also: dirty bloodbag of a husband - vielleicht sollten wir jetzt
einmal Maggie aufsuchen um sie über die Vokabel zu befragen\ldots{}
falls ihr euch n.~daran erinnern könnt. Stellt ihr doch einmal die
Frage.

\end{document}
