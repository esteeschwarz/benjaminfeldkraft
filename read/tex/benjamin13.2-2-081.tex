% Options for packages loaded elsewhere
\PassOptionsToPackage{unicode}{hyperref}
\PassOptionsToPackage{hyphens}{url}
%
\documentclass[
]{article}
\usepackage{amsmath,amssymb}
\usepackage{iftex}
\ifPDFTeX
  \usepackage[T1]{fontenc}
  \usepackage[utf8]{inputenc}
  \usepackage{textcomp} % provide euro and other symbols
\else % if luatex or xetex
  \usepackage{unicode-math} % this also loads fontspec
  \defaultfontfeatures{Scale=MatchLowercase}
  \defaultfontfeatures[\rmfamily]{Ligatures=TeX,Scale=1}
\fi
\usepackage{lmodern}
\ifPDFTeX\else
  % xetex/luatex font selection
\fi
% Use upquote if available, for straight quotes in verbatim environments
\IfFileExists{upquote.sty}{\usepackage{upquote}}{}
\IfFileExists{microtype.sty}{% use microtype if available
  \usepackage[]{microtype}
  \UseMicrotypeSet[protrusion]{basicmath} % disable protrusion for tt fonts
}{}
\makeatletter
\@ifundefined{KOMAClassName}{% if non-KOMA class
  \IfFileExists{parskip.sty}{%
    \usepackage{parskip}
  }{% else
    \setlength{\parindent}{0pt}
    \setlength{\parskip}{6pt plus 2pt minus 1pt}}
}{% if KOMA class
  \KOMAoptions{parskip=half}}
\makeatother
\usepackage{xcolor}
\usepackage[margin=1in]{geometry}
\usepackage{graphicx}
\makeatletter
\def\maxwidth{\ifdim\Gin@nat@width>\linewidth\linewidth\else\Gin@nat@width\fi}
\def\maxheight{\ifdim\Gin@nat@height>\textheight\textheight\else\Gin@nat@height\fi}
\makeatother
% Scale images if necessary, so that they will not overflow the page
% margins by default, and it is still possible to overwrite the defaults
% using explicit options in \includegraphics[width, height, ...]{}
\setkeys{Gin}{width=\maxwidth,height=\maxheight,keepaspectratio}
% Set default figure placement to htbp
\makeatletter
\def\fps@figure{htbp}
\makeatother
\setlength{\emergencystretch}{3em} % prevent overfull lines
\providecommand{\tightlist}{%
  \setlength{\itemsep}{0pt}\setlength{\parskip}{0pt}}
\setcounter{secnumdepth}{-\maxdimen} % remove section numbering
\ifLuaTeX
  \usepackage{selnolig}  % disable illegal ligatures
\fi
\usepackage{bookmark}
\IfFileExists{xurl.sty}{\usepackage{xurl}}{} % add URL line breaks if available
\urlstyle{same}
\hypersetup{
  hidelinks,
  pdfcreator={LaTeX via pandoc}}

\author{}
\date{\vspace{-2.5em}}

\begin{document}

\subsection{W}\label{w}

Einmal im Jahr findet jene Lesung statt, die langsam in den Tunnel
einführen soll, den sie damals gegraben haben \emph{über der Spree.}
Aber n.~ist die Gruppe der Nicht richtig auf mich aufmerksam geworden,
ich muß ein bißchen dreister werden in der Verhandlung des
Gegenstandes:\\
"Es gibt jetzt das Buch über die Benjaminfeldkraft in einer Ausgabe, die
sehr bescheiden den Namen des Autors verschweigt, so daß also alles
darin Allgemeingut werden dürfte zu Forschungszwecken. Und wenn jemand
schon in der letzten Edition darauf aus war, die Theorie mittels des
dafür angefertigten Schlüssels zu überprüfen, so steht ihm nun
jedenfalls mit dem zusätzlichen n.~unvollständigen Supplementband ein
hülfreiches Instrument zur Verfügung, das netzunabhängig alle bisher
aufgeworfenen Fragen hinreichend zu klären vermag. Nichtsdestotrotz sei
angemerkt, daß auch in diesem Stadium der Erforschung der Kraft eine
eingehende Beschäftigung mit den Erkenntnissen der neuen und neueren
Physik unbedingt empfohlen wird, schon, um die immanenten Auswirkungen
der Textanalyse abschätzen zu können, bevor Sie sich auf die Suche nach
dem Drachen begeben werden."\\
Die Forderungen sind klar, es gibt keine Verhandlungsbasis über
gerechten Ablauf der Geschichte. Wir haben gelernt uns an die Tatsachen
zu halten und sind Empiriker geworden eher als Historiker. H.B. ist der
letzte wirkliche \emph{polyhistor, }der mir in meinem Leben bisher
begegnete und ich habe im Archiv alles was ich nun in den Nächten
durchgraben auf der Suche nach etwas, das mir die Kraft erklären kann.
Ich habe sie in seiner Gegenwart ganz stark gefühlt und wenn ich eins
der Blätter mit den notenartigen Niederschriften zur Hand nehme, ist sie
sofort da. Stenographisch verkürzt und auf den absoluten Symbolcharakter
reduziert; aber etwas kann man daraus lesen lernen: jede Notiz, die
einem am Rande einfällt zum Textkorpus ist es wert, darin eingeschmolzen
zu werden. Erst am Ende, wenn wir das Konglomerat vor uns sehen und den
Abstand ermessen, der uns selbst von der denkenden Struktur die wir
abbildeten trennt, ist es uns möglich überhaupt einzuschätzen, wie
wichtig die matris lectionis zwischen, über, unter und neben den Zeilen
waren die er und jetzt auch wir so sorglos übernahmen. Aber wie
selbstverständlich unsere Anteilnahme an seinen Überlegungen auch
daherkommen mag; immer, immer leider \emph{ist uns gegeben auf keiner
Stätte zu ruhn\ldots{}} erst das nächste Wort schafft Klarheit.
Hölderlin war eine erste Information, die wir erhalten haben, jedoch:
bis zum Abbruch der Beziehung zu ihm durch den Tod ist eine zu kurze
Zeit, um darin sicher zu werden. Und so bleibt die Aufgabe für danach
gestellt - vor dem Verschwinden.

\end{document}
