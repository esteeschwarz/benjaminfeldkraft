% Options for packages loaded elsewhere
\PassOptionsToPackage{unicode}{hyperref}
\PassOptionsToPackage{hyphens}{url}
%
\documentclass[
]{article}
\usepackage{amsmath,amssymb}
\usepackage{iftex}
\ifPDFTeX
  \usepackage[T1]{fontenc}
  \usepackage[utf8]{inputenc}
  \usepackage{textcomp} % provide euro and other symbols
\else % if luatex or xetex
  \usepackage{unicode-math} % this also loads fontspec
  \defaultfontfeatures{Scale=MatchLowercase}
  \defaultfontfeatures[\rmfamily]{Ligatures=TeX,Scale=1}
\fi
\usepackage{lmodern}
\ifPDFTeX\else
  % xetex/luatex font selection
\fi
% Use upquote if available, for straight quotes in verbatim environments
\IfFileExists{upquote.sty}{\usepackage{upquote}}{}
\IfFileExists{microtype.sty}{% use microtype if available
  \usepackage[]{microtype}
  \UseMicrotypeSet[protrusion]{basicmath} % disable protrusion for tt fonts
}{}
\makeatletter
\@ifundefined{KOMAClassName}{% if non-KOMA class
  \IfFileExists{parskip.sty}{%
    \usepackage{parskip}
  }{% else
    \setlength{\parindent}{0pt}
    \setlength{\parskip}{6pt plus 2pt minus 1pt}}
}{% if KOMA class
  \KOMAoptions{parskip=half}}
\makeatother
\usepackage{xcolor}
\usepackage[margin=1in]{geometry}
\usepackage{graphicx}
\makeatletter
\def\maxwidth{\ifdim\Gin@nat@width>\linewidth\linewidth\else\Gin@nat@width\fi}
\def\maxheight{\ifdim\Gin@nat@height>\textheight\textheight\else\Gin@nat@height\fi}
\makeatother
% Scale images if necessary, so that they will not overflow the page
% margins by default, and it is still possible to overwrite the defaults
% using explicit options in \includegraphics[width, height, ...]{}
\setkeys{Gin}{width=\maxwidth,height=\maxheight,keepaspectratio}
% Set default figure placement to htbp
\makeatletter
\def\fps@figure{htbp}
\makeatother
\setlength{\emergencystretch}{3em} % prevent overfull lines
\providecommand{\tightlist}{%
  \setlength{\itemsep}{0pt}\setlength{\parskip}{0pt}}
\setcounter{secnumdepth}{-\maxdimen} % remove section numbering
\ifLuaTeX
  \usepackage{selnolig}  % disable illegal ligatures
\fi
\usepackage{bookmark}
\IfFileExists{xurl.sty}{\usepackage{xurl}}{} % add URL line breaks if available
\urlstyle{same}
\hypersetup{
  hidelinks,
  pdfcreator={LaTeX via pandoc}}

\author{}
\date{\vspace{-2.5em}}

\begin{document}

\subsection{X}\label{x}

Der Alte sprach im Wahn von einem Lügengebilde, das um ihn herum
aufgebaut war, alle Menschen seien daran beteiligt u. manche wüßten es
selbst nicht. Es gäbe jedoch Hinweise darauf, daß die Sache von einer
zentralen Instanz gesteuert wird. Er war während seiner Forschung in
Informationskanäle eingedrungen und hatte Nachrichten abfangen können,
die dieses beweisen würden - wenn jemand daran interessiert wäre. Er
fragte zuerst mich. Ich hatte ihm Antworten geliehen und immer darauf
verzichtet, sie zurückzubekommen, ich hatte genug. (Jetzt, nachdem er
gestorben ist, sehne ich mich nach ihnen, aber natürlich ist es zu spät,
etwa die Antworten zu fordern.) Er fragte so: ob ich die kleine
Meßeinheit bemerkte, die er um den rechten Finger trug. Als ich nicht
wußte, was er damit meinte, zischte er und legte seine Uhr auf den
Tisch. Dies! zum Messen der Zeit, das kennst du doch, oder? Ich sagte
ja, aber es sei das Handgelenk, um das sie gebunden war, nicht ein
Finger.\\
- ja, ist es. Und es spielt keine Rolle mehr, wieviel Grad man
verändert, jeder sieht zuerst nur das ganze Wort, und mit dem
durchschnittlichen Gehalt von 5 Buchstaben wird sie leider schnell
langweilig. Dann versuche ich zu parallelisieren und denke mir einen
neuen Zusammenhang aus in den es paßt und siehe da: die Meßeinheit
verrutscht um eine Potenz und zeigt plötzlich nicht mehr die Zeit an,
sondern die relative Entfernung zum Horizont. Damit kann man dann eine
Weile herumexperimentieren bis die Checksumme stimmt (wir befinden uns
immer n.~im Gebiet um den neunten Planeten\ldots) - und schon sind die
Gesetze weiter gültig, die Invarianz wurde gelöst auf einer höheren
Ebene. Es gibt sie immer, von jedem Niveau aus; nur die
Selbsteinschätzung geht dabei fehl, daß man glaubt, oben zu sein. Aber
wäre denn oben wie unten und unten wie oben\ldots{}

\end{document}
