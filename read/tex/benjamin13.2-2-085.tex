% Options for packages loaded elsewhere
\PassOptionsToPackage{unicode}{hyperref}
\PassOptionsToPackage{hyphens}{url}
%
\documentclass[
]{article}
\usepackage{amsmath,amssymb}
\usepackage{iftex}
\ifPDFTeX
  \usepackage[T1]{fontenc}
  \usepackage[utf8]{inputenc}
  \usepackage{textcomp} % provide euro and other symbols
\else % if luatex or xetex
  \usepackage{unicode-math} % this also loads fontspec
  \defaultfontfeatures{Scale=MatchLowercase}
  \defaultfontfeatures[\rmfamily]{Ligatures=TeX,Scale=1}
\fi
\usepackage{lmodern}
\ifPDFTeX\else
  % xetex/luatex font selection
\fi
% Use upquote if available, for straight quotes in verbatim environments
\IfFileExists{upquote.sty}{\usepackage{upquote}}{}
\IfFileExists{microtype.sty}{% use microtype if available
  \usepackage[]{microtype}
  \UseMicrotypeSet[protrusion]{basicmath} % disable protrusion for tt fonts
}{}
\makeatletter
\@ifundefined{KOMAClassName}{% if non-KOMA class
  \IfFileExists{parskip.sty}{%
    \usepackage{parskip}
  }{% else
    \setlength{\parindent}{0pt}
    \setlength{\parskip}{6pt plus 2pt minus 1pt}}
}{% if KOMA class
  \KOMAoptions{parskip=half}}
\makeatother
\usepackage{xcolor}
\usepackage[margin=1in]{geometry}
\usepackage{graphicx}
\makeatletter
\def\maxwidth{\ifdim\Gin@nat@width>\linewidth\linewidth\else\Gin@nat@width\fi}
\def\maxheight{\ifdim\Gin@nat@height>\textheight\textheight\else\Gin@nat@height\fi}
\makeatother
% Scale images if necessary, so that they will not overflow the page
% margins by default, and it is still possible to overwrite the defaults
% using explicit options in \includegraphics[width, height, ...]{}
\setkeys{Gin}{width=\maxwidth,height=\maxheight,keepaspectratio}
% Set default figure placement to htbp
\makeatletter
\def\fps@figure{htbp}
\makeatother
\setlength{\emergencystretch}{3em} % prevent overfull lines
\providecommand{\tightlist}{%
  \setlength{\itemsep}{0pt}\setlength{\parskip}{0pt}}
\setcounter{secnumdepth}{-\maxdimen} % remove section numbering
\ifLuaTeX
  \usepackage{selnolig}  % disable illegal ligatures
\fi
\usepackage{bookmark}
\IfFileExists{xurl.sty}{\usepackage{xurl}}{} % add URL line breaks if available
\urlstyle{same}
\hypersetup{
  hidelinks,
  pdfcreator={LaTeX via pandoc}}

\author{}
\date{\vspace{-2.5em}}

\begin{document}

\subsection{1. GUHL}\label{guhl}

Lassen Sie mich an einem Beispiel folgendes demonstrieren: Wenn man
Reflexion stets voraussetzen darf, wo wir uns begegnen - das Guhl-Ich
und die herbeigeführten Protagonisten, diese sich begegnenden
Unendlichkeiten (wie die zwei der Himmel und des Brunnen in sich
starrenden) - wenn wir jene also einmal setzen; Reflexion nicht im Sinne
eines Charakterzuges gemeint oder menschlicher Fähigkeit sondern nur als
passives Spiel zweier Ansichten, zweier Intentionen sich zu äußern dann
kommen wir bald dahin sie zwischen beiden anzusehen als Bindeglied
Austauschsphäre oder besser schon: Medium, in welchem die Begegnung
stattfindet stattfinden kann könnte oder würde, wenn von den zwei Seiten
die Erlaubnis dazu gegeben ist. Wie aber und in welcher Hierarchie wird
die Erlaubnis erteilt? Gleichzeitig so daß es keine Kausalität mehr
gibt? Gleichzeitig und trotzdem kausal verknüpft in einer verminderten
Ebene; gleich\emph{zeitig} ohne konkreten zeitlichen Bezug also nur für
unsere Draufsicht zeitlich gebunden, nicht jedoch im eigenen Komplex?
Oder ohne jeden Zeitbezug? \emph{Wo} wäre es dann? Also genau \emph{wo?
}Denn nach den Invarianzprinzipien ist uns ja gestattet \emph{einen}
Parameter zu vernachlässigen, was wir also jetzt im Sinne der
T-Invarianz getan haben. Damit ist unter sonst gleichen Bedingungen die
zeitliche Einordnung für die Hierarchie unerheblich, es gibt keine
derartige Reihung. Wäre also nur n.~mit den alten Worten zu klären wann
wir hier sind wenn wir hier sind an diesem Ort. Denn jener ist das
nahezu fixe Moment der Schreibe. Wir haben die Konstante bestimmt,
können es jeden (Augenblick) tun und legen den Ort damit fest wo wir uns
befinden wenn wir uns hier befinden. Die Gegenläufigkeit eueres
Leseverhaltens zu dem meinen der ich ja erst rückwärtsgewandt euch
anblicken kann läßt in dieser hier konstanten Mitte unserer
Ortsverknüpfung die Zeit entstehen \emph{zu} welcher sie sich begeben
die drei Momente:

\end{document}
