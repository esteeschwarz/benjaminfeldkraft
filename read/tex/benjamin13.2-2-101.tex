% Options for packages loaded elsewhere
\PassOptionsToPackage{unicode}{hyperref}
\PassOptionsToPackage{hyphens}{url}
%
\documentclass[
]{article}
\usepackage{amsmath,amssymb}
\usepackage{iftex}
\ifPDFTeX
  \usepackage[T1]{fontenc}
  \usepackage[utf8]{inputenc}
  \usepackage{textcomp} % provide euro and other symbols
\else % if luatex or xetex
  \usepackage{unicode-math} % this also loads fontspec
  \defaultfontfeatures{Scale=MatchLowercase}
  \defaultfontfeatures[\rmfamily]{Ligatures=TeX,Scale=1}
\fi
\usepackage{lmodern}
\ifPDFTeX\else
  % xetex/luatex font selection
\fi
% Use upquote if available, for straight quotes in verbatim environments
\IfFileExists{upquote.sty}{\usepackage{upquote}}{}
\IfFileExists{microtype.sty}{% use microtype if available
  \usepackage[]{microtype}
  \UseMicrotypeSet[protrusion]{basicmath} % disable protrusion for tt fonts
}{}
\makeatletter
\@ifundefined{KOMAClassName}{% if non-KOMA class
  \IfFileExists{parskip.sty}{%
    \usepackage{parskip}
  }{% else
    \setlength{\parindent}{0pt}
    \setlength{\parskip}{6pt plus 2pt minus 1pt}}
}{% if KOMA class
  \KOMAoptions{parskip=half}}
\makeatother
\usepackage{xcolor}
\usepackage[margin=1in]{geometry}
\usepackage{graphicx}
\makeatletter
\def\maxwidth{\ifdim\Gin@nat@width>\linewidth\linewidth\else\Gin@nat@width\fi}
\def\maxheight{\ifdim\Gin@nat@height>\textheight\textheight\else\Gin@nat@height\fi}
\makeatother
% Scale images if necessary, so that they will not overflow the page
% margins by default, and it is still possible to overwrite the defaults
% using explicit options in \includegraphics[width, height, ...]{}
\setkeys{Gin}{width=\maxwidth,height=\maxheight,keepaspectratio}
% Set default figure placement to htbp
\makeatletter
\def\fps@figure{htbp}
\makeatother
\setlength{\emergencystretch}{3em} % prevent overfull lines
\providecommand{\tightlist}{%
  \setlength{\itemsep}{0pt}\setlength{\parskip}{0pt}}
\setcounter{secnumdepth}{-\maxdimen} % remove section numbering
\ifLuaTeX
  \usepackage{selnolig}  % disable illegal ligatures
\fi
\usepackage{bookmark}
\IfFileExists{xurl.sty}{\usepackage{xurl}}{} % add URL line breaks if available
\urlstyle{same}
\hypersetup{
  hidelinks,
  pdfcreator={LaTeX via pandoc}}

\author{}
\date{\vspace{-2.5em}}

\begin{document}

\subsection{I}\label{i}

Vielleicht ist S. dann nur möglich, wenn wir i. unsere Bedingungen
anpassen, nicht jener der anderen, denen wir nacheifern oder sie
bewundern. Man soll sie bewundern und hochschätzen, soll ihnen Lob
spenden und vielleicht auch vergöttern. Aber nicht versuchen, mit den
Farben der Meistersinger die Bücher auszumalen; denn ich weiß, daß der
ernsthafte und wichtige Leser sich dann bepinkelt, daran führt kein Weg
vorbei, mit den bekannten Figuren. Meine müssen von jenen lernen und ich
es ihnen beibringen. Dann ergibt sich das Gerüst, welches im Bauwerk
auszuführen war: Geist der Geschichte. Also nähern wir uns ihm. Ich fing
an, Korrektur zu lesen, nachdem die sechs Wochen verstrichen waren und
fand nicht viel zu verhindern. Das war ein schlechtes Zeichen. Man
sollte aber das Regelwerk ändern für den weiteren Fortgang, es war nicht
möglich, nach dem bisher bekannten zu verfahren. Immanent hieße
demzufolge: wir sprechen im Text selbst zu uns, ohne uns dabei zu
überanstrengen oder zu langweilen. Gelingen wird mir das nur, wenn ich
euch dabei anschauen darf, die mir zuhören wollten und nicht wußtet, was
euch erwartet. Aber ihr seid mir ja gefolgt und das ist mehr, als mich
der Moment glauben läßt. Ich schreibs weiter und versuche, euch an den
ersten Teil der Aufzeichnungen zu erinnern, als es n.~viel schwerer war
als jetzt, meine Gedanken zu haben; nicht nur, weil beständig die Leute
dazwischenquatschten in ihren verschiedenen Bewußtseinsströmen sondern
auch wegen der Geschichte von Ewa und der 10. Symphonie. Die hat es ja
nie gegeben, Ewa Laplace nicht und auch keine wirkliche 10. Symphonie
die über das hinausgegangen wäre, was schon öfter versucht wurde. Doch
es gab eine Ewa, eine zierliche unversuchte Unschuldsewa, die zwei Jahre
lang das Kind von Gustav Mahler gewesen ist und schnell starb, bevor sie
ihn hätte verlieren können als 51jährigen Vater mit einem unausführbaren
Vermächtnis. Natürlich ist das relativ und die Kausalität nicht
berechenbar, die zu \emph{seinem} Verlust führte, sicher ist aber, daß
ohne ihrn leichten Tod das damit aprupt einsetzende Unvermögen sich
weiter zu bewegen über den Horizont des bisher Erreichten hinaus nicht
zu verstehen ist und wenn man darin eine Konsequenz sehen will, dann nur
die seiner absolut notwendigen Dichtung. \emph{Absolut:} weil sie ihren
Abbruch vorausdeutet, \emph{notwendig:} weil es außerhalb seiner Macht
stand, den Tod zu verhindern und \emph{Dichtung:} weil es ihm gelang,
diese Information in einer jeden \emph{nicht notwendigen} Zusatz
ausschließenden Struktur zu versammeln, die als Fragment erhalten ist.
Fragmentarisch können also nur alle Zusammenhänge erklärt werden, die
sich jetzt, im späten Danach ergeben. Doch nicht als wenn es wirklich
gewollt gewesen wäre, was sich Mahler da aus den hörenden Fingern seiner
Klavierhände schüttelte. Doch nein, das werde ich nicht zugeben, auch
wenn ich Ihnen mit \emph{Dem Vergangen }zeigte, sich nicht in einem
Meisterwerk zu befinden, wenn Sie richtig zuhörten und die harmonischen
Brüche also wahrnehmen. Ihm selbst jedoch bleibt es ein \emph{essai
absolu} der nicht über sich hinausweisen darf. Die Bedingungen sind
geklärt\ldots{}\\
\emph{I} setzten wir bisher nur als Stromstärke ein. Aber was kann es
n.~alles sein, das wir erlauben: die Begrifflichkeiten rattern durch
unds bleibt ein Wort stehen, eigentlich kein böses Wort, aber gemessen
an den übrigen ein überdurchschnittlich schlecht konnotiertes:
Intelligenz. Warum aber bleibt das trotzdem hängen? Kein anderes nimmt
sofort mehr Raum in Anspruch des Denkens wie jenes, das unheimlich
vertraut wirkt aber an Abstraktheit kaum zu übertreffen ist. Versuchen
wir zu konkretisieren: die schachtelartig ineinandergeschobenen Ansätze
bilden etwas aus sich heraus, wenn man sie aufzieht: ich sehe ein
Konstrukt, sehe wirklich eines. Aber soll ich das auch beschreiben? Ivy,
Ivich, Eve-Lilith Ewa-Alma-magna-mater-Mahler-Laplace: Johannes ev. der
Stückeschreiber mit ihm der Thomaszwilling Jokaanan (also ich, Seamus,
weil ich deine Initiante bin) und Orpheus-HB/Eurydike. Jene alle also
auf diesem Weg, Hermes führt, (Gadamer) geht voraus - \ldots hätte ich
HB länger kennengelernt, wäre mehr (von dessen) Kraft in mich
übergegangen, jetzt muß ich sie selbst suchen im Archiv, das die
Überlieferung rechtfertigt. Ein wenig nahm ich n.~auf, ein wenig\ldots{}
das müßte reichen in dem supraleitfähigen Material, das die Ideen
beherbergt. Und den Anschluß bildet das \emph{I} der unerlaubten Weihe -
eine harmlose aber wirksame Droge, die erst im Alter zur Geltung kommt,
wenn nichts mehr hilft zur \emph{apathia} der Vorgängigen. Das verstand
ich. An ihr haftete die Beweisführung allen Materials und ging auch
daran zugrunde, wenn jemand es wollte, aber wollen mußte er es,
zumindest das war ein Gewinn für mich.

\end{document}
