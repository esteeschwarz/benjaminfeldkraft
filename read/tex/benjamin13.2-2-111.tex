% Options for packages loaded elsewhere
\PassOptionsToPackage{unicode}{hyperref}
\PassOptionsToPackage{hyphens}{url}
%
\documentclass[
]{article}
\usepackage{amsmath,amssymb}
\usepackage{iftex}
\ifPDFTeX
  \usepackage[T1]{fontenc}
  \usepackage[utf8]{inputenc}
  \usepackage{textcomp} % provide euro and other symbols
\else % if luatex or xetex
  \usepackage{unicode-math} % this also loads fontspec
  \defaultfontfeatures{Scale=MatchLowercase}
  \defaultfontfeatures[\rmfamily]{Ligatures=TeX,Scale=1}
\fi
\usepackage{lmodern}
\ifPDFTeX\else
  % xetex/luatex font selection
\fi
% Use upquote if available, for straight quotes in verbatim environments
\IfFileExists{upquote.sty}{\usepackage{upquote}}{}
\IfFileExists{microtype.sty}{% use microtype if available
  \usepackage[]{microtype}
  \UseMicrotypeSet[protrusion]{basicmath} % disable protrusion for tt fonts
}{}
\makeatletter
\@ifundefined{KOMAClassName}{% if non-KOMA class
  \IfFileExists{parskip.sty}{%
    \usepackage{parskip}
  }{% else
    \setlength{\parindent}{0pt}
    \setlength{\parskip}{6pt plus 2pt minus 1pt}}
}{% if KOMA class
  \KOMAoptions{parskip=half}}
\makeatother
\usepackage{xcolor}
\usepackage[margin=1in]{geometry}
\usepackage{graphicx}
\makeatletter
\def\maxwidth{\ifdim\Gin@nat@width>\linewidth\linewidth\else\Gin@nat@width\fi}
\def\maxheight{\ifdim\Gin@nat@height>\textheight\textheight\else\Gin@nat@height\fi}
\makeatother
% Scale images if necessary, so that they will not overflow the page
% margins by default, and it is still possible to overwrite the defaults
% using explicit options in \includegraphics[width, height, ...]{}
\setkeys{Gin}{width=\maxwidth,height=\maxheight,keepaspectratio}
% Set default figure placement to htbp
\makeatletter
\def\fps@figure{htbp}
\makeatother
\setlength{\emergencystretch}{3em} % prevent overfull lines
\providecommand{\tightlist}{%
  \setlength{\itemsep}{0pt}\setlength{\parskip}{0pt}}
\setcounter{secnumdepth}{-\maxdimen} % remove section numbering
\ifLuaTeX
  \usepackage{selnolig}  % disable illegal ligatures
\fi
\usepackage{bookmark}
\IfFileExists{xurl.sty}{\usepackage{xurl}}{} % add URL line breaks if available
\urlstyle{same}
\hypersetup{
  hidelinks,
  pdfcreator={LaTeX via pandoc}}

\author{}
\date{\vspace{-2.5em}}

\begin{document}

\subsection{N - VI.}\label{n---vi.}

Das Rasterbild der Elektronenröhre, wenn das Medium ermüdet ist von
unsteten Emissionen ist gegenläufig zu den Ausschüttungserscheinungen
der den Dopaminhaushalt regelnden (Druse) nach einem körperlich
anstrengenden Tag im Freien. Das bedeutet hier einen Anstieg der
Motivation, immer n.~etwas abbilden zu wollen gegen den unleugbaren
Erschöpfungszustand der Moleküle. Die Nebelkammer im einen Fall und der
Sprachapparat im anderen benötigen aber zur Abbildung beide den
ausführenden Gedanken, welcher die Resultate verzeichnet, so lange sie
sich ereignen. Jener ist dann seinerseits abhängig von so vielen
Gegebenheiten der Situation, daß nicht mehr d.~Grad d.~Erschöpfung
allein entscheidend für seinen Fortgang oder seine Beendigung wäre
sondern auch zum Beispiel ein einfacher Wille, frei und los von einer
äußerlich bestimmten Richtung ausschlaggebend sein könnte für den
Verlauf der Aufzeichnung. Ich nahm diesen Willen schon einmal an als
eine vierte konstitutive Kraft im Atomverband, damals als regulierendes
Glied zwischen Elementarteilchen, das n.~keinen Vektor und keine Größe
besitzt aber vorhanden sein könnte in beiden Richtungen und frei von
äußeren Bestimmungen, nur als ideeller Ausdruck der Wollensrichtung des
einen im Bezug zum anderen Teilchen. Selbst nun, wenn nichts konkret
wird, gibt es die Intention, auch verändert sich nichts durch ihren
Wegfall außerhalb ihrer\ldots{}

\end{document}
