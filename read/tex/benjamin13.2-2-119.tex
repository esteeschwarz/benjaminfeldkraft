% Options for packages loaded elsewhere
\PassOptionsToPackage{unicode}{hyperref}
\PassOptionsToPackage{hyphens}{url}
%
\documentclass[
]{article}
\usepackage{amsmath,amssymb}
\usepackage{iftex}
\ifPDFTeX
  \usepackage[T1]{fontenc}
  \usepackage[utf8]{inputenc}
  \usepackage{textcomp} % provide euro and other symbols
\else % if luatex or xetex
  \usepackage{unicode-math} % this also loads fontspec
  \defaultfontfeatures{Scale=MatchLowercase}
  \defaultfontfeatures[\rmfamily]{Ligatures=TeX,Scale=1}
\fi
\usepackage{lmodern}
\ifPDFTeX\else
  % xetex/luatex font selection
\fi
% Use upquote if available, for straight quotes in verbatim environments
\IfFileExists{upquote.sty}{\usepackage{upquote}}{}
\IfFileExists{microtype.sty}{% use microtype if available
  \usepackage[]{microtype}
  \UseMicrotypeSet[protrusion]{basicmath} % disable protrusion for tt fonts
}{}
\makeatletter
\@ifundefined{KOMAClassName}{% if non-KOMA class
  \IfFileExists{parskip.sty}{%
    \usepackage{parskip}
  }{% else
    \setlength{\parindent}{0pt}
    \setlength{\parskip}{6pt plus 2pt minus 1pt}}
}{% if KOMA class
  \KOMAoptions{parskip=half}}
\makeatother
\usepackage{xcolor}
\usepackage[margin=1in]{geometry}
\usepackage{graphicx}
\makeatletter
\def\maxwidth{\ifdim\Gin@nat@width>\linewidth\linewidth\else\Gin@nat@width\fi}
\def\maxheight{\ifdim\Gin@nat@height>\textheight\textheight\else\Gin@nat@height\fi}
\makeatother
% Scale images if necessary, so that they will not overflow the page
% margins by default, and it is still possible to overwrite the defaults
% using explicit options in \includegraphics[width, height, ...]{}
\setkeys{Gin}{width=\maxwidth,height=\maxheight,keepaspectratio}
% Set default figure placement to htbp
\makeatletter
\def\fps@figure{htbp}
\makeatother
\setlength{\emergencystretch}{3em} % prevent overfull lines
\providecommand{\tightlist}{%
  \setlength{\itemsep}{0pt}\setlength{\parskip}{0pt}}
\setcounter{secnumdepth}{-\maxdimen} % remove section numbering
\ifLuaTeX
  \usepackage{selnolig}  % disable illegal ligatures
\fi
\usepackage{bookmark}
\IfFileExists{xurl.sty}{\usepackage{xurl}}{} % add URL line breaks if available
\urlstyle{same}
\hypersetup{
  hidelinks,
  pdfcreator={LaTeX via pandoc}}

\author{}
\date{\vspace{-2.5em}}

\begin{document}

\subsection{S - XI.}\label{s---xi.}

Aber in den anderen sind den nichtkanonischen Gesängen Kräfte versteckt,
die sich am Sabbath dem neu entfalten, der die alten Schriften offen
ansieht. Mit offenem Herzen heißt das, nicht nur mit wachem und
aufnahmefähigen Geist, auch mit dem nicht zu steuernden Auge das ins
Verborgne schaut liest man sie. Manchmal tritt einem daraus etwas
schauerlich entgegen, das ängstigen kann was man sich n.~gerade erlaubt.
Diese Abschweifung ist immanent, ihr lest davon und ich versuche den
Faden immer anzuknüpfen von dem das Buch abhängt.\\
Tut es das aber immer n., und ist nicht der Inhalt mächtig durch das
Wort, mit dem ich ihn zu verheimlichen suche? Blättert Seiten vor und
vielleicht steht dann schon irgendwannwas, das euren Fragen
entgegenkommt, die ihr meiner Geschichte stellt. Ich werd die
Schlüsselfiguren auffordern, an ihrer Entwicklung mitzuarbeiten, aber
zwingen kann ich sie nicht. Oft lassen sie auf sich warten, teiln nur
dem Gehör sich mit.\\
Doch etwas im Ohr ist stumpf geworden in den Tagen, so als befänden sich
die schwingenden Teile plötzlich in einem anderen Medium als der
restliche Körper. Wir können das natürlich mit seiner Realität der
Wasserorgel ganz einfach erklären. Aber ich befinde mich ja nicht mehr
unter Wasser sondern habe den See wieder verlassen nach der Legende von
den sechs Brüdern und ihrer Schwester. Die beständigen Tage werden
kürter, aber n.~isthelle Nacht über dem Uurainen und seine Insel von den
Nebeln umzogen, die Felsen ragen heraus und die kleinen Felstische wo du
den Fisch zerteiltest \emph{und mich also aufgeweckt hast.} Ich frage
dich tief und ehrlich nach dem Grund der Erweckung. Ist es dir um das
Buch geschehen? Denn wenn du mit einem zuinnerst gefundenen Ja! laut
aufantworten kannst, dann werden ich und die anderen unseren Weg zu dir
finden, um die Kraft zu erneuern verspreche ich dir hiermit. Ist es dir
aber nicht im ganzen ernst damit und du hast geringe Zweifel an deiner
Berechtigung mich zu rufen, wird dich die Aufgabe zerschmettern wie sie
etliche vor dir aus dem Leben abberufen hat ohne Ansicht ihrer
Charaktere.\\
Der erste erschien ohne Grund. Ohne Vorwissen, ohne fiktionale Ursache
und hatte Wirkung nur in der Erscheinung, nicht in der Substanz. Aber
war daraus untrüglich hervorgegangen, so daß es eines zweiten bedurfte,
um ihn zu erblicken. Der ist schon gleich der Mensch der Menschen
gewesen und viel Zeit blieb ihm nicht allein. Denn auch er wünschte sich
ein Gegenüber und bekam ein Wesen untertan, das sich bis heute in uns
erhält - das Dritte: War die große Mutter und jeder kann sie n.~fühlen,
so lang der Körper sich n.~nicht entfremdet hat dem anderen dem Menschen
aus zweitem Grund, den wir nach uns zogen. Ihm blicken wir ins
aufwachsende Gesicht - Das Vierte Glied, so will es das unvermeidlich
Große Wort, das wir immer umgehen werden müssen, wenn es bezeichnet
werden will, denn wer vermags, wer darf es nennen, das sich ja nur
selbst aussprechen kann in seiner Wirklichkeit und nicht wiederholt,
nicht nachgesprochen oder gedacht werden - das Vierte Glied erst wird
zum Vater der Gedanken und verhilft aus der ihm gegebenen Kraft allen
anderen in ihr Dasein, aber den Vergangenen genauso wie den zukünftigen
so wie es alles die Macht hat in sein Gegenteil zu verkehren und niemand
vor ihm sicher ist. Das ist sein Charakter und wird nur aufgehalten von
dem also auch ihm selbst entsprungenen Fünften wie die linke Hand: weil
es schwach ist und müde von allem Leben das ihm immer entrinnt, unfähig
es zu bewahren aber glücklich es weitergeben zu können. \emph{Es ist
seine heimliche Grenze.} Darüberhinaus kann es nur n.~Vermutungen geben,
am nahesten kommt man ihm durch das geflossene Blut, wenn es im Kelch
als Weiheopfer dargereicht wird; dann mag sich seines etwas mitteilen,
das sich ebensolange verhindern ließ, \emph{auch darum also, liebe
patres\ldots{} bis zum sechsten, das sich unaufhaltsam nähert\ldots{}}
schweige ich über die anderen mir glücklich hinter dem Schleier
verborgenen Gesetze, um die sich niemand kümmern sollte. Es gibt sie,
das muß zur Gewißheit über die hier anstehenden fragwürdigen Charaktere
ausreichen und mich ihnen vermitteln, so wie sie sich mir vermittelt
haben.\\
Mich: seine Mitteilung muß ausbleiben, weil sie über sich hinausweist.\\
Thomasich: verzweifelt wegen der angezeigten Höhenlinien in der
Rasterauflösung, die eine Ortung erst im dritten Raum ermöglichen.\\
Ewa: kennt bereits die Koordinaten der Vollendung, kann aber über deren
Transfer nicht verfügen, hofft also, daß\\
Johannes: hat alle notwendigen Weisungen erhalten, um die Weihe
vollziehen zu dürfen, allein\\
Orpheus: ist sich selbst ein Geweihter und beansprucht das Gebiet für
sich, das wir ihm eigentlich öffnen wollten. Das bedeutet den Konflikt
mit der hermetischen Handlung.\\
Und damit wären wir endlich da angelangt wohin es von selbst strebte: in
den Bereich der Geburt - in das Fahrwasser der Tragödie - aus dem Geist
der Musik.

\end{document}
