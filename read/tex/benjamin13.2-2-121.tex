% Options for packages loaded elsewhere
\PassOptionsToPackage{unicode}{hyperref}
\PassOptionsToPackage{hyphens}{url}
%
\documentclass[
]{article}
\usepackage{amsmath,amssymb}
\usepackage{iftex}
\ifPDFTeX
  \usepackage[T1]{fontenc}
  \usepackage[utf8]{inputenc}
  \usepackage{textcomp} % provide euro and other symbols
\else % if luatex or xetex
  \usepackage{unicode-math} % this also loads fontspec
  \defaultfontfeatures{Scale=MatchLowercase}
  \defaultfontfeatures[\rmfamily]{Ligatures=TeX,Scale=1}
\fi
\usepackage{lmodern}
\ifPDFTeX\else
  % xetex/luatex font selection
\fi
% Use upquote if available, for straight quotes in verbatim environments
\IfFileExists{upquote.sty}{\usepackage{upquote}}{}
\IfFileExists{microtype.sty}{% use microtype if available
  \usepackage[]{microtype}
  \UseMicrotypeSet[protrusion]{basicmath} % disable protrusion for tt fonts
}{}
\makeatletter
\@ifundefined{KOMAClassName}{% if non-KOMA class
  \IfFileExists{parskip.sty}{%
    \usepackage{parskip}
  }{% else
    \setlength{\parindent}{0pt}
    \setlength{\parskip}{6pt plus 2pt minus 1pt}}
}{% if KOMA class
  \KOMAoptions{parskip=half}}
\makeatother
\usepackage{xcolor}
\usepackage[margin=1in]{geometry}
\usepackage{graphicx}
\makeatletter
\def\maxwidth{\ifdim\Gin@nat@width>\linewidth\linewidth\else\Gin@nat@width\fi}
\def\maxheight{\ifdim\Gin@nat@height>\textheight\textheight\else\Gin@nat@height\fi}
\makeatother
% Scale images if necessary, so that they will not overflow the page
% margins by default, and it is still possible to overwrite the defaults
% using explicit options in \includegraphics[width, height, ...]{}
\setkeys{Gin}{width=\maxwidth,height=\maxheight,keepaspectratio}
% Set default figure placement to htbp
\makeatletter
\def\fps@figure{htbp}
\makeatother
\setlength{\emergencystretch}{3em} % prevent overfull lines
\providecommand{\tightlist}{%
  \setlength{\itemsep}{0pt}\setlength{\parskip}{0pt}}
\setcounter{secnumdepth}{-\maxdimen} % remove section numbering
\ifLuaTeX
  \usepackage{selnolig}  % disable illegal ligatures
\fi
\usepackage{bookmark}
\IfFileExists{xurl.sty}{\usepackage{xurl}}{} % add URL line breaks if available
\urlstyle{same}
\hypersetup{
  hidelinks,
  pdfcreator={LaTeX via pandoc}}

\author{}
\date{\vspace{-2.5em}}

\begin{document}

\subsection{T - XIII.}\label{t---xiii.}

Einmal war erstes Erwachen der Sterne für den Menschen. Den Kopf hob er,
aufrecht gehend, ohne Grund zum Firmament und fing den Überschuß zu
denken an, der keine Notwendigkeit hatte neben jagen, schlafen, sich
vermehren. Dann wurde er Wagender. Er sollte nur helfen, den Tod zu
überwinden. Aber in jedem Kunstwerk tat er n.~mehr als das: erschuf
Leben und dabei lehrte uns. Dahin gingen sämtliche Schritte und wir
lernten sehr schnell.\\
Jedoch wir befanden uns irgendwann mitten in der Partitur und wurden
unterbrochen. Ich muß zurückfallen, um euch zu der Stelle zu bringen, wo
wir uns trennten. Dann werden wir Hermes n.~einmal gemeinsam folgen und
ich blicke mich nicht mehr nach euch um weil ich darauf vertraue daß ihr
nachkommt; ihr habt gelernt die Grundlagen zur Ausübung der
Wissenschaft, wendet sie also selbstbewußt an und geht meinen Spuren
nach, sie sind sicher gestreut vor euch und H. leuchtet voran. Auch wenn
ihr mich nicht mehr seht ist der Weg gangbar und ich nur ein Vorgänger
so wie die genannten für mich es waren - und auch sie haben keinen
verloren.\\
Was uns erwartete: \emph{damit die Zeichen, die sie umstanden, nicht zu
Omen würden werden können,} hat man mir eines klar gegeben - die
ungeschriebene Zukunft und das Buch\ldots{} war in der Theorie
begriffen. Der Chor von Protagonisten in Position. Wie lange? - So lange
das Vorspiel eben n.~dauert. \emph{Sie befinden sich Hier!} können wir
jetzt schon raunen hören, bevor der Vorhang überhaupt aufgezogen wurde,
das ist mehr als eine Ahnung des Kommenden. Dann sieh hin wenn es so
weit ist und verpasse jetzt nicht\\
Die erste Szene: \emph{Ewa sitzt an einem einfachen kleinen Holztisch
auf der Veranda v. der Hütte. Vor ihr liegt ein Stapel Manuskriptpapier,
in dem sie mit einem Bleistift in der Hand blättert. Sie fährt damit
fort bis ein Geräusch sie aufhorchen läßt. Da es außerhalb der V. sehr
dunkel ist und die Kerze nur den Tisch erleuchtet, muß sie die Augen
dagegen abschirmen, um hinausblicken zu können. Sie schaut angestrengt
in das Dunkel, sieht nichts, kehrt zu ihrer Arbeit zurück. }\\
Ewa: - da machts mich fertig, daß ers nicht zu ende bringt. Seht euch
das an, S beginnt so vielversprechend, baut sich auf und trägt einen
hinein in das Stück. Und dann? Nichts. Großes dumpfes und eindeutiges
Nichts. Das erschlägt einn doch\ldots{}\\
\emph{Während sie weiter murmelt hören wir den Chor flüstern, nur
flüstern, aber laut, kanonisch:}\\
Ewa: \emph{- \ldots wenn ich mich nur an den ersten Satz erinnern
könnte, seinen ersten Satz\ldots{} }laut: es gab eine Fuge, die er nicht
angerührt hat, soweit erinner ich mich n.~und er wollte sie nie
angespielt wissen. Inhaltlich. Der Form nach schon sehe ich jetzt, im
Faksimile, aber inhaltlich? Nein, daran war ihm nichts gelegen, die
Vollkommenheit des Materials bestand erst in der Veredelung nach ihm.
Ihm war es nur um sein Steinchen zu diesem Mosaik zu tun, er wußte unser
Wissen sei Stückwerk. \emph{fährt sich durch die Haare - }an die Arbeit!
Er muß etwas hinterlassen haben, das mehr ist als Fragment, aber was
soll das sein\ldots{} ich muß den Stein fragen.\\
Ewa legt den Fragestein vor sich hin und tippt ihn mit dem Bleistift
an:\\
Stein, Stein, laß das Grautier ein. wartet\\
Stein Stein, laß das Grautier ein\ldots{} wartet, dannn stärker,
akzentuiert\\
\emph{Beim letzten Wort springt ein Funke vom Stein, erschrocken läßt
Ewa den Bleistift fallen. }\\
Stein: - wenn du mich fragen würdest, könnte ich auch antworten.\\
Ewa: - weil es gegen das Gesetz ist. Es sei denn, der Chor würde sie
n.~einmal für mich stellen. Chor?\\
Stein: - ich habe es verstanden und gebe die Antwort, wenn sie mich
erreicht. So lange fasse dich in Geduld und deine Worte kurz, daß du
nicht im Gefasel endest.\\
\emph{Ewa steht vom Tisch auf und geht unruhig hin und her, steigt die
Stufen von der Veranda auf den knirschenden Kies, geht in die Hocke und
schaut unter den von der Veranda gebildeten seitlichen Bodenverschlag,
der dunkel und unheimlich anmutet. - }brrrr\ldots! \emph{schüttelt sich.
Steigt wieder nach oben, schleicht um den Stein auf dem Tisch herum. }-
da liegt er und schweigt also. So so\ldots{} weißt auch keine Antwort,
Stein, \emph{pause,} Steinhn\ldots{} \emph{pause}\\
- wenn du mehr wärst als eine Siliziumbrückenverbindung, hätte ich auch
Respekt vor dir. Aber du bist nur ein sprechender Kiesel und von dem
lass ich mir keine Angst einjagen - \emph{wird unterbrochen}\\
Stein: - ich habe eine Antwort für dich, willst du sie jetzt hören?\\
Stein: - Also gut: Du hast die Begrenzung der zweiten Ebene erreicht.
Innerhalb dieser Dimension wird es keine neuen Erkenntnisse mehr für
dich geben. Du mußt dich also für den Sprung in das nächste Niveau
bereithalten, wo es dann andere Regeln für dich gibt. Nimm deinen Willen
und deine Freiheit fest an dich, ich werde dir ein Zeichen geben, wenn
es so weit ist. Sei gefaßt, denn dann mußt du springen.

\end{document}
