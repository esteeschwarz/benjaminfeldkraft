% Options for packages loaded elsewhere
\PassOptionsToPackage{unicode}{hyperref}
\PassOptionsToPackage{hyphens}{url}
%
\documentclass[
]{article}
\usepackage{amsmath,amssymb}
\usepackage{iftex}
\ifPDFTeX
  \usepackage[T1]{fontenc}
  \usepackage[utf8]{inputenc}
  \usepackage{textcomp} % provide euro and other symbols
\else % if luatex or xetex
  \usepackage{unicode-math} % this also loads fontspec
  \defaultfontfeatures{Scale=MatchLowercase}
  \defaultfontfeatures[\rmfamily]{Ligatures=TeX,Scale=1}
\fi
\usepackage{lmodern}
\ifPDFTeX\else
  % xetex/luatex font selection
\fi
% Use upquote if available, for straight quotes in verbatim environments
\IfFileExists{upquote.sty}{\usepackage{upquote}}{}
\IfFileExists{microtype.sty}{% use microtype if available
  \usepackage[]{microtype}
  \UseMicrotypeSet[protrusion]{basicmath} % disable protrusion for tt fonts
}{}
\makeatletter
\@ifundefined{KOMAClassName}{% if non-KOMA class
  \IfFileExists{parskip.sty}{%
    \usepackage{parskip}
  }{% else
    \setlength{\parindent}{0pt}
    \setlength{\parskip}{6pt plus 2pt minus 1pt}}
}{% if KOMA class
  \KOMAoptions{parskip=half}}
\makeatother
\usepackage{xcolor}
\usepackage[margin=1in]{geometry}
\usepackage{graphicx}
\makeatletter
\def\maxwidth{\ifdim\Gin@nat@width>\linewidth\linewidth\else\Gin@nat@width\fi}
\def\maxheight{\ifdim\Gin@nat@height>\textheight\textheight\else\Gin@nat@height\fi}
\makeatother
% Scale images if necessary, so that they will not overflow the page
% margins by default, and it is still possible to overwrite the defaults
% using explicit options in \includegraphics[width, height, ...]{}
\setkeys{Gin}{width=\maxwidth,height=\maxheight,keepaspectratio}
% Set default figure placement to htbp
\makeatletter
\def\fps@figure{htbp}
\makeatother
\setlength{\emergencystretch}{3em} % prevent overfull lines
\providecommand{\tightlist}{%
  \setlength{\itemsep}{0pt}\setlength{\parskip}{0pt}}
\setcounter{secnumdepth}{-\maxdimen} % remove section numbering
\ifLuaTeX
  \usepackage{selnolig}  % disable illegal ligatures
\fi
\usepackage{bookmark}
\IfFileExists{xurl.sty}{\usepackage{xurl}}{} % add URL line breaks if available
\urlstyle{same}
\hypersetup{
  hidelinks,
  pdfcreator={LaTeX via pandoc}}

\author{}
\date{\vspace{-2.5em}}

\begin{document}

\subsection{U - 1}\label{u---1}

Wenn man einmal außer acht läßt was uns die Schule über
Wahrscheinlichkeit lehren wollte und das andere, was wir später
dazulernten von Heisenberg und das, was irgendwann dazwischen schon
einmal aufgetaucht war als \emph{Heart of Gold,} 1. ein Raumschiff mit
Wahrscheinlichkeitsantrieb und 2. eigentlich die unvermeidliche Gitarre
Neil Youngs - dann können wir an der jetzigen Stelle nicht mehr tun als
uns darauf verlassen, das Ewa gut zugehört hat und ich sie so aufmerksam
wie möglich die Zukunft verbringen ließ, damit sie ihren Befehl zu
springen nicht verpassen sollte. Würde sie das, so wäre alles hinfällig,
was ich bisher geschrieben habe über meine und ihre Vollendung der
Symphonie des Ersten Bandes, es könnte so wie ich es schilderte nicht
stattgefunden haben. Hat es aber und deshalb muß jemand Zeugnis ablegen
darüber, wie sie von der Reling sprang des Passagierfrachtschiffes P
mitten in die Ostsee auf der Hälfte der Strecke zwischen Helsinki und
Rostock. Es war mir zu jenem Zeitpunkt als ich das ersann klar daß
jemand dabei gewesen ist.\\
War er dann auch, aber daß er ihr so nahe kam während der Zeiten mit ihr
in den Finnlandsommern habe ich nicht geahnt. Also auch nicht, daß
dieses Buch geschrieben werden würde, was ja als Zettel gereicht hätte
und alles beglaubigt was ich von ihr zu wissen brauchte: die Koordinaten
genau in Raum und Zeit des Sprunges. Seine protagoniste Nähe zu diesem
Ewamoment ist der Schlüssel ihrer Geschichte; die sie ja in die See
mitgenommen hatte - aber nicht als in ihren Tod. Das hatte ihr der
Kiesel versucht zu vermitteln und sie verstand es scheinbar wirklich.
Sonst wär sie nicht mehr unter uns. Das frage ich euch: ist sie denn
nicht hier? Spürt ihr sie denn nicht? Wenn es so ist: wenn ihr dieses
verneinen müßt also mir sagt, daß ihr sie nicht gesehen habt und immer
n.~seht, dann ist auch für sie alles zu spät gewesen, was sie lernte und
vollbringen wollte, dann bleibt auch ihre Symphonie absolut Fragment.
Und wenn das von Mahler so gedacht war, dann habt ihr recht sie nicht
anzuerkennen. Wer aber aufrichtig weiß, daß aneinemder Todestage im März
2011 mehr dazu passieren wird als nur versandender Applaus über den
jeweiligen Dirigenten im jeweiligen Saal irgendeiner beliebigen
Weltstadt, der ist hiermit aufgefordert, die Schrift dahingehend zu
befördern, daß sie zu jenem Datum in der Welt sei. Damit wäre das
Zusammentreffen dieser beiden Menschen, Ewa-Alma mater Mahler-Laplace
und Gustav Mahler an dem n.~in dieser Schrift zu bestehenden Termin
endlich durchgesetzt und bewahrheitet. Es wird dazu kommen, die
dramatische Form hat ihren Eingang schon gefunden. N. ein paar solcher
Begegnungen in der ersten Person und beide werden sich hier wie zu hause
fühlen.

\end{document}
