% Options for packages loaded elsewhere
\PassOptionsToPackage{unicode}{hyperref}
\PassOptionsToPackage{hyphens}{url}
%
\documentclass[
]{article}
\usepackage{amsmath,amssymb}
\usepackage{iftex}
\ifPDFTeX
  \usepackage[T1]{fontenc}
  \usepackage[utf8]{inputenc}
  \usepackage{textcomp} % provide euro and other symbols
\else % if luatex or xetex
  \usepackage{unicode-math} % this also loads fontspec
  \defaultfontfeatures{Scale=MatchLowercase}
  \defaultfontfeatures[\rmfamily]{Ligatures=TeX,Scale=1}
\fi
\usepackage{lmodern}
\ifPDFTeX\else
  % xetex/luatex font selection
\fi
% Use upquote if available, for straight quotes in verbatim environments
\IfFileExists{upquote.sty}{\usepackage{upquote}}{}
\IfFileExists{microtype.sty}{% use microtype if available
  \usepackage[]{microtype}
  \UseMicrotypeSet[protrusion]{basicmath} % disable protrusion for tt fonts
}{}
\makeatletter
\@ifundefined{KOMAClassName}{% if non-KOMA class
  \IfFileExists{parskip.sty}{%
    \usepackage{parskip}
  }{% else
    \setlength{\parindent}{0pt}
    \setlength{\parskip}{6pt plus 2pt minus 1pt}}
}{% if KOMA class
  \KOMAoptions{parskip=half}}
\makeatother
\usepackage{xcolor}
\usepackage[margin=1in]{geometry}
\usepackage{graphicx}
\makeatletter
\def\maxwidth{\ifdim\Gin@nat@width>\linewidth\linewidth\else\Gin@nat@width\fi}
\def\maxheight{\ifdim\Gin@nat@height>\textheight\textheight\else\Gin@nat@height\fi}
\makeatother
% Scale images if necessary, so that they will not overflow the page
% margins by default, and it is still possible to overwrite the defaults
% using explicit options in \includegraphics[width, height, ...]{}
\setkeys{Gin}{width=\maxwidth,height=\maxheight,keepaspectratio}
% Set default figure placement to htbp
\makeatletter
\def\fps@figure{htbp}
\makeatother
\setlength{\emergencystretch}{3em} % prevent overfull lines
\providecommand{\tightlist}{%
  \setlength{\itemsep}{0pt}\setlength{\parskip}{0pt}}
\setcounter{secnumdepth}{-\maxdimen} % remove section numbering
\ifLuaTeX
  \usepackage{selnolig}  % disable illegal ligatures
\fi
\usepackage{bookmark}
\IfFileExists{xurl.sty}{\usepackage{xurl}}{} % add URL line breaks if available
\urlstyle{same}
\hypersetup{
  hidelinks,
  pdfcreator={LaTeX via pandoc}}

\author{}
\date{\vspace{-2.5em}}

\begin{document}

\subsection{503.254}\label{section}

Natürlich wird das keiner lesen können, der vor 2045 anfing zu
schreiben. Es sagt uns etwas für heute: Zeit ist nicht immer relativ
gewesen. Das besagt weiter: sie ist es n.~nicht, wenn wir uns an
bestehende Regeln halten über unsere Umwelt. Zeit ist ein konstitutives
Element der menschlichen Wahrnehmung das Sinnzusammenhänge zwischen
Erfahrungen herstellt, indem es sie physisch verknüpft auf einem ganz
herkömmlichen Weg: dem chemischer Reaktionen. Was soll das heißen? Daß:
wenn wir dem Gehirn ein paar Geheimnisse über genau diese konstitutive
Funktion entschleiern wollen, müssen wir es sich selbst
gegenüberstellen. Die alchemistischen Verfahrensweisen zur Nachbildung
jener chemischen Prozesse sind nicht zu Unrecht als esoterisches
Grundkonzept jeglicher Naturwissenschaft bezeichnet worden, sie
versuchen auf dem Weg der Informationsvermittlung Relationen zwischen
Ereignissen herzustellen, die zeitlich und räumlich realiter ohne
Zusammenhang sind; im Medium aber unserer gemeinsamen
Kohlenstoffverbindungen gibt es ihn und wir erfahren ihn jeden
handelnden Moment neu. Und neu hieß einmal im politisch angemessenen
Sinne: daß zB Gedichte nach dem Krieg nicht mehr zu schreiben waren.
Eine Menge Autoren hat sich daran gehalten, sie schrieben an Texten
statt Gedichte. Aber irgendwann konnten sie schleichend zu unserem
Hölderlin zurückkehren und sogar zu Rilke, ohne schuldig zu werden. Ich
verlasse mich auf meine Intuition dieses Buch hier zu schreiben in der
unheiligen Verfassung damit den unbekannten Gedichten von heute keinen
großen Gefallen zu tun; sie sind fast ausnahmslos schlechtes Deutsch das
sich darin gefällt durch seine Unvollkommenheit nicht mehr als gerade
n.~zu provozieren - ohne den Mut jedoch zu haben, den Grund dafür zu
bekennen. Den vermute ich da, wo mein Gr. für solche Gedichte auch sein
würde: im zurückgedrängten Nationalstolz (Federvieh), der dann nicht
anders geht als in Haß ("Erinnerungskultur") sublimiert auf alles
ehrliche, auf alles was aufrichtig schön ist, so wie es Gedichte einmal
wirklich auch waren. Es gibt Ausnahmen. Toter wie lebender Dichter. Aber
sie zu kennen kostet Mut und auch ein gerüttelt Maß ataraktischen
Zufriedenseins mit der eigenen Sprache, so daß man nicht im Kauderwelsch
endet des angeblichen Versagens höherer Zwecke der Dichtung. Dichtung
will immer Hohes. Sonst wär sie nur Wissenschaft. Ok, das wars schon.

\end{document}
