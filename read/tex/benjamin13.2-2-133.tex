% Options for packages loaded elsewhere
\PassOptionsToPackage{unicode}{hyperref}
\PassOptionsToPackage{hyphens}{url}
%
\documentclass[
]{article}
\usepackage{amsmath,amssymb}
\usepackage{iftex}
\ifPDFTeX
  \usepackage[T1]{fontenc}
  \usepackage[utf8]{inputenc}
  \usepackage{textcomp} % provide euro and other symbols
\else % if luatex or xetex
  \usepackage{unicode-math} % this also loads fontspec
  \defaultfontfeatures{Scale=MatchLowercase}
  \defaultfontfeatures[\rmfamily]{Ligatures=TeX,Scale=1}
\fi
\usepackage{lmodern}
\ifPDFTeX\else
  % xetex/luatex font selection
\fi
% Use upquote if available, for straight quotes in verbatim environments
\IfFileExists{upquote.sty}{\usepackage{upquote}}{}
\IfFileExists{microtype.sty}{% use microtype if available
  \usepackage[]{microtype}
  \UseMicrotypeSet[protrusion]{basicmath} % disable protrusion for tt fonts
}{}
\makeatletter
\@ifundefined{KOMAClassName}{% if non-KOMA class
  \IfFileExists{parskip.sty}{%
    \usepackage{parskip}
  }{% else
    \setlength{\parindent}{0pt}
    \setlength{\parskip}{6pt plus 2pt minus 1pt}}
}{% if KOMA class
  \KOMAoptions{parskip=half}}
\makeatother
\usepackage{xcolor}
\usepackage[margin=1in]{geometry}
\usepackage{graphicx}
\makeatletter
\def\maxwidth{\ifdim\Gin@nat@width>\linewidth\linewidth\else\Gin@nat@width\fi}
\def\maxheight{\ifdim\Gin@nat@height>\textheight\textheight\else\Gin@nat@height\fi}
\makeatother
% Scale images if necessary, so that they will not overflow the page
% margins by default, and it is still possible to overwrite the defaults
% using explicit options in \includegraphics[width, height, ...]{}
\setkeys{Gin}{width=\maxwidth,height=\maxheight,keepaspectratio}
% Set default figure placement to htbp
\makeatletter
\def\fps@figure{htbp}
\makeatother
\setlength{\emergencystretch}{3em} % prevent overfull lines
\providecommand{\tightlist}{%
  \setlength{\itemsep}{0pt}\setlength{\parskip}{0pt}}
\setcounter{secnumdepth}{-\maxdimen} % remove section numbering
\ifLuaTeX
  \usepackage{selnolig}  % disable illegal ligatures
\fi
\usepackage{bookmark}
\IfFileExists{xurl.sty}{\usepackage{xurl}}{} % add URL line breaks if available
\urlstyle{same}
\hypersetup{
  hidelinks,
  pdfcreator={LaTeX via pandoc}}

\author{}
\date{\vspace{-2.5em}}

\begin{document}

\subsection{34839 + 1 Wort:}\label{wort}

ist immer irgendwann das erste Wort gewesen. Hier erscheint es mir a.
besonders wichtig, auf den statistischen Wert zu verweisen, den die
Schöpfung mittlerweile angenommen hatte. Wir kommen an eine alte Grenze:
das ist, wenn wie jetzt die letzten 5.000 Wörter zu schreiben übrig sind
n.~nicht so gefährlich. Mit jeder Seite aber und ihren 800 Wörtern
rücken wir ein Stück ab vom Rand und steuern auf das Zentrum zu in
dieser platten fast konzentrische Kreise aber unaufhaltsamen Spiralbahn.
Und es ist immern. eine Seite denken wir aber bald werden die 6.25 auch
voll sein und dann muß etwas geschehen sein fürchte ich für ihre
Berechtigung. Es sind nicht nur Jahre, die einem vergehen oder die stets
sich ähnlicher werdenden Sommer und Winter im Garten. Kaum merklich
gleichen sich auch alle Menschen einander an sofern sie nicht selbst
ihre Berechtigung erfahren und die eigene Geschichte, das sie in den
Rang von Schöpfungen erhebt - für mich unverwechselbar. Sie sind es doch
auch muß ich mich erinnern und daß Sie jeder einmal einzigartig waren:
aber doch, daß sie verschmelzen. Diejenigen, welche sich zu retten
vermochten werden bleiben auch wenn sie gleich tot sind. Dann lassen wir
ein paar von ihnen jetzt eintreten, damit sie nicht verloren gehen.\\
Neben jenen, die so schon oft genug vorhanden, sind Ewa-Alwa-Alma mater
Mahler-Laplace alsauch Mignon und die Thomaszwillinge sowie
Jean-Jokaanan und HB/Orpheus-Eurydike, bleibt dann hier nur n.~zu klären
von Seamus und Maggie. Das sind vielleicht einfach die Eltern von allem
gewesen (halbkatholisch). Damit ist die Genesis vertan und ich könnte
mich schon ein paar Seiten früher aus dem Staube machen als ihr es
vielleicht erwartet hättet. Es gibt aber ein Problem: die Aufgabe war
gestellt den Lebenden zu helfen und nicht den Toten nachzutrauern. Die
erste Erfüllung erfuhr ich somit durch die Arbeit in besagtem Archiv,
welches die Hinterlassenschaft jenes orphischen Gelehrten HB war. Warum
ich damit angefangen habe, seine hermeneutischen Schriften zu zitieren
als von der Benjaminfeldkraft n.~gar nichts gewußt werden konnte (weil
ich sie ja erst fand aus ihnen!) war mir lange ein Rätsel. Doch in der
weiteren Beschäftigung mit den Notizen und der Entscheidung, den
musikkritischen Teil von jenem Ewamädchen ausführen zu lassen das ich
als Komponistin Laplace kennenlernte reifte auch Bewußtsein heran seiner
Idee der unmöglichen Kompositionen. Sie dann vollführte das Kunststück,
ein nicht spielfähiges/fertiges Orchester zu antizipieren oder vielmehr:
ein nicht spielbares Orchester zu benutzen zur Erschaffung der
x-Mahler-Symphonie. Daß auch Glück dabei war, ihren Zeitpunkt der
Vollendung mit seinem 100. Todestag zusammenzubringen, wußte sie nur
selbst. Für mich sah es aus wie eine professionelle Anordnung aller zum
Experiment gehörigen Module und das Eintreten der erwarteten
Wahrscheinlichkeit am event horizon dank ihrer genialen Erfindungsgabe.
Zum Genius gehört ja Glück dazu wie nichts anderes sonst; sogar so sehr,
daß man schon in weiteren Dingen total verzweifeln muß um seiner nur in
Ansätzen nicht habhaft zu werden. Wir sind also bis hierher gefolgt und
werden es bis zum Ende halten. Es braucht jedoch Zäsuren, wir sehnen uns
nach Unterbrechungen des konzentrierten Prozesses. Und nachdem in diesem
zweiten Band verschiedene Möglichkeiten erlebt wurden, Schnitte
darzustellen komme ich jetzt zur letzten einzigen realen Bedingung, die
Einschnitte rechtfertigt: daß ich \emph{outtime} vom Gang meiner
persönlichen Geschichte bestimmt werde und diese Abschnitte also nun zu
spüren sein sollen für euch. Wodurch wurde getrennt im Vorigen?
Taktstriche vielleicht oder Szenenwechsel, Umbaupausen; schließlich
sogar Akte nach dem Gefühl von Grunderschütterungen in der Ereignisbahn.
Nun also eine eigentlich streng chronologische Struktur, die aber für
die innere Chronologie unerheblich sein muß. Doch wirkt sie ja, nicht zu
verleugnen, wirkte ja auch bisher aber ich durftes nicht zugeben für den
Erhalt jener Dimension. Da wir uns aber zum Ende begeben werden, wird
nun einiges klar. Dem Wiener Lektor verprach ich den Schlüssel zum
Verständnis; daß aber die Aufgabe durch den Entwurf eines zweiten Bandes
gelöst werden müßte der sich selbst zum Inhalt hat und seinen succesiven
Charakter bis ins kleinste hinausarbeitet - das war im Anfang mir nicht
bewußt, ist es seit wenigen Einheiten und ich muß dem also n.~gerecht
werden. Beenden wir es jetzt. Die Nacht war gestern und der Rest ist
gelogen, diesmal sind es heute die Geschichten des morgigen Tages,
obwohl morgen immer erst der nächste Tag gewesen ist. Springen wir also.

\end{document}
