% Options for packages loaded elsewhere
\PassOptionsToPackage{unicode}{hyperref}
\PassOptionsToPackage{hyphens}{url}
%
\documentclass[
]{article}
\usepackage{amsmath,amssymb}
\usepackage{iftex}
\ifPDFTeX
  \usepackage[T1]{fontenc}
  \usepackage[utf8]{inputenc}
  \usepackage{textcomp} % provide euro and other symbols
\else % if luatex or xetex
  \usepackage{unicode-math} % this also loads fontspec
  \defaultfontfeatures{Scale=MatchLowercase}
  \defaultfontfeatures[\rmfamily]{Ligatures=TeX,Scale=1}
\fi
\usepackage{lmodern}
\ifPDFTeX\else
  % xetex/luatex font selection
\fi
% Use upquote if available, for straight quotes in verbatim environments
\IfFileExists{upquote.sty}{\usepackage{upquote}}{}
\IfFileExists{microtype.sty}{% use microtype if available
  \usepackage[]{microtype}
  \UseMicrotypeSet[protrusion]{basicmath} % disable protrusion for tt fonts
}{}
\makeatletter
\@ifundefined{KOMAClassName}{% if non-KOMA class
  \IfFileExists{parskip.sty}{%
    \usepackage{parskip}
  }{% else
    \setlength{\parindent}{0pt}
    \setlength{\parskip}{6pt plus 2pt minus 1pt}}
}{% if KOMA class
  \KOMAoptions{parskip=half}}
\makeatother
\usepackage{xcolor}
\usepackage[margin=1in]{geometry}
\usepackage{graphicx}
\makeatletter
\def\maxwidth{\ifdim\Gin@nat@width>\linewidth\linewidth\else\Gin@nat@width\fi}
\def\maxheight{\ifdim\Gin@nat@height>\textheight\textheight\else\Gin@nat@height\fi}
\makeatother
% Scale images if necessary, so that they will not overflow the page
% margins by default, and it is still possible to overwrite the defaults
% using explicit options in \includegraphics[width, height, ...]{}
\setkeys{Gin}{width=\maxwidth,height=\maxheight,keepaspectratio}
% Set default figure placement to htbp
\makeatletter
\def\fps@figure{htbp}
\makeatother
\setlength{\emergencystretch}{3em} % prevent overfull lines
\providecommand{\tightlist}{%
  \setlength{\itemsep}{0pt}\setlength{\parskip}{0pt}}
\setcounter{secnumdepth}{-\maxdimen} % remove section numbering
\ifLuaTeX
  \usepackage{selnolig}  % disable illegal ligatures
\fi
\usepackage{bookmark}
\IfFileExists{xurl.sty}{\usepackage{xurl}}{} % add URL line breaks if available
\urlstyle{same}
\hypersetup{
  hidelinks,
  pdfcreator={LaTeX via pandoc}}

\author{}
\date{\vspace{-2.5em}}

\begin{document}

\subsection{La nuit douzieme}\label{la-nuit-douzieme}

In dieser Nacht kam es dazu, daß Ewa so schnell ihrem vorausgesagten
Bestimmungsort entgegeneilte wie wir uns in entgegengesetzter Richtung
von dem Entschluß dazu entfernt hatten, eine Sitzung lang nichts anderes
zu machen als gerade hin (und wenn ich sage hin so meine ich ihn!!!
ihn!!!) voranzubringen. Er wurde in der Nacht zum treibenden Faktor der
Geschichte und ließ mich teilhaben an seinem Fortgang. Dadurch also
verloren wir langsam aus den Augen was wir eigentlich angestrebt hatten,
versucht von der leichten Erfassbarkeit seines Wesens. Nur, daß es
leicht war ließ mich hier nicht auf beschränkte Dimensionen schließen
sondern auf eine außergewöhnlich innere Abgeschlossenheit den ihn
hervorbringenden Umständen gegenüber. Jene hatten ihn endlich das
gelehrt, was ich ja unbedingt mir erschließen wollte: die Gelassenheit,
die Urteilsfreiheit - das Ideal der Ataraxie. \emph{Ich} ließ es ihn
finden. Und nur deshalb müssen wir die Seitenpfade verfolgen, weil eben
Prioritäten keine Rolle mehr spielen. Die Bezüge kommen unverwechselbar
regelmäßig und durchleuchten den Text an entsprechenden Stellen. Ich muß
nur wachsam genug die Einschlüsse wahrnehmen und dann darauf verweisen
(Hypertextvariablen). Was verliert man schon im Tagesgeschehen ständig
an Gedanken, das sollte hier keinen Einzug halten. Also bewahren also
aufschreiben immer n.~und die protagoniste Seamusgestaltung am Weggang
hindern. Ich kann ihn ja sehen (wehender Rock, Fontanebüßer) und seine
Gründe sind zu überzeugend uns zu verlassen. Dann werde ich mit ihm
reden. Etwa jetzt ihn ansprechen?\\
la nuit, c\textquotesingle est la nuit du dixieme. Tu sais que les
suivantes sont occupé avec les demands d\textquotesingle une voyage sans
fin.\\
Il y avait un jour quand tu m\textquotesingle appelais La Princesse
Marie-Astrid. Ou est-ce que les temps?\\
je ne suis pas le Jean qui tu pense a connaitre. Je suis un autre. Je
suis adult.\\
quand même: Ou sont les nuits? Je ne le veux savoir\ldots{}\\
Mais je le toi dis: ils sont passé, ils sont passé - mais je
n\textquotesingle oubliais pas une. Ils reste dans mon coeur pour siecle
et siecle comme tu dirais. Mais moi - il n\textquotesingle y aura plus.
C\textquotesingle est sure et je le toi disai.\\
simplement. Tres bien a comprendre. Je me veux vous presenter: Mon nom
est Benjamin, comme la force\ldots{}

\end{document}
