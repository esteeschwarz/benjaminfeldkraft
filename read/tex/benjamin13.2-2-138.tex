% Options for packages loaded elsewhere
\PassOptionsToPackage{unicode}{hyperref}
\PassOptionsToPackage{hyphens}{url}
%
\documentclass[
]{article}
\usepackage{amsmath,amssymb}
\usepackage{iftex}
\ifPDFTeX
  \usepackage[T1]{fontenc}
  \usepackage[utf8]{inputenc}
  \usepackage{textcomp} % provide euro and other symbols
\else % if luatex or xetex
  \usepackage{unicode-math} % this also loads fontspec
  \defaultfontfeatures{Scale=MatchLowercase}
  \defaultfontfeatures[\rmfamily]{Ligatures=TeX,Scale=1}
\fi
\usepackage{lmodern}
\ifPDFTeX\else
  % xetex/luatex font selection
\fi
% Use upquote if available, for straight quotes in verbatim environments
\IfFileExists{upquote.sty}{\usepackage{upquote}}{}
\IfFileExists{microtype.sty}{% use microtype if available
  \usepackage[]{microtype}
  \UseMicrotypeSet[protrusion]{basicmath} % disable protrusion for tt fonts
}{}
\makeatletter
\@ifundefined{KOMAClassName}{% if non-KOMA class
  \IfFileExists{parskip.sty}{%
    \usepackage{parskip}
  }{% else
    \setlength{\parindent}{0pt}
    \setlength{\parskip}{6pt plus 2pt minus 1pt}}
}{% if KOMA class
  \KOMAoptions{parskip=half}}
\makeatother
\usepackage{xcolor}
\usepackage[margin=1in]{geometry}
\usepackage{graphicx}
\makeatletter
\def\maxwidth{\ifdim\Gin@nat@width>\linewidth\linewidth\else\Gin@nat@width\fi}
\def\maxheight{\ifdim\Gin@nat@height>\textheight\textheight\else\Gin@nat@height\fi}
\makeatother
% Scale images if necessary, so that they will not overflow the page
% margins by default, and it is still possible to overwrite the defaults
% using explicit options in \includegraphics[width, height, ...]{}
\setkeys{Gin}{width=\maxwidth,height=\maxheight,keepaspectratio}
% Set default figure placement to htbp
\makeatletter
\def\fps@figure{htbp}
\makeatother
\setlength{\emergencystretch}{3em} % prevent overfull lines
\providecommand{\tightlist}{%
  \setlength{\itemsep}{0pt}\setlength{\parskip}{0pt}}
\setcounter{secnumdepth}{-\maxdimen} % remove section numbering
\ifLuaTeX
  \usepackage{selnolig}  % disable illegal ligatures
\fi
\usepackage{bookmark}
\IfFileExists{xurl.sty}{\usepackage{xurl}}{} % add URL line breaks if available
\urlstyle{same}
\hypersetup{
  hidelinks,
  pdfcreator={LaTeX via pandoc}}

\author{}
\date{\vspace{-2.5em}}

\begin{document}

\subsection{G}\label{g}

Ich allein sammelte alle ihre Erinnerungen ein, die sie auf den Zetteln
zertreut wenn sie ins Bett ging hinterließ und am Boden fanden sich
manchmal kleine Zeichnungen wie bunte Vögel der Anderen, (Planeten?) Sie
waren ja eingesperrt, morgen ließ sie sie wieder fliegen hats ihnen
versprochen, hielt auch ihre Versprechen. Also kümmere ich mich nicht
weiter um die Vögel und ihr glückliches Dasein. Wenn ich grabe, sitzen
sie eben daneben, immer auf der Suche nach einem Insekt, und stochern in
den Zeiln; deshalb mag ich sie und sie kommen ganz nah an einn ran.\\
Es dauerte als die Sonne schließlich erloschen ist, n.~gerade 8 Minuten
19,6 Sekunden bis es hier dunkel ward. Die Relativität des substanzlosen
Masseträgers, hier: Photonenstrahlen braucht sich also selbst auf dem
Weg zurück ins Hirn - aber dann: hinter dem Doppelspalt auf der
Elektronenröhrenmattscheibe bildet es fast verlustfrei den Zenit des
Gedankens aus, den es zu überschreiten gilt. Ließe man sich ein auf ihn
- es gäbe ihn wirklich nur einmal, an einem Ort, zu einer unbestimmten
aber einzigen Zeit\ldots{} wie: Keimzeit seiner Frische, wie: Endzeit
seiner Jugend aber wie: Zelle eines unabwendbar und für immer
verschlossenen (Todes), der auf Abruf in ihm bereit ist. Der Moment der
sinkenden Sonne ist dann ein Festhalten des Körpers am letzten Warum -
und immer ist es uns in der Zeit voraus 8min19,6sec\ldots{} Jetzt also
der kurze Schauer, da sie eigentlich schon hinter dem Berg verschwunden
ist, und das Licht und die verspürte Wärme nur der Abgesang dessen, was
so gnädig uns verwirrt? So also: daß ja, wenn sie im Jetzt nur n.~6min
tatsächlich ganz verschwunden sein wird also nicht mehr zu sehen - doch
der letzte Rest Wärme vorgestellt wird, aber die Masse an Wärme ja schon
seit 2 Minuten stetig abnimmt. Wie erklärt man das einem Kind\ldots{} am
besten mit seiner Zeichnung:

\end{document}
