% Options for packages loaded elsewhere
\PassOptionsToPackage{unicode}{hyperref}
\PassOptionsToPackage{hyphens}{url}
%
\documentclass[
]{article}
\usepackage{amsmath,amssymb}
\usepackage{iftex}
\ifPDFTeX
  \usepackage[T1]{fontenc}
  \usepackage[utf8]{inputenc}
  \usepackage{textcomp} % provide euro and other symbols
\else % if luatex or xetex
  \usepackage{unicode-math} % this also loads fontspec
  \defaultfontfeatures{Scale=MatchLowercase}
  \defaultfontfeatures[\rmfamily]{Ligatures=TeX,Scale=1}
\fi
\usepackage{lmodern}
\ifPDFTeX\else
  % xetex/luatex font selection
\fi
% Use upquote if available, for straight quotes in verbatim environments
\IfFileExists{upquote.sty}{\usepackage{upquote}}{}
\IfFileExists{microtype.sty}{% use microtype if available
  \usepackage[]{microtype}
  \UseMicrotypeSet[protrusion]{basicmath} % disable protrusion for tt fonts
}{}
\makeatletter
\@ifundefined{KOMAClassName}{% if non-KOMA class
  \IfFileExists{parskip.sty}{%
    \usepackage{parskip}
  }{% else
    \setlength{\parindent}{0pt}
    \setlength{\parskip}{6pt plus 2pt minus 1pt}}
}{% if KOMA class
  \KOMAoptions{parskip=half}}
\makeatother
\usepackage{xcolor}
\usepackage[margin=1in]{geometry}
\usepackage{graphicx}
\makeatletter
\def\maxwidth{\ifdim\Gin@nat@width>\linewidth\linewidth\else\Gin@nat@width\fi}
\def\maxheight{\ifdim\Gin@nat@height>\textheight\textheight\else\Gin@nat@height\fi}
\makeatother
% Scale images if necessary, so that they will not overflow the page
% margins by default, and it is still possible to overwrite the defaults
% using explicit options in \includegraphics[width, height, ...]{}
\setkeys{Gin}{width=\maxwidth,height=\maxheight,keepaspectratio}
% Set default figure placement to htbp
\makeatletter
\def\fps@figure{htbp}
\makeatother
\setlength{\emergencystretch}{3em} % prevent overfull lines
\providecommand{\tightlist}{%
  \setlength{\itemsep}{0pt}\setlength{\parskip}{0pt}}
\setcounter{secnumdepth}{-\maxdimen} % remove section numbering
\ifLuaTeX
  \usepackage{selnolig}  % disable illegal ligatures
\fi
\usepackage{bookmark}
\IfFileExists{xurl.sty}{\usepackage{xurl}}{} % add URL line breaks if available
\urlstyle{same}
\hypersetup{
  hidelinks,
  pdfcreator={LaTeX via pandoc}}

\author{}
\date{\vspace{-2.5em}}

\begin{document}

\subsection{Huit}\label{huit}

8 minutes, 19,6 secondes passent lentement si on attend le fin
d\textquotesingle une heure du travail mais passent tres vite quand on a
besoin des quelques minutes pour finir une exercise dans un examen.
C\textquotesingle est l\textquotesingle example tres simplement pour la
relativité du temp perdre. Nous en pouvons retrouver dans les livres
d\textquotesingle ancien sciences et aussi dans le plus nouvelle
recherche.\\
Ich habe mir das nicht ausgedacht, sondern meine Beobachtungen
(Alpenglühen) an natürlichen Vorgängen gemacht und beschrieben. So ich
am Rande des Sonnenstudiums vom eigentlichen Zweck (die Fl. zu zeichnen)
abgekommen bin und meiner körperlichen Empfindung der abnehmenden
Strahlunghier die Schlußfolgerung nachstelle, daß es sich also bei dem
Ausgesandten um Pakete von Energiebündeln handelt, die verlustlos gleich
einer Pendelreihe den Anstoßimpuls durch den Sender weitergeben. Es muß
sich aber um P.einheiten handeln, weil beim Wegfall des Senders des
jeweilig nächsten Paketes ein Ausfall in der ganzen Reihe eintritt, so
als würde man in der Pendelreihe nach dem letzten Anstoß die von rechts
initiierenden Kugeln nach dem Stoß entfernen. Dann schwänge zwar die
erste Kugel (links) zurück, jedoch hier nur bedingt durch die
Schwerkraftwirkung im Modell. Ausgeführt des Sonnenpendels ginge die
letzte Sendung ins Leere und es folgte ihr keine nach. Und wohin führt
dies uns? Ich meine, auf zwei bewegte Elemente von der Sonne her: einmal
das Licht, das wir eben n.~als freischwingende letzte Kugel ohne Verlust
wahrnehmen können, bis ihr Impuls aus Lichtgeschwindigkeit x des
zurückgelegten Weges verbraucht ist und zweitens eine stehende Welle von
8:19,6 min, deren letztes Maximum eintritt, wenn die Sonne realiter grad
unter den Horizont getaucht ist und die dann langsam abnehmend endet
nach dem Ablauf der halben Periode. Wie in der Musik, auch deren Finale
sich lange vorbereitet. Wie hat er das nur gemacht\ldots{} wie konnte er
nur wissen. Aber er wußte es. Wir sollten hier nicht mehr weiter
nachforschen\ldots{}

\end{document}
