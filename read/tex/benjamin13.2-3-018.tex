% Options for packages loaded elsewhere
\PassOptionsToPackage{unicode}{hyperref}
\PassOptionsToPackage{hyphens}{url}
%
\documentclass[
]{article}
\usepackage{amsmath,amssymb}
\usepackage{iftex}
\ifPDFTeX
  \usepackage[T1]{fontenc}
  \usepackage[utf8]{inputenc}
  \usepackage{textcomp} % provide euro and other symbols
\else % if luatex or xetex
  \usepackage{unicode-math} % this also loads fontspec
  \defaultfontfeatures{Scale=MatchLowercase}
  \defaultfontfeatures[\rmfamily]{Ligatures=TeX,Scale=1}
\fi
\usepackage{lmodern}
\ifPDFTeX\else
  % xetex/luatex font selection
\fi
% Use upquote if available, for straight quotes in verbatim environments
\IfFileExists{upquote.sty}{\usepackage{upquote}}{}
\IfFileExists{microtype.sty}{% use microtype if available
  \usepackage[]{microtype}
  \UseMicrotypeSet[protrusion]{basicmath} % disable protrusion for tt fonts
}{}
\makeatletter
\@ifundefined{KOMAClassName}{% if non-KOMA class
  \IfFileExists{parskip.sty}{%
    \usepackage{parskip}
  }{% else
    \setlength{\parindent}{0pt}
    \setlength{\parskip}{6pt plus 2pt minus 1pt}}
}{% if KOMA class
  \KOMAoptions{parskip=half}}
\makeatother
\usepackage{xcolor}
\usepackage[margin=1in]{geometry}
\usepackage{graphicx}
\makeatletter
\def\maxwidth{\ifdim\Gin@nat@width>\linewidth\linewidth\else\Gin@nat@width\fi}
\def\maxheight{\ifdim\Gin@nat@height>\textheight\textheight\else\Gin@nat@height\fi}
\makeatother
% Scale images if necessary, so that they will not overflow the page
% margins by default, and it is still possible to overwrite the defaults
% using explicit options in \includegraphics[width, height, ...]{}
\setkeys{Gin}{width=\maxwidth,height=\maxheight,keepaspectratio}
% Set default figure placement to htbp
\makeatletter
\def\fps@figure{htbp}
\makeatother
\setlength{\emergencystretch}{3em} % prevent overfull lines
\providecommand{\tightlist}{%
  \setlength{\itemsep}{0pt}\setlength{\parskip}{0pt}}
\setcounter{secnumdepth}{-\maxdimen} % remove section numbering
\ifLuaTeX
  \usepackage{selnolig}  % disable illegal ligatures
\fi
\usepackage{bookmark}
\IfFileExists{xurl.sty}{\usepackage{xurl}}{} % add URL line breaks if available
\urlstyle{same}
\hypersetup{
  hidelinks,
  pdfcreator={LaTeX via pandoc}}

\author{}
\date{\vspace{-2.5em}}

\begin{document}

\subsection{Neuf - Des 3. Beginn}\label{neuf---des-3.-beginn}

liegt schon zwei Nächte hinter uns, unseren Vorgängigen und vor aller
existierenden Handlung. Die hat es nie gegeben und wird es auch in
diesem Band: dem Dritten -- nicht geben. Anfangen ist das einzige
Verdienst gewesen, das ich für mich beanspruchen durfte, tat es also und
fing nochmal an. Mit den letzten Worten der zweiten Einheit: die ist
gezählt, gezählt, gewogen und zerteilt (mene mene tekel upharsin: Die
Konstante wird nach seiner Berechnung heute 5\emph{,}26785 betragen).
Das wird das absolut letzte Wort dieses Buches gewesen sein.\\
Sein Manuskript läßt mich an dieser Stelle innehalten: ich hatte eine
Steigung vermutet vom Niveau des Bandes I in jenes, was sich hier gerade
beendet hat. Sie wirkt aber ausgedrückt nur knapp über 0,026e! hinaus.
Das ist zwar eine Steigung doch von solch geringem Ausmaß, daß es fast
scheint als wäre die Arbeit umsonst gewesen. Vielleicht wählte ich einen
simplen Faktor, daß es verständlicher werden sollte als bisher was die
Theorie aufdeckt. Ich verstehe sie ja und weiß um die Nachrichten, aber
habe ich so etwas vermittelt? Könnt ihr euch erinnern, was anfangs
erzählt wurde? Es hieß: \emph{rot ist, wenn der Schatten länger
geblieben ist als sein} \emph{Licht es vermag ihn zu halten. }Ich habe
versucht von den Schatten etwas zu erhalten, das ich an euch weitergeben
wollte. Darum mußte also der dritte Band folgen, auch weil einige Ideen
wahrgeworden sind. Laßt mich erzä hlen, Genien!\\
Dann fangen wir an, zu übertragen. Es ist eine weitere Maschine nötig,
um das Primärmaterial zu sammeln, das sie auf den Zetteln wenn sie ins
Bett geht hinterläßt. Ich lese mich in ihre Handschriften ein. Ich
studiere ihre Charakter. Ich werde etwas aus ihrem Leben so lernen, als
wenn Sie es mir selbst beigebracht hätten.\\
Und damit habe ich \emph{Ihnen} jetzt gezeigt, wie ich mit den Schriften
fortan verfahren will. Ich stelle sie als den offen zugänglichen
Quelltext zur Verfügung, der allein sich selbst zu erklären hat. Ich
nehme mich in der Interpretation zurück und schenke dir mein und das
Vertrauen von HB in mich, daß er, ich und du gemeinsam als analoges
computing grid diese Arbeit schaffen können: ihm zu seiner Stimme zu
verhelfen, die ihm im Leben nur halb gegeben war. Warum das, was spricht
dafür, den \emph{unbekannten} \emph{polyhistor} einer Öffentlichkeit
zugänglich zu machen, die er sich scheinbar nicht selbst verdient hat zu
zeiten? Ist es wirklich so? Sein Leben war ja bestimmt von der steten
Wendung auch an die Menschen und wirkte da überall fort zu jedem Anlaß
der sich bot, sie mit dem vertraut zu machen, was sie anging: das war
immer die Kunst, war die Philosophie, war Religion und Leben und Tod,
Liebe und Kampf um Liebe und Kunst und Freiheit. Das war alles
zusammengenommen immer n.~nicht genug, was er ihnen tatsächlich
überbrachte -- da war ein Moment, das ich selbst auch nur fassen konnte,
aber weiterzugeben mir hier an dich jede Möglichkeit fehlt. Erst wenn
ich selbst auch ein Lehrender geworden sein werde, kann ich das tun, was
er mit aller Hingabe an die ihn Umgebenden tat: sie für sich selbst
begeistern, wenn sie wie kleine Kinder staunend vor ihren eigenen
Erkenntnissen auf einmal stark waren und nichts mehr eigentlich wollten
als nach Hause gehen und endlich die Bücher alle lesen, die er ihnen
preisgab.\\
Aber dann wirklich; und er kam immer fast zu spät, um n.~atemholen zu
können für den nächsten Abend, der zu bestehen war. So lernte ich ihn
kennen und so ging er: daß selbst der Tod warten mußte, bis die
Schlußworte ausformuliert waren. Wenn ich jetzt den ersten Absatz mit
seinen Worten also beende so tue ich das in der Gewißheit, daß ich mich
später daran erinnern werde wie er mich darauf brachte, jedem Ding seine
angemessene Bedeutung zukommen zu lassen. Ich glaubte lange nur dieser
Satz wäre von mir. Jetzt bin ich weiter und weiß woher er stammt: Es
gibt ein Buch, aus welchem ich ihn unbewußt herausgenommen habe und zu
meinem eigenen umformte. Das war \emph{Das Innerste Tor:} Ich habe
dieses irgendwann beschrieben und wußte nicht, daß er hindurchgegangen
war bevor wir uns kennenlernten. Was aber von ihm in meine Welt
hineinreichte war genug um den Funken weiterzugeben, den H. ihm
abgenommen hatte. So wurde ich davon angeleuchtet und dieses zuerst hier
will \emph{meinen} Reiser8 erzählen von den Sternen der Philosophie zu
den Gründen unserer Existenz.\\
Damit ihr nicht mit leeren Händen in den Hörsaal eintreten müßt gibt es
einen Anhang mit entsprechenden Hilfen. Laßt euch gut beraten und
\emph{wählt weise.}\\
"\ldots{} Und alles: ein einziges Versprechen, das auch nur im Ansatz
einzulösen denen vorbehalten bleiben muß, die nach mir kommen.

\end{document}
