% Options for packages loaded elsewhere
\PassOptionsToPackage{unicode}{hyperref}
\PassOptionsToPackage{hyphens}{url}
%
\documentclass[
]{article}
\usepackage{amsmath,amssymb}
\usepackage{iftex}
\ifPDFTeX
  \usepackage[T1]{fontenc}
  \usepackage[utf8]{inputenc}
  \usepackage{textcomp} % provide euro and other symbols
\else % if luatex or xetex
  \usepackage{unicode-math} % this also loads fontspec
  \defaultfontfeatures{Scale=MatchLowercase}
  \defaultfontfeatures[\rmfamily]{Ligatures=TeX,Scale=1}
\fi
\usepackage{lmodern}
\ifPDFTeX\else
  % xetex/luatex font selection
\fi
% Use upquote if available, for straight quotes in verbatim environments
\IfFileExists{upquote.sty}{\usepackage{upquote}}{}
\IfFileExists{microtype.sty}{% use microtype if available
  \usepackage[]{microtype}
  \UseMicrotypeSet[protrusion]{basicmath} % disable protrusion for tt fonts
}{}
\makeatletter
\@ifundefined{KOMAClassName}{% if non-KOMA class
  \IfFileExists{parskip.sty}{%
    \usepackage{parskip}
  }{% else
    \setlength{\parindent}{0pt}
    \setlength{\parskip}{6pt plus 2pt minus 1pt}}
}{% if KOMA class
  \KOMAoptions{parskip=half}}
\makeatother
\usepackage{xcolor}
\usepackage[margin=1in]{geometry}
\usepackage{graphicx}
\makeatletter
\def\maxwidth{\ifdim\Gin@nat@width>\linewidth\linewidth\else\Gin@nat@width\fi}
\def\maxheight{\ifdim\Gin@nat@height>\textheight\textheight\else\Gin@nat@height\fi}
\makeatother
% Scale images if necessary, so that they will not overflow the page
% margins by default, and it is still possible to overwrite the defaults
% using explicit options in \includegraphics[width, height, ...]{}
\setkeys{Gin}{width=\maxwidth,height=\maxheight,keepaspectratio}
% Set default figure placement to htbp
\makeatletter
\def\fps@figure{htbp}
\makeatother
\setlength{\emergencystretch}{3em} % prevent overfull lines
\providecommand{\tightlist}{%
  \setlength{\itemsep}{0pt}\setlength{\parskip}{0pt}}
\setcounter{secnumdepth}{-\maxdimen} % remove section numbering
\ifLuaTeX
  \usepackage{selnolig}  % disable illegal ligatures
\fi
\usepackage{bookmark}
\IfFileExists{xurl.sty}{\usepackage{xurl}}{} % add URL line breaks if available
\urlstyle{same}
\hypersetup{
  hidelinks,
  pdfcreator={LaTeX via pandoc}}

\author{}
\date{\vspace{-2.5em}}

\begin{document}

\subsection{III. Ton}\label{iii.-ton}

Aber n., ich: denke n.~immer über den Wald nach; ich habe Angst. Es war
ein Märchenwald glaube ich, der selbst durchlebt werden wollte, seine
Geschichten sind nicht anders als mit dem Körper erlernbar. Die Psyche
allein, die immer neue ewige Bettlerin, wenn ich auf den Weggang schaue,
steht sie später an der selben Stelle fest, als hätte ich ihr nicht
längst Tribut gezahlt. Sie kann mir nichts mehr beibringen über die Welt
und den Wald soll ich ertragen, niemand kann ihr das abnehmen. Aber
langsam kenne ich mich aus. Wenn es über Abend geht, haben sich in den
Fußspuren genug Tropfen des Nieselregens gesammelt, daß ich mein Gesicht
waschen kann. Auf dem See: Hier ist jetzt Wind, ein großer, starker
Wind, der über das Wasserdunkel hinzieht und die Birken rauschen, die
dünnen, und die Pappeln, ein paar Kiefern stehen steif und
widerspenstig. Am Ufer schlagen Wellen an die Findlingssteine, das Boot
schaukelt, halb an Land liegend, man glaubt: wie an Worte. Das kann man
hören; was es heißen könnte, hier wird es angeschwemmt. Auch kleine Seen
haben ihr Treibgut. Ich warte also, immer sitzend, immer mit dem
Ausblick zwischen den Birken hindurch, wo das alles herkommt. Es gibt
natürlich diesen Ursprung auf irgendeiner Seite oder die Insel, deren
Bäume ein paar Schatten ins Wasser werfen. Aber was davon übrig bleibt
zu hören, wenn ich das wacklige Ruderboot betrachte, das sind die
quietschenden Riemen und ein glucksender Hohlraum, ganz bestimmte Töne.
Vielleicht habe ich ja etwas davon aufgenommen? Ich kann mich nur an den
Regen erinnern, immer stärker und barfuß auf dem nassen Steg. Und an
Ewa, schlafend. Sie bewegt sich. Ich möchte sie weiter anschauen, doch
dreht sie sich gerade um. Ein kleines Ohr, das in die Nacht hinaussteht
und alles wahrnehmen kann, längst mehr, als ich n.~jemals vorstellen
möchte. Darum bleibe ich ruhig und ich weiß, daß sie mein Herz trotzdem
schlagen hört und, wenn es schneller geht, unruhig wird. Doch das ist
nur der schwarze Tee. Ich will einen Blick auf sie tun, stehe von der
Bank auf und setze mich neben dem Kopfende ihres Bettes auf einen
Klappstuhl. Man blickt herunter, ein wenig Mond schien herein? Nein, es
war die umsonst hell gebliebene Nacht des Sees, mit der er mich dazu
bringen wollte, hier zu bleiben. Doch das ist her und vielleicht wird es
einmal doch Mond gewesen sein, der ins Fenster schien und ein paar
Schatten machte. Ich konnte das kleine Ohr kaum sehen zwischen Kissen
und Decke, wie es lauschte aus dem Schlaf in mein Herz hinein. Aber als
es hörte, wie das schlug und daß es ihm gut erging, war es ruhig und
schlief bis morgens. Dann war ein neuer Tag und die Geschichten dieses
Tages waren die von morgen, weil morgen immer der nächste Tag gewesen
ist.

\end{document}
