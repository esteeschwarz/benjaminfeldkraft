% Options for packages loaded elsewhere
\PassOptionsToPackage{unicode}{hyperref}
\PassOptionsToPackage{hyphens}{url}
%
\documentclass[
]{article}
\usepackage{amsmath,amssymb}
\usepackage{iftex}
\ifPDFTeX
  \usepackage[T1]{fontenc}
  \usepackage[utf8]{inputenc}
  \usepackage{textcomp} % provide euro and other symbols
\else % if luatex or xetex
  \usepackage{unicode-math} % this also loads fontspec
  \defaultfontfeatures{Scale=MatchLowercase}
  \defaultfontfeatures[\rmfamily]{Ligatures=TeX,Scale=1}
\fi
\usepackage{lmodern}
\ifPDFTeX\else
  % xetex/luatex font selection
\fi
% Use upquote if available, for straight quotes in verbatim environments
\IfFileExists{upquote.sty}{\usepackage{upquote}}{}
\IfFileExists{microtype.sty}{% use microtype if available
  \usepackage[]{microtype}
  \UseMicrotypeSet[protrusion]{basicmath} % disable protrusion for tt fonts
}{}
\makeatletter
\@ifundefined{KOMAClassName}{% if non-KOMA class
  \IfFileExists{parskip.sty}{%
    \usepackage{parskip}
  }{% else
    \setlength{\parindent}{0pt}
    \setlength{\parskip}{6pt plus 2pt minus 1pt}}
}{% if KOMA class
  \KOMAoptions{parskip=half}}
\makeatother
\usepackage{xcolor}
\usepackage[margin=1in]{geometry}
\usepackage{graphicx}
\makeatletter
\def\maxwidth{\ifdim\Gin@nat@width>\linewidth\linewidth\else\Gin@nat@width\fi}
\def\maxheight{\ifdim\Gin@nat@height>\textheight\textheight\else\Gin@nat@height\fi}
\makeatother
% Scale images if necessary, so that they will not overflow the page
% margins by default, and it is still possible to overwrite the defaults
% using explicit options in \includegraphics[width, height, ...]{}
\setkeys{Gin}{width=\maxwidth,height=\maxheight,keepaspectratio}
% Set default figure placement to htbp
\makeatletter
\def\fps@figure{htbp}
\makeatother
\setlength{\emergencystretch}{3em} % prevent overfull lines
\providecommand{\tightlist}{%
  \setlength{\itemsep}{0pt}\setlength{\parskip}{0pt}}
\setcounter{secnumdepth}{-\maxdimen} % remove section numbering
\ifLuaTeX
  \usepackage{selnolig}  % disable illegal ligatures
\fi
\usepackage{bookmark}
\IfFileExists{xurl.sty}{\usepackage{xurl}}{} % add URL line breaks if available
\urlstyle{same}
\hypersetup{
  hidelinks,
  pdfcreator={LaTeX via pandoc}}

\author{}
\date{\vspace{-2.5em}}

\begin{document}

\subsection{GUHL fr.th.}\label{guhl-fr.th.}

\begin{enumerate}
\def\labelenumi{\alph{enumi}.}
\setcounter{enumi}{2}
\tightlist
\item
  \emph{Funken} aber, ein Regen von ineinanderstrebenden Gedanken ihrer
  und meiner, die wir sich befruchten lassen. Geben Sie sich selbst frei
  -- vielleicht gelingt eine Wende n.~bevor wir unser gemeinsames Ende
  erreichen. Es hieße, sich allen Bedingungen fügen, die nur das Werk
  befördern. Ich frage Sie einmal: was haben wir kennengelernt? Nur des
  anderen Schriften? Das tut er auch, seiner selbst Leser, der uns
  folgt. Aber was lernen wir? Sie überhaupt n.~etwas wo Sie jetzt sich
  auch immer aufhalten? Ich vermute einen Ort, an dem nicht viel
  gesprochen wird etwa nicht viel gesprochen werden kann, weil die
  Stimmen in Watte gehüllt erscheinen und sie deshalb alle Worte
  vermeiden, die an ihnen selbst so kleben blieben. Also war etwas neu?
  Ja. Ich schrieb den Anfang, nur den Anfang. Aber Sie drängten heraus
  plötzlich mit der Sprache als wäre ich lange ein erwarteter Katalyt
  und Ihnen jetzt Anlaß endlich sich gehört zu fühlen. Und mein
  Versprechen? -- war ja mehr mir selbst gegeben, daß \emph{mein} Werk
  beendet wird. Wenn ich also jetzt Ihres dazu nehme, es wie den
  überreifen Fruchtkörper einer unbekannten Pflanze zur Gärung zu
  bringen und darauf vertrauend daß es mich nicht blind macht- mit dem
  Destillat meine lebendigen Schriften ansetze: dann soll jenes nur der
  Träger sein? So sieht es aus und so habe ich es mir gedacht. Sie
  müssen auch mir vertrauen, daß es damit für \emph{Ihr} Werk ein Gutes
  hat: Sie werden gelesen werden ohne die Beschränkung meiner
  interpretatorischen Versuche. Alles, was dahin von mir verstanden
  wird, läßt sich in diesen Randnotizen zusammenhalten und ist Lesehilfe
  zum Primärtext. Vielleicht kommen wir so zusammen. Ich jedenfalls
  hatte an der Idee gefallen gefunden und so haben wir den ersten
  Abschnitt bestanden. Es war Zeit für weiteres gekommen.
\end{enumerate}

\end{document}
