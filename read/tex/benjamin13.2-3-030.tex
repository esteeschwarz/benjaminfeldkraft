% Options for packages loaded elsewhere
\PassOptionsToPackage{unicode}{hyperref}
\PassOptionsToPackage{hyphens}{url}
%
\documentclass[
]{article}
\usepackage{amsmath,amssymb}
\usepackage{iftex}
\ifPDFTeX
  \usepackage[T1]{fontenc}
  \usepackage[utf8]{inputenc}
  \usepackage{textcomp} % provide euro and other symbols
\else % if luatex or xetex
  \usepackage{unicode-math} % this also loads fontspec
  \defaultfontfeatures{Scale=MatchLowercase}
  \defaultfontfeatures[\rmfamily]{Ligatures=TeX,Scale=1}
\fi
\usepackage{lmodern}
\ifPDFTeX\else
  % xetex/luatex font selection
\fi
% Use upquote if available, for straight quotes in verbatim environments
\IfFileExists{upquote.sty}{\usepackage{upquote}}{}
\IfFileExists{microtype.sty}{% use microtype if available
  \usepackage[]{microtype}
  \UseMicrotypeSet[protrusion]{basicmath} % disable protrusion for tt fonts
}{}
\makeatletter
\@ifundefined{KOMAClassName}{% if non-KOMA class
  \IfFileExists{parskip.sty}{%
    \usepackage{parskip}
  }{% else
    \setlength{\parindent}{0pt}
    \setlength{\parskip}{6pt plus 2pt minus 1pt}}
}{% if KOMA class
  \KOMAoptions{parskip=half}}
\makeatother
\usepackage{xcolor}
\usepackage[margin=1in]{geometry}
\usepackage{graphicx}
\makeatletter
\def\maxwidth{\ifdim\Gin@nat@width>\linewidth\linewidth\else\Gin@nat@width\fi}
\def\maxheight{\ifdim\Gin@nat@height>\textheight\textheight\else\Gin@nat@height\fi}
\makeatother
% Scale images if necessary, so that they will not overflow the page
% margins by default, and it is still possible to overwrite the defaults
% using explicit options in \includegraphics[width, height, ...]{}
\setkeys{Gin}{width=\maxwidth,height=\maxheight,keepaspectratio}
% Set default figure placement to htbp
\makeatletter
\def\fps@figure{htbp}
\makeatother
\setlength{\emergencystretch}{3em} % prevent overfull lines
\providecommand{\tightlist}{%
  \setlength{\itemsep}{0pt}\setlength{\parskip}{0pt}}
\setcounter{secnumdepth}{-\maxdimen} % remove section numbering
\ifLuaTeX
  \usepackage{selnolig}  % disable illegal ligatures
\fi
\usepackage{bookmark}
\IfFileExists{xurl.sty}{\usepackage{xurl}}{} % add URL line breaks if available
\urlstyle{same}
\hypersetup{
  hidelinks,
  pdfcreator={LaTeX via pandoc}}

\author{}
\date{\vspace{-2.5em}}

\begin{document}

\subsection{und als Geschichte meiner verlorenen
Geschichte}\label{und-als-geschichte-meiner-verlorenen-geschichte}

\begin{enumerate}
\def\labelenumi{\alph{enumi}.}
\setcounter{enumi}{1}
\tightlist
\item
  HB sprach einmal von der Wissenschaft und meiner Aufgabe dazu sie
  irgendwann zu benennen. Ich bin n.~nicht zu einem Namen gereift, der
  dafür in Frage käme; es gibt aber diese Sicherheit, daß ich ihn in mir
  aufbewahre. Man müßte die Fragen alle entschlüsseln können, die sie
  einem fortwährend stellt; nicht, daß man antworten soll, nein, nur
  ersteinmal die Worte finden, Formeln jener geisterhaften Erinnerung an
  das Archetypenmaterial aller Fragen. Woraus bestünden sie denn
  sonst\ldots{} Bevor unseren Worten was soll es da gegeben haben, sich
  des Zweifels zu erwehren und schließlich: wo soll der Zweifel selbst
  hergekommen sein wenn nicht aus einem allerersten Wort, einem
  beständig wiederholten Urlaut ohne anfänglichen Sinn, aber er
  \emph{erlangte} sich ihn. Und mit dem Bezug einher ging sofort der
  Verlust des Bezuges und damit war ja schon der Zweifel in der Welt -
  und schon, aus der tatsächlichen Möglichkeit einer einfachen
  Bestellung der Sache durch das Archewort: die Doppelsicht auf
  \emph{jeden} Wortanfang. Daß jetzt alle Anfänge längst außerhalb
  unserer bewußten Reichweite liegen macht meine Bemühungen sie zu
  erhalten nur n.~wichtiger. Die Mittel dazu habe ich gezeigt wie ich
  sie mir vorstelle. Eines würde der Stein im Uurainen sein, der dort
  liegengeblieben ist für wie lange man eine Ewigkeit annehmen mag. Das
  andere befindet sich hier, unter meinen Fingern und ich nenne es
  "Standardschriftprozessor" zur Erfassung aller erkannten Funkenmuster,
  die vom selbsttätig arbeitenden Geist ausgesandt werden. Seine
  Funktionen sind auf wesentliche Merkmale beschränkt. Es geht nicht
  über Schriftzusammenstellung, Ausdruckmöglichkeit und den Export in
  zeitlich kohärente Raster hinaus. Jedoch genügt das zur Aufarbeitung
  des allein verfügbaren Manuskriptmaterials. HB hätte auch nicht anders
  gearbeitet und ich bin nah an seinen Quellen, viel näher als ich es
  mit anderen Mitteln hätte sein können. Dann also endlich gibt es den
  Widerstand des Textes und solches ist viel wert. Da ist Leben in der
  Bude\ldots{}
\end{enumerate}

\end{document}
