% Options for packages loaded elsewhere
\PassOptionsToPackage{unicode}{hyperref}
\PassOptionsToPackage{hyphens}{url}
%
\documentclass[
]{article}
\usepackage{amsmath,amssymb}
\usepackage{iftex}
\ifPDFTeX
  \usepackage[T1]{fontenc}
  \usepackage[utf8]{inputenc}
  \usepackage{textcomp} % provide euro and other symbols
\else % if luatex or xetex
  \usepackage{unicode-math} % this also loads fontspec
  \defaultfontfeatures{Scale=MatchLowercase}
  \defaultfontfeatures[\rmfamily]{Ligatures=TeX,Scale=1}
\fi
\usepackage{lmodern}
\ifPDFTeX\else
  % xetex/luatex font selection
\fi
% Use upquote if available, for straight quotes in verbatim environments
\IfFileExists{upquote.sty}{\usepackage{upquote}}{}
\IfFileExists{microtype.sty}{% use microtype if available
  \usepackage[]{microtype}
  \UseMicrotypeSet[protrusion]{basicmath} % disable protrusion for tt fonts
}{}
\makeatletter
\@ifundefined{KOMAClassName}{% if non-KOMA class
  \IfFileExists{parskip.sty}{%
    \usepackage{parskip}
  }{% else
    \setlength{\parindent}{0pt}
    \setlength{\parskip}{6pt plus 2pt minus 1pt}}
}{% if KOMA class
  \KOMAoptions{parskip=half}}
\makeatother
\usepackage{xcolor}
\usepackage[margin=1in]{geometry}
\usepackage{graphicx}
\makeatletter
\def\maxwidth{\ifdim\Gin@nat@width>\linewidth\linewidth\else\Gin@nat@width\fi}
\def\maxheight{\ifdim\Gin@nat@height>\textheight\textheight\else\Gin@nat@height\fi}
\makeatother
% Scale images if necessary, so that they will not overflow the page
% margins by default, and it is still possible to overwrite the defaults
% using explicit options in \includegraphics[width, height, ...]{}
\setkeys{Gin}{width=\maxwidth,height=\maxheight,keepaspectratio}
% Set default figure placement to htbp
\makeatletter
\def\fps@figure{htbp}
\makeatother
\setlength{\emergencystretch}{3em} % prevent overfull lines
\providecommand{\tightlist}{%
  \setlength{\itemsep}{0pt}\setlength{\parskip}{0pt}}
\setcounter{secnumdepth}{-\maxdimen} % remove section numbering
\ifLuaTeX
  \usepackage{selnolig}  % disable illegal ligatures
\fi
\usepackage{bookmark}
\IfFileExists{xurl.sty}{\usepackage{xurl}}{} % add URL line breaks if available
\urlstyle{same}
\hypersetup{
  hidelinks,
  pdfcreator={LaTeX via pandoc}}

\author{}
\date{\vspace{-2.5em}}

\begin{document}

\subsection{CHOR}\label{chor}

Auch Erfahrungen die wir uns scheuen zu machen lassen uns reifen,
spätestens wenn man sie tatsächlich begeht; nämlich dann, wenn wir über
die Gründe nachdenken warum wir uns ihnen verweigerten. So kann es mit
der Kunst sein. Einem Werk gegenüber aufgeschlossen sein heißt nicht, es
mögen zu müssen. Wir können aber kein Werk nicht mögen, das wir nicht
kennen. Und kennen heißt nie, es studiert zu haben, kennen heißt auch,
seinen Schatten wahrgenommen zu haben den das Licht anderer Werke von
ihm wirft. Vielleicht werden wir sogar so lange etwas nicht wirklich
hassen können, bis wir es ganz verstanden haben und die Tiefe des
Gefühls dadurch begründen. Es ist beängstigend sich vorzustellen man
müßte die schlechten Gedanken der Menschen zuerst aufnehmen, um diesen
dann gleichgültig gegenübertreten zu dürfen; ataraktisch, aber auch
gleichgültig im Sinne einer sie unbeachtet lassenden Teilnahmslosigkeit.
Doch es wird wohl so sein und vorher wird man immer auch Regungen
zeigen, die sich nicht vermeiden lassen. Solange versteht ihr niemanden
ganz.

\end{document}
