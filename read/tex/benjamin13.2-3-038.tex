% Options for packages loaded elsewhere
\PassOptionsToPackage{unicode}{hyperref}
\PassOptionsToPackage{hyphens}{url}
%
\documentclass[
]{article}
\usepackage{amsmath,amssymb}
\usepackage{iftex}
\ifPDFTeX
  \usepackage[T1]{fontenc}
  \usepackage[utf8]{inputenc}
  \usepackage{textcomp} % provide euro and other symbols
\else % if luatex or xetex
  \usepackage{unicode-math} % this also loads fontspec
  \defaultfontfeatures{Scale=MatchLowercase}
  \defaultfontfeatures[\rmfamily]{Ligatures=TeX,Scale=1}
\fi
\usepackage{lmodern}
\ifPDFTeX\else
  % xetex/luatex font selection
\fi
% Use upquote if available, for straight quotes in verbatim environments
\IfFileExists{upquote.sty}{\usepackage{upquote}}{}
\IfFileExists{microtype.sty}{% use microtype if available
  \usepackage[]{microtype}
  \UseMicrotypeSet[protrusion]{basicmath} % disable protrusion for tt fonts
}{}
\makeatletter
\@ifundefined{KOMAClassName}{% if non-KOMA class
  \IfFileExists{parskip.sty}{%
    \usepackage{parskip}
  }{% else
    \setlength{\parindent}{0pt}
    \setlength{\parskip}{6pt plus 2pt minus 1pt}}
}{% if KOMA class
  \KOMAoptions{parskip=half}}
\makeatother
\usepackage{xcolor}
\usepackage[margin=1in]{geometry}
\usepackage{graphicx}
\makeatletter
\def\maxwidth{\ifdim\Gin@nat@width>\linewidth\linewidth\else\Gin@nat@width\fi}
\def\maxheight{\ifdim\Gin@nat@height>\textheight\textheight\else\Gin@nat@height\fi}
\makeatother
% Scale images if necessary, so that they will not overflow the page
% margins by default, and it is still possible to overwrite the defaults
% using explicit options in \includegraphics[width, height, ...]{}
\setkeys{Gin}{width=\maxwidth,height=\maxheight,keepaspectratio}
% Set default figure placement to htbp
\makeatletter
\def\fps@figure{htbp}
\makeatother
\setlength{\emergencystretch}{3em} % prevent overfull lines
\providecommand{\tightlist}{%
  \setlength{\itemsep}{0pt}\setlength{\parskip}{0pt}}
\setcounter{secnumdepth}{-\maxdimen} % remove section numbering
\ifLuaTeX
  \usepackage{selnolig}  % disable illegal ligatures
\fi
\usepackage{bookmark}
\IfFileExists{xurl.sty}{\usepackage{xurl}}{} % add URL line breaks if available
\urlstyle{same}
\hypersetup{
  hidelinks,
  pdfcreator={LaTeX via pandoc}}

\author{}
\date{\vspace{-2.5em}}

\begin{document}

\subsection{f:}\label{f}

Fast wäre ich an diesem Vorabend des Feiertags in meinen Gedanken über
Mond und Gedächtnis hängengeblieben. Allein daß ich aber weiterdenken
konnte (weil es sich einsam so ergeben hat) und weiterschreiben wollte
(weil mir Ihre Gedanken dazukamen) bewirkte schon einen Spannungsanstieg
im Textmuster. Bis hierher folgen seine Augen. Und dann Abbruch,
abbruch, abbruch des Eingewohnten jetzt abbruch der netzaugen abbruch
eines gewöhnlichen todeskandidaten, der n.~sein recht zu leben nicht
erlangt hatte, abbruch eines und aufbruch ungewohnten erlebens. Abbruch.
Schwöre ab, schwör ab, sonst bist du hier nicht richtig, heute.\\
Mitmurmeln Mt.6,9; Aber was läßt mich so sicher sein, daß das andere das
richtige Gebet ist? - weil der Gedanke frei dahinfindet, wenn ich ihn
gehen lasse? Helmut hat vielleicht keine Zeit mehr gehabt, sich dazu
eines auszusuchen als der T. drängte. Ich: vor allen Prüfungen, vor
allen Erschwernissen, ohne krank zu sein stelle mich diesem und jedesmal
intuitiv weiß ich, daß es das echte ist, sein wird, wo immer es hingeht.
Die Schultern sind gerade durchgedrückt, im ganzen Körper liegt
Spannung. Als müßt ich gleich abspringen liegen in den n.~immer
gefalteten Händen meine beiden Gesichter. Das eine, G.t hast du
vielleicht geschaffen. Das andere aber: ist mein Eigentum und ich werde
es nicht n.~einmal verleugnen. Irgendjemand, sicher, ist damals für
menschensünden gestorben und schrecklich war sein Tod für uns heute die
solches nicht mehr sehen können (westlich). Aber der alte, den er nicht
genannt hat oder man hat es nicht richtig vernommen war eigentlich viel
eher gemeint als er den Elia rief wie manche dachten. Vielleicht ist das
prophetische übertrieben worden über die Jahrhunderte so daß nicht mehr
übrig davon geblieben ist. Aber n.~im selben alten Buch sprechen sie
immerw. sorgenvoll von den kommenden Zeiten, die das Volk durchmachen
müßte, weil wir uns vom rechten Glauben entfernen tun. Und jetzt,
Helmut, sagen Sie mir, daß das nicht wichtig sei, nur, daß man Religion
habe, weil man in der Kunst lebt \emph{und wer keine Kunst hat, der habe
welche.}\\
Da fand ich mich doch längst zurecht mit meinen strebenden Figuren, der
Kunst, in die ich Sie mitaufgenommen habe als kanonische Quelle, also
non \emph{sola gratia}. Was zählt, sind die guten Eigenschaften. Alles
andere Stückwerk und wir können aufhören nach Vollendung zu streben;
wenn es sich will, wird es das tun ohne uns. Schrift erlaubt keine
Fehler, Geschrieben ist über den Zweifel erhaben. Deshalb sehen wir das
Ende nicht, auch wennes längst feststeht. Aber ich nähere mich und
hinterlasse hier Zeugnis des Weges für euch zu verfahren, so wie ich mit
den Zeugnissen meines Vorgängigen HB: deuten, danken und weiterdenken.
In der Wiederholung liegt ein rhythmisches Geheimnis und verborgen heißt
nicht, daß es nicht sich uns zeigen möchte. Es verliert nur mit jedem
Aufscheinen in den Gedanken an Intensität, welche endlich ist. Also
schont es sich und wir sind ungewiß darüber, ob der Weg richtig ist. An
jedem Kreuzweg aber werde \emph{ich} mich für euch entscheiden. Das
verspreche ich und die Worte sind gezählt, gezählt, gewogen und
zerteilt: zum Ende haben wir erneut eine Konstante, die das Buch
hervorbrachte. Aber bis dort sind es n.~Zeiten, und unwiderruflich, mir
zu folgen. Vielleicht seid ihr einmal unsicher gewesen, doch wenn es so
war, haben wir schon einen neuen Bund geschlossen, der eure fernen
Gelübde der Heimat heimholt und mich darüber einzusetzen in der
Weltengeduld von Schöpferschöpfern ausharren läßt bis ihr mich findet.
Und in den Köpfen lese ich: daß ihr gar nicht so weit entfernt seid wie
ihr mich glauben machen wollt. Eine kleine Dunkelheit n.~aushalten.

\end{document}
