% Options for packages loaded elsewhere
\PassOptionsToPackage{unicode}{hyperref}
\PassOptionsToPackage{hyphens}{url}
%
\documentclass[
]{article}
\usepackage{amsmath,amssymb}
\usepackage{iftex}
\ifPDFTeX
  \usepackage[T1]{fontenc}
  \usepackage[utf8]{inputenc}
  \usepackage{textcomp} % provide euro and other symbols
\else % if luatex or xetex
  \usepackage{unicode-math} % this also loads fontspec
  \defaultfontfeatures{Scale=MatchLowercase}
  \defaultfontfeatures[\rmfamily]{Ligatures=TeX,Scale=1}
\fi
\usepackage{lmodern}
\ifPDFTeX\else
  % xetex/luatex font selection
\fi
% Use upquote if available, for straight quotes in verbatim environments
\IfFileExists{upquote.sty}{\usepackage{upquote}}{}
\IfFileExists{microtype.sty}{% use microtype if available
  \usepackage[]{microtype}
  \UseMicrotypeSet[protrusion]{basicmath} % disable protrusion for tt fonts
}{}
\makeatletter
\@ifundefined{KOMAClassName}{% if non-KOMA class
  \IfFileExists{parskip.sty}{%
    \usepackage{parskip}
  }{% else
    \setlength{\parindent}{0pt}
    \setlength{\parskip}{6pt plus 2pt minus 1pt}}
}{% if KOMA class
  \KOMAoptions{parskip=half}}
\makeatother
\usepackage{xcolor}
\usepackage[margin=1in]{geometry}
\usepackage{graphicx}
\makeatletter
\def\maxwidth{\ifdim\Gin@nat@width>\linewidth\linewidth\else\Gin@nat@width\fi}
\def\maxheight{\ifdim\Gin@nat@height>\textheight\textheight\else\Gin@nat@height\fi}
\makeatother
% Scale images if necessary, so that they will not overflow the page
% margins by default, and it is still possible to overwrite the defaults
% using explicit options in \includegraphics[width, height, ...]{}
\setkeys{Gin}{width=\maxwidth,height=\maxheight,keepaspectratio}
% Set default figure placement to htbp
\makeatletter
\def\fps@figure{htbp}
\makeatother
\setlength{\emergencystretch}{3em} % prevent overfull lines
\providecommand{\tightlist}{%
  \setlength{\itemsep}{0pt}\setlength{\parskip}{0pt}}
\setcounter{secnumdepth}{-\maxdimen} % remove section numbering
\ifLuaTeX
  \usepackage{selnolig}  % disable illegal ligatures
\fi
\usepackage{bookmark}
\IfFileExists{xurl.sty}{\usepackage{xurl}}{} % add URL line breaks if available
\urlstyle{same}
\hypersetup{
  hidelinks,
  pdfcreator={LaTeX via pandoc}}

\author{}
\date{\vspace{-2.5em}}

\begin{document}

\subsection{guhl}\label{guhl}

Ich aber mußte ihm antworten, wie ich es gelernt hatte. Und die Stimme
allein dieses Windhauches war mir nicht mehr so fremd wie damals als ich
sie zum ersten mal vernommen hatte. Und ja, sicher knarrte es in den
Balken und ja, auch die Bäume rauschten herüber. Aber die See war anders
und nachts kamen mir die Töne der Welt nicht so leicht vor wie ich den
ganzen Tag dachte. Ich hatte mich in die Farben begeben, so wie es
vorgesehen war. Das große Denken wartete unten vor den Häusern und
selbst wenn ich nicht immer verstand warum es mich dorthin zog, machte
ich doch jeden Tag die Türen auf und trat hinaus, auch wenn das hieß
sich einmal am Tag die Schuhe anziehen zu müssen und den Hut aufsetzen.
Das war ich ihr schuldig, der Welt. Und einen Punkt hatten wir jetzt
geklärt: warum wir hier waren wenn wir hier waren. Es gab zwar den
Hintergrund n.~nicht auf welchem sich die Geschichte zutragen würde, es
gab auch für die Handelnden n.~keinen Anlaß, sich zu zeigen; was es aber
gab waren 1.: mein unheimliches Wissen der steten Zeugung, 2.: daß ich
den Stoff angefaßt hatte, von dem man sich nicht trennen kann und der
3.: sich zu einer Handlung verdichten ließe, sobald eine Person anfinge
zu sprechen. Der jetzt spricht ist ja nur das verzeichnende Ich, das
Autorenich, mit dem ich mich für die Dauer der Bücher bekleiden muß und
das ich ablege wenn ich den Deckel der Schreibmaschine zuklappe. Was
dann mit ihm geschieht kann ich nur ungefähr nachdenken, indem ich
meinen eigenen Tagesablauf versuche mit dem seinen zu verschalten.
Jedoch werde ich Gleichklang wohl nie erreichen. Selten gelingt es aber
in einem Gedanken ihn zu entdecken, oder auch für die Dauer eines
Flusses mit ihm zu verschwimmen bis man uns trennt. Denn die anderen
sehen immer zwei in uns: einer der lebt, einer der schreibt. Aber wie
oft ging uns das ineins; daß ich, der Schreibende, ihm, dem so fern
Lebenden das aussprach was er in seinen Denkpausen erfand.\\
Wir sind ja zu mancher Zeit auch wirklich zusammen in einer Figur --
doch diese Ebene ist so weit, daß sie n.~nicht in einem Buch stattfinden
kann, nicht in solchem das mit euren Buchstaben geschrieben werden muß.
Ich fing ja an zu bauen bereits und habe die Kathedrale immer n.~im
Auge, die euch vorschwebte Meister. Ihr habt nur von einem Burgfried
gesprochen und nach außen eine kleine Kapelle gemeint. Davon ist nie
etwas verwirklicht worden. Auf dem Boden des Schloßhofes hat man
nachgebildet im Plattenmuster den Grundriß jener \emph{kleinen Kapelle.}
Und was soll ich sagen? Natürlich würde es eine Kathedrale werden, ganz
sicher gehen irgendwann auch Menschen darin ein und aus. Ich bin ja nur
einer Ihrer Baurn, n.~ein sehr junger zudem\ldots{} \emph{etwa in ihrem
Alter mein König\ldots,} aber die feuchte Bergluft tut auch meinen
Lungen gut und für die Tage, die (der andere in seinem Herbst) ich hier,
weit weg von jeder Höhe schreibend in ihrem Märchen verbringe, lege ich
meine ganze Kraft in den Dienst an Ihrer Verwirklichung bis ich
leergeschrieben habe diese Gegend. N. gibt Ihre Geschichte so viel zum
eigenen dazu, daß ich \emph{danken für alles lern und verstehe die
Freiheit aufzubrechen wohin }ich will, Hölderlin, vielleicht auch über
meine unbestimmte Sehnsucht diesem Tale hier angesichts vor dessen
Verlassen es mich schon graut, weil ich weiß wie sehr es mich herziehen
wird nach dem Berg, zum Schloß hinauf, mit dem griechischen Tee. Wir
erfahren: es gibt den Augenblick. Und warum Hölderlin darin auftauchen
muß, erkläre, indem ich die anderen beteiligten Funken einlasse: Richard
Wagner und Nietzsche, Schopenhauer und die Geburt der Tragödie
schließlich die Demut der Schwabensöhne und Stuttgarts Krone, den Gang
aufs Land und Benn, der ihm fortwährend Denkmäler setzte\ldots{} Wenn
wir das alles verstanden haben, dürfen wir uns auch zum Ursprung und HB
das Feld übergeben so weit, wie wir uns fortzubewegen wagen, er wird uns
zurückholen und seine Lektionen um die fehlenden Kenntnisse erweitern,
die wir hier und jetzt - am Fuße von Tegelberg und Säuling\ldots{} - zu
verzeichnen Seine gnädige Erlaubnis vorraussetzten, des toten Königs, es
lebe der König!

\end{document}
