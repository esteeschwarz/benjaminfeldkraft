% Options for packages loaded elsewhere
\PassOptionsToPackage{unicode}{hyperref}
\PassOptionsToPackage{hyphens}{url}
%
\documentclass[
]{article}
\usepackage{amsmath,amssymb}
\usepackage{iftex}
\ifPDFTeX
  \usepackage[T1]{fontenc}
  \usepackage[utf8]{inputenc}
  \usepackage{textcomp} % provide euro and other symbols
\else % if luatex or xetex
  \usepackage{unicode-math} % this also loads fontspec
  \defaultfontfeatures{Scale=MatchLowercase}
  \defaultfontfeatures[\rmfamily]{Ligatures=TeX,Scale=1}
\fi
\usepackage{lmodern}
\ifPDFTeX\else
  % xetex/luatex font selection
\fi
% Use upquote if available, for straight quotes in verbatim environments
\IfFileExists{upquote.sty}{\usepackage{upquote}}{}
\IfFileExists{microtype.sty}{% use microtype if available
  \usepackage[]{microtype}
  \UseMicrotypeSet[protrusion]{basicmath} % disable protrusion for tt fonts
}{}
\makeatletter
\@ifundefined{KOMAClassName}{% if non-KOMA class
  \IfFileExists{parskip.sty}{%
    \usepackage{parskip}
  }{% else
    \setlength{\parindent}{0pt}
    \setlength{\parskip}{6pt plus 2pt minus 1pt}}
}{% if KOMA class
  \KOMAoptions{parskip=half}}
\makeatother
\usepackage{xcolor}
\usepackage[margin=1in]{geometry}
\usepackage{graphicx}
\makeatletter
\def\maxwidth{\ifdim\Gin@nat@width>\linewidth\linewidth\else\Gin@nat@width\fi}
\def\maxheight{\ifdim\Gin@nat@height>\textheight\textheight\else\Gin@nat@height\fi}
\makeatother
% Scale images if necessary, so that they will not overflow the page
% margins by default, and it is still possible to overwrite the defaults
% using explicit options in \includegraphics[width, height, ...]{}
\setkeys{Gin}{width=\maxwidth,height=\maxheight,keepaspectratio}
% Set default figure placement to htbp
\makeatletter
\def\fps@figure{htbp}
\makeatother
\setlength{\emergencystretch}{3em} % prevent overfull lines
\providecommand{\tightlist}{%
  \setlength{\itemsep}{0pt}\setlength{\parskip}{0pt}}
\setcounter{secnumdepth}{-\maxdimen} % remove section numbering
\ifLuaTeX
  \usepackage{selnolig}  % disable illegal ligatures
\fi
\usepackage{bookmark}
\IfFileExists{xurl.sty}{\usepackage{xurl}}{} % add URL line breaks if available
\urlstyle{same}
\hypersetup{
  hidelinks,
  pdfcreator={LaTeX via pandoc}}

\author{}
\date{\vspace{-2.5em}}

\begin{document}

\subsection{VI. Tode}\label{vi.-tode}

\begin{enumerate}
\def\labelenumi{\arabic{enumi}.}
\tightlist
\item
  Benn meinte: spinnen die Nebel sich ein, und: wie vom Berg im Wind
  schluchzt ein Schatten wie vom Berg imWind. Vielleicht hat er einiges
  gesehen das widerfährt. N. nacht und der Berg bis auf die Silhouette
  gegen den Sternenhimmel. Davor wirklich Geisterschwaden und zum Schloß
  verteilen sie .\\
\item
  Aber da die Weideklä nge, Glocken selbst die schwarze Nacht lang
  irgendwo am Hang. obwohl eine Erinnerung erst jetzt bestehtgeht mir
  ein, wie den die von hier stammen weit weg sind und wenn sie ihn nach
  Jahren erneut vernehmen von diesem Klang eine sehr tiefe Sehnsucht
  endlich befriedigt werden muß. Wie erste die Nacht in der Heimat und
  hört man davon. Vielleicht nie einen passenderen Vergleich finden. Ich
  habe für H. nichts, oder? Kohlegeruch im Winter, der durch die
  Stadt\ldots{} und da wie der berliner Benn sagt, sich wehmütig
  vergangener Armuten erinnert. Wenn wir zurü ck ist Zeit zu heizen und
  die Luft einem dann jenes arme b. Gefühl von zuhause und warm von und
  die Zeit für die man a. geht. Aber die Weideklänge, wiegt die auf? Ich
  habe n.~nicht , weil ich die Stadt nie verließ. Vielleicht kommt es ja
  einmal dazu und werde wissen.\\
  3 . Was haben wir heute gelernt: Dann war es Zeit, daß wir ihn wecken
  gehen. Ihn, das sei jetzt: Der Märchenkönig\\
  Im Schloß brennt Licht da oben, sag ich dir,\\
  wie sehr sind wir euch nah, die n.~den Ansturm fühlen\\
  Warum gehen sie denn nicht und sind sich ehrlich\ldots{}\\
  aber in den Hüften der Schwung, der n.~den Schutzmantel\\
  der Madonnen wölbte: wie man sich gut aufgehoben fühlte,\\
  Und ich vor dem Berg dem Schloß und Himmeln unheimlicher Sterne voll
  und eine Straß e rauscht Laternen, eine Gastwirtschaft wo Teller
  klappern durch offene Kü chen , Herr und Hund, selten ein Auto. ich
  auf der Seite wo ich jetzt in drei Balkons zum Ort, zum Berg, zum
  Schloß hin ö ffnet. Der Tee der schwarzen ist der, Was sehe ich- und
  bin so voll davon: nur das Schloß aus dem Dunkel hell auf mögen zwei
  Kilometer die mich von ihm in der Luft; aber was sollen zwei Kilometer
  gegen die nicht zu leugnende Nähe, mit der es sich anträgt?\\
  Wenn im Harz die , Faust und Übernatur , die ich spürte so sind hier
  ganz andere Gewalten, ursprünglichere und vormenschliche tätig, Wind
  und der Luft, dem Polarstern und sehr deutlich von der Lage der Berge
  zueinander bestimmte Schwingungszustände der Elementarkräfte. Ein
  wenig abseits in einer Wiese steht St.~Coloman aber, eine
  Wallfahrtskirche.\\
  Vielleicht n.~einmal regnen. Vielleicht also dann wird es regnen. Aber
  vielleicht regnet es n.. Ich jedenfalls hätte daran eine große Freude.
  Zu ist der Sternen und die klare Nachtnebelnacht schon erlebt, jetzt
  gibt es sie nicht.\\
  Und dann hat es gereg und den Laternen, wie Erinnerungen an ande
  fremde Straßen. Nie nehme ich das so stark wahr wie wenn ich unterwegs
  bin irgen Natürlich sind auch die berliner Straßen dann naß, aber es
  sind ja immer Straß en die ich kenne; hier sind es neue Anblickewas
  ich dann suche? Ich weiß es nicht\ldots{} weiß nur, daß das überall so
  war.\\
  Der Bruder, wenn er die Nacht und den Berghang hinab welcher ist er
  dann wirklich? Nur blinder Nebel? Der entschieden für einen Ausgang
  und ihn uns zeigen damit er am Geheimnis seine Freude ? Das er nur
  bewahren kann wenn eresuns entleitet? Ich gebe dem freiwillig nach, es
  sind keine großen Forderungen damit und der Verdienst ungemein. Wir
  blieben uns wo immer wir uns begegnet und die Zeit nur eine Varianz
  die Gegenwart. Der Raum, mit demselben Körper wie die anderen
  Lebewesen auch, war glücklicherweise neutral; sonst hätten sich längst
  Probleme eingestellt im Umgang mit seinen Rechten.\\
  Aber vielleicht gab es ja auch Probleme. Ich werde mir z.B. nicht
  sicher darüber wie ich das Feld beibehalten soll. Einen Eingang
  gefunden hieß ja nicht, den Eintritt gleich erworben zu haben. Hatte
  ich doch aber, oder? Ludwig jedenfalls war wohlgesinnt, so viel. Das
  Tagewerk zeigt es. Wir sind uns irgendwie einig geworden. Und wenn ich
  das auch nur von hier behaupten kann, wo ihm jede Zusage unmöglich ist
  im erlaubten Medium, so bleibt n.~eine Sicherheit: ich spüre ihn mich
  ermutigen, fühle ein starkes Ja, du sollst! zu meinen Worten über ihn.
  Genügt es ihm weiter zuzuhören? Muß ich mehr tun?\\
  hier: Der Nordenstern steht hoch über dem Berg. Nebel fließen unter
  dem Schloß entlang und fallen zwischen den beiden Hängen die man vom
  Westturm aus sieht ab in den Alpsee. Vermute ich, sicher über die
  klimatischen Verhältnisse hier oben sind nur die Einheimischen. Jene
  könnte ich befragen mit einer jener typischen Fragen; kenne jedoch
  keine davon, müßte mir eine aneignen. Und so lange verrät mir niemand
  etwas über Fön/Hangauftrieb und Kammleuchten. Manches ist klar, und
  selbst ein Städter kann sich was denken. - Jetzt zum Beispiel sieht es
  aus als brenne etwas unterhalb der Nebeldecke und würde sie rot
  anleuchten vor dem Hang. Wenn ich mich erinnere waren dort keine
  Häuser mehr oder anderes Licht. Also Feuer da drüben, in der Nähe von
  Coloman? Soll man hinübergehen?
\end{enumerate}

\end{document}
