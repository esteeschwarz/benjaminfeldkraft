% Options for packages loaded elsewhere
\PassOptionsToPackage{unicode}{hyperref}
\PassOptionsToPackage{hyphens}{url}
%
\documentclass[
]{article}
\usepackage{amsmath,amssymb}
\usepackage{iftex}
\ifPDFTeX
  \usepackage[T1]{fontenc}
  \usepackage[utf8]{inputenc}
  \usepackage{textcomp} % provide euro and other symbols
\else % if luatex or xetex
  \usepackage{unicode-math} % this also loads fontspec
  \defaultfontfeatures{Scale=MatchLowercase}
  \defaultfontfeatures[\rmfamily]{Ligatures=TeX,Scale=1}
\fi
\usepackage{lmodern}
\ifPDFTeX\else
  % xetex/luatex font selection
\fi
% Use upquote if available, for straight quotes in verbatim environments
\IfFileExists{upquote.sty}{\usepackage{upquote}}{}
\IfFileExists{microtype.sty}{% use microtype if available
  \usepackage[]{microtype}
  \UseMicrotypeSet[protrusion]{basicmath} % disable protrusion for tt fonts
}{}
\makeatletter
\@ifundefined{KOMAClassName}{% if non-KOMA class
  \IfFileExists{parskip.sty}{%
    \usepackage{parskip}
  }{% else
    \setlength{\parindent}{0pt}
    \setlength{\parskip}{6pt plus 2pt minus 1pt}}
}{% if KOMA class
  \KOMAoptions{parskip=half}}
\makeatother
\usepackage{xcolor}
\usepackage[margin=1in]{geometry}
\usepackage{graphicx}
\makeatletter
\def\maxwidth{\ifdim\Gin@nat@width>\linewidth\linewidth\else\Gin@nat@width\fi}
\def\maxheight{\ifdim\Gin@nat@height>\textheight\textheight\else\Gin@nat@height\fi}
\makeatother
% Scale images if necessary, so that they will not overflow the page
% margins by default, and it is still possible to overwrite the defaults
% using explicit options in \includegraphics[width, height, ...]{}
\setkeys{Gin}{width=\maxwidth,height=\maxheight,keepaspectratio}
% Set default figure placement to htbp
\makeatletter
\def\fps@figure{htbp}
\makeatother
\setlength{\emergencystretch}{3em} % prevent overfull lines
\providecommand{\tightlist}{%
  \setlength{\itemsep}{0pt}\setlength{\parskip}{0pt}}
\setcounter{secnumdepth}{-\maxdimen} % remove section numbering
\ifLuaTeX
  \usepackage{selnolig}  % disable illegal ligatures
\fi
\usepackage{bookmark}
\IfFileExists{xurl.sty}{\usepackage{xurl}}{} % add URL line breaks if available
\urlstyle{same}
\hypersetup{
  hidelinks,
  pdfcreator={LaTeX via pandoc}}

\author{}
\date{\vspace{-2.5em}}

\begin{document}

\subsection{Chor}\label{chor}

Also HB wird nicht reden haben wir gesehen, auch wenn er zu wissen
scheint was den jungen Thomas antrieb der nicht nur uns verließ um
höherer Wahrheiten willen sondern der auch sein bereits erlangtes Wissen
mißbraucht hatte um sich auf den Akt der Erkenntnisübergabe
vorzubereiten wie es bei uns ja jedem freisteht zu tun. Damit wird er
sich zu einem Geächteten gemacht haben glaube ich. Natürlich kommt nicht
jede Mißachtung der Gesetze vor den Rat und ebenso ist nicht jede ihrer
Übertretungen Grund für eine Untersuchung. Was aber diesen Fall angeht
so erfahren wir damit eine ganz neue Dimension von Verletzungen unserer
Regeln. Es könnte so gewesen sein, daß Thomas schon zu Beginn der
Verhandlungen über seinen Eintritt darüber Bewußtsein hatte wie die
Gesetze der congrégation beschaffen waren. Anders ist sonst nicht zu
erklären wie es ihm damals gelungen war seine Beitrittserklärungen so
abzufassen, daß sie uns jetzt eine Anfechtung vielleicht unmöglich
machen. Es wird uns schwerfallen eine Begründung für seinen Ausschluß zu
finden obwohl er leicht zu sehen ein Unrecht begangen hat. Jedoch warum
oder zu welchem Erfolge er die Gemeinschaft betrog wird sich wohl nicht
rekapitulieren lassen. Wir müssen annehmen, daß er dem unbedingten
Seelenheil, das unsere Zugehörigkeit versprach, die Erfolgsvariante
\emph{linientreu und volksnah} vorzog. Thomas war uns abhanden
gekommen..

\end{document}
