% Options for packages loaded elsewhere
\PassOptionsToPackage{unicode}{hyperref}
\PassOptionsToPackage{hyphens}{url}
%
\documentclass[
]{article}
\usepackage{amsmath,amssymb}
\usepackage{iftex}
\ifPDFTeX
  \usepackage[T1]{fontenc}
  \usepackage[utf8]{inputenc}
  \usepackage{textcomp} % provide euro and other symbols
\else % if luatex or xetex
  \usepackage{unicode-math} % this also loads fontspec
  \defaultfontfeatures{Scale=MatchLowercase}
  \defaultfontfeatures[\rmfamily]{Ligatures=TeX,Scale=1}
\fi
\usepackage{lmodern}
\ifPDFTeX\else
  % xetex/luatex font selection
\fi
% Use upquote if available, for straight quotes in verbatim environments
\IfFileExists{upquote.sty}{\usepackage{upquote}}{}
\IfFileExists{microtype.sty}{% use microtype if available
  \usepackage[]{microtype}
  \UseMicrotypeSet[protrusion]{basicmath} % disable protrusion for tt fonts
}{}
\makeatletter
\@ifundefined{KOMAClassName}{% if non-KOMA class
  \IfFileExists{parskip.sty}{%
    \usepackage{parskip}
  }{% else
    \setlength{\parindent}{0pt}
    \setlength{\parskip}{6pt plus 2pt minus 1pt}}
}{% if KOMA class
  \KOMAoptions{parskip=half}}
\makeatother
\usepackage{xcolor}
\usepackage[margin=1in]{geometry}
\usepackage{graphicx}
\makeatletter
\def\maxwidth{\ifdim\Gin@nat@width>\linewidth\linewidth\else\Gin@nat@width\fi}
\def\maxheight{\ifdim\Gin@nat@height>\textheight\textheight\else\Gin@nat@height\fi}
\makeatother
% Scale images if necessary, so that they will not overflow the page
% margins by default, and it is still possible to overwrite the defaults
% using explicit options in \includegraphics[width, height, ...]{}
\setkeys{Gin}{width=\maxwidth,height=\maxheight,keepaspectratio}
% Set default figure placement to htbp
\makeatletter
\def\fps@figure{htbp}
\makeatother
\setlength{\emergencystretch}{3em} % prevent overfull lines
\providecommand{\tightlist}{%
  \setlength{\itemsep}{0pt}\setlength{\parskip}{0pt}}
\setcounter{secnumdepth}{-\maxdimen} % remove section numbering
\ifLuaTeX
  \usepackage{selnolig}  % disable illegal ligatures
\fi
\usepackage{bookmark}
\IfFileExists{xurl.sty}{\usepackage{xurl}}{} % add URL line breaks if available
\urlstyle{same}
\hypersetup{
  hidelinks,
  pdfcreator={LaTeX via pandoc}}

\author{}
\date{\vspace{-2.5em}}

\begin{document}

\subsection{guhl-ich}\label{guhl-ich}

Ich aber lebte längst in ihnen weiter als ich mir vorstellen konnte und
hatte die vergangenen Zellstrukturen mir angeeignet, die mich nun als
Tote behandelten. Rilke sprach davon und bestimmte n.~andere mehr als
ich gelesen haben konnte vor seinem Tod, jener aber war der erste, der
es für mich erwähnte und die Idee ist geblieben: tot bei den Toten zu
bleiben und nicht an sie zu rühren, wenn sie einen erwecken wollen, egal
was man erwartete vom zukünftigen Leben. Also folgte ich IHM (?), IBM
(HAL?), ihsous oder sonst einem Lehrer (rabbi), der eine Geschichte zu
erzählen hatte, fand auch Ihn, der Worte besaß wirkliche, die man nicht
nur hören konnte sondern auch lesen. Wohin er sein Buch schrieb war mir
nicht eröffnet worden, daß er aber es herausgebracht hatte Gewißheit,
denn es las ja vor mir schon jener andere von dem es angeblich handelte
- die HB-Besetzung des nicht zu vergessenden Polyhistoren.
Festgeschnallt am Schreibtisch vor dem Standardschriftprozessor war
diesem n.~kaum von der Möglichkeit bewußt, etwas aus sich
herauszupressen das mehr war als Stottern, wie es der Hölderlin tat
seiner ganzen Vergangenheit; aber Stottern hörte sich nicht gut an also
wollte er es lassen und anfangen da wo jener aufgehört hatte mit seinem
Buch: Schlug es auf und hatte die 117\emph{.} Seite beendet als
Fragment, aber auch jede folgende\ldots{} blieb Fraktur von dem
eigentlichen Text den man als Antwortrauschen darüberhören mußte um
irgendwas verstehen zu können. Und Fragmente erzeugen ein fantastisches
Rauschen wenn man sie nur richtig zu lesen weiß. Der guhl-Phantast ließ
eine der Anweisungen (unbeabsichtigt?) im Primärtext auftauchen, so daß
ich die Lesart kannte, die der andere voraussetzte. Ich werde sie an
geeigneter Stelle öffentlich machen, so daß die Manuskripte auch in
diesem hier Zusammenhange zu entschlüsseln sind ohne den Zugriff auf das
Archiv. Was n.? Ach ja: wir sind uns begegnet. Auf dem täglich zu
bestehenden Weg von Arbeitsbeginn bis Ende sind doch irgendwann die
gewohnten Strukturen eingebrochen und die Trennung nicht mehr
aufrechtzuerhalten gewesen. Was heißt das? Im Guten: das die Krankheit
an ihr Ende gelangt langsam, im Schlechten: daß man jetzt auch in den
anderen Sphären man selbst sein muß mit allen Attributen, die es dann
wie sonst anders ebenso zu verteidigen gilt; das sei neben der Sache
erwähnt\ldots{}\\
\ldots und das dreipunktelicht längst zum Endeinläuten geworden der
gegen mich gesetzen \emph{topoi,} alles daran hat sich zu mir umgekehrt
und ich laufe nicht mehr Gefahr etwas verschwenderisch preiszugeben um
der Anerkennung willen. Die Buchstabenwörter als schwarzer Bestandteil
fortgeführt, träge Masse die nur schwer zu bewegen war - jene also
weiter lesen zu lassen was sie euch dann zu sagen haben könnten, immer
erst lautgedacht (Rauschzustand) und später auch gegenständlich
verwirklichen meine Idee. Am Anfang (Jean) sagte man n.~Wort dazu, ich
schon nenne es \emph{idea} und was ich meine ist euch nicht unbedingt
klar? Muß es verdeutlichen. Es gab da ein Leben vor dem Worthauch,
glaube ich. Kann nur versuchen, es in mein sechs Tönen auszudrücken:\\
1. Das Kind ist den Weisen näher als es jeder spätere Mensch sein
kann.\\
2. Das Kind ist seiner Weisheit näher als Kind wie als späterer
Mensch.\\
3. Als Kind ist er seiner Weisheit näher wie als Mensch später.\\
4. Ein Kind ist seinem späteren Menschen näher in der Weisheit wie er
selbst.\\
5. Der spätere Mensch ist sich nur so nahe wie er als Kind weise war.\\
6. Später ist der Mensch nur so weise wie als Kind nah an Weisheit.-
Verstanden?\\
Der kleine Mensch hat das große Wissen um die Sache verlernt
auszudrücken und seine Entwicklung geht nur in der Richtung von dessen
Vermittlung. Dazu hat er alle Fähigkeiten, die es zu fördern gilt. Man
bringt ihm sprechen bei, lesen, Schreiben. Die Auseinandersetzung mit
den Versuchen der anderen hilft ihm, seine Methode zu verbessern. Hört
man ihm zu, liest man und kritisiert ihn so wächst er heran und kommt
seinem Urwissen näher. Alles ist in ihm vorhanden und muß nur von ihm
nach der Regel entdeckt werden, \emph{in den Bildern der Außenwelt nach
Übereinstimmung suchend.} Stellt er solche fest, so werden jene
gemeinsamen \emph{ideen} als schön empfunden usw. aber das habe\\
ich mir dann doch nicht ausgedacht sondern nur vergessen gehabt und
hielt es fürs Eigene. Das war die Archetypenlyrik aber mehr nicht. Mein
Eigentum ist daran nur die mich von ihr entfernende Konstante des
Benjaminfeldes. Soviel ich zu ihrer Entdeckung beitragen konnte habe ich
was in meiner Macht stand --

\end{document}
