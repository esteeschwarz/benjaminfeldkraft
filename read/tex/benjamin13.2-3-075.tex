% Options for packages loaded elsewhere
\PassOptionsToPackage{unicode}{hyperref}
\PassOptionsToPackage{hyphens}{url}
%
\documentclass[
]{article}
\usepackage{amsmath,amssymb}
\usepackage{iftex}
\ifPDFTeX
  \usepackage[T1]{fontenc}
  \usepackage[utf8]{inputenc}
  \usepackage{textcomp} % provide euro and other symbols
\else % if luatex or xetex
  \usepackage{unicode-math} % this also loads fontspec
  \defaultfontfeatures{Scale=MatchLowercase}
  \defaultfontfeatures[\rmfamily]{Ligatures=TeX,Scale=1}
\fi
\usepackage{lmodern}
\ifPDFTeX\else
  % xetex/luatex font selection
\fi
% Use upquote if available, for straight quotes in verbatim environments
\IfFileExists{upquote.sty}{\usepackage{upquote}}{}
\IfFileExists{microtype.sty}{% use microtype if available
  \usepackage[]{microtype}
  \UseMicrotypeSet[protrusion]{basicmath} % disable protrusion for tt fonts
}{}
\makeatletter
\@ifundefined{KOMAClassName}{% if non-KOMA class
  \IfFileExists{parskip.sty}{%
    \usepackage{parskip}
  }{% else
    \setlength{\parindent}{0pt}
    \setlength{\parskip}{6pt plus 2pt minus 1pt}}
}{% if KOMA class
  \KOMAoptions{parskip=half}}
\makeatother
\usepackage{xcolor}
\usepackage[margin=1in]{geometry}
\usepackage{graphicx}
\makeatletter
\def\maxwidth{\ifdim\Gin@nat@width>\linewidth\linewidth\else\Gin@nat@width\fi}
\def\maxheight{\ifdim\Gin@nat@height>\textheight\textheight\else\Gin@nat@height\fi}
\makeatother
% Scale images if necessary, so that they will not overflow the page
% margins by default, and it is still possible to overwrite the defaults
% using explicit options in \includegraphics[width, height, ...]{}
\setkeys{Gin}{width=\maxwidth,height=\maxheight,keepaspectratio}
% Set default figure placement to htbp
\makeatletter
\def\fps@figure{htbp}
\makeatother
\setlength{\emergencystretch}{3em} % prevent overfull lines
\providecommand{\tightlist}{%
  \setlength{\itemsep}{0pt}\setlength{\parskip}{0pt}}
\setcounter{secnumdepth}{-\maxdimen} % remove section numbering
\ifLuaTeX
  \usepackage{selnolig}  % disable illegal ligatures
\fi
\usepackage{bookmark}
\IfFileExists{xurl.sty}{\usepackage{xurl}}{} % add URL line breaks if available
\urlstyle{same}
\hypersetup{
  hidelinks,
  pdfcreator={LaTeX via pandoc}}

\author{}
\date{\vspace{-2.5em}}

\begin{document}

\subsection{II. Die Prophezeiung,}\label{ii.-die-prophezeiung}

Eine Kathedrale ist errichtet worden, die nun ihre Standfestigkeit
erweisen muß. Da werden Gewitter sein, Stürme und Hagel wird darauf
niedergehen. Haben wir die Baumeister an alles gedacht? Es scheint so,
daß nur n.~die Namen einzusetzen sind, für die wir den Bau beanspruchen.
Seiner ist schon gewiß, war es seit dem ersten Mal, da wir das neue
Gebäude betreten haben. Finden müssen wir n.~den Ihren, die Sie ihn bis
hierher begleitet haben und zusammen waren sie jenes unheimlich
anmutende Paar, das hinter unseren Worten steckte. Also, Thomas, wie
heißen Sie wirklich?\\
Aber`s war der Aquinate und hatte seinen Namen nicht aufgegeben. Jemand
rüttelte wohl an ihm und den Grundfesten des Werkes, aber das drang
nicht bis zu ihm weil es genug Verteidiger gab in den Vorhöfen. Es nahm
aus unseren Reihen zuviel mit\ldots{} und die fehlen mir jetzt beim
Aufbau. Die Glücklichen unter ihnen sind n.~von der eigenen Geschichte
ermordet worden, und wer sich nicht rechtzeitig zu seiner Tat
entschlossen hatte, verpaßte die letzte Möglichkeit seine Gesinnung zu
beweisen. Auch ich - aber ich lebte n.~im Unterschied zu den meisten
anderen. Und jene ebenso fehlen zum Aufbau, der hier stattfinden soll.
Aber laßt uns anfangen mit dem was zur Verfügung steht. Wer ist der
Eckstein?\\
Hierwieder meldete sich der Aquinate zu Wort. Auch mein zweiter Versuch
ihn zum Schweigen zu bringen, scheiterte. Sein um ihn gebreiteter Mantel
jenes größeren Schweigens schloß mich mit ein und aus dieser ihm
innewohnenden Position konnte ich nicht handeln, ich mußte mitansehen
wie er seine Gegner zermalmte - wie er uns Menschen für Menschen
vernichtete. Wo kam ihm aber das große Schweigen her? Natürlich von
unserem Gott\ldots{} was sonst wäre solch eine Erfüllung wert, wenn sie
nicht vorausgesehen worden wäre, was wäre sein Gericht in jenen Tagen
ohne seine Prophezeiung. Es war also geweissagt worden und der
Steinleser mußte nur sorgfältig zuhören seinem Lektor, Mentor und dessen
faustische Eingebungen zu deuten wissen für die rechte Sache als die er
seine ja ansah. Daß sie es nicht war, nicht sein konnte - wissen nur die
sich an ihr verloren haben durch die "Infektion" wie es vor kurzem Chr.
Schlingensief in \emph{Metanoia }nannte, der jetzt auch ein Vorgängiger
ist\ldots{} aber daran nahm ich nicht teil. Vielleicht wird es nötig,
die Namen auszusprechen, die unbedingt dazugehören. In den Geschichten
fingen wir damit an, auf dem Weg in dies Buch jedoch sind seine (des
Mentoren) Namensgeber irgendwie nach unten durchgerutscht. Ich werde sie
einsammeln müssen auf dem Boden der G. die heute spielt. Aber heute ist
schon fast morgen und die Gn von morgen werden die des nächsten Tages
sein - weil Morgen immer der nächste Tag gewesen ist.\\
Vielleicht reichten ja die drei Stunden von Mitternach. Haben Sie das
jemals versucht, wachzubleiben? Was könnte schlimmstenfalls
passieren\ldots{} Eins könnte so sein: daß mir die Stunden jetzt
dazugezählt werden - weil ich ja zwar im Wachzustand aber ebenfalls
Aufarbeitung betreibe und Traum und den Geist beruhigend ausschlafe. Es
ist so, es soll so sein und beides wäre gleichberechtigt nebeneinander
Schlaf, Schlaf als Tod als Zeitlosigkeit, die man ebensogut sich im
Wachzustand herstellen kann. Und das tue ich gerade. Was soll euch aber
das bedeuten\ldots? Warum sei es wertvoll, wozu Äußerungen. Nur weil die
Nacht so* schön* ist? Was, wenn keine Tage mehr kämen nach ihr; dann
wären dies immer dies die letzten Worte. Es ist ein lächerlicher
Versuch, das eigene am Werk nur feststellen zu wollen. Es zeigt sich ja
nicht vor dem bewußten Umgang überhaupt mit dem Tag. Der ist doch immer
erst träumend zu erkennen. Aber etwas dringt durch spüre ich, obwohl ich
wach bin. Merkst du das nicht? Als wenn sich alles von selbst ergäbe was
sich je ereignete und eine Schuld nur dann zu vergleichen wäre mit der
von denen, die eines jeden Tag für sich benutzen, wenn ich aber auch
eines anderen Tag für mich benutze um dieses hier zu äußern\ldots{} Und
da ist die Fuge, der dein Licht entströmt\ldots{} auch sie, eine
unheimliche Ephemerenquelle die ich gefunden haben werde, wenn das hier
an sein Ende kommt. Es gibt scheinbar immer eine zweite Hälfte des
Lebens. Möglich, daß sie heut nacht anbrach. Ich wäre nicht enttäuscht
von mir n.~verbleibenden\ldots{} Jahren und verspreche hiermit
Gewissenhaftigkeit: im Glauben (und hier würde ich gerne folgen hören an
die Auferstehung etc., aber es wird glaube ich eher heißen) an den Wert
des vollbrachten Lebens unter der Prämisse seiner eigentlich nicht zu
vollendenden Aufgabe - es wirklich zu bestehen und nicht zu fliehen.\\
Manches war mystisch. Dann wird man von irgendwo einem Verstehen
angehaucht und braucht das Blattwerk nicht mehr aufstören in der Suche
nach den verschwiegenen Zusätzen. Aber das Mystische bleibt erhalten,
auch über die erkennende Arroganz hinaus, die Ataraxie, mit welcher man
bald gesegnet ist. Es liegt außerhalb der menschlichen Momente. Kommt
man dahinter, was sie so konstituiert hat, daß sie sich erkennen lassen
müssen ist es schon zu spät um durch die gewohnte Blindenbrille schauen
zu dürfen, man wird als Erkennender gezeichnet und es teilt sich allen
mit. Allein was man gesehen hat bleibt einem als Geheimnis, das mag ein
Trost sein; manchem ist aber auch das gerade der größte Fluch, weil er
sich Heilung davon versprach ein Prophet zu werden um das Geschaute. Ich
jedenfalls wußte etwas und hatte mit dem Wissen meinen Frieden gemacht,
so daß es nicht mehr unbedingt herausdrängte an alle Ufer. Es gab auch
so genug zu erzählen, nur finden mußte ich es. Die Enden waren
gestaltet, die Anfänge hatten begonnen, es war Zeit, mit den Fäden zu
arbeiten. Wie Spinnen\ldots{} eigentlich schon; selbst die Opfer kannte
ich jetzt, denen das Netz gehalten würde. Nach ihnen mußte ich sehr
lange suchen, hatte aber schließlich Glück gehabt als ich meine Blicke
seitwärts schweifen ließ in Unergründetem; da grasten sie - die
angschwollnen Lämmer, gescheckt u. zur Schur bereitet. Das Messer
scharf, ein ungetrübter Blick auf die Uhr sagt es mir: Zeit, rauszugehn,
Zeit die Übungen zu beenden. Le Livre du travail commence içi.
Maintenant.

\end{document}
