% Options for packages loaded elsewhere
\PassOptionsToPackage{unicode}{hyperref}
\PassOptionsToPackage{hyphens}{url}
%
\documentclass[
]{article}
\usepackage{amsmath,amssymb}
\usepackage{iftex}
\ifPDFTeX
  \usepackage[T1]{fontenc}
  \usepackage[utf8]{inputenc}
  \usepackage{textcomp} % provide euro and other symbols
\else % if luatex or xetex
  \usepackage{unicode-math} % this also loads fontspec
  \defaultfontfeatures{Scale=MatchLowercase}
  \defaultfontfeatures[\rmfamily]{Ligatures=TeX,Scale=1}
\fi
\usepackage{lmodern}
\ifPDFTeX\else
  % xetex/luatex font selection
\fi
% Use upquote if available, for straight quotes in verbatim environments
\IfFileExists{upquote.sty}{\usepackage{upquote}}{}
\IfFileExists{microtype.sty}{% use microtype if available
  \usepackage[]{microtype}
  \UseMicrotypeSet[protrusion]{basicmath} % disable protrusion for tt fonts
}{}
\makeatletter
\@ifundefined{KOMAClassName}{% if non-KOMA class
  \IfFileExists{parskip.sty}{%
    \usepackage{parskip}
  }{% else
    \setlength{\parindent}{0pt}
    \setlength{\parskip}{6pt plus 2pt minus 1pt}}
}{% if KOMA class
  \KOMAoptions{parskip=half}}
\makeatother
\usepackage{xcolor}
\usepackage[margin=1in]{geometry}
\usepackage{graphicx}
\makeatletter
\def\maxwidth{\ifdim\Gin@nat@width>\linewidth\linewidth\else\Gin@nat@width\fi}
\def\maxheight{\ifdim\Gin@nat@height>\textheight\textheight\else\Gin@nat@height\fi}
\makeatother
% Scale images if necessary, so that they will not overflow the page
% margins by default, and it is still possible to overwrite the defaults
% using explicit options in \includegraphics[width, height, ...]{}
\setkeys{Gin}{width=\maxwidth,height=\maxheight,keepaspectratio}
% Set default figure placement to htbp
\makeatletter
\def\fps@figure{htbp}
\makeatother
\setlength{\emergencystretch}{3em} % prevent overfull lines
\providecommand{\tightlist}{%
  \setlength{\itemsep}{0pt}\setlength{\parskip}{0pt}}
\setcounter{secnumdepth}{-\maxdimen} % remove section numbering
\ifLuaTeX
  \usepackage{selnolig}  % disable illegal ligatures
\fi
\usepackage{bookmark}
\IfFileExists{xurl.sty}{\usepackage{xurl}}{} % add URL line breaks if available
\urlstyle{same}
\hypersetup{
  hidelinks,
  pdfcreator={LaTeX via pandoc}}

\author{}
\date{\vspace{-2.5em}}

\begin{document}

\subsection{2. The Gathering}\label{the-gathering}

\begin{enumerate}
\def\labelenumi{\alph{enumi}.}
\tightlist
\item
  Man nehme an, daß es eine wirkliche Zeit gegeben hat. Und auch wenn
  das vor uns war und unsere Vorstellungskraft überschreitet, nehme man
  n.~dazu an, daß sich etwas herausgebildet hat, das jene Periode
  überdauerte, die wir genannt haben Phase I oder Eintrittsphase. Nichts
  eigentlich, was von dort n.~existieren könnte bis heute, wenn wir
  daran glauben, was uns die Alten erzählten: daß es eine Neuerschaffung
  gegeben hat mit einer völlig anderen Struktur des Denkens die zu
  derjenigen des selbst, das damals bestanden haben mag in keiner Weise
  kompatibel sein dürfte weil diese laut unserer Analysen nicht als in
  einem dualen System darstellbarer Zahlencode vermittelt wurde sondern
  höchstens als Lichtbrechung an den Perspektivachsen in Erscheinung
  trat. Das war schon seine ganze Struktur: Brechung, von akustischen
  und optischen Signaturen zu einer Interpretation durch den
  synaptischen Apparat wie er wohl existiert haben muß. Was hat uns dazu
  geführt, diese Struktur anzunehmen? Vielleicht müssen wir anders
  fragen, damit jeder Doppelsinn ausgeschlossen werden kann: Was hat uns
  zu \emph{den} Gedanken geführt, die man nur in jener scheinbar doch
  unmöglichen Denkstrukur überhaupt zu fassen erst in der Lage ist, in
  die Lage versetzt sein wird eigentlich, wenn die Welt sich
  weitergedreht hätte? Und weitergedreht hat sie sich nach der
  Eintrittsphase, das konnte man messen und die Daten sind zugänglich
  gewesen lange bevor jemand n.~daran dachte, sie für diese Forschung
  jetzt zu verwenden (-sie sind damit neutral.) Wir können auch
  annehmen, daß sie sich ein weiteres Mal drehen wird und damit eine
  vierte Phase passiert, die wir natürlich n.~nicht vorstellen können,
  genausowenig die behandelnde Struktur unserer momentanen Situation
  vorstellbar war auf dem Weg aus der ersten Wahrnehmung. Woher wir das
  wissen? Es gibt keine Aufzeichnungen, es gibt keine Notizen Fragmente
  Restbestände die lesbar wären mit unseren Mitteln. Das einzige
  Material auf das wir uns stützen können sind jene in einem wie
  Ausbruchstadium steckengebliebenen Beobachtungsprotokolle der
  täglichen Geschehnisse um die fiktive Struktur herum, die sie aber
  nicht berührten n.~Hinweise gaben über ein Motiv ihres Vorhandenseins
  \emph{am Ort der Entscheidung über Dringliches}. Aber vorhanden war da
  etwas\ldots{} das über die Aufzeichnung hinausging und reine
  Beobachtung, es muß ein Moment gewesen sein das wir ehestens mit
  unserem Begriff von Energie vergleichen können, also Willen oder
  Anspannung, Zielstrebigkeit vielleicht Intention. Doch was soll
  solches gewesen sein?\\
  Die Archekataloge geben uns verschiedenes an, über solche
  \emph{Willenskraft} wie sie auch hieß, Daten zu erheben und dem
  Menschen dahingehend zu gebieten der eventuell über sie verfügt.
  Obwohl nur wenige ein Potential haben sich darin zu entwickeln, treten
  fallweise Häufungen in bestimmten Gebieten auf, die wir dann
  Gegebenheitscluster nennen weil an ihnen zwei der Invarianzprinzipien
  (des Ortes und dessen, was unserer Zeit entspricht) zusammentreffen
  und die Signale erzeugen, die wir hier hören können. Ob es n.~sinnvoll
  ist in ihnen nach Bedeutungen zu suchen ist eine Frage auch die wir
  aufschieben müssen zum besseren Verständnis der damaligen Verhältnisse
  und der unmittelbaren Umgebung in welcher jene Struktur verankert
  schien. So lange es keine Sicherheit gibt über jedweden Zustand in dem
  sie sich manifestiert haben könnte sind Aussagen nur ungewiß über
  Inhalte oder Attribute. Eines aber wissen wir schon:\\
  Die Töchter der Magie (das war ihr Name) haben sie weitervererbt und
  so lange aufrechterhalten wie das Blut folgte. Danach wurde sie von
  ihren Männern in Besitz genommen und konnte nur auf Umwegen erworben
  werden. Diese zu beschreiten, sie erst zu finden und dann zum Ende zu
  gehen, war eine Lebensweise in der viele verzweifelten; manche jedoch
  haben s. erreicht.
\end{enumerate}

\end{document}
