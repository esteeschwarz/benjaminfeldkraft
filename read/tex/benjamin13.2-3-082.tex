% Options for packages loaded elsewhere
\PassOptionsToPackage{unicode}{hyperref}
\PassOptionsToPackage{hyphens}{url}
%
\documentclass[
]{article}
\usepackage{amsmath,amssymb}
\usepackage{iftex}
\ifPDFTeX
  \usepackage[T1]{fontenc}
  \usepackage[utf8]{inputenc}
  \usepackage{textcomp} % provide euro and other symbols
\else % if luatex or xetex
  \usepackage{unicode-math} % this also loads fontspec
  \defaultfontfeatures{Scale=MatchLowercase}
  \defaultfontfeatures[\rmfamily]{Ligatures=TeX,Scale=1}
\fi
\usepackage{lmodern}
\ifPDFTeX\else
  % xetex/luatex font selection
\fi
% Use upquote if available, for straight quotes in verbatim environments
\IfFileExists{upquote.sty}{\usepackage{upquote}}{}
\IfFileExists{microtype.sty}{% use microtype if available
  \usepackage[]{microtype}
  \UseMicrotypeSet[protrusion]{basicmath} % disable protrusion for tt fonts
}{}
\makeatletter
\@ifundefined{KOMAClassName}{% if non-KOMA class
  \IfFileExists{parskip.sty}{%
    \usepackage{parskip}
  }{% else
    \setlength{\parindent}{0pt}
    \setlength{\parskip}{6pt plus 2pt minus 1pt}}
}{% if KOMA class
  \KOMAoptions{parskip=half}}
\makeatother
\usepackage{xcolor}
\usepackage[margin=1in]{geometry}
\usepackage{graphicx}
\makeatletter
\def\maxwidth{\ifdim\Gin@nat@width>\linewidth\linewidth\else\Gin@nat@width\fi}
\def\maxheight{\ifdim\Gin@nat@height>\textheight\textheight\else\Gin@nat@height\fi}
\makeatother
% Scale images if necessary, so that they will not overflow the page
% margins by default, and it is still possible to overwrite the defaults
% using explicit options in \includegraphics[width, height, ...]{}
\setkeys{Gin}{width=\maxwidth,height=\maxheight,keepaspectratio}
% Set default figure placement to htbp
\makeatletter
\def\fps@figure{htbp}
\makeatother
\setlength{\emergencystretch}{3em} % prevent overfull lines
\providecommand{\tightlist}{%
  \setlength{\itemsep}{0pt}\setlength{\parskip}{0pt}}
\setcounter{secnumdepth}{-\maxdimen} % remove section numbering
\ifLuaTeX
  \usepackage{selnolig}  % disable illegal ligatures
\fi
\usepackage{bookmark}
\IfFileExists{xurl.sty}{\usepackage{xurl}}{} % add URL line breaks if available
\urlstyle{same}
\hypersetup{
  hidelinks,
  pdfcreator={LaTeX via pandoc}}

\author{}
\date{\vspace{-2.5em}}

\begin{document}

\subsection{Euryd.e ist zurück?}\label{euryd.e-ist-zuruxfcck}

\begin{enumerate}
\def\labelenumi{\alph{enumi}.}
\setcounter{enumi}{1}
\tightlist
\item
  Nehmen wir weiteres an: der Hermesbote hätte kein Verbot ausgesprochen
  damals und mit diesem Or/Eu auf ewig getrennt- wie lange wäre es dann
  nur gewesen, daß der Sänger endlich aufhörte in der Erfüllung, die
  Leier zu schlagen, glücklich verstummt. Ich wage nicht daran zu
  zweifeln, daß die Liebeslieder es tief meinten, was er sich ersehnte.
  Was ich jedoch nicht in Frieden sein lassen kann, sind seine
  Ausflüchte in die menschlicher Seelen Niederungen wie es heißt wo er
  etwas finden würde wenn er nur suchte das Bedeutung hätte.29 Und wenn
  wir jetzt schon dahin abgeschweift sind seinen Erinnerungen nach
  (Jahreszeiten\ldots) wo er überall schon gewesen ist seit den ersten
  Sommern die allein zu verbringen waren- dann können wir uns auch den
  Ausblick erlauben über den Horizont jener einen Wahrheit die ihn mir
  schuf: das war er, wie ihn die Wissenschaft mir hinterlassen hatte und
  nichts habe ich hinzugefügt oder weggenommen, er ist so wie ich es
  euch hier überliefere auf mich gekommen.\\
\item
  Sternfelder: denn über die Sternfelder kam es zuerst. Man mochte
  meinen, daß es sie unendlich viel Energie gekostet hat, so weit
  vorzudringen. Wir glauben jedoch eher an ihre Fähigkeit diese Energie
  aus den überschüssigen Elementaren der Zustände herzuleiten die nicht
  eintraten u. die aber genug Intention dazu aufgebracht hatten daß der
  Nichteintritt ebenfalls Wirkung zeitigte. Die Ladung oder Richtung der
  I. würde auf der nächsthöheren Ebene keine unterschiedlichen Vektoren
  erzeugen, die uns hier interessierten. Es gab sie ohne Frage, nur sie
  sind jetzt nicht mehr wichtig wo wir die erste Ebene verlassen
  können.\\
  Vielleicht habe ich einmal Nebensätze final enden lassen so daß keine
  weitere Entscheidung mehr darüber notwendig erschien ob sie
  tatsächlich an ihr Ende gelangten oder nur die Syntax es vorgab der
  Geschichte nicht weiter folgen zu wollen. Vielleicht nehmen wir einmal
  weiter an: daß jene syntaktischen Gegebenheiten des natürlichen
  Redeflusses wie ich sie vom Lehrer HB wie er sie lernte von Gadamer u.
  jener d.~Heidegger und wie jener seinen begeisterten Meisterschülern
  selbst in alle Winde verstreut die Welt mit ursprünglichen Zitaten
  hermeneutisch aufzuschließen verstand -- daß diese also genau dorthin
  auf jenen immer Mann zurückführen, der sie als letzte auszusprechen
  wagte mit seinen spärlichen Mitteln und von dem wir sie als erstes
  vernommen haben: am Anfang der Philosophie und hatten Parmenides über
  Heraklit und Plato verfolgt bis zu diesem letzten großen Zauderer nun:
  Hölderlin selbst, die Mutter seiner Botschaften über das All hinaus,
  der große Schlotternde. Aber es war niemals genug gewesen nur darin zu
  lesen. Man sollte es hören können was er nicht vermochte, und im Gehör
  bildeten sich die Bezüge dann, im Wasser des Labyrinths -- dem weichen
  Vermittler zwischen den schwingenden zwei (ovalen und runden)
  Fenstern. Da mischt sich bekanntes mit numinösem, das alltägliche mit
  den Ahnungen die uns die Archetypenlyrik eingibt. Und obwohl H. weit
  entfernt von uns heute n.~lallenden in seiner Sprache eingeschlossen
  war so ist das Gefängnis selbst (sie) kein anderes als damals und auch
  kein anderes als jenes des Parmeniden. Doch so tief muß ich gar nicht
  eindringen, es reicht hier, der eigenen Stimme zuzuhören beim ihrem
  Stammeln auf der Suche nach den \emph{sich im Laufe der Rede von
  selbst verfertigenden Gedanken, }wie von Kleist gefordert. Manchmal
  hört man sich dabei, dann werden es glückliche Gespräche. Dazu braucht
  es aber auch ein Gegenüber das den Faden zu verlieren nicht sofort es
  aufgeben läßt, sondern das sich traut, über dem Rauschen während der
  Pausen die weiteren optischen Entsprechungen im Textkorpus
  wahrzunehmen und sie in sich aufzunehmen. Es gibt ein unseren Worten
  entsprechendes Politikum \emph{Integration} genannt das genaues an
  Menschen vollzieht:\\
  Wir trauen uns, einen Fremdkörper im Sprachfluß mit aufzunehmen um den
  Preis der abfallenden Sinnebene; gewinnen aber oft daraus keinen Schub
  weil der Fehler (Schreib/Sprechfehler) uns zum Umdenken nötigt und
  eine Richtung vorgibt, die wir nicht bedacht hatten. Draußen nannte
  man die entstehende Bereicherung eine \emph{multikulturelle
  }Gesellschaft. Hier nenne ich die Möglichkeiten, die ein solcher
  Eindringling oder Fremdling bietet vielleicht: \emph{spontane
  Mutation} oder ähnliches aus dem Bereich der Biologie. Der Text ist
  eigentlich ein homogenes Gewebe und die mir darin passierenden Fehler
  sind Keimzellen für Wucherungen. Man soll nur nicht versuchen, sie
  über ihr negatives Potential zu definieren. Es gibt zu wenig
  Wortfehler, die sich nicht zu einem phantastischen Gebilde umformen
  lassen oder wenigstens umdeuten zu einem optischen Gefüge, als daß ich
  die Geschichte ihrer Zuwanderung verschließen wollte. Jene wenigen
  fallen so von selbst aus, daß sie als das was sie sind sich zeigen:
  Einfluß. Sozialisation. Fehlsublimation. Der echte Text trägt sie.\\
  E. ein weiterer Sabbath war notwendig, um die Schrift weiter zu
  \emph{begehen.} Aber vielleicht sind wir uns nicht immer einig über
  angestrebte Verläufe der Beziehungen gewesen in welche wir uns ja
  niemals freiwillig fügten; so ich z.B.: der ihr EL-S vorgenannten und
  n.~nicht \emph{Vorgängigen} wie immer es auch heißen mag sie also fast
  überfallen hatte mit meinem Wunsch nach einer Verständigung über ihr
  Schweigen hinweg; das sie also brach.Und endlich: wenn ich vor zwei
  Tagen die nicht von ungefähr passierte Reise schließlich bezwang im
  Gleichklang wohl, im einigenden und läuternden Einvernehmen mit der
  Autorin wie ich mir Frau Schüler wohl vorstelle:\\
  el: als vorangenommenes Präfix des Namens weil er als existierend
  n.~gar nicht eingeführt werden darf - hier also nur der Weißbruch der
  an seinem zwar immer vorbeiführt aber mich doch innen heranleitet,
  wenn ich die richtige Sprache gewählt habe. Ob das jetzt schon so ist
  oder irgendwann erst, wenn vielleicht n.mal Antwortet kam, mehr als
  durchs stagnierende Responsorium; dann jedenfalls könnten auch die
  kleineren Zeiteinheiten meiner Zwischenrufe (ins Totenreich wo die
  wirklich Vorgängigen sie hören) von ihr gehört werden, um mir ihre
  Verwandlungen derselben zuzustellen. Doch\ldots* wie lange hält man
  aus* die Zurückhaltung aller anderen die sich um unsere Freiheit mühen
  unsere Mündigkeit im Auge behalten und für den gesunden Geist im
  gesunden Körper soviel Unrecht auf sich laden aber in \emph{unserem
  Namen} und wenn wir uns von ihnen lossagten es nur Abspaltung wäre,
  keine Überwindung, nur Aufschub, nie eine wirkliche Beendigung jenes
  unwissenden Zustandes über das Leid und was es uns wirklich kostete,
  und ob wirs jdfs. schon linderten oder nur davon träumten. Was wir
  auch verdienten konnte nicht genug sein um eine große Revolution damit
  zu haushalten. Irgendwann würden wir den Pfad der reinen Meinung
  verlassen müssen und habhaft werden, es wäre damit ein geringes getan
  aber dringend ein Anfang gemacht den Sie nicht den ich n.~nicht
  voraussehen werde; davor bewahrt sind wir nur weil Sie ein weil ich
  ein reines Gewissen haben n., solange wir nicht über einen Sabbathweg
  aus dem Haus gingen bis zum Dreisternenlicht morgen; aber was taten
  wir nicht alles um ihm zu gefallen, darum. Und es ist nicht leichter
  geworden. Der unsere Last trug jetzt selbst eine Last mit der wir
  umzugehen haben.
\end{enumerate}

\end{document}
