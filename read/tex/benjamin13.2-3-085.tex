% Options for packages loaded elsewhere
\PassOptionsToPackage{unicode}{hyperref}
\PassOptionsToPackage{hyphens}{url}
%
\documentclass[
]{article}
\usepackage{amsmath,amssymb}
\usepackage{iftex}
\ifPDFTeX
  \usepackage[T1]{fontenc}
  \usepackage[utf8]{inputenc}
  \usepackage{textcomp} % provide euro and other symbols
\else % if luatex or xetex
  \usepackage{unicode-math} % this also loads fontspec
  \defaultfontfeatures{Scale=MatchLowercase}
  \defaultfontfeatures[\rmfamily]{Ligatures=TeX,Scale=1}
\fi
\usepackage{lmodern}
\ifPDFTeX\else
  % xetex/luatex font selection
\fi
% Use upquote if available, for straight quotes in verbatim environments
\IfFileExists{upquote.sty}{\usepackage{upquote}}{}
\IfFileExists{microtype.sty}{% use microtype if available
  \usepackage[]{microtype}
  \UseMicrotypeSet[protrusion]{basicmath} % disable protrusion for tt fonts
}{}
\makeatletter
\@ifundefined{KOMAClassName}{% if non-KOMA class
  \IfFileExists{parskip.sty}{%
    \usepackage{parskip}
  }{% else
    \setlength{\parindent}{0pt}
    \setlength{\parskip}{6pt plus 2pt minus 1pt}}
}{% if KOMA class
  \KOMAoptions{parskip=half}}
\makeatother
\usepackage{xcolor}
\usepackage[margin=1in]{geometry}
\usepackage{graphicx}
\makeatletter
\def\maxwidth{\ifdim\Gin@nat@width>\linewidth\linewidth\else\Gin@nat@width\fi}
\def\maxheight{\ifdim\Gin@nat@height>\textheight\textheight\else\Gin@nat@height\fi}
\makeatother
% Scale images if necessary, so that they will not overflow the page
% margins by default, and it is still possible to overwrite the defaults
% using explicit options in \includegraphics[width, height, ...]{}
\setkeys{Gin}{width=\maxwidth,height=\maxheight,keepaspectratio}
% Set default figure placement to htbp
\makeatletter
\def\fps@figure{htbp}
\makeatother
\setlength{\emergencystretch}{3em} % prevent overfull lines
\providecommand{\tightlist}{%
  \setlength{\itemsep}{0pt}\setlength{\parskip}{0pt}}
\setcounter{secnumdepth}{-\maxdimen} % remove section numbering
\ifLuaTeX
  \usepackage{selnolig}  % disable illegal ligatures
\fi
\usepackage{bookmark}
\IfFileExists{xurl.sty}{\usepackage{xurl}}{} % add URL line breaks if available
\urlstyle{same}
\hypersetup{
  hidelinks,
  pdfcreator={LaTeX via pandoc}}

\author{}
\date{\vspace{-2.5em}}

\begin{document}

\subsection{1. GUHL}\label{guhl}

Ihnen abverlangt, weil alles er gelesen haben möchte. Doch ich versprach
euch eine Geschichte, oder\ldots{} aber ach, was ist das\ldots{}
\emph{jeder Zoll, den die Menschheit weiterrückt }kostet Ströme Bluts
und jedes Wort beherbergt eine Welt, die sich euch verbirgt. \emph{Und
jedes Wort ein Stück Welt, mit dem wir allein sind. }Meine Geschichte
wird gewesen sein, daß ich euch dahin führe, die Worte endlich fassen zu
können, wie ich sie mir vorstelle in der ebenen Weise, von eurem Geist
gesehen. Wie das geht? Damit haben Bücher voller Weiser ihre Randnotizen
in meiner Schrift hinterlassen und sind nicht wir n.~nicht zum Ende
damit gelangt, nur immer Stücke weiter hinein Holzwege in den Wald und
Glück ist nur, daß es Tag und Nacht gibt woran wir klar sehen können,
daß die Zeit überhaupt vergeht und nicht ein einziges Ding ist beides
und der Morgen immerunaufhörlich, unaufschiebbar unnachgiebig. Aber
dieser Morgen ist der nächste Tag und seine Geschichten waren die des
nächsten Tages weil morgen immer der nächste Tag gewesen ist. So
schließe ich auch heute das Buch zur Epiphanienelfuhr nach 1500 Wörtern,
die mir dieser Abend überlassen hat. Das ist viel für Abende, die nicht
offen sind und woher dies kam, weiß ich nicht genau, ahne nur
irgendetwas zwischen Thinkpad und Cassandra, das mir hilft, wie mir das
Fernste manchmal hilft: in mir.\\
Aber auch hier die Grenze, klamm und kältlich und vielleicht nur deshalb
können wir sie auch nur mehr als ahnen- weil was sie ausmacht unserem
Empfinden so fremd ist wie alles andere auch, was nicht vorgestellt
werden kann. Und die Grenze ahnen ist eben n.~nicht sie vorstellen,
dafür bräuchte es ein Datum, wo sie zum Beispiel in Kraft getreten wäre
oder Koordinaten, eine Adresse\ldots{} einen festen Punkt einmal nur
einmal, an welchem sie bewiesen war. Aber das ging über die
metaphysischen Parameter hinaus, mit welchen allein unser Glaube
hantieren darf. Vielleicht hat es einmal so ausgesehen als wenn ich
Nebensächliches in Finalsätzen auslaufen ließ. So zu verfahren ist
falsch.\\
Warum aber der Glaube nur nicht unter ausschließlich physikalischen
Bedingungen erlaubt ist, die Naturwissenschaften jedoch schon und fast
gut ohne die Metaphysik auskommen ist als seine Fragestellung schon
beinahe tautologisch, also unbedingt wahr. Es müßte in der
n.~verbleibenden Hälfte dieses dritten Bandes aufgeklärt werden welches
denn der geforderte Glaube wäre dem die Protagonisten sich ausgesetzt
fühlen und so seine Berechtigung erhalten sei es erst in der Verneinung
aber ganz sicher in der Reflexion.\\
Lassen Sie mich an einem Beispiel folgendes demonstrieren: Wenn man
Reflexion stets voraussetzen darf, wo wir uns begegnen - das Guhl-Ich
und die herbeigeführten Protagonisten, diese sich begegnenden
Unendlichkeiten (wie die zwei der Himmel und des Brunnen in sich
starrenden) - wenn wir jene also einmal setzen; Reflexion nicht im Sinne
eines Charakterzuges gemeint oder menschlicher Fähigkeit sondern nur als
passives Spiel zweier Ansichten, zweier Intentionen sich zu äußern dann
kommen wir bald dahin sie zwischen beiden anzusehen als Bindeglied
Austauschsphäre oder besser schon: Medium, in welchem die Begegnung
stattfindet stattfinden kann könnte oder würde, wenn von den zwei Seiten
die Erlaubnis dazu gegeben ist. Wie aber und in welcher Hierarchie wird
die Erlaubnis erteilt? Gleichzeitig so daß es keine Kausalität mehr
gibt? Gleichzeitig und trotzdem kausal verknüpft in einer verminderten
Ebene; gleich\emph{zeitig} ohne konkreten zeitlichen Bezug also nur für
unsere Draufsicht zeitlich gebunden, nicht jedoch im eigenen Komplex?
Oder ohne jeden Zeitbezug? \emph{Wo} wäre es dann? Also genau \emph{wo?
}Denn nach den Invarianzprinzipien ist uns ja gestattet \emph{einen}
Parameter zu vernachlässigen, was wir also jetzt im Sinne der
T-Invarianz getan haben. Damit ist unter sonst gleichen Bedingungen die
zeitliche Einordnung für die Hierarchie unerheblich, es gibt keine
derartige Reihung. Wäre also nur n.~mit den alten Worten zu klären wann
wir hier sind wenn wir hier sind an diesem Ort. Denn jener ist das
nahezu fixe Moment der Schreibe. Wir haben die Konstante bestimmt,
können es jeden (Augenblick) tun und legen den Ort damit fest wo wir uns
befinden wenn wir uns hier befinden. Die Gegenläufigkeit eueres
Leseverhaltens zu dem meinen der ich ja erst rückwärtsgewandt euch
anblicken kann läßt in dieser hier konstanten Mitte unserer
Ortsverknüpfung die Zeit entstehen \emph{zu} welcher sie sich begeben
die drei Momente:\\
Halten wir das einmal fest: \emph{Es muß sie geben.} Sonst wäre das Buch
nicht entstanden das ihr jetzt zur Hand habt. Es muß sie also
geben\ldots{} das ist merkwürdig genug. Denn ich wußte es irgendwie
bevor ich mich entschieden hatte sie aufzuschreiben als hätten sie Leben
ohne mich schon gehabt und bemerkte ihr Vorhandensein in meiner
\emph{realwelt} die nichts mehr mit euer \emph{intime} zu tun hat oder
besser vielleicht n.~nicht. Aber ich soll mich wegscheren\ldots{} sagen
manche Gedanken protagoniste Fehler, (wie auch nicht- die Einzelfälle
umgehe ich geschickt.) Die momentane Konstante wäre so ein
unaufhaltbarer Einzelfall den ich einflechten kann, eine herausragende
Singularität sozusagen. Nennen wir sie: 5,2982. Zur letzten Zählung eine
Steigerung wie erwartet, klein, wenig
\emph{(}3\emph{,}035**\emph{10}-\emph{2})\emph{, aber steigt an, nie
fällt sie ab. Darum allein sollte es schon genügen daß wir fortfahren zu
schreiben - zu lesen; und richtig zu lesen hatten wir wahrlich erst
angefangen, je vous le dis en vérite!\\
Es ist die unvermeidliche Überschrift, die jenes erscheinen läßt als
hätte es schon Anspruch. Sollten wir Daten nennen? Vielleicht wenn wir
etwas sicherer geworden sind im Umgang jener altvorderen Strukturen,
also eigentlich Ihr die es werden müßt weil ihr euch schon so weit
entfernt habt, daß Erinnerung nur n.~vermittelt möglich ist jedoch nicht
mehr in wirklichen Strukturen die über die Flüssigkristallspiegelungen
hinausgingen durch die }ich* sie mir euch vergegenwärtige. Aber immerhin
wächst mir daraus Erinnerung zu, meine. Aber eure eigene? Was macht sie,
wenn ihr eure Geräte ausgeschaltet habt und einmal wirklich allein seid?
Was bleibt übrig wenn ihr das Buch "zugeschlagen" habt? Hier ist Papier
alles Papier und kein Grund über meine Zukunft unruhig zu werden wo es
das nicht mehr geben wird jedenfalls nicht für diese Art Buch.\\
Dann wird es auch immer dringender über die Manuskripte nachzudenken.
Die optische Abbildung war schon lang abgesichert, was akustisch bis zu
euch gelangt wage ich n.~nicht auszudenken, schwanke zwischen
euphorischem Selbstlob und demütigem Erschauern vor dem eigentlich
Möglichen wenn man nun alles ausschöpfte was schon mal da war. Da ist
n.~eine große Nische in die das passen würde. Ich nannte das
Archetypenlyrik nach der Verbindung von Höhlenmalerei und der orphischen
Gelehrsamkeit. Es muß aber erwachsen werden. Dafür m. ich mich
davonscheren und dem protagonisten Auditorium seine Plätze überlassen.
Ich saß zu lange so gut (deutsche Opernloge, Karajansaal F) als daß ich
mich mit einem Hörplatz begnügen würde. (Schluß der Metaphysik:
arbeiten!)

\end{document}
