% Options for packages loaded elsewhere
\PassOptionsToPackage{unicode}{hyperref}
\PassOptionsToPackage{hyphens}{url}
%
\documentclass[
]{article}
\usepackage{amsmath,amssymb}
\usepackage{iftex}
\ifPDFTeX
  \usepackage[T1]{fontenc}
  \usepackage[utf8]{inputenc}
  \usepackage{textcomp} % provide euro and other symbols
\else % if luatex or xetex
  \usepackage{unicode-math} % this also loads fontspec
  \defaultfontfeatures{Scale=MatchLowercase}
  \defaultfontfeatures[\rmfamily]{Ligatures=TeX,Scale=1}
\fi
\usepackage{lmodern}
\ifPDFTeX\else
  % xetex/luatex font selection
\fi
% Use upquote if available, for straight quotes in verbatim environments
\IfFileExists{upquote.sty}{\usepackage{upquote}}{}
\IfFileExists{microtype.sty}{% use microtype if available
  \usepackage[]{microtype}
  \UseMicrotypeSet[protrusion]{basicmath} % disable protrusion for tt fonts
}{}
\makeatletter
\@ifundefined{KOMAClassName}{% if non-KOMA class
  \IfFileExists{parskip.sty}{%
    \usepackage{parskip}
  }{% else
    \setlength{\parindent}{0pt}
    \setlength{\parskip}{6pt plus 2pt minus 1pt}}
}{% if KOMA class
  \KOMAoptions{parskip=half}}
\makeatother
\usepackage{xcolor}
\usepackage[margin=1in]{geometry}
\usepackage{graphicx}
\makeatletter
\def\maxwidth{\ifdim\Gin@nat@width>\linewidth\linewidth\else\Gin@nat@width\fi}
\def\maxheight{\ifdim\Gin@nat@height>\textheight\textheight\else\Gin@nat@height\fi}
\makeatother
% Scale images if necessary, so that they will not overflow the page
% margins by default, and it is still possible to overwrite the defaults
% using explicit options in \includegraphics[width, height, ...]{}
\setkeys{Gin}{width=\maxwidth,height=\maxheight,keepaspectratio}
% Set default figure placement to htbp
\makeatletter
\def\fps@figure{htbp}
\makeatother
\setlength{\emergencystretch}{3em} % prevent overfull lines
\providecommand{\tightlist}{%
  \setlength{\itemsep}{0pt}\setlength{\parskip}{0pt}}
\setcounter{secnumdepth}{-\maxdimen} % remove section numbering
\ifLuaTeX
  \usepackage{selnolig}  % disable illegal ligatures
\fi
\usepackage{bookmark}
\IfFileExists{xurl.sty}{\usepackage{xurl}}{} % add URL line breaks if available
\urlstyle{same}
\hypersetup{
  hidelinks,
  pdfcreator={LaTeX via pandoc}}

\author{}
\date{\vspace{-2.5em}}

\begin{document}

\subsection{CC}\label{cc}

Der Stein wird sichtbar grün wie die Maienostsee und schillert
weißgeädert etwas durch nach oben. Sie dringen ein. Ich finde sie
heraus, habe den Filter diesmal richtig eingestellt. Erst nur die
Stimme, ist welke Brache. Zweitens: Notenfolgen, die mir recht geben: es
handelte sich um keine wirkliche, sondern um eine Phantasiesprache.
Regeln gab es fast nicht, nur: Atemanweisungen; Probleme erst, als man
das Stück in einem anderen Medium aufführen wollte, wurden aber mit der
Erfindung der Wasserorgel gelöst und ihrer Einstellung auf unser
Innenohr. Also lesen wir nach: der Auftrag der Sinfonie fand statt und
im Diagramm lassen sich \emph{jederzeit} die Entwicklungsstufen
erkennen, die ich genommen habe. Anfangen mußte ja jeder an derselben
Stelle und es waren nur die fehlenden Koordinaten aber die Systemenergie
erhielt sich. (Milchreis im Federbett.) Und was ist mir wichtig gewesen?
Zu zeigen: da, da, da fängt man an mit der Geschichte, wo Seamus seine
Sprache verlor um nicht mehr mit ihr aufzuhören, bis das
Spannungspotential zum ersten Band völlig verbraucht war und der Aufbau
rückläufig. Das ist das Zentrum gewesen und wir können jetzt ungefähr
sagen, wo es liegt. Wir wissen von deiner Kraft und ich habe immer
gemeint: behalte sie, vermehre sie. Aber es war ja dann viel wertvoller,
sie von sich zu geben und die ersten Resultate ereigneten sich langsam.
Vielleicht jedoch ist es nicht immer besser, die Wirkungen
vorausbestimmen zu wollen, die eine Geschichte nehmen soll; vielleicht
müssen wir uns auch manchmal damit begnügen, im kürzest erreichbaren
Zustand eines Gegenübers soviel auszulösen daß gerade n.~verhindert wird
uns für langweilig zu erklären. Mit diesen Worten der Warnung an den
Leser der Schrift will ich mich erkenntlich zeigen für die Geduld und
Zuversicht die er immer wartend auf Geschehnisse bereit war mir zu
opfern und werde einen Faden wieder aufnehmen, den ich vor einem halben
Buch verloren ließ in der Hoffnung es trüge sich von selbst hier her.
Hat es, habe ich. Das heißt es überlebt - und gibt mir die Schuld am
folgenden.\\
Schreiben: einfach anfangen, atmen und langsam. Was wäre sonst der Sinn
jenes Zwanges? Nicht, die angstbeseelte Vernunft endlich zu überwinden
nach freier Sicht? Dahin soll\textquotesingle s gehen, aller Anfang sei
wie immer Hölderlin. (Ich möchte mir, Benn: \emph{ein Stichwort borgen,
allein bei wem\ldots)} Es hat las ich also auch jener den Zugang gesucht
Eingang, seiner Zeit über Schiller und dessen Größe. Nur in unserem Fall
verhält es sich etwas anders: wir sind ja zwei Suchende, zwei Unbekannte
und einer schon tot (vier Jahre) und ich nur versuche jetzt uns beide
irgendwohin zu ziehen, wo wir besser gelesen würden als es deine HB
Lebzeit mir erlaubte. Sind ja auch immer lesbarer geworden, ihre die
weichen Manuskripte und meine Quartärschrift z.B. zu den Vorsokratikern
oder auch schon in der 6. oder 7. Generation; nur wie es mich
unmittelbar erreicht viertensüber HB/Gadamer/Heidegger und dann die
\emph{große Leere zwischen den Jahrhunderten bis zu meinen
Wortschatzzitaten (Benn), }soviel kann keiner dazugeben. Aber langsam
leben wir uns gegen unsere Zeiten aus (die Wache), sicher bleibt nur das
fühlbare Drängen zum Gehirnmittelpunkt oder dem was ich mir als ihn
vorstelle mit der allzu schwachen Fähigkeit zur Imagination echter
Zusammenhänge. Darum nur: schreiben - atmen und langsam, damit wir mit
lautem Hecheln nicht die Muse aufschrecken sie schläft so schön tief.
Das ist ein echter Zusammenhang der nicht schwer zu verstehen ist, das
Wort Sublimation habe ich schon irgendwo verwendet, tue es aber n.mal
zum Zwecke sichtbarer Defizite, die ihr selbst ausräumen müßt. Niemand
kann das hier lektorieren der nicht der Revolution sich auszusetzen
mutig genug ist. Die geschieht natürlich nicht wirklich, aber Lektor ist
mir auch kein wirklicher Mensch sondern höchstens die Zukunft einer
Sage, die sich um das Buch zu ranken beginnt, etwa sie hätten es erst
vom Index nehmen müssen, um es wirklich verbieten zu können. Aber wie
hatten wir das Buch indizieren lassen? Bestimmt, daß es irgendwo
zwischenbuchstaben gab, die eine andere Lesart vorschlugen als
gewöhnlich, \emph{matris lectionis,} die keiner wahrhaben will die aber
unwiderlegbar ihren Augen drohten sich vom Text zu entfernen, so oder
ähnlich muß es geklungen haben als er mich davon überzeugen wollte, daß
es sich um ein gefährliches Buch handle das man verbieten mußte. Aber
mater lectionis ist "die Mutter des lauten Lesens" und damit haben wir
schon zugegeben, daß die Gefahr einzig darin bestünde die Wahrheiten die
das Buch verkündete auszusprechen- weil das einen Kollaps hervorrufen
könnte, der sich nicht auf die Buchwelt beschränkte sondern die von ihr
eigentlich unabhängigen Medienwelten der musikalischen und
gestaltbildnerischen Literatur ebenfalls ergreifen würde. Damit wäre
aber dem Buch seinem Autor schon zuviel gedient wenn es wirklich
Revolution war sein Ziel. Wie werden wir das wissen? Das geht nur in
einer Simulation der Wirkkräfte unter den geschützten Bedingungen, die
uns die Benjaminfeldkraft in dem von der Geschichte aufgespannten Feld
(der Kathedrale) bietet. Warum also treten wir nicht ein und sehen uns
diese Kräfteeinmal genau an. Verlassen wir die Imagination. Begeben wir
uns ins Feld. Keine Sorge, wir können Hölderlin nicht vergessen, das ist
unsere Sicherheit.\\
m. Yoda: \emph{- nur, was du mit dir nimmst. -- Deine Waffen\ldots{}
nicht brauchen wirst du sie an jenem Ort.}\\
Ich stecke aber Stein, Schwert undת die Flammeniris unter den
Schildmantel in dem die Kn.en ruhen. Die Maske verbirgt Feuerholz und
vom uisce batha genau für den rettenden Schluck.\\
Doch Jean ist trotzdem nicht zurückgekehrt. Sie ging uns verloren auf
dem Weg über die Augen. Die \emph{congrégation} konnte die Spur nicht
finden und wir schließen dieses Kapitel mit der Warnung an den Leser er
möge sich sicher sein über den Verbleib seiner Emotionen denn sie wären
das einzige, was ihn am Leben erhalten würde wenn sein Mensch gegangen
ist. Ich zähle die Bewegungen der Korpuskeln die von der Mattscheibe
abprallen zu den eher seltenen Phänomenen meiner Existenz, darum
verzeichne ich sie hier so genau. Was ihr damit anfangt, bleibt bei
euch. Die geflohene aber, die Zeit, lasse ich. Denn das ist Lebenszeit
und die gibt es nur einmal. Sie sagt etwas aus über die Tiefe eures
Schlafes: die Unterbrechung, mit der ihr haushalten müßt als einem
knappen Mittel zur Erkenntnis dessen was euch umgibt. Und da wird sie
schon zu Tagträumen, die uns nur kurz herauszureißen vermögen aus der
aufkeimenden Bedrängnis über unseren Zustand. Aber das ist immer nur
morgens. Morgen sind andere Geschichten, die des nächsten Tages - weil
morgen immer der nächste Tag gewesen ist.\\
Und auch die Nacht ist nah, auch die Nacht ist immer jung und öffnet
sich an jedem Sabbath in Unendliches. Nah sind die Geschichten der
Väter, vielleicht wird man n.~mehr lernen daraus als wir jetzt denken
können. Deshalb sollen wir die \emph{Reise }antreten als hätten wir
n.~nie und wären dieselben Kinder wie zum ersten Mal als wir den Pfad
beschritten oder \emph{das Feld.} Dahin erinnere ich mich jetzt. In die
Zwischenbuchstaben der frühesten Versuche. Also gehen Sie\ldots{} gehen
sie voran, H.; vielleicht werde ich Ihnen nachfolgen nachdem die Toten
bestattet sind. Durch den entstehenden Raum ist etwas geflossen das mehr
war als ein Wille sich zu äußern. Ich kann seine Richtung nicht deuten,
n.~nicht, aber der Fluß findet statt, das merken auch Sie, oder? Es ist
Zeit, den nächsten der Namen zu nennen, die wir empfangen. Nur ein
unbedeutender Reisender glaube ich zwischen den Polen; stellt aber im
Moment jetzt die einzige Verbindung dar zu Ihnen. Das ist Anto"ne.
Retten wir sie. Einmal.

\end{document}
