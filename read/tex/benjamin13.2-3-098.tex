% Options for packages loaded elsewhere
\PassOptionsToPackage{unicode}{hyperref}
\PassOptionsToPackage{hyphens}{url}
%
\documentclass[
]{article}
\usepackage{amsmath,amssymb}
\usepackage{iftex}
\ifPDFTeX
  \usepackage[T1]{fontenc}
  \usepackage[utf8]{inputenc}
  \usepackage{textcomp} % provide euro and other symbols
\else % if luatex or xetex
  \usepackage{unicode-math} % this also loads fontspec
  \defaultfontfeatures{Scale=MatchLowercase}
  \defaultfontfeatures[\rmfamily]{Ligatures=TeX,Scale=1}
\fi
\usepackage{lmodern}
\ifPDFTeX\else
  % xetex/luatex font selection
\fi
% Use upquote if available, for straight quotes in verbatim environments
\IfFileExists{upquote.sty}{\usepackage{upquote}}{}
\IfFileExists{microtype.sty}{% use microtype if available
  \usepackage[]{microtype}
  \UseMicrotypeSet[protrusion]{basicmath} % disable protrusion for tt fonts
}{}
\makeatletter
\@ifundefined{KOMAClassName}{% if non-KOMA class
  \IfFileExists{parskip.sty}{%
    \usepackage{parskip}
  }{% else
    \setlength{\parindent}{0pt}
    \setlength{\parskip}{6pt plus 2pt minus 1pt}}
}{% if KOMA class
  \KOMAoptions{parskip=half}}
\makeatother
\usepackage{xcolor}
\usepackage[margin=1in]{geometry}
\usepackage{graphicx}
\makeatletter
\def\maxwidth{\ifdim\Gin@nat@width>\linewidth\linewidth\else\Gin@nat@width\fi}
\def\maxheight{\ifdim\Gin@nat@height>\textheight\textheight\else\Gin@nat@height\fi}
\makeatother
% Scale images if necessary, so that they will not overflow the page
% margins by default, and it is still possible to overwrite the defaults
% using explicit options in \includegraphics[width, height, ...]{}
\setkeys{Gin}{width=\maxwidth,height=\maxheight,keepaspectratio}
% Set default figure placement to htbp
\makeatletter
\def\fps@figure{htbp}
\makeatother
\setlength{\emergencystretch}{3em} % prevent overfull lines
\providecommand{\tightlist}{%
  \setlength{\itemsep}{0pt}\setlength{\parskip}{0pt}}
\setcounter{secnumdepth}{-\maxdimen} % remove section numbering
\ifLuaTeX
  \usepackage{selnolig}  % disable illegal ligatures
\fi
\usepackage{bookmark}
\IfFileExists{xurl.sty}{\usepackage{xurl}}{} % add URL line breaks if available
\urlstyle{same}
\hypersetup{
  hidelinks,
  pdfcreator={LaTeX via pandoc}}

\author{}
\date{\vspace{-2.5em}}

\begin{document}

\subsection{- C -}\label{c--}

So trat es auf, daß einer der (troubadores) im Kampf mit der Stimme
zufällig einen Laut hervorbrachte, der heute nicht mehr reproduziert
werden kann. Wir haben nicht die Mittel die Aufzeichnungen (die ersten,
die wir über das T.tr.gr.ton überhaupt fanden) dahin zu interpretieren,
was ihre eigentliche Verlautbarung gewesen sein könnte. Jeder z.B., der
schon einmal versucht hat einen Tierlaut nachzumachen wird zum selben
Ende geführt: eine Deutung ist nur in unseren eigenen Kategorien
möglich, den menschlichen, zeitlich bedingten, den persönlichen. Was wir
hören, wenn wir meinen die troubadores sprächen zu uns ist nicht ihre
Stimme, ist nicht Wesen oder Charakter den sie vermitteln. Es bleibt
unser ganz von uns selbst gefärbtes Klangbild der Notation. Woher nehmen
wir aber seine Stimme, wenn sie nicht uns eingegeben wurde? Wie klingen
diese Gedanken in deinen Ohren??? Zum Beispiel Hölderlin, Andenken:
\emph{Nicht ist es gut, seellos von sterblichen Gedanken zu sein. Doch
gut ist ein Gespräch und zu sagen des Herzens Meinung, zu hören viel von
Tagen der Lieb, und Taten, welche geschehen}. Das ist Archetypenlyrik
meine ich, die uns angeht wo immer wir auf sie treffen. Die
Vergangenheit (der Klassiker) bietet fast unendlich viel solchen
Stimmaterials an welchem wir das Lesen, nur das Lesen! schulen können.
Aber gehen wir mit jenem Potential dann daran, die eigene Stimme zu
erlernen, so erlebt mancher wohl eine Enttäuschung wenn er meinte etwa
so auch schreiben zu können wie dieser und jener der \emph{Vorgängigen.}
Dann bleibt er mit einem mal bei den Lebenden hängen, liest sich ein und
gewinnt vielleicht auch von dort Zuwachs. Nicht von allem was der
Spiegel einem vorschreibt aber von jenen zumindest, die auch ohne
Spiegel Klassik machen können. (Spiegel an der Wand.) Die sind es dann,
denen man zuschauen muß, auch wenn sie zuweilen halbe Jahrhunderter
älter sind als man selbst, zu lernen gilt es von ihnen; möglicherweise
auch gerade weil sie diesen Vorsprung schon haben, den ich mir erst
mühsam zusammenerinnern muß aus aller Leute Munde (die n.~erzählen
wollen.) Aber das hat ja auch den Namen (Erinnerungskultur), den ich für
mich nicht beanspruchen möchte. Es gibt nur die persönliche Erinnerung
die wichtig sein kann, jede andere ist bei jedem anderen immer die
gleiche und läßt sich umreißen in wenigen pol. korrekten Worten. Aber
das sind nicht meine, die ich hier zu verwenden gedenke und bleibe also
(Fahnenneid war ja schon) für die Gedenkpolitik - erstmal eine
angefochtene Seelenlandschaft mit der kein Krieg zu gewinnen ist. So
stand es in meiner Verweigerung und daß man dazu 2,20 irgendwo
rumzuliegen haben mußte den Brief abzuschicken im Überformat spielte in
meiner Zeit keine große Rolle denn wir hatten ja was jeder nur anders
als obsolet bezeichnen kann der es nicht selbst erlebt hat 1989 die
Systeme gewechselt. Zum Glück werde ich heute sagen, aber damals waren
die Jugendjahre andere als heute es sie sind, ohne eine Reaktion
hervorrufen zu wollen. Unsere Religion jedenfalls hat mich von da an
wenig gekümmert weil ja das System ein anderes war plötzlich und
\emph{politisch interessanter gestaltet} als das herkömmliche. N. immer
waren nicht die Aufgaben leichter zu lösen als vorher, aber die
Mathematik hatte sie befreit aus den Sachzwängen die im Dialog steckten.
(Ich jedenfalls liebte Bloch sehr\ldots(aimais) und das ging weiter als
es die alten Tonbandaufnahmen einer leipziger Vorlesung irgendwem
bezeugen dürften.) Wir sind ganz bestimmt hierher zurückgekehrt zu den
Wendepunkten der Geschichtsauffassung um aus jedem Satz den Doppelsinn
so weit wie möglich zu tilgen, und ob es gelungen ist wird man erst im
Lichte neuer Berechnung sehen. Bis dahin ist aber n.~Zeit, die die
heutige Kindheit ersteinmal verleben will. Es wird n.~genug gerechnet
werden\ldots{} Nur einen dringenden Vorausgriff will ich mir gestatten
und ihn als Frage formulieren, von deren ehrlicher Beantwortung sehr
viel für den Fortgang dieser Geschichte abhängen kann (wenn die Antwort,
die ich erhoffe, zutrifft): Was, wenn Ihr ein Kathole wäret würdet Ihr
tun? Sie haben einmal gesagt, das Überleben der Menschheit hinge schon
jetzt davon ab welchem Glauben sie in der Zunkunft nachgehen würde. Und
mit Zukunft meinten sie sicher nicht das Nachleben, oder? - Wir haben
uns dagegen entschieden, unfruchtbaren Mitgliedern die gleichen Rechte
zuzugestehen wie den anderen. -\\
\emph{Das war bestimmt die ehrliche Antwort, mit dem Sinn der
ursprünglichen Fragestellung hat sie aber nicht viel zu tun. Jene sollte
sich richten auf die gegenwärtigen Konflikte die wir austragen weil wir
uns religiös an diametral entgegengesetzen Punkten einer an sich
einheitlichen Werteskala befinden, deren Gemeinsamkeiten aber nur von
denen durchschaut werden, die sich schon nicht mehr in der
Kirche/Institution befinden sondern nur }n.* in der Religion.* fr. Th.
Guhl, COSTA Berlin

\end{document}
