% Options for packages loaded elsewhere
\PassOptionsToPackage{unicode}{hyperref}
\PassOptionsToPackage{hyphens}{url}
%
\documentclass[
]{article}
\usepackage{amsmath,amssymb}
\usepackage{iftex}
\ifPDFTeX
  \usepackage[T1]{fontenc}
  \usepackage[utf8]{inputenc}
  \usepackage{textcomp} % provide euro and other symbols
\else % if luatex or xetex
  \usepackage{unicode-math} % this also loads fontspec
  \defaultfontfeatures{Scale=MatchLowercase}
  \defaultfontfeatures[\rmfamily]{Ligatures=TeX,Scale=1}
\fi
\usepackage{lmodern}
\ifPDFTeX\else
  % xetex/luatex font selection
\fi
% Use upquote if available, for straight quotes in verbatim environments
\IfFileExists{upquote.sty}{\usepackage{upquote}}{}
\IfFileExists{microtype.sty}{% use microtype if available
  \usepackage[]{microtype}
  \UseMicrotypeSet[protrusion]{basicmath} % disable protrusion for tt fonts
}{}
\makeatletter
\@ifundefined{KOMAClassName}{% if non-KOMA class
  \IfFileExists{parskip.sty}{%
    \usepackage{parskip}
  }{% else
    \setlength{\parindent}{0pt}
    \setlength{\parskip}{6pt plus 2pt minus 1pt}}
}{% if KOMA class
  \KOMAoptions{parskip=half}}
\makeatother
\usepackage{xcolor}
\usepackage[margin=1in]{geometry}
\usepackage{graphicx}
\makeatletter
\def\maxwidth{\ifdim\Gin@nat@width>\linewidth\linewidth\else\Gin@nat@width\fi}
\def\maxheight{\ifdim\Gin@nat@height>\textheight\textheight\else\Gin@nat@height\fi}
\makeatother
% Scale images if necessary, so that they will not overflow the page
% margins by default, and it is still possible to overwrite the defaults
% using explicit options in \includegraphics[width, height, ...]{}
\setkeys{Gin}{width=\maxwidth,height=\maxheight,keepaspectratio}
% Set default figure placement to htbp
\makeatletter
\def\fps@figure{htbp}
\makeatother
\setlength{\emergencystretch}{3em} % prevent overfull lines
\providecommand{\tightlist}{%
  \setlength{\itemsep}{0pt}\setlength{\parskip}{0pt}}
\setcounter{secnumdepth}{-\maxdimen} % remove section numbering
\ifLuaTeX
  \usepackage{selnolig}  % disable illegal ligatures
\fi
\usepackage{bookmark}
\IfFileExists{xurl.sty}{\usepackage{xurl}}{} % add URL line breaks if available
\urlstyle{same}
\hypersetup{
  hidelinks,
  pdfcreator={LaTeX via pandoc}}

\author{}
\date{\vspace{-2.5em}}

\begin{document}

\subsection{2. Anto''ne}\label{antone}

\begin{itemize}
\tightlist
\item
  wir sind uns nicht einig darüber, es deutet aber alles darauf hin wenn
  du mich fragst.\\
\item
  es war an der Zeit, einfach es war zeit dafür und das wußte er. Die
  unendlichen Gewitter haben aufgehört und der Schneefall begonnen. Es
  war die richtige Entscheidung, herzukommen.\\
\item
  von uns wurde er damals nur der Alte genannt; Jean nannte ihn
  Smithson. Sein Name ist nicht wichtig aber wenn du ihn brauchst: für
  dich wird er Anto"ne heißen. Wie du das aussprichst ist deine Sache.\\
\item
  es gab schon mal so jemanden glaube ich, das war n.~vor deiner Zeit.
  Ich habe jedoch vergessen, wo der hin ist.\\
\item
  ich kann dir helfen: es war Anton, der dich daran erinnert, wie wir
  hier saßen dicht aufeinander weinende Protagonisten mit nichts als
  unserer schäbischen Bezeichnung, keine Kleider, keine Eigenschaften
  keine Bedeutungen die über das Symbol hinausgingen. Wir sind sechs
  oder sieben gewesen und wurden immer nur die Anderen genannt. Bis wir
  uns trennen konnten von der Geschichte, dann waren wir eigene Figuren.
  Es gab aber nichts mehr was uns hielt, und so begannen wir erneut in
  den Schriften nach uns zu suchen damit wir Zeugnisse hatten unserer
  Existenz. Damit brachten wir die nächsten Geschichten ins rollen und
  immer so weiter bis heute, wo es nun die erste Gelegenheit gibt mit
  ihm, dem Schöpfer, in Kontakt zu treten. Du wirst ihn zu dir bitten,
  wenn ich dir das Zeichen gebe seines Namens.\\

  Fangen wir also mit den Vorbereitungen an. Wir haben die Wörter
  gezählt und die Zeichen und den Quotienten gebildet nach Abzug der
  Leerstellen. Was kam als nächstes?

  die Pfeiler. Jeder bekam nun eine geheime Bezeichnung in welcher sich
  ablesen ließe, wie er sich zum Fundament verhält. Oft ist sie nicht
  mehr zurückzuverfolgen, da müssen wir dann ausweichen auf eine
  zufällig vom Schriftprozessor vergebene Adresse, die die tatsächlichen
  Eigenschaften vertritt.

  Haben wir sie ermittelt, können wir nun versuchen, die Sprache
  festzulegen. Sollte es deutsch sein?

  vielleicht ist es besser, die Muttersprache anzunehmen für dieses
  System. Die wäre irgendwo \emph{zwischen} deutsch/französisch/mhd,
  aber die Motivation liegt auf dem Deutschen.

  Damit ist der Interpreter eingerichtet und folgt der Standardschrift.
  Wie lautet ihre Bezeichnung?

  Line Printer\_IBM. Aber das läßt sich jederzeit n.~ändern, es ist nur
  für uns hier wichtig, weil wir ja etwas ablesen müssen. Wie es aufs
  Papier übertragen wird soll dich nicht weiter kümmern, in jedem Fall
  wird aus der Rasterschriftart eine relationale werden und die derzeit
  farbigen Formatierungen in absolute umgesetzt.

  Was ist mit den dateirelevanten Angaben zu Beginn der Schrift?

  sie werden ebenfalls aus dem Manuskript getilgt und ausgeführt. Sie
  stehen nur in prima, für die anderen wird es sein als hätte es sie nie
  gegeben. Entspricht das deinen Vorstellungen?

  Ja, nur, daß ja einiges davon wichtig wäre an den Leser zu übergeben
  um sich in der Hierarchie zurechtzufinden. Können wir einen Kompromiß
  machen und eine Umschrift der Dateianweisungen in lesbarer Sprache
  dort einfügen, wo Erklärungen benötigt werden weil man sonst vom
  Interpreter fehlgeleitet würde?

  Viele Male schon war ich mit meinen Worten zu ihm gelangt (HB) und
  wußte nicht mehr mir sein dazu Schweigen anders zu erklären als daß
  ich ihn nicht erreichte. Etwas sagte aber mir ganz sicher, daß er mich
  vernahm und selbst er wäre dort drüben taub für jegliche Dichtung
  geworden so waren wir ja einmal so übereingekommen daß die Dichtung
  Wahrheit sei und auch wenn er für alle anderen Wahrheiten
  zugeschlossen worden wäre, meine müßte ihn immer n.~ankommen - denn
  ich vernahm ja seine ebenso: die längst aus dem Leben gezogenen
  Schriften. Und vielleicht nur, weil wir dieses beide wußten konnte
  überhaupt in der Vergangenheit das entstehn, was schließlich seine
  Entdeckung wurde; die er mir nie zugab, von der mir nie auch nur eine
  Idee kam bevor ich nach seinem Tod das Archiv übereignet bekam. Ich
  weiß n.~sehr genau wann ich zum ersten Mal mit der Theorie in
  Berührung kam. Es war eine schon sehr dunkle Spätsommernacht am
  Uurainen in Finnland wohin ich die Jahre über Berlin zu dieser Zeit
  regelmäßig verließ und auch dann wären die Fische geköpft, die Pilze
  geputzt und das Wasser getrunken worden drei Monate lang, und es wurde
  allg. dunkler und man sah also Sterne. Sitzend da und hinausroch auf
  den See vor der Hütte und ich werde nicht mehr müde weil sein Wasser
  einen wachbleiben läßt gegen alle körperlichen Stimmungen höre ich
  eine Cellomusik auf dem Kopfhörer (Atterberg) vielleicht nur weil
  damit gerade er wachgerufen ist, der das Cello so liebt und ich an ihn
  denken muß der in Berlin krank ist; denke jedenfalls über einen
  kryptischen Satz nach den er mir später vonjemandem hinterlassen wird
  und den ich über seine ersten Schriften setzen werde: Was wir haben
  haben wir nicht aus uns, was wir sind, sind wir nicht aus eigenem;
  wenn wir dies, nur dies endlich lernen würden. Da tritt ein Funke in
  mich ein den ich irgendwie erahnte als ich in die Sterne hinaufschaute
  aber ich habe ihn trotzdem nicht kommen sehn. Aber der Satz, der sich
  da plötzlich aufbaute gegen diesen ungeheuren Himmel, gegen diesen
  Weltraum, der machte ihn ganz klein. Denn ich stand auf und konnte
  mich erheben in ihn und ca. denken: da ist ja nichts über mir, das
  mich vor ihm schützt, das mich von ihm abhält, das mich von ihm
  zurückweisen kann. Ich stehe ja mitten darin mit meinen 1,78m über dem
  Erdboden und nur was mich davor schützt in ihn hinauszutreiben, ist
  meine innerste treibende Kraft die in jeder Zelle so tätig sei wie in
  jenem Raum der sich da an meinem Gesicht schon mir entgegenbreitet.
  Ich kann also hoch und meine Hände ausstrecken auf einen der Sterne zu
  und warte nur eine Weile dann bin ich bei ihm. Aber wie will man
  dieses denn verkraften daß da nichts ist zwischen uns, zwischen mir
  und dem Stern und zwischen mir und dem gewaltig sich ausdehnenden
  Schwarz in alle Richtungen\ldots-- Das war der Moment wo mir aus
  seinen Worten zum ersten Mal eine Ahnung von der kleinen Kraft zuteil
  wurde die unser eigen ist um mit dem was uns umgibt irgendwie
  zurechtzukommen. Er hatte ihr dann auch irgendwann einen Namen
  zugeteilt unter seinen Entdeckungen und weil es seine letzte sein
  sollte und die kleinste unter allen aber wesentlich: eine unter
  Schmerzen errungene und denn. nur die eine zurecht genannte nach ihm,
  dem großen unserem schließlich gemeinsam gewonnenen Vordenker den sie
  alle jetzt ruhig erfahren können - sie würde tatsächlich auf W.B.
  recurrieren; nur, daß \emph{ich} es jetzt selbst n.~gar nicht wußte,
  weil ich sie n.~nicht zum Ende erkannt habe. Ich werde mich dann
  weiterhin davor hüten Gemeinsamkeiten zu suchen und über B. nur
  sekundäres mir zu erfahren gestatten (Sholem) bis es vielleicht
  unvermeidlich ist doch einen Blick in sein Werk zu tun um nicht allen
  Vermutungen der anderen die mit mir dieses schreiben völlig schutzlos
  ausgeliefert zu sein, ahnungslos. Aber dieses später, denn morgen sind
  die Toten und das sind Geschichten von morgen und das ist der nächste
  Tag. Morgen war immer der nächste Tag. Das kann ein ganz schön
  stauchen.
\end{itemize}

\end{document}
