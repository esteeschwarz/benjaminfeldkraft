% Options for packages loaded elsewhere
\PassOptionsToPackage{unicode}{hyperref}
\PassOptionsToPackage{hyphens}{url}
%
\documentclass[
]{article}
\usepackage{amsmath,amssymb}
\usepackage{iftex}
\ifPDFTeX
  \usepackage[T1]{fontenc}
  \usepackage[utf8]{inputenc}
  \usepackage{textcomp} % provide euro and other symbols
\else % if luatex or xetex
  \usepackage{unicode-math} % this also loads fontspec
  \defaultfontfeatures{Scale=MatchLowercase}
  \defaultfontfeatures[\rmfamily]{Ligatures=TeX,Scale=1}
\fi
\usepackage{lmodern}
\ifPDFTeX\else
  % xetex/luatex font selection
\fi
% Use upquote if available, for straight quotes in verbatim environments
\IfFileExists{upquote.sty}{\usepackage{upquote}}{}
\IfFileExists{microtype.sty}{% use microtype if available
  \usepackage[]{microtype}
  \UseMicrotypeSet[protrusion]{basicmath} % disable protrusion for tt fonts
}{}
\makeatletter
\@ifundefined{KOMAClassName}{% if non-KOMA class
  \IfFileExists{parskip.sty}{%
    \usepackage{parskip}
  }{% else
    \setlength{\parindent}{0pt}
    \setlength{\parskip}{6pt plus 2pt minus 1pt}}
}{% if KOMA class
  \KOMAoptions{parskip=half}}
\makeatother
\usepackage{xcolor}
\usepackage[margin=1in]{geometry}
\usepackage{graphicx}
\makeatletter
\def\maxwidth{\ifdim\Gin@nat@width>\linewidth\linewidth\else\Gin@nat@width\fi}
\def\maxheight{\ifdim\Gin@nat@height>\textheight\textheight\else\Gin@nat@height\fi}
\makeatother
% Scale images if necessary, so that they will not overflow the page
% margins by default, and it is still possible to overwrite the defaults
% using explicit options in \includegraphics[width, height, ...]{}
\setkeys{Gin}{width=\maxwidth,height=\maxheight,keepaspectratio}
% Set default figure placement to htbp
\makeatletter
\def\fps@figure{htbp}
\makeatother
\setlength{\emergencystretch}{3em} % prevent overfull lines
\providecommand{\tightlist}{%
  \setlength{\itemsep}{0pt}\setlength{\parskip}{0pt}}
\setcounter{secnumdepth}{-\maxdimen} % remove section numbering
\ifLuaTeX
  \usepackage{selnolig}  % disable illegal ligatures
\fi
\usepackage{bookmark}
\IfFileExists{xurl.sty}{\usepackage{xurl}}{} % add URL line breaks if available
\urlstyle{same}
\hypersetup{
  hidelinks,
  pdfcreator={LaTeX via pandoc}}

\author{}
\date{\vspace{-2.5em}}

\begin{document}

\subsection{b.}\label{b.}

Damit haben wir den ersten Grund erreicht. Von hier führen jetzt nur
n.~unsere eigenen Hurenkinder über den bemessenen Abstand von deinem
Auge nach meinem Kopf. Was auf dieser kurzen Bahn stattfinden kann ist
eigentlich alles bisher Erzählte, als Checksumme im ersten Absatz
gespeichert und in jeder Lesung von mir aktualisiert. Wenn \emph{du}
aber am Anfang nachschaust, ist es der alte geblieben. So funktioniert
das Wesen dieses \emph{Buches} wie wir es genannt haben, \emph{bevor die
Welt sich weitergedreht hatte. Ich benutzte bewußt diese von stephen
king gelernt Wendung die hier ausdrücken soll was von ihm unbedingt
gesagt werden muß, der Generationen von Schreiberlingen jung und sich so
frei fühlenden daß sie einen ihrer Triebe in Sprache zu drängen wußten:
}Sie jedenfalls anregte, über ihr Innerstes nachzudenken als wäre es
selbst die fern zu entdeckende Welt, die sich irgendwann von ihnen
wegdrehte und bis dahin soll man sie so gut es geht kennenlernen um sie
unter den zehntausend die einem n.~bevorstünden ausmachen zu können,
wenn diese sich auch drehten. Und mir lag der eigenen Welt stephen kings
eine vertraute Heimlichkeit zugrunde, die mich ihm innig verbunden hatte
sobald ich etwas sah das mehr gewesen ist als seine Schrift: es war die
erste \emph{Kursive} und bezeichnete ein Lied was ich damals n.~nicht
wußte von den Beatles: \emph{hey jude. }Bis ich jedoch auch dieses Lied
auf einer alten Hebräerplatte von Chava Alberstein hörte als \emph{Hey
Ruth} glaubte ich (später) wäre alsin einer bösen Anwandlung
hingeschrieben worden von ihm. Es taucht in allen seinen Büchern oft
auf, irgendwo kursive aber an sicher verborgenen Stellen vor dem
querlesenden Auge. Und so soll es auch hier seinen Platz gefunden haben,
hat\textquotesingle s ja schon im letzten Band aber hier erinnert etwas
daran: es gab scheinbar einmal einen Bruch zur Entscheidung gegen
dasWesen, das sich mit einem Davidstern an der Kette geschmückt und
verwickelt in pubertären Disputen über Schwein oder nicht davon befreien
wollte ewig als Täter dazustehn und die leichteste Übung vollzog sich
selbst in die Opferrolle begebend. Wir hatten ich und der Zweite dieses
Kapitels: Anto"ne aber nicht mit der Geistesgegenwart des Jungen
gerechnet und der hatte plötzlich die Waffe des Alten in der Hand und
richtete sie auf mich. Ich wußte mir nicht anders zu helfen als in
dessen leblosem Körper Zuflucht zu suchen und so haben wir
glücklicherweise das Loch in der Stirn des ersten magischen
Protagonisten überlebt und konnten uns weiterentwickeln. Daß das Buch
\emph{Poseidon} eine Wendung einläutete haben nur Jean, der Junge und
Ich bemerkt und verbargen es klug vor den Anderen.\\
Damit erfüllten wir das Zeugnis des Ersten Bandes das sich bis jetzt
erhalten hat. Sie können jederzeit darauf zugreifen und sich
vergewissern, daß nichts \emph{dazugetan oder weggelassen wurde} so wie
es abgesprochen war zwischen uns. Die schmale Sammlung überlassener
Fragmente fangen wir gerade an zu überblicken und sind auf unsere
Muttersprache verwiesen immer n., wo es andere besser übersetzt haben
als es uns möglich war aus dem Urtext. Nehmen wir dies zum Anlaß,
endlich die Studien voranzutreiben, die wir mit HB an der Seite
begannen: von Parmenides, der uns lange vorher begegnete, zu Heraklit,
den wir jetzt aufsuchen wollen. Gehn wir es an.

\end{document}
