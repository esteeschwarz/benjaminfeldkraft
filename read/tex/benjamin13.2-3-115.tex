% Options for packages loaded elsewhere
\PassOptionsToPackage{unicode}{hyperref}
\PassOptionsToPackage{hyphens}{url}
%
\documentclass[
]{article}
\usepackage{amsmath,amssymb}
\usepackage{iftex}
\ifPDFTeX
  \usepackage[T1]{fontenc}
  \usepackage[utf8]{inputenc}
  \usepackage{textcomp} % provide euro and other symbols
\else % if luatex or xetex
  \usepackage{unicode-math} % this also loads fontspec
  \defaultfontfeatures{Scale=MatchLowercase}
  \defaultfontfeatures[\rmfamily]{Ligatures=TeX,Scale=1}
\fi
\usepackage{lmodern}
\ifPDFTeX\else
  % xetex/luatex font selection
\fi
% Use upquote if available, for straight quotes in verbatim environments
\IfFileExists{upquote.sty}{\usepackage{upquote}}{}
\IfFileExists{microtype.sty}{% use microtype if available
  \usepackage[]{microtype}
  \UseMicrotypeSet[protrusion]{basicmath} % disable protrusion for tt fonts
}{}
\makeatletter
\@ifundefined{KOMAClassName}{% if non-KOMA class
  \IfFileExists{parskip.sty}{%
    \usepackage{parskip}
  }{% else
    \setlength{\parindent}{0pt}
    \setlength{\parskip}{6pt plus 2pt minus 1pt}}
}{% if KOMA class
  \KOMAoptions{parskip=half}}
\makeatother
\usepackage{xcolor}
\usepackage[margin=1in]{geometry}
\usepackage{graphicx}
\makeatletter
\def\maxwidth{\ifdim\Gin@nat@width>\linewidth\linewidth\else\Gin@nat@width\fi}
\def\maxheight{\ifdim\Gin@nat@height>\textheight\textheight\else\Gin@nat@height\fi}
\makeatother
% Scale images if necessary, so that they will not overflow the page
% margins by default, and it is still possible to overwrite the defaults
% using explicit options in \includegraphics[width, height, ...]{}
\setkeys{Gin}{width=\maxwidth,height=\maxheight,keepaspectratio}
% Set default figure placement to htbp
\makeatletter
\def\fps@figure{htbp}
\makeatother
\setlength{\emergencystretch}{3em} % prevent overfull lines
\providecommand{\tightlist}{%
  \setlength{\itemsep}{0pt}\setlength{\parskip}{0pt}}
\setcounter{secnumdepth}{-\maxdimen} % remove section numbering
\ifLuaTeX
  \usepackage{selnolig}  % disable illegal ligatures
\fi
\usepackage{bookmark}
\IfFileExists{xurl.sty}{\usepackage{xurl}}{} % add URL line breaks if available
\urlstyle{same}
\hypersetup{
  hidelinks,
  pdfcreator={LaTeX via pandoc}}

\author{}
\date{\vspace{-2.5em}}

\begin{document}

\subsection{VII. Von Johannes
Barrabas}\label{vii.-von-johannes-barrabas}

Nachdem die Fische geköpft, die Tauben ausgenommen und die Schafe
gehäutet waren, sind wir einen weitern Weg gemeinsam gegangen. Der
sollte uns eigentlich über Aquino zusammenführen, brachte uns aber doch
nur an die alte Grenze. Das war hier: lautes Unbewußtes und glich der
Mauer, die mich einmal vor der Welt weggesperrt hatte (Berlinrand) als
die Seherängste zu groß und bedrohlich geworden waren. Jedoch standen
wir diesmal auf ihrer anderen Seite und waren nicht mehr innen vor der
Welt geschützt sondern diesmal war sie es, die in uns gefangen war und
wir selbst ihr außen, vor dem wir sie beschützen mußten und deshalb die
Mauer hochgezogen hatten, waren also ihr Überschuß und das Mehrsein, das
sie gebar wenn wir sie zwangen, waren ihr "Kind": herrisch und sehr,
sehr launisch.\\
Nachdem die Fische gefüttert, die Tauben erzogen und die Haare geschoren
waren begaben wir uns nahezu gemeinsam auf den nächsten Weg, eine
weitere Etappe auf dem Weg zu den Schwestern von Santiago, die man
vielleicht nie sehen würde, das war auch nicht wichtig außer zu ihnen
unterwegs zu sein egal wohin. Jakob hat das Ziel an jeden Ort
gesteckt.\\
Wir sollten zusammengeführt werden zu th., doch keiner konnte jene
Grenze diesmal überwinden, die der aquinate vor den Unzeiten setzte. Es
war eines, die Geliebte immer nur zu singen als sie wirklich jede Nacht
zu beherbergen und ihr Eingang und Ausgang zu sein dieser Welt. Sie
wußte von mir ja nur durch die Schrift: die Braut, ewig junge,
gerechte\ldots{} die niemand verstünde wenn sie es einmal erzählen wird.
Aber so haben wir uns kennengelernt und wenn sie nicht wirklich Mignon
geheißen hätte, wäre sie nie meine Begleiterin geworden. Daß sie es aber
wurde ist verbürgt und nur ihr, weil ihr nichts davon gelesen habt (weil
niemand den Mut hatte es zu verzeichnen) könnt ihre Geschichte
umschreiben, neu schreiben wenn sie sich diesmal euch entfaltet, nur
ihr\ldots{} \emph{mit eurer ewigen Jugend. }Nachdem also: die Stiere
gejagt, die Widder geschlachtet und die Fische geräuchert waren,
entschuppt auf dem Stein im Uurainenfinnland, der daraus den
eigentlichen Akt erst machte dem ich nachzudenken pflege immer wenn das
Labyrinthwasser mich ertauben läßt (taub gegen die Empfindungen der
eigentlichen seiner \emph{Seele}.) Dann nur dann ergaben sich wirklich
Möglichkeiten. Und möglich ist manches geworden seitdem. Die Fische sind
ebenso beendet mit den anderen Gezeiten vorher, man sagte nur n.~nicht
good bye \& thanks\ldots{} sondern half sich wo man konnte im jeweilgen
Element dem Gegenüber. Lange sprachen wir nicht miteinander aber haben
das Wort n.~nicht verloren merken wir weil wir uns n.~hören können -
weil wir lesen, in uns, über uns, gegen die Sehnsucht gegen jedes
Verlangen das dich mir entfernen könnte. Darum bleiben auch Worte übrig
wenn das hier beendet ist, die \emph{mir} die Erinnerung ermöglichen
werden. Aber wirst du dich auch erinnern? Schlag es zu das Buch, so
lange es n.~nicht geschrieben ist und verrate all seine Bedingungen.
Dann wirst du gewonnen haben über manchen Zwang und vielleicht bist du
gerettet, kannst die Arbeitleisten die jener dir verhindert, kannst alle
Orte besuchen von denen es handelt und finden, wie wahr es doch gewesen
ist nur daß du nicht daran glaubtest. Doch kommt das Wissen zu spät und
deine Einsichten werden mich nicht mehr erreichen, ich bin ein zu
kleiner Teil deiner Zukunft als daß ich selbst Wirkung hätte. Es wird
nur diese Schrift sein und was du an ihr doch lerntest: von Parmenides,
Heraklit, den kabb. Studien, zur Benjaminfeldkraft, der ir.-kath.
Gemeinschaft und ihren ältesten Gebäuden (immer ein Lichthof um die
Mitte\ldots) und weiteres n.~über jeden Stein, den ich zertrümmer. Aber
von jenen sollte geschwiegen werden, denn was solltest du dir daraus
schon weben? In deiner Sprache hieß es, daß die Toten sich selbst
begraben sollten und wir hätten dem zu folgen der ein G der Lebenden war
weil wir auch leben (sollen\ldots) Jedoch: die großen Fehler waren immer
n.~die der anderen, die sie machten, weil sie nicht weit genug
vorausgingen in ihren Deutungen. Verließen sich also darauf, daß ich sie
interpretieren würde und die Arbeit leiste, metaphysische Denkarbeit zu
ihrem Ruhm. Aber nur weniges ist so groß geschrieben, daß meine Energie
darin gut angelegt ist. \emph{Und so verhalt ich mich denn,} verschlucke
was zu ihnen n.~zu sagen wäre, lasse meine immerwährenden Gedanken um
den Prozeß kreisen und die Anteile sinken, die den Auftrieb hemmen. Wie
lange man damit beschäftigt war, also wirklich beschäftigt wie man die
Mittel herstellen und erhalten könnte die zur Erschaffung jener
\emph{homunculi} nötig waren und jetzt hätte ich sie einfach im Zufall
entdeckt: einer der Momente als ich nicht ganz anwesend war enthüllte es
mir: alle Mittel stehen bereit zur Mischung, ich sollte nur n.~einen
Spruch mir ausdenken, der sie vereinte. \emph{Aber zuerst mußte ich mich
in die Farben begeben. }Dann lernte ich sie kennen.

\end{document}
