% Options for packages loaded elsewhere
\PassOptionsToPackage{unicode}{hyperref}
\PassOptionsToPackage{hyphens}{url}
%
\documentclass[
]{article}
\usepackage{amsmath,amssymb}
\usepackage{iftex}
\ifPDFTeX
  \usepackage[T1]{fontenc}
  \usepackage[utf8]{inputenc}
  \usepackage{textcomp} % provide euro and other symbols
\else % if luatex or xetex
  \usepackage{unicode-math} % this also loads fontspec
  \defaultfontfeatures{Scale=MatchLowercase}
  \defaultfontfeatures[\rmfamily]{Ligatures=TeX,Scale=1}
\fi
\usepackage{lmodern}
\ifPDFTeX\else
  % xetex/luatex font selection
\fi
% Use upquote if available, for straight quotes in verbatim environments
\IfFileExists{upquote.sty}{\usepackage{upquote}}{}
\IfFileExists{microtype.sty}{% use microtype if available
  \usepackage[]{microtype}
  \UseMicrotypeSet[protrusion]{basicmath} % disable protrusion for tt fonts
}{}
\makeatletter
\@ifundefined{KOMAClassName}{% if non-KOMA class
  \IfFileExists{parskip.sty}{%
    \usepackage{parskip}
  }{% else
    \setlength{\parindent}{0pt}
    \setlength{\parskip}{6pt plus 2pt minus 1pt}}
}{% if KOMA class
  \KOMAoptions{parskip=half}}
\makeatother
\usepackage{xcolor}
\usepackage[margin=1in]{geometry}
\usepackage{graphicx}
\makeatletter
\def\maxwidth{\ifdim\Gin@nat@width>\linewidth\linewidth\else\Gin@nat@width\fi}
\def\maxheight{\ifdim\Gin@nat@height>\textheight\textheight\else\Gin@nat@height\fi}
\makeatother
% Scale images if necessary, so that they will not overflow the page
% margins by default, and it is still possible to overwrite the defaults
% using explicit options in \includegraphics[width, height, ...]{}
\setkeys{Gin}{width=\maxwidth,height=\maxheight,keepaspectratio}
% Set default figure placement to htbp
\makeatletter
\def\fps@figure{htbp}
\makeatother
\setlength{\emergencystretch}{3em} % prevent overfull lines
\providecommand{\tightlist}{%
  \setlength{\itemsep}{0pt}\setlength{\parskip}{0pt}}
\setcounter{secnumdepth}{-\maxdimen} % remove section numbering
\ifLuaTeX
  \usepackage{selnolig}  % disable illegal ligatures
\fi
\usepackage{bookmark}
\IfFileExists{xurl.sty}{\usepackage{xurl}}{} % add URL line breaks if available
\urlstyle{same}
\hypersetup{
  hidelinks,
  pdfcreator={LaTeX via pandoc}}

\author{}
\date{\vspace{-2.5em}}

\begin{document}

\subsection{T - XIII.}\label{t---xiii.}

Warum \emph{ich} diese dann benutze, obwohl mir eigentlich das Recht
dazu nicht gehört? Der sie einfügte oder besser ausließ hat sich einmal
vielleicht auch darüber geäußert; wie mit denen umzugehen sei die sich
zwar bekannten (Schuld) und nicht beklagten (Vergebung) aber trotzdem
offene Rechnungen dem Volk gegenüber nicht beglichen (Trägheit, Gier,
Neid etc.). Ich habe Kommentare gefunden die das belegen und auch wenn
die Schriften nicht alles sein müssen (sola scriptura) was an
Überlieferung auf uns gekommen ist; wenn wir also weiterhin auch dem
Archetypenmaterial eine Berechtigung zugestehen - dann sollten wir einen
Aspekt in Betracht ziehen unserer Konstitution: wir wären nicht hier
wenn wir nichts davon hätten, sei es auch nur die Möglichkeit sich
darüber zu beschweren. Ich habe viele Grabsteine gesehen und auch viel
kaputte, weil niemand eine Erinnerung interessiert war zu bewahren.
Einer, der schön ist trotz dem kantig ins Gesicht eingravierten Spruch,
der irgendwann wohl Konjunktur hatte hält scheinbar, was er verspricht
indem er den Stein sein läßt: "Wer im Gedenken seiner Lieben bleibt ist
nicht tot, er ist nur fern. Tot ist nur, wer vergessen wird." Und, wozu
steht das hier und auf der Steinkante, selbst verlassen in der
Landschaft? Fragen Sie doch, warum man das tut, fragen Sie einmal nach
Gründen für diese so sehr gewünschte Steinexistenz. Es gibt scheinbar
nur einen: Wir lassen uns aus dem Garten nicht mehr vertreiben, so sehr
sie es wollen, die anderen. Wir haben dort unser Ewigkeitsrecht. Unsere
Lieben sind sie selbst: \emph{let thans dauthans ga filhan seinans
dauthans,}37 ich spreche nicht nur mit Ihnen. Also weiter graben bis zu
finden, das euch eben berührt und die T. mitnehmen aber wohin
eigentlich. Totes ist nicht immer die richtige Essenz, aber in diesem
Buch wird sublimiert unds geht ins Leben ein als\ldots- ja als was
e\ldots{} Wenn ich das wüßte, wären wir nicht erst hier. So müßt ihr
weiterlesen und wir kommen an das Ende bis wohin sich n.~manches geklärt
haben wird über den Zustand, je nachdem wie gut ihr aufpaßt und mich in
euerm Tag. Denn der wird anders sein nach meinen Erfahrungen, nicht
anders weil ihr neues lerntet, das habt ihr umsonst. Anders aber weil
ich selbst eure Erweiterung geworden bin. Das habt ihr meinem Leben zu
verdanken. Und wohin das führt haben wir schon einmal gesehen: da stand
einer auf und schleuderte Buch und Stift, Zettel von sich um nicht mehr
daran zu gehen weil es schon zu gespenstisch. Klappezu, affedot. War
ganz einfach. Aber wir können weiterlesen und egal wie viel
n.~aussteigen, \emph{im Aschengrund von allen Weltenfeuern sind immer
Seelen, die das Glück erneuern.} Haushofer, in Moabit, das jetzt
überfüllt ist und notiert an der Wand: habe mich bepißt vor Lachen. Dann
hat man alle anderen auch hingerichtet und 2006 am Lietzensee das
Reichskammergericht zu edlen Wohnungen "restauriert." Das kann man nicht
in zwei kurze Sätze packen, das geht nicht. Also schneiden sie, schwarze
Balken drüber, aber so, daß man die Gesichter n.~erkennt bitte schön.
Ist ja unsere Vergangenheit, oder; also die meiner Protagonisten jdfs.\\
Wie kam ich darauf? Gespenstisch\ldots{} war ein kurzer Reigen, wenn man
eine Musik fände die das ausleuchtet (danse macabre), aber natürlich
weil sie ungeduldig sind und nicht in der Lage alles auf einmal zu lesen
wie ICH (dämonischer Hall) konntes mir ja gar nicht gelingen sie bei
Laune zuhalten. Gehn sie also, das macht mir schlechte Stimmung, mit dem
Erbe unserer Väter und so. Wir haben auch einen Totenk. und der
gestattet ihnen nicht, Sie einfach auszutilgen, wie Ägypten den
Altherrschern aus allen Schriften, von allen Bildern und allen Statuen
die Gesichter zerbiss. Ob aber das gut sein wird, wenn es irgendwann
dazu kommt, daß man sich nicht erinnert an du-weißt-schon-wen? Aber bis
dahin ist hoffentlich der Geist ausgestorben, sonst wäre esnur eine
fatale Entwicklung derheutigen Menschheit.\\
Wir sind ja aber immer n.~nach der Hybridtheorie dazu in der Lage
Vorstellungen zu evozieren und tun das auch die ganze Zeit, die ich hier
mit euch verbringen wollte in meiner persönlichen
Erinnerungs\emph{arbeit}. Zumal man mir nicht gestattet öffentlich zu
werden (Ein Land schafft sich ab\ldots) in jenen offenen Bezügen auf
eine Vergangenheit, die sich eben nicht offen interpretieren lassen
sollte sondern (donde?) in unseren jetzigen Wertekanon eingebettet
hoffentlich immer als nicht fragwürdig erscheint. Wie lange dieser
Zustand aber aufrechtzuerhalten ist, wenn die letzten der sich
freiwillig dafür Interessierenden und nicht weil sie vom Schulsystem
gezwungen werden die Letzten sind ist keine Frage der Zeit sondern eine
Frage \emph{unserer, der jetzigen} Ehre. Können wir darauf bauen, daß
ein Teil unseres Gewissens immer genug tätig bleiben wird, die
Verdrängung, die Auslöschnung, die Banalisierung oder schlimmstenfalls
die Leugnung zu verhindern? Aber zu Parmenides hat HB eigentlich gute
Arbeit geleistet als es darum ging mich auf diese Lehrtätigkeit
vorzubereiten. In der letzten Zeit allerdings sind wir vom hundertsten
ins tausendste gekommen und wir müssen nun also die Glieder, die uns
irgendwohin abhanden gekommen sind zusammenführen damit ihr euch zum
Ende dieser Nacht - die erst mit dem Morgen beginnen wird - dort
wiederfindet, wo ich euch hab stehengelassen in 1/8/3/4. Damit hätten
wir das Sigel und sehn wir hätten an vergangenen Stellen für heut
aufhören können, wenn es nach mir gegangen wäre. Jemand anderes aber,
der die Führung übernehmen wollte ist stärker gewesen und vielleicht nur
deshalb ist einmal ein Absatz weniger beglückend, aber wir wissen was
gemeint war und \emph{Deine Meinung ist uns wichtig.}\\
Vielleicht habe ich Bedingungen unnötig in Finalsätzen ausklingen lassen
- dieses sei aber dem Umstand geschuldet, daß eben jene Bedingungen
zeitweise eine Endlichkeit anzunehmen schienen die es mir unmöglich
machte, sich darüberhinaus eine Fortsetzung der Gedanken vorzustellen
die den Anfang der Kette bildeten. Die Kette aber, wenn sie nicht von
selbst abreißen wollte mußte ich als \emph{Bedingung} eben in jene
finalen Sätze führen die mir erst ermöglichten, den Gedanken überhaupt
zu denken; ist das zu kompliziert?\\
Wir sind nicht immer mit den Protagonisten einig, daß sich n.~etwas
erzählen läßt das über die letzten Worte hinausging, die der alleinige
Joh. ans Ende \emph{seiner} Erzählung stellte. Für mich ist diese Frage
nicht zu klären und wenn ich zu streiten anfange mit den Gestalten (mich
also in ihr Terrain begebe) geht es schnell so daß sie mir die Worte
entgegenschleudern ohne nur scheinbar über meine Fragen nachdenken zu
müssen. Weil aber ich selbst meistens n.~nicht einmal diese richtig weiß
ist es mir ein Rätsel wie sie dazu kommen können so selbstsicher und
unabhängig. Ich vermute einen Quell sozusagen ewiger Kenntnisse von gut
und böse auf den sie dauernden Zugriff haben und nicht mehr erschrecken
vor diesem Wissen wie es für uns eigentlich normal war \emph{bevor die
Welt sich weitergedreht hatte.} Heute habe ich schon viel von dem
verloren was an Demut einmal in mir war und die Wissenschaft anging.
Leicht geht jetzt manches über die Lippen wofür man Jahre des Lernens
auf sich nahm früher. Ich bin geübt und die Sprache ist zu flüssjem
Material vor mir geworden (blackbox) das ich nur anschauen muß um es zu
formen. Wenn wir irgendwann so weit sind, daß wir über die
Elektronenröhre miteinander Kontakt aufnehmen kann es endlich geschehen:
die dauerhafte Umformung zu Gedankenmaterial ist möglich. Aber bis dann
ist n.~unbestimmte Zeit die vergangen sein will. Wir lassen uns hier
nicht vertreiben. Nicht n.~einmal.

\end{document}
