% Options for packages loaded elsewhere
\PassOptionsToPackage{unicode}{hyperref}
\PassOptionsToPackage{hyphens}{url}
%
\documentclass[
]{article}
\usepackage{amsmath,amssymb}
\usepackage{iftex}
\ifPDFTeX
  \usepackage[T1]{fontenc}
  \usepackage[utf8]{inputenc}
  \usepackage{textcomp} % provide euro and other symbols
\else % if luatex or xetex
  \usepackage{unicode-math} % this also loads fontspec
  \defaultfontfeatures{Scale=MatchLowercase}
  \defaultfontfeatures[\rmfamily]{Ligatures=TeX,Scale=1}
\fi
\usepackage{lmodern}
\ifPDFTeX\else
  % xetex/luatex font selection
\fi
% Use upquote if available, for straight quotes in verbatim environments
\IfFileExists{upquote.sty}{\usepackage{upquote}}{}
\IfFileExists{microtype.sty}{% use microtype if available
  \usepackage[]{microtype}
  \UseMicrotypeSet[protrusion]{basicmath} % disable protrusion for tt fonts
}{}
\makeatletter
\@ifundefined{KOMAClassName}{% if non-KOMA class
  \IfFileExists{parskip.sty}{%
    \usepackage{parskip}
  }{% else
    \setlength{\parindent}{0pt}
    \setlength{\parskip}{6pt plus 2pt minus 1pt}}
}{% if KOMA class
  \KOMAoptions{parskip=half}}
\makeatother
\usepackage{xcolor}
\usepackage[margin=1in]{geometry}
\usepackage{graphicx}
\makeatletter
\def\maxwidth{\ifdim\Gin@nat@width>\linewidth\linewidth\else\Gin@nat@width\fi}
\def\maxheight{\ifdim\Gin@nat@height>\textheight\textheight\else\Gin@nat@height\fi}
\makeatother
% Scale images if necessary, so that they will not overflow the page
% margins by default, and it is still possible to overwrite the defaults
% using explicit options in \includegraphics[width, height, ...]{}
\setkeys{Gin}{width=\maxwidth,height=\maxheight,keepaspectratio}
% Set default figure placement to htbp
\makeatletter
\def\fps@figure{htbp}
\makeatother
\setlength{\emergencystretch}{3em} % prevent overfull lines
\providecommand{\tightlist}{%
  \setlength{\itemsep}{0pt}\setlength{\parskip}{0pt}}
\setcounter{secnumdepth}{-\maxdimen} % remove section numbering
\ifLuaTeX
  \usepackage{selnolig}  % disable illegal ligatures
\fi
\usepackage{bookmark}
\IfFileExists{xurl.sty}{\usepackage{xurl}}{} % add URL line breaks if available
\urlstyle{same}
\hypersetup{
  hidelinks,
  pdfcreator={LaTeX via pandoc}}

\author{}
\date{\vspace{-2.5em}}

\begin{document}

\subsection{P - 6 (statt N)}\label{p---6-statt-n}

1 . Vielleicht habe ich irgendwann angefangen nebensächliches in
Finalsätzen ausklingen zu lassen, eine Technik, die mir durchaus bewußt
ist aber ebensowenig gewollt war wie das Pathos das sie ausläutet.
Auf\textquotesingle s pathetische habe ich mich lange genug gut genug
verstanden und bin mit der Zeit sogar an ihm gereift, in ihm
möglicherweise aufgewachsen und zwar derart daß natürlich eine
anstehende Wahrheit ganz anders klingt mit seiner Stimme als mit der
eigenen. Man benützt sie dann und spricht mit den großen Worten einer
fremden Sprache die Dinge aus, die man ja doch tief innen spürt, die
aber weil es im Leben selbst gar keine Möglichkeit gibt für ihre
Äußerung, nur den Weg über das Gefühl können, über das Pathos. Also ließ
ich ihm seinen Frieden und schenkte mir Entzückung über so viel
Gefundenes. Nichts anderes war ja die Ursache dieses pathetischen
Empfindens: daß man etwas großes gefunden habe, daß man zu etwas
vordrang, das einen nun zwingt es auszusprechen weil man es sonst nicht
behalten darf. Und darum ging es: der Erkenntnisse habhaft zu werden,
die sich da aufschlossen. Was wir benennen können, gehört uns. \emph{Was
wir haben, haben wir nicht aus uns, was wir sind, sind wir nicht aus
eigenem. Wenn wir dies, nur dies, endlich lernen wollten.}\\
2 .Dieser durchaus pathetische Satz sollte stehen am Ende der Schriften
des polyhistors wenn er sie hätte herausgebracht. Nun ist mir diese
Aufgabe zugefallen und bisher habe ich nicht viel mehr heben können als
ich euch auf dem Weg über das hier mitgeteilt habe. Mir werden jetzt
n.~6646 Worte bleiben für den Anfang des Prozesses, sein Abschluß ist so
weit wie es das Konvolut an hinterlassenen Blättern nur vermuten läßt:
es geht in Dekaden voran, aber das ist nicht etwa beunruhigend, sondern
weil mir ja meine Zeit wirklich unendlich erscheint wie sie mir von
einem sh. zum nächsten vergeht ist auch die Auseinandersetzung mit dem
Berg Material durch kein Limit bedroht außer meiner Lebensspanne; die
aber sozusagen erstmal ebenso unbegrenzt ist. Darüber mögen andere s.
entsetzen wie auch über die angedeutete Idee. Ich jdfs. weiß wie lange
ich leben werde und das liegt mir fern.39\\
Also gehen wir langsam voran, taste mich auf jenem Brückensteg durch die
Nacht zurück. Wie lange werden sie brauchen mich zu finden? Hat man
schon angefangen zu suchen? Vorwärts tasten und fühlen was da bewegt
wird. Der Sturm innen hört sich an als würde das Meer jetzt zwischen den
Häusern hindurchfluten. Vielleicht ist es ja so. In den Straßen liegen
die Sandsäcke vor den Türen und die Läden sind leer. Aber ich habe
glaubich alle Dinge dabei für die einsame Insel: Musik (160 Gbyte),
Schreibbuch (IBM), Voigtländer, Schweizermesser, Wettersachen, Rucksack
und Bücher (Phänomenologie d.~Geister, Hebräisch, Siegfried Unselds
Briefe an seine Autoren, die fioretti des F. v. Assisi und 1 Buch über
dicke Jesuiten sowie außerdem eine hilfsgebundene Ausgabe der Ersten
Benjaminfeldkraft.) Das reicht um die anhängigen Schriften hier zu
endigen und nach Hause (irgendwann) zu kommen mit dem Zyklus und (wenn
lebend) offen für Anfänge (die mir im Zug schon eingefallen waren; so :
Nebel stand über den Feldern meiner Heimat, als ich sie zum letzten Mal:
(gesehen hatte = hätte ich von meiner jetzigen Position aus darüber zu
urteilen, daß ich sie wiedersah.) (sah = schloß ich schon in einer mir
jetzt n.~Zukunft damit ab, sie jemals wiederzusehen.) Ich entschied mich
für die zweite Möglichkeit und begebe mich damit jeden Zwanges zur
Verantwortung für was ich weiter schreibe.) Ich bin damals nicht nach
Berlin gekommen weiß ich jetzt, wo mir die Rückkehr nach\ldots{} schon
als die Idee einer Sehnsucht klar wird, die mich dem Gewohnten
entfliehen läßt und jeder Form der Beschränkung. Nicht, daß es nötig
gewesen wäre sich aufzulehnen oder mit Dingen zu brechen, die in der
Stadt um mich ihre Kreise gezogen hatten, nein das nicht. Aber eine
Schwere ging so wie Erdenschwere mit ihnen einher, die mir Trägheit
verursachte; Schwere, die meine Bücher festhielt an mir; ich würde sie
wohl nie losgeben. Es mußte also dazu kommen. So bin ich hergefahren,
und die Felder auf dem Weg sind mir nie ferner gewesen wie heute und in
den kommenden Tagen werden sie immer weiter von mir abrücken bis ich
Deutschland ganz vergessen habe. 1 schöne Vorstellung.

\end{document}
