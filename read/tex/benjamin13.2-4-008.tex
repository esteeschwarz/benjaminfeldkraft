% Options for packages loaded elsewhere
\PassOptionsToPackage{unicode}{hyperref}
\PassOptionsToPackage{hyphens}{url}
%
\documentclass[
]{article}
\usepackage{amsmath,amssymb}
\usepackage{iftex}
\ifPDFTeX
  \usepackage[T1]{fontenc}
  \usepackage[utf8]{inputenc}
  \usepackage{textcomp} % provide euro and other symbols
\else % if luatex or xetex
  \usepackage{unicode-math} % this also loads fontspec
  \defaultfontfeatures{Scale=MatchLowercase}
  \defaultfontfeatures[\rmfamily]{Ligatures=TeX,Scale=1}
\fi
\usepackage{lmodern}
\ifPDFTeX\else
  % xetex/luatex font selection
\fi
% Use upquote if available, for straight quotes in verbatim environments
\IfFileExists{upquote.sty}{\usepackage{upquote}}{}
\IfFileExists{microtype.sty}{% use microtype if available
  \usepackage[]{microtype}
  \UseMicrotypeSet[protrusion]{basicmath} % disable protrusion for tt fonts
}{}
\makeatletter
\@ifundefined{KOMAClassName}{% if non-KOMA class
  \IfFileExists{parskip.sty}{%
    \usepackage{parskip}
  }{% else
    \setlength{\parindent}{0pt}
    \setlength{\parskip}{6pt plus 2pt minus 1pt}}
}{% if KOMA class
  \KOMAoptions{parskip=half}}
\makeatother
\usepackage{xcolor}
\usepackage[margin=1in]{geometry}
\usepackage{graphicx}
\makeatletter
\def\maxwidth{\ifdim\Gin@nat@width>\linewidth\linewidth\else\Gin@nat@width\fi}
\def\maxheight{\ifdim\Gin@nat@height>\textheight\textheight\else\Gin@nat@height\fi}
\makeatother
% Scale images if necessary, so that they will not overflow the page
% margins by default, and it is still possible to overwrite the defaults
% using explicit options in \includegraphics[width, height, ...]{}
\setkeys{Gin}{width=\maxwidth,height=\maxheight,keepaspectratio}
% Set default figure placement to htbp
\makeatletter
\def\fps@figure{htbp}
\makeatother
\setlength{\emergencystretch}{3em} % prevent overfull lines
\providecommand{\tightlist}{%
  \setlength{\itemsep}{0pt}\setlength{\parskip}{0pt}}
\setcounter{secnumdepth}{-\maxdimen} % remove section numbering
\ifLuaTeX
  \usepackage{selnolig}  % disable illegal ligatures
\fi
\usepackage{bookmark}
\IfFileExists{xurl.sty}{\usepackage{xurl}}{} % add URL line breaks if available
\urlstyle{same}
\hypersetup{
  hidelinks,
  pdfcreator={LaTeX via pandoc}}

\author{}
\date{\vspace{-2.5em}}

\begin{document}

\subsection{Neuf - Des 3. Beginn}\label{neuf---des-3.-beginn}

\%\%end of \&filedescription\& as ``Band II''\\
womit der letzte Band beendet war, befand sich immer n.~auf dem Meer,
der kleinen Binnensee zwischen Deutschland und Skandinaivien, doch umso
getriebener, weil sies nicht aus sich heraus konnte, also aus dem Norden
heraus. Darum sah ich mich genauso gefangen und alles an ihr
widerspiegelte meinen Drang, vom jüngsten Wellenkräuseln bis zum
Gischtumschäumen des Bug der groß wie der letzte Wal mich hintrug auf
dem Weg von Helsinki nach Rostock; und selbst als ich endlich
abzuspringen bereit war von seinem Zwölfgeschoß über meinen Verlust,
hinderte mich ihre (der See) Abgeschlossenheit. Und Ewa ist immer
gesprungen, sagt jemand im Tonfall der Jakobneunerin, sich verwundernd
über die Wendung und jener schließlich können wir aber jetzt dankbar
sein, daß diese hier sich im Geiste der meisten bewahrt hat. Denn wo
wäre ich mit meinen Gedanken schon überall hingekommen, wenn nicht über
dem einzigen Anfang der ruhig genannt werden darf die ernsthafte Frage
nach der Arbeit gestanden hätte, die ein gewisser vielleicht der
bedeutendste Mann hier irgendallen einmal gestellt hatte, um sie ja am
Laufen zu halten.\\
Und um diesen Band zu machen kann es nur n.~darum gehn: wie lernen wir
das protagoniste Auditorium so gut kennen, daß es nicht bevor der Tag
widder nechdern geworden ist, schon vergessen hat, wozu ich es einlud:
den hermeneutischen Zirkel mit mir zu schließen oder sich seiner
Möglichkeit dazu bewußt zu werden, seiner einzigen Funktion innerhalb
der Ereignisse. Ich verspreche nicht das Absolute. Aber ich verhalte
mich still zu ihm, wie zum brütenden Vogel über dem Meermeerurmeer, daß
er sein Ei nicht verliere auf dem Weg in die Schöpfung.

\emph{1. Wenn ich das Orakel beschreiben könnte} würden die Gesichter,
die mir unterwegs begegneten hierher in die Mitte der Dinge, nicht so
unheimlich erscheinen; ich wüßte um ihre Bestimmung und ungefähr, warum
sie nur mich dazu anhalten, weiterzusehen, als ich es sonst tue, und
etwas davon wissen sie. Das wichtige\ldots{} sei aber, nicht immer in
ihnen Zukunft zu erahnen oder schicksalhafte Wendungen im nächsten
Augenblick, sondern mehr das darin Offene wahrzunehmen, grad das
Unbestimmte, das den meisten anhaftet, während sie selbst unberührt
bleiben. Jenes besitzt die strukturellen Potenzen, aus denen zukünftige
Ereignisse kristallisieren. Da ich lerne, die Gemeinsamkeiten zu
erkennen unter meinen eigenen und den Gesichtern der anderen, wenn diese
sich zu mir bewegen, finde ich den Moment ihrer Nähe heraus, der das
verbirgt, was sie nicht sind. Das schreibt die erste Konstante und
sobald eines das andere überwiegt unbestimmter Zeichen, muß man sich
darauf einlassen und darf nur n.~Katalysator sein für das was entsteht:
das Benjaminfeld, dessen Kraft wir niemals unterschätzen dürfen. Wer
sich mit ihr aufhält, lernt in ihm zu handeln; aber wer sie meidet,
kommt darin um. Das Feld kümmert sich nicht darum was wir ihm entnehmen,
die Kraft aber muß benutzt werden sonst vergeht sie einem. Dann wird die
Zurückhaltung schädlich und die verfälschten Wirkungen des Feldes werden
überall als seine Zeugen genannt. Dagegen soll man darin das ablegen,
was einem selbst als unwirklich erscheint. Unwirkliches oder auch der
Anderen Realität, die man für etwas eigenes nahm. Nur so wird
ausgedrückt, was die Bewegung hervorrief.\\
Und Euer Ziel? Sie sind vielleicht, die fr.m., über die schwindeligen
Höhen schon hinaus, und selbst, sie erinnerten sich an ein paar der
Worte, die ich ihnen\ldots{} schickte, kommen doch von hier, aus dem
B.feld immer n.~mehr geflossen; und es ist ein großer Strand an dem sie
landen; es war auch Ihr Strand. Wann sie ihn einmal verließen, wage ich
nicht zu denken, vermute aber, wenn mich mein Hirn in Ruhe läßt, die
Jahre zwischen 30 und 40, in welchen man sich n.~unter Kontrolle hatte.
Das haben Sie schon erlangt und sind also glücklich darin gewordene
Zustände beschreiben zu können, die ich mir erst n.~verdienen muß als
Historie. Das andere, an welchem sie dann manchmal teilhaben, wird nicht
mehr ausgehn glaube ich, und wenn die Gedanken schwer, das Brot trocken
und die Hühner, ach, ein Nichts, sind, so bleiben uns, wo wir dicht
dran, immer genug Leben übrig, daß wir das heutige getrost verschieben
können in die auf uns wartenden Fernen. Wo diese beginnen also wirklich
einmal beginnen könnten, danach - auch nach unserem hiesigen Dasein als
wüßten wir nicht schon immer von ihrem Auftauchen: da jedenfalls prägen
sich Spuren in das, was wir jetzt n.~Welt nennen, das vielleicht aber
bald schon Geschichte ist und dem eigenen Ausgang nachfolgt. Sowie ich
meine Erinnerungen dahin zu konzentrieren versuche, wo sie mir einmal
helfen könnten, dessen etwas zu verstehn, was schließlich aus mir wurde,
so sind auch, wenn es gelingt, diese Aufzeichnungen hier des
Benjaminfeldes mehr als fiktive Fragmente, sie nehmen weil ich sie als
Züge der allgemeinen Ansichten aus dem Leben der Anderen weiß, deren
Gesichter vollends an, werden ihr Gut, werden Orte des Geschehns:
arbeiten sich durch Sloterdijks \emph{Reich der Notwendigkeit in das der
Freiheit.}

Vielleicht aber liegt den fratres im Betteln, in Ihrem Schweigen endlich
das versteckt, das mich ja immer hinderte, meine eigenen Auslassungen
w.zuentdecken\ldots, unter den Worten des Übrigen, wenn es versuchte,
mich zu hören. Irgendeiner muß es einmal verstanden haben, was ich da
suchte, sonst wäre es weiterhin in Heften und Aktendeckeln abgeschlossen
vergraben, was ja aber nun gelesen wird.\\
Also haben Sie mir doch geholfen? Haben Sie, weiß ich; denn Ihre
Schriften schlafen nicht umsonst bei mir an dunkeln Orten fest, wo ich
nur nachts herangeh, zum Ende der Kursive.

\emph{``In order to achieving a stable and integer state of mind
improving it`s history the development of an individual desease through
greater eyes allows the protagonist. Still the main principle of}
Melancholia \emph{to be was 1 conclusion to the philosophic discurse in
former times.''} (Harry Potter, Traceback memories; in: Defense against
darks arts; Faculty Press, London 2045)

Die Phönizier dachten, daß wir nicht wissen werden, wohin uns die
unbekannten Regeln bringen wenn wir gegen sie verstoßen und ließen also
zwei Wege, das herauszufinden, offen. Erstens indem man etwas tue und es
erfährt. Der zweite, langsamere und sehr schwere und unwiderrufliche
Versuch sei, sich für ein eigenes Regelwerk zu bemühen und in seinen
Gleichnissen mit jenem der Welt diese zu erschließen. Daß wir unsere
Zeit aber nur soweit voraussehn, wie die Vorstellung erlaubt, macht es
fast unmöglich, die zu fassenden Gesetze nicht doch irgendwie auf die
Welt zu beziehn, die wir ja eigentlich mit ihnen ergründen wollten. Es
besteht jedoch immerhin ein Unterschied zwischen dem hermeneutischen
Zirkel und einer Gefangenschaft in der Tautologie. Was uns schließlich
aus dieser zu jenem befreien konnte, habe ich versucht durch das Studium
der nachgelassenen Schriften des \emph{polyhistors }aufzudecken und kam
tatsächlich zu neuen Antworten. Es ist aber n.~ungewiß ob etwas davon
bestehen blieb bis zu dem, was sich jetzt heranbildet. Weil die
Modalitäten der Texterstellung deshalb grau sind und ich selbst in
irgendwelchen zukünftigen Belangen mir meine Erinnerung daran schwer
zurückrufen mußte, erhält sich nur einiges. Ich schreibe in den reinen
Bezug vielleicht, den der Eine meinte, als er fantasiert hat von unseren
Zeiten und es bleibt ein Rest unbewußte Deutungen dessen, was erzählt
werden muß. Ob man wird sagen können es hat sich übertragen bis heute
(also bis dahin\ldots), sei eine äußerst mühsam nur bestandene
Feststellung aber unser hier jedenfalls und jetzt gültigen Variablen.
Sie werden nicht über das hinausweisen, was hier steht, u. sind also nur
primär erhaltenes Gedankenmaterial zur Großschrift der
Benjaminfeldkraft, die wir uns damit die ich mir hiermit erlaubte
fortzusetzen -- gerade weil das heißt, allen bisher verwendeten
Schriftverfahren entgegen sich auf solches zu beschränken, das einen
Wahrheitsanspruch schon daraus ableitet, im ersten Absatz im ersten
Abend über den Versuch hinausgekommen zu sein, Schönheit aus dem
Vergänglichen zu sublimieren; Schönheit im Sinne mathematischer
Ästhetik, vergänglich verstanden als zeitlich und räumlich begrenzter
atomarer Zusammenhang selbst auch der für meinn Bewußtseinsstand
verantwortlichen chemischen oder physikalischen Vorgänge, die uns
n.~beschäftigen werden. Darüber hinaus:

\end{document}
