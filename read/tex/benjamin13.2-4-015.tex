% Options for packages loaded elsewhere
\PassOptionsToPackage{unicode}{hyperref}
\PassOptionsToPackage{hyphens}{url}
%
\documentclass[
]{article}
\usepackage{amsmath,amssymb}
\usepackage{iftex}
\ifPDFTeX
  \usepackage[T1]{fontenc}
  \usepackage[utf8]{inputenc}
  \usepackage{textcomp} % provide euro and other symbols
\else % if luatex or xetex
  \usepackage{unicode-math} % this also loads fontspec
  \defaultfontfeatures{Scale=MatchLowercase}
  \defaultfontfeatures[\rmfamily]{Ligatures=TeX,Scale=1}
\fi
\usepackage{lmodern}
\ifPDFTeX\else
  % xetex/luatex font selection
\fi
% Use upquote if available, for straight quotes in verbatim environments
\IfFileExists{upquote.sty}{\usepackage{upquote}}{}
\IfFileExists{microtype.sty}{% use microtype if available
  \usepackage[]{microtype}
  \UseMicrotypeSet[protrusion]{basicmath} % disable protrusion for tt fonts
}{}
\makeatletter
\@ifundefined{KOMAClassName}{% if non-KOMA class
  \IfFileExists{parskip.sty}{%
    \usepackage{parskip}
  }{% else
    \setlength{\parindent}{0pt}
    \setlength{\parskip}{6pt plus 2pt minus 1pt}}
}{% if KOMA class
  \KOMAoptions{parskip=half}}
\makeatother
\usepackage{xcolor}
\usepackage[margin=1in]{geometry}
\usepackage{graphicx}
\makeatletter
\def\maxwidth{\ifdim\Gin@nat@width>\linewidth\linewidth\else\Gin@nat@width\fi}
\def\maxheight{\ifdim\Gin@nat@height>\textheight\textheight\else\Gin@nat@height\fi}
\makeatother
% Scale images if necessary, so that they will not overflow the page
% margins by default, and it is still possible to overwrite the defaults
% using explicit options in \includegraphics[width, height, ...]{}
\setkeys{Gin}{width=\maxwidth,height=\maxheight,keepaspectratio}
% Set default figure placement to htbp
\makeatletter
\def\fps@figure{htbp}
\makeatother
\setlength{\emergencystretch}{3em} % prevent overfull lines
\providecommand{\tightlist}{%
  \setlength{\itemsep}{0pt}\setlength{\parskip}{0pt}}
\setcounter{secnumdepth}{-\maxdimen} % remove section numbering
\ifLuaTeX
  \usepackage{selnolig}  % disable illegal ligatures
\fi
\usepackage{bookmark}
\IfFileExists{xurl.sty}{\usepackage{xurl}}{} % add URL line breaks if available
\urlstyle{same}
\hypersetup{
  hidelinks,
  pdfcreator={LaTeX via pandoc}}

\author{}
\date{\vspace{-2.5em}}

\begin{document}

\subsection{Das Neue Buch}\label{das-neue-buch}

Es gab also schon eine Stoffsammlung und leergeschriebenes Beimaterial
zur Großschrift? Daß diese fortgesetzt wurde im folgenden ist eine
leicht vorausgesehne Tatsache die aus den jetzt zu nennenden Gründen
hier erscheint; das ist einmal weil der Fortschritt schon da ist um die
von hier zukünftigen 18.483 Wörter die sich in der dritten
Standardschriftprozessur ereigneten, ferner, weil wir wiedermal durch
die Übertragung von Text in den aktuellen Prozess, davon befreit sind,
uns über den Erhalt der Kursiven, die doch wesentlicher Bestandteil
geworden waren, zu sorgen und vollkommen hemmungslos sie setzen können,
wo immer uns der Zuspruch des Absoluten (des Zwillings) wichtig
erscheint. Wenn ich aber sagte \emph{Stoffsammlung,} haben wir ja den
dankbaren Umstand meiner (dem Prozessor untergeordneten) Tätigkeit als
der Herausgeber des immerhin n.~zu vollbringenden smaragdgrünen Teils
der Schrift, die heute ins fünfte Jahr geht\ldots{} stoff also\ldots,
den es für dies hier zu bedenken gibt, gewiß zu sein. Daß wir einen
Aufbau in Stufen irgendwann verlangt haben, macht es nun, wenn wir uns
n.~daran halten wollen, notwendig, die erste Variable erneut zu
bestimmen, von der wir uns weiterbewegen werden. 5.4149 ist die das
Konstrukt über uns (also vor uns) tragende nun Konstante, die wir
auslesen. Sie liegt unterhalb der Erwartung und das gibt uns die
Aufgabe, der wir ohnehin folgten: die füllenden Halbvokabeln, die euch
nur unterhalten, einzuschränken zugunsten der länger und wirksameren
Vollworte. Daß damit das Geschriebene nicht besser lesbar wird, versteht
sich von selbst und weil "dabei gut schlafen können" als Kriterium v.
Literaturen nicht gilt, werde man v.Z.z.Z. Blitze einstreun, die euch zu
mir erwecken solln. Immer wenn ihr ein schon bekanntes Motiv hört, ist
das auch so ein Blitz, der eure Wachsamkeit herstellt und euch mir
weiter folgen läßt. Und da sind schon ein Haufen solcher Motive über die
ganze Großschrift verteilt - wie: Weihwasser zum Beispiel und daß wir es
nur zuweilen anders nennen, um nicht häretisch angeklagt zu werden. Man
bringt mir jedes Jahr eine Flasche immer aus der Nähe vom Ursprungssee
..rainen mit, das ich dann eine Weile mit dem Teewieder trinke. Für
andere mögen es andere Wasser und von meinem verschiedene Erlebnisse
sein, die sich darauf gründen; mein lymphatischer Haushalt jedenfalls
wird belebt durch den Teil leviathaner Materie vom Seengrund.
Wahrscheinlich würdet ihr auch gar nicht mehr weitergelesen haben, wenn
sich nicht von Ohr zu Ohr das hinzöge: die Seewassermoleküle im
Labyrinthwasser des Innenohrs, die mich euch hören machen. Was aber
sprecht ihr dann, wenn ihr sprechen tut? Lange Zeit\ldots{} \emph{je me
suis couché de bonne heure?} Das wäre die Nacht einmal wenn dieses
bewahrt ist und übertragen in eurem Sinne. Ich legte allein mein Herz
nicht darein, was zu finden wie Einklang oder Harmonie, das nicht aus
sich selbst erwachsen wäre; das tat ja jener anders ältere schon lange
vor mir und ihr lest nur ein paar Zeichen ihn übersetzen von heute auf
morgen. Aber er der \emph{Troubador, }wie sie ihn nannten und seine
Helligkeit, ist n.~nicht ausgegangen, und einige Worte sind wirklich
entstanden sehn wir. Vielleicht drang manches durch und ich habe es nun
vernommen, vielleicht drang manches durch und ich könnte es weitergeben.
Wenn wir Sinne versuchen zu erkennen aus dem Gezeichneten, verdeuten wir
uns das, was es von allein aussagen will. Es ist immer ausgerichtet auf
das Zentralhirn und versorgt die absterbenden Nervenzellen mit \emph{Mut
könnte man vielleicht sagen,} aber richtiger wäre wohl \emph{der Kraft
der Resignation}, der Anerkenntnis der "Macht als Anwesenheit am Ort der
Entscheidung über Dringliches" wieschoneinmal von Sloterdijk, aus den
Poetikvorlesungen. Und n.~ein Motiv: \emph{Die Synthese von Zweck, Stoff
und Form,}6 die es neu zu erringen galt auch in diesem Text, beinahe
beiläufig von der Trompeterin erzählerischer Poetik erreicht\ldots{} das
bleibende also\ldots{} ging damit in die Schöpfung ein. Das solln für
eben ein paar aufgenommene Tonlagenmotive sein und ob ich werde mich
daran erinnern können in den vorhergegangen Versätzen und ihrer jetzt
sich langsam natürlich regelnden Chronologie der Ereignisse, steht
n.~aus zu entscheiden; wenn ich mich dazu bewege, wird es vermerkt. Bis
dahin ist dieser Absatz n.~voranzutreiben so daß wir heute die erste
Hälfte halten und ruhig ein Datum angeben können. Es standen aber nur
4940 Zeichen zur Verfügung und bis da sind n.~182, bevor die
erforderlichen 20.000 Worte sich erfüllen, d.h. allerhand Beschränkung
auf sich zu nehmen, um nicht die konsonanten Pfeiler gegenständig zu
überlasten.

\end{document}
