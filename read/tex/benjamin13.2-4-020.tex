% Options for packages loaded elsewhere
\PassOptionsToPackage{unicode}{hyperref}
\PassOptionsToPackage{hyphens}{url}
%
\documentclass[
]{article}
\usepackage{amsmath,amssymb}
\usepackage{iftex}
\ifPDFTeX
  \usepackage[T1]{fontenc}
  \usepackage[utf8]{inputenc}
  \usepackage{textcomp} % provide euro and other symbols
\else % if luatex or xetex
  \usepackage{unicode-math} % this also loads fontspec
  \defaultfontfeatures{Scale=MatchLowercase}
  \defaultfontfeatures[\rmfamily]{Ligatures=TeX,Scale=1}
\fi
\usepackage{lmodern}
\ifPDFTeX\else
  % xetex/luatex font selection
\fi
% Use upquote if available, for straight quotes in verbatim environments
\IfFileExists{upquote.sty}{\usepackage{upquote}}{}
\IfFileExists{microtype.sty}{% use microtype if available
  \usepackage[]{microtype}
  \UseMicrotypeSet[protrusion]{basicmath} % disable protrusion for tt fonts
}{}
\makeatletter
\@ifundefined{KOMAClassName}{% if non-KOMA class
  \IfFileExists{parskip.sty}{%
    \usepackage{parskip}
  }{% else
    \setlength{\parindent}{0pt}
    \setlength{\parskip}{6pt plus 2pt minus 1pt}}
}{% if KOMA class
  \KOMAoptions{parskip=half}}
\makeatother
\usepackage{xcolor}
\usepackage[margin=1in]{geometry}
\usepackage{graphicx}
\makeatletter
\def\maxwidth{\ifdim\Gin@nat@width>\linewidth\linewidth\else\Gin@nat@width\fi}
\def\maxheight{\ifdim\Gin@nat@height>\textheight\textheight\else\Gin@nat@height\fi}
\makeatother
% Scale images if necessary, so that they will not overflow the page
% margins by default, and it is still possible to overwrite the defaults
% using explicit options in \includegraphics[width, height, ...]{}
\setkeys{Gin}{width=\maxwidth,height=\maxheight,keepaspectratio}
% Set default figure placement to htbp
\makeatletter
\def\fps@figure{htbp}
\makeatother
\setlength{\emergencystretch}{3em} % prevent overfull lines
\providecommand{\tightlist}{%
  \setlength{\itemsep}{0pt}\setlength{\parskip}{0pt}}
\setcounter{secnumdepth}{-\maxdimen} % remove section numbering
\ifLuaTeX
  \usepackage{selnolig}  % disable illegal ligatures
\fi
\usepackage{bookmark}
\IfFileExists{xurl.sty}{\usepackage{xurl}}{} % add URL line breaks if available
\urlstyle{same}
\hypersetup{
  hidelinks,
  pdfcreator={LaTeX via pandoc}}

\author{}
\date{\vspace{-2.5em}}

\begin{document}

\subsection{III. Ton}\label{iii.-ton}

\emph{Die Melancholier aber dachten,} daß auch sie nicht wissen wohin
sie unsere ihnen unbekannten Regeln bringen wenn sie gegen sie verstoßen
und daß es nur zwei Wege das herauszufinden gibt. Erstens indem sie
etwas tun und es erfahren. Der zweite, langsamere und sehr schwere und
unwiderrufliche Versuch wäre sich für ein eigenes Regelwerk zu bemühen
und in seinen Gleichnissen mit jenem der Welt diese sich zu erschließen.
Daß sie, also wir unserer Zeit aber nur soweit voraus sind, wie die
Vorstellung erlaubt macht es fast unmöglich sich bei den zu fassenden
Gesetzen nicht doch irgendwie auf die Welt zu beziehn, die wir ja
eigentlich mit ihnen erst gründen wollten. Es besteht dann immerhin ein
Unterschied zwischen dem hermeneutischen Zirkel und einer Gefangenschaft
in der Tautologie. Was uns schließlich aus dieser zu jenem befreien
konnte, habe ich durch das Studium der nachgelassenen Schriften des
Polyhistors entdeckt und bin zu unglaublichen Antworten gekommen. Es ist
aber durchaus ungewiß ob etwas seines bestehen wird in Relation zu dem,
was ich jetzt heranbilde. Denn weil die Modalitäten der Texterstellung
grau waren und ich selbst in irgendwelchen kommenden Bezügen mir meine
Erinnerung daran schwer zurückrufen mußte wird nur einiges haften. Ich
schrieb in den reinen B. vielleicht den er meinte als fantasierend von
unseren Zeiten. Es bleibt ein Rest unbewußte Deutungen dessen was
erzählt werden muß und ob man wird sagen können es hat sich erhalten bis
heute (also bis dahin\ldots) ist vielleicht ja eine äußerst mühsam nur
bestehende Feststellung unserer bis jetzt aber jedenfalls gültig
gewesenen Variablen. Sie werden heute nicht mehr über das hinausweisen,
was ihr gelesen habt, sind also nur primär gehaltenes Gedankenmaterial
zur Großschrift der Benjaminfeldkraft, die wir uns hiermit die ich mir
hier erlaubt hatte fortzusetzen - gerade weil das hieß allen bisher
verwendeten Schriften entgegen sich auf diese zu beschränken, die ihren
Wahrheitsanspruch schon daraus ableitete, daß sie in Erster Imagination
über den Versuch hinauskam, Schönheit aus dem Vergänglichen zu
sublimieren; Schönheit im Sinne mathematischer Ästhetik, vergänglich
verstanden als zeitlich und räumlich begrenzte atomare Zusammenhänge
selbst auch der für ihren Bewußtseinsstand verantwortlichen chemischen
oder physikalischen Vorgänge, die uns beschäftigen. Darüber hinaus nur
encore une foi le champ du force sur certains plaines.\\
2\emph{. Bilde etwas und rede nicht!} Also mußte ich mir einen Stoff
anlesen und fand Yalom Meyrink kaum weit genug von meinen eigenen
homunculi entfernt. Wie lange aber wielange muß man wie lange halten wir
halte ich aus die Zurückhaltung aller anderen wenn sie sich gegen einen
sorgen und damit gutes tun wolln aber die Zurückhaltung nur dazu führt
daß überall \emph{ihre} Wirkungen als Zeugen für meine Veränderung
genannt werden. Irgendwann wird ihnen der Kopf abfalln.\\
Aber n., ich: denke n.~immer über den Wald nach; ich habe Angst. Es war
ein Märchenwald glaube ich, der selbst durchlebt werden wollte, seine
Geschichten sind nicht anders als mit dem Körper erlernbar. Die Psyche
allein, die immer neue ewige Bettlerin, wenn ich auf den Weggang schaue,
steht sie später an der selben Stelle fest, als hätte ich ihr nicht
längst Tribut gezahlt. Sie kann mir nichts mehr beibringen über die Welt
und den Wald soll ich ertragen, niemand kann ihr das abnehmen. Aber
langsam kenne ich mich aus. Wenn es über Abend geht, haben sich in den
Fußspuren genug Tropfen des Nieselregens gesammelt, daß ich mein Gesicht
waschen kann. Auf dem See: Hier ist jetzt Wind, ein großer, starker
Wind, der über das Wasserdunkel hinzieht und die Birken rauschen, die
dünnen, und die Pappeln, ein paar Kiefern stehen steif und
widerspenstig. Am Ufer schlagen Wellen an die Findlingssteine, das Boot
schaukelt, halb an Land liegend, man glaubt: wie an Worte. Das kann man
hören; was es heißen könnte, hier wird es angeschwemmt. Auch kleine Seen
haben ihr Treibgut. Ich warte also, immer sitzend, immer mit dem
Ausblick zwischen den Birken hindurch, wo das alles herkommt. Es gibt
natürlich diesen Ursprung auf irgendeiner Seite oder die Insel, deren
Bäume ein paar Schatten ins Wasser werfen. Aber was davon übrig bleibt
zu hören, wenn ich das wacklige Ruderboot betrachte, das sind die
quietschenden Riemen und ein glucksender Hohlraum, ganz bestimmte Töne.
Vielleicht habe ich ja etwas davon aufgenommen? Ich kann mich nur an den
Regen erinnern, immer stärker und barfuß auf dem nassen Steg. Und an
Ewa, schlafend. Sie bewegt sich. Ich möchte sie weiter anschauen, doch
dreht sie sich gerade um. Ein kleines Ohr, das in die Nacht hinaussteht
und alles wahrnehmen kann, längst mehr, als ich n.~jemals vorstellen
möchte. Darum bleibe ich ruhig und ich weiß, daß sie mein Herz trotzdem
schlagen hört und, wenn es schneller geht, unruhig wird. Doch das ist
nur der schwarze Tee. Ich will einen Blick auf sie tun, stehe von der
Bank auf und setze mich neben dem Kopfende ihres Bettes auf einen
Klappstuhl. Man blickt herunter, ein wenig Mond schien herein? Nein, es
war die umsonst hell gebliebene Nacht des Sees, mit der er mich dazu
bringen wollte, hier zu bleiben. Doch das ist her und vielleicht wird es
einmal doch Mond gewesen sein, der ins Fenster schien und ein paar
Schatten machte. Ich konnte das kleine Ohr kaum sehen zwischen Kissen
und Decke, wie es lauschte aus dem Schlaf in mein Herz hinein. Aber als
es hörte, wie das schlug und daß es ihm gut erging, war es ruhig und
schlief bis morgens. Dann war ein neuer Tag und die Geschichten dieses
Tages waren die von morgen, weil morgen immer der nächste Tag gewesen
ist.

\end{document}
