% Options for packages loaded elsewhere
\PassOptionsToPackage{unicode}{hyperref}
\PassOptionsToPackage{hyphens}{url}
%
\documentclass[
]{article}
\usepackage{amsmath,amssymb}
\usepackage{iftex}
\ifPDFTeX
  \usepackage[T1]{fontenc}
  \usepackage[utf8]{inputenc}
  \usepackage{textcomp} % provide euro and other symbols
\else % if luatex or xetex
  \usepackage{unicode-math} % this also loads fontspec
  \defaultfontfeatures{Scale=MatchLowercase}
  \defaultfontfeatures[\rmfamily]{Ligatures=TeX,Scale=1}
\fi
\usepackage{lmodern}
\ifPDFTeX\else
  % xetex/luatex font selection
\fi
% Use upquote if available, for straight quotes in verbatim environments
\IfFileExists{upquote.sty}{\usepackage{upquote}}{}
\IfFileExists{microtype.sty}{% use microtype if available
  \usepackage[]{microtype}
  \UseMicrotypeSet[protrusion]{basicmath} % disable protrusion for tt fonts
}{}
\makeatletter
\@ifundefined{KOMAClassName}{% if non-KOMA class
  \IfFileExists{parskip.sty}{%
    \usepackage{parskip}
  }{% else
    \setlength{\parindent}{0pt}
    \setlength{\parskip}{6pt plus 2pt minus 1pt}}
}{% if KOMA class
  \KOMAoptions{parskip=half}}
\makeatother
\usepackage{xcolor}
\usepackage[margin=1in]{geometry}
\usepackage{graphicx}
\makeatletter
\def\maxwidth{\ifdim\Gin@nat@width>\linewidth\linewidth\else\Gin@nat@width\fi}
\def\maxheight{\ifdim\Gin@nat@height>\textheight\textheight\else\Gin@nat@height\fi}
\makeatother
% Scale images if necessary, so that they will not overflow the page
% margins by default, and it is still possible to overwrite the defaults
% using explicit options in \includegraphics[width, height, ...]{}
\setkeys{Gin}{width=\maxwidth,height=\maxheight,keepaspectratio}
% Set default figure placement to htbp
\makeatletter
\def\fps@figure{htbp}
\makeatother
\setlength{\emergencystretch}{3em} % prevent overfull lines
\providecommand{\tightlist}{%
  \setlength{\itemsep}{0pt}\setlength{\parskip}{0pt}}
\setcounter{secnumdepth}{-\maxdimen} % remove section numbering
\ifLuaTeX
  \usepackage{selnolig}  % disable illegal ligatures
\fi
\usepackage{bookmark}
\IfFileExists{xurl.sty}{\usepackage{xurl}}{} % add URL line breaks if available
\urlstyle{same}
\hypersetup{
  hidelinks,
  pdfcreator={LaTeX via pandoc}}

\author{}
\date{\vspace{-2.5em}}

\begin{document}

\subsection{CHOR}\label{chor}

Wenn ich über die Kursive hinaus jemals etwas schrieb das nicht von mir
stammte, dann war es von mir aufgeschrieben dh nicht geschrieben sondern
mir zur Erinnerung im Text aufgehoben. Das ist nicht selten notwendig
gewesen da mein Gedächtnis oder besser der im Kopf gelegene Bestand all
dessen was ich je las stetig zunehmen mußte zum weiterschreiben. Die
Maschine nahm mir ja die Kreation nicht ab sondern übernahm gerade ihre
Objektwerdung. Diese Arbeit jedoch von jemand anderem leisten zu lassen
der sich mit den Worten besser auskannte als sie fiel mir irgendwann
ein. Ich hatte beim toten Polyhistor eines jener alten Geräte stehn sehn
und wußte damals aber n.~nicht wofür sie wirklich dienten. Also
kellerten wir fast alles was nicht Schrift war aus der mit ihr eng
gewordenen und den Büchern bis zu Decke gereihten Wohnung ein und erst
später erkannte ich was ich dort eingelagert hatte: einen aus der ersten
Generation stammenden Standardschriftprozessor von 1985. Kein wirklich
bekannt gewordenes Ding denn die einen besaßen verzichteten darauf
überhaupt von ihm zu sprechen. Auch ich wenn ich euch davon hier erzähle
werde nicht den richtigen Namen nennen können, wer einen will wird ihn
bekommen. Nur zu ihm will ich etwas weitergeben was ich in darauf
gespeicherten Aufzeichnungen HBs fand. Vielleicht waren es auch nicht
seine und waren n.~viel früher darauf gekommen als HB es hätte
niederschreiben können wenn er erst Mitte der 80er solchen Prozessors
habhaft werden konnte. Die Sprache könnte seine gewesen sein, der Umgang
mit den Worten als Zeichen lebendigen Daseins innerhalb der Sprache. Und
dieser war soweit ich den Prozeß verstand nur mit seiner Hilfe möglich.
Das Fluidum aus Wort, Gedanke und dem Textkörper darüber so zu bilden
daß (---) eben nicht zu sich selber erstarrt sondern gleichsam
weiterzufließen scheint ist das Vermächtnis seiner Erfinder, die notiert
als N.U., als M.H., als F.M. Chiffren bleiben bis zu meinen Vorgängigen,
die jene Initialen dann austragen dürfen. Kleist hat dazu einmal (und
auch er wird genannt) in einem sehr kurzen Vortrage über "Die
allmähliche Verfertigung der Gedanken beim Reden" gesprochen oder sich
verewigt bei Gadamer (Hölderlin/Heidegger) auf der einen und Gershom
Sholem und Martin Buber auf zwei anderen Seiten.\\
Als ich mit fünfzehn Kilo auf dem Rücken und dem Jakobneunerstab in der
rechten Hand meinen Weg auf dem Moselhöhenweg damit verbrachte mir eben
jenes Reden das hier das rhetorische meinte heranzubilden im lebendigen
Selbstgespräch und dabei zwar sehr langsam aber immerhin doch vorankam
wußte ich, daß mir das alleinige Sprechen bald abhanden kommen würde für
ein geläutertes Reden aus den wirklichen Gedanken heraus. Das mußte
geübt werden und bedurfte einiger harter Zusammenstöße mit dem
unbekannten Gegner doch hier stehe ich jetzt und wenn ich Kleist nach
dieser Zeit und die Aufzeichnungen n.mal lese habe ich etwas mehr meiner
eigenen Gedanken verstanden als schon durchgebildet scheint. Die
Wortalchemie/mystik/kabbala welche danach stets ihre Wirkung zeigt da,
wo wir sie lebendig fassen und gleichzeitig nichtfassen also freilassen
können und nur ihren Weg beschreiben (wie zb den beschleunigter
Teilchen); diese entspringt ja doch den urgemeinsamen Quellen von
Hö/Hei/Gadamer (dessen Schüler HB dann schließlich auch mein Lehrer
wurde), den "Studien"texten Sholems, den Excerzitien Loyolas bis wir und
das schließt den Kreis zum vorigen Band schließlich bei Aristoteles und
endlich auch den so glücklich begonnenen Vorsokratikern und ihren
Selbstverwurzelungen fernab jeglicher Theologie enden. Was für ein
Kreis? - Das war genau das Duell das sich die Schüler hier unter dem
jüngsten Bruder (Benjamin) und dem jüngsten Großmeister (Stockton)
lieferten als sogenannte "Prüfung zur höheren Tauglichkeit" eines der
beiden die für die Aufnahme in den nächsten Grad vorgeschrieben war. Ich
kann auch so ehrlich sein zuzugeben daß ich selbst die Prüfung zweimal
nicht bestand und jemand anderen vortreten ließ. Aber zum dritten Gegner
war die Rede flüssig geworden und der Geist befestigt, ihn schlug ich
mit zwei Reden die mir aus dem Herzen kamen. Eine ging über meine
geliebten W.(oyzeck?agner?eidenhaus?) und der zweite Protagonist war mir
schon damals so vertraut obwohl sein Erfinder heute n.~mein größtes
Vorbild ist, das war der homo faber, der seinerzeit n.~zur Schule
gelesen ward\ldots{}

\end{document}
