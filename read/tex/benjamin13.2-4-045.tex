% Options for packages loaded elsewhere
\PassOptionsToPackage{unicode}{hyperref}
\PassOptionsToPackage{hyphens}{url}
%
\documentclass[
]{article}
\usepackage{amsmath,amssymb}
\usepackage{iftex}
\ifPDFTeX
  \usepackage[T1]{fontenc}
  \usepackage[utf8]{inputenc}
  \usepackage{textcomp} % provide euro and other symbols
\else % if luatex or xetex
  \usepackage{unicode-math} % this also loads fontspec
  \defaultfontfeatures{Scale=MatchLowercase}
  \defaultfontfeatures[\rmfamily]{Ligatures=TeX,Scale=1}
\fi
\usepackage{lmodern}
\ifPDFTeX\else
  % xetex/luatex font selection
\fi
% Use upquote if available, for straight quotes in verbatim environments
\IfFileExists{upquote.sty}{\usepackage{upquote}}{}
\IfFileExists{microtype.sty}{% use microtype if available
  \usepackage[]{microtype}
  \UseMicrotypeSet[protrusion]{basicmath} % disable protrusion for tt fonts
}{}
\makeatletter
\@ifundefined{KOMAClassName}{% if non-KOMA class
  \IfFileExists{parskip.sty}{%
    \usepackage{parskip}
  }{% else
    \setlength{\parindent}{0pt}
    \setlength{\parskip}{6pt plus 2pt minus 1pt}}
}{% if KOMA class
  \KOMAoptions{parskip=half}}
\makeatother
\usepackage{xcolor}
\usepackage[margin=1in]{geometry}
\usepackage{graphicx}
\makeatletter
\def\maxwidth{\ifdim\Gin@nat@width>\linewidth\linewidth\else\Gin@nat@width\fi}
\def\maxheight{\ifdim\Gin@nat@height>\textheight\textheight\else\Gin@nat@height\fi}
\makeatother
% Scale images if necessary, so that they will not overflow the page
% margins by default, and it is still possible to overwrite the defaults
% using explicit options in \includegraphics[width, height, ...]{}
\setkeys{Gin}{width=\maxwidth,height=\maxheight,keepaspectratio}
% Set default figure placement to htbp
\makeatletter
\def\fps@figure{htbp}
\makeatother
\setlength{\emergencystretch}{3em} % prevent overfull lines
\providecommand{\tightlist}{%
  \setlength{\itemsep}{0pt}\setlength{\parskip}{0pt}}
\setcounter{secnumdepth}{-\maxdimen} % remove section numbering
\ifLuaTeX
  \usepackage{selnolig}  % disable illegal ligatures
\fi
\usepackage{bookmark}
\IfFileExists{xurl.sty}{\usepackage{xurl}}{} % add URL line breaks if available
\urlstyle{same}
\hypersetup{
  hidelinks,
  pdfcreator={LaTeX via pandoc}}

\author{}
\date{\vspace{-2.5em}}

\begin{document}

\subsection{guhl}\label{guhl}

N. schreibe ich in den reinen Bezug, aber welche Phase die gesicherten
Erkenntnisse einleiten könnte wage ich nicht mir vorzustellen. Es ist
möglich, daß ich um mich zu bewegen nicht mal mehr jemanden anhalten
muß, der mir zuhört, vielleicht geben auch hiermit zu diesen Schriften
Dinge Anlaß, die über das Verständnis aller anderen so weit hinausgehn,
daß mich zu hören in ihren Zusammenhängen nicht nur unvorstellbar ist
sondern ebensowenig erlaubt wie erwünscht. Wir müssen fromm bleiben. Und
wenn jetzt die Sonne untergeht, bin ich schon lange nicht mehr hier
sondern so tot bei den Toten wie nur ihr es wissen könnt, weil jede euer
Wochen n.~so geendigt hat: Aber dieses Stück vom reinn Bezug werde ich
hinüberretten ins Kommende, das mich ertragen hat so lange ich hier war
immer in der Nähe der Gräber zwischen Findus und Hauk und dem
beständigen institut francais. Das ist der Ort und die Zeit ist heute,
Märzenneumond, es könnte 2011 gewesen sein. Nun ich aber beschloß lieber
zu den Tagen zu beginnen da ich es so empfinde, haben wir den ersten
Neumond des 1. Freijahrs. Wir begeben uns in die Schrift die weiterhin
der Benjaminfeldkraft. Machen wir also ernst mit dem was uns beschäftigt
und bringen es heraus: Das Vierte Hauptstück: Die Standardschrift. Und
falls ich aufgehört haben sollte euch anzusprechen, lest n.~einmal den
kanonischen Gruß vom Anfang, vom Verlauf, und vom Ende, den ihr da
findet wo die W. sich weitergedreht hätte wenn -- vielleicht ihr diesmal
mehr aufnehmt. Jedoch\ldots{} wenn ich den Leser bis in diese Seiten zu
binden vermochte was bleibt mir zu tun ab jetzt? - Die Geschichte
fortzuerzählen der fr.m. sagen wohl sie wolln es bestimmt nicht tun und
unnötig wäre das mir zu sagen ich müßte damit aufhörn: denn das hättich
ja längst schon irgendwann zwischen oben und dem 1518. Wort oft genug
also getan, bisher, und würde es immer wieder \textless tun\textgreater.
Was sich als \textless Geschichte\textgreater{} dann zu euch übertrug
wenn ihr das Buch wirklich keuftet werde ich vielleicht als Leser einmal
erfahren. Aber daran jetzt schon zu denken\ldots{} was das wohl mit der
Schrift macht? Es hebt sie an glaube ich, obwohl dies ja nur ein Sockel
ist und nicht mehr Funktion erfüllen darf als die vorigen für das
Folgende (Grundfläche auf welcher (cubique?) sich der Dirigent bewege um
nicht abzufallen). Aber deshalb Geschichte nicht schreiben wollen, weil
jener sich weigerte diese zu besteigen? Ich war scheint es zu wenig Herr
darüber und wäre ja selbst ans Pult getreten mit den benötigten
Fähigkeiten, mais ils me manquent et a cause de cela nous nous
retrouvions a centre d\textquotesingle histoire, maintenant! Was später
sein wird wenn ihr es wirklich in Händen haltet, was dann also meine
Geschichte gewesen sein wird möge man betrachten aus der Perspektive des
demütigen Eleven seiner (HB-Gadamer-Heidegger-\ldots{}
Hölderlinelfenbeintürmer?\ldots) und jener zu ephemerer Himmelslust
verstiegenen Majorität meines Wortschatzes die sie nicht mehr zu
verhindern weiß (beede). Und, fragen sie: arbeiten Sie n., sind Sie
n.~original? Ich antworte ruhig, denn ich habe die Gewißheit, daß das
Wasser sich erinnert. Was darüberhinaus mir zugeflossen also wäre aus
den fremden meine ich ebenfalls Wortschatzzitaten (u.a.) muß natürlich
irgendwann angehängt werden. Aber von jenen überhaupt nie zu schweigen
könnte was auslösen, das es mir n.~schwerer macht euch zu erreichen:
weil ja jene die alle Namen des Codex/Annex so laut reden, daß sie schon
abstoßend wirken bevor ihr einfach angefangen hättet zulesen. Was machen
wir also? Verschieben\ldots{} und verzerren wir das Bild etwas: irgendwo
werdet ihr auf einen Anhang stoßen, der sich nicht sofort als solcher zu
erkennen gab aber mehr Erklärung beherbergt als jeder bisher gelieferte
Schlüssel. Das hättich dann hiermit versprochen und die Kursiven können
wieder hervor als meine nichteigenen aber geliebten, gebrauchten euch zu
beweisenden Zusprüche meiner Meister die ich euch dann nenne wenn es
dazu kommt daß einmal Namen genannt werden müssen. Da sind wir n.~nicht
und begeben uns abermals in das erste Kapitel. Denn dieses war nur
Anfang. Und am Anfang war? Wie immer alles:

\end{document}
