% Options for packages loaded elsewhere
\PassOptionsToPackage{unicode}{hyperref}
\PassOptionsToPackage{hyphens}{url}
%
\documentclass[
]{article}
\usepackage{amsmath,amssymb}
\usepackage{iftex}
\ifPDFTeX
  \usepackage[T1]{fontenc}
  \usepackage[utf8]{inputenc}
  \usepackage{textcomp} % provide euro and other symbols
\else % if luatex or xetex
  \usepackage{unicode-math} % this also loads fontspec
  \defaultfontfeatures{Scale=MatchLowercase}
  \defaultfontfeatures[\rmfamily]{Ligatures=TeX,Scale=1}
\fi
\usepackage{lmodern}
\ifPDFTeX\else
  % xetex/luatex font selection
\fi
% Use upquote if available, for straight quotes in verbatim environments
\IfFileExists{upquote.sty}{\usepackage{upquote}}{}
\IfFileExists{microtype.sty}{% use microtype if available
  \usepackage[]{microtype}
  \UseMicrotypeSet[protrusion]{basicmath} % disable protrusion for tt fonts
}{}
\makeatletter
\@ifundefined{KOMAClassName}{% if non-KOMA class
  \IfFileExists{parskip.sty}{%
    \usepackage{parskip}
  }{% else
    \setlength{\parindent}{0pt}
    \setlength{\parskip}{6pt plus 2pt minus 1pt}}
}{% if KOMA class
  \KOMAoptions{parskip=half}}
\makeatother
\usepackage{xcolor}
\usepackage[margin=1in]{geometry}
\usepackage{graphicx}
\makeatletter
\def\maxwidth{\ifdim\Gin@nat@width>\linewidth\linewidth\else\Gin@nat@width\fi}
\def\maxheight{\ifdim\Gin@nat@height>\textheight\textheight\else\Gin@nat@height\fi}
\makeatother
% Scale images if necessary, so that they will not overflow the page
% margins by default, and it is still possible to overwrite the defaults
% using explicit options in \includegraphics[width, height, ...]{}
\setkeys{Gin}{width=\maxwidth,height=\maxheight,keepaspectratio}
% Set default figure placement to htbp
\makeatletter
\def\fps@figure{htbp}
\makeatother
\setlength{\emergencystretch}{3em} % prevent overfull lines
\providecommand{\tightlist}{%
  \setlength{\itemsep}{0pt}\setlength{\parskip}{0pt}}
\setcounter{secnumdepth}{-\maxdimen} % remove section numbering
\ifLuaTeX
  \usepackage{selnolig}  % disable illegal ligatures
\fi
\usepackage{bookmark}
\IfFileExists{xurl.sty}{\usepackage{xurl}}{} % add URL line breaks if available
\urlstyle{same}
\hypersetup{
  hidelinks,
  pdfcreator={LaTeX via pandoc}}

\author{}
\date{\vspace{-2.5em}}

\begin{document}

\subsection{II. Die Prophezeiung,}\label{ii.-die-prophezeiung}

Wie weit können wir aber n.~gehn, scientes bonum et malum oder besser:
was ist darüberhinaus nötig?25 Das Wissen einzusetzen, zurückzuführen
(religere) an seinen Bestimmungort welcher ja nicht in uns liegen dürfte
kleinen Menschen ohne den Bezug. Ich habe einmal ausgedacht wo wir ihn
n.~notdürftig herstelln können: Kochkunst. Das ist nicht ironisch
gemeint, sondern tatsächlich ist den meisten dort das eigen, was
ursprünglich in allen Lebensbereichen gegenwärtig war, ein Gefühl für
Proportionen, Maß und Gewicht, für Harmonie, Schönheit und selbst für
ethische Aspekte. Da wäre, wenn man den richtigen Träger isoliert hätte
sozusagen eine Lötstelle für neue Ansätze von denen aus sich
weiterdenken läßt oder mindestens den Denkprozeß einmal etwas anstoßen.
Die Aufgabe hieße nur jene Verknüpfung herbeizuführen zwischen dem
wirklichen alten Wissen über die Ernährung in concreto und ihren
chemischen Bezug zur damaligen Realität in abstracto, genauer: zur in
uns abgebildeten Realität dessen was war. Da dann die wahrgenommenen
Aspekte ihrer spiegelbildlich das widergeben sollten was sie auf
chemischem Wege hervorzurufen immer n.~in der Lage (dazu verdammt) sind
(bewußtseinslos) gewinnt man dadurch eine ziemliche deutliche
körperliche Impression der archischen Umstände. Vom Doktor (Sargnagel)
zwar dafür entlohnt aber aufgrund seiner Situation trotzdem dazu
gezwungen frißt das "Subjekt Woyzeck" exklusiv Erbsen in sich hinein und
wird zu einer Uriniermaschine degradiert, die man mit ihrer eigenen
Ausscheidung tränken kann und in der nächsten Phase damit am Leben
erhalten, was man erzeugt auf diesem Wege. Und Erbsen (soy) lassen sich
wie wir wissen ganz gut maschinell erzeugen\ldots{} aber wie weit können
wir n.~gehn war ja als Absatzidee dafür entworfen worden, einen
Worthorizont auszuloten, der doch gegeben ist, vorgegeben längst von
anderen die vor mir den Gedanken der Biobatterie schon verworfen haben,
weil sich eben das Bewußtsein noch nich auf nur chemischem Wege
austreiben läßt. Sie brauchen auch eine intellektuelle Anleitung in
ihrem Gefüge die ihnen die Zeit nicht lang werden läßt. Denn nichts für
den Lebenswillen ist schädlicher als Langeweile. Und nichts bietet sich
dem Angriff der Langeweile so bereitwillig an wie ein einseitig
ernährtes Gehirn. Und mit einseitig ernährt sind sowohl chemische als
ebenso physikalische also von reizenden Umständen ausgehende
neurologische Erregungen gemeint. Und um den verfassenden
bioelektronischen Dilletantismus zu beenden: ich glaube zuletzt auch
daran, daß die Woyzeckvision nicht nur wegen ihrer im Stück berechtigten
Verwurzelung innerhalb der dramatischen Realität sondern auch im darüber
weit hinaus gesehenen esoterischen Wissen um seine Gegebenheit also um
die Gegebenheit des pathologischen Zustands sich diesen plötzlich
erklären kann, sich als Figur dazu ermächtigt sieht wes Spielball sie
ist und "wes Geistes Kind".\\
W. ist nicht am Schluß umsonst völliger Herr der Lage seines und der
Geliebten Ermordung, er ist es für die Herstellung der erzählenden
Gerechtigkeit Büchners umso mehr als jener in die politischen Zeitläufte
ja auch aktiv eingreift und der Revolution seine Stimme verleiht; so
leid es den heutigen Deutschfanatikern tut, aber da werden sich die
Geister immer scheiden\ldots{} bis das Politische völlig aus der
historischen deutschen Dichtung verdrängt worden ist und über allem nur
n.~Europa-Welt-und-Universum steht an das man sich erinnern kann und die
Freiheitskriege genauso antik sind wie die Antike die zu spalten ja
n.~wenigen gelingt. Doch wir waren über Parmenides26 schon hinweg und
nur um einen Absatz final enden zu lassen muß kein gewolltes Plädoyer
her. Nur: P. hat seine Bedeutung erst mit den Parallelaktionen richtig
ermessen über welche wir heut bescheid wissen. Frage aber: Und
Hölderlin? Wie weit seine Kenntnisse gegangen sein können und wie tief
seine Quellen reichten ist zwar ziemlich sicher, unsicher bleibt doch
trotzdem das Ausmaß seines lyrischen Einblicks in den parmeniden
Erschaffungstyp. Daß er etwa die antiken Gewalten tatsächlich in sich
vernehmen konnte muß ich gar nicht mehr anzweifeln weil auch ich in
anderen Bezügen zwar aber und mit ähnlicher Dringlichkeit als es ihm
wohl war meine Gewalten vernehme, die ebenso historisiert sein mögen
aber nicht weniger evident als es zb sein Kampf der Titanen war. Wir
legen uns nicht mehr unbedingt körperlich mit ihnen an aber die
Herausforderung ihrer Macht ist ja schon dort immer gegeben wo wir etwa
einen Konflikt nicht klassisch (modern) zu lösen uns entscheiden sondern
in der Manier eines Zauberers, eines Adepten, eines Eingeweihten (eines
Mäeutikers). Und was darüber schon alles geschrieben wurde sprengt
sämtliche Gattungsbezeichnungen. Wie soll man es auch nennen, wenn einem
bei der Lektüre simpler Zeitungsartikel auf einmal die Gegenseite
entsteht also der Journalist in eine Spalte rutscht, die nichts mit dem
oberflächlich dargestellten zu tun hat sondern auf einer Ebene handelt
der "Innerlichkeit" des lesenden Individuums, beschreibend seine
Zustände gleich von da aus wo er nur selbst sein dürfte und nicht der
Fremde. Da keimt natürlich Angst auf und die höl.sche Beruhigung über
Stottern und Stammeln eigener fester Verse hilft nur langsam wo einem
die Grenze anfängt zu verschwimmen zwischen den Traumweltwesen der
Geschichte und jener realen Bedrohung durch Faulheit, Nachlässigkeit und
Ignoranz und Toleranz -- den heutigen Reitern der Apokalypse. Ich hab
sie schon gesehn glaubt mir und sie schwingen nicht laut Ketten und
Rasseln sondern schleichen sich an wie man es wörtlich sagt. Es sind nur
ein paar Schritte vom ersten egal zum letzten Finale. Verhindern: durch
Armut/Keuschheit/Gehorsam!

\end{document}
