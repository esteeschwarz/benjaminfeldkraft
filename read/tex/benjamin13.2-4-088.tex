% Options for packages loaded elsewhere
\PassOptionsToPackage{unicode}{hyperref}
\PassOptionsToPackage{hyphens}{url}
%
\documentclass[
]{article}
\usepackage{amsmath,amssymb}
\usepackage{iftex}
\ifPDFTeX
  \usepackage[T1]{fontenc}
  \usepackage[utf8]{inputenc}
  \usepackage{textcomp} % provide euro and other symbols
\else % if luatex or xetex
  \usepackage{unicode-math} % this also loads fontspec
  \defaultfontfeatures{Scale=MatchLowercase}
  \defaultfontfeatures[\rmfamily]{Ligatures=TeX,Scale=1}
\fi
\usepackage{lmodern}
\ifPDFTeX\else
  % xetex/luatex font selection
\fi
% Use upquote if available, for straight quotes in verbatim environments
\IfFileExists{upquote.sty}{\usepackage{upquote}}{}
\IfFileExists{microtype.sty}{% use microtype if available
  \usepackage[]{microtype}
  \UseMicrotypeSet[protrusion]{basicmath} % disable protrusion for tt fonts
}{}
\makeatletter
\@ifundefined{KOMAClassName}{% if non-KOMA class
  \IfFileExists{parskip.sty}{%
    \usepackage{parskip}
  }{% else
    \setlength{\parindent}{0pt}
    \setlength{\parskip}{6pt plus 2pt minus 1pt}}
}{% if KOMA class
  \KOMAoptions{parskip=half}}
\makeatother
\usepackage{xcolor}
\usepackage[margin=1in]{geometry}
\usepackage{graphicx}
\makeatletter
\def\maxwidth{\ifdim\Gin@nat@width>\linewidth\linewidth\else\Gin@nat@width\fi}
\def\maxheight{\ifdim\Gin@nat@height>\textheight\textheight\else\Gin@nat@height\fi}
\makeatother
% Scale images if necessary, so that they will not overflow the page
% margins by default, and it is still possible to overwrite the defaults
% using explicit options in \includegraphics[width, height, ...]{}
\setkeys{Gin}{width=\maxwidth,height=\maxheight,keepaspectratio}
% Set default figure placement to htbp
\makeatletter
\def\fps@figure{htbp}
\makeatother
\setlength{\emergencystretch}{3em} % prevent overfull lines
\providecommand{\tightlist}{%
  \setlength{\itemsep}{0pt}\setlength{\parskip}{0pt}}
\setcounter{secnumdepth}{-\maxdimen} % remove section numbering
\ifLuaTeX
  \usepackage{selnolig}  % disable illegal ligatures
\fi
\usepackage{bookmark}
\IfFileExists{xurl.sty}{\usepackage{xurl}}{} % add URL line breaks if available
\urlstyle{same}
\hypersetup{
  hidelinks,
  pdfcreator={LaTeX via pandoc}}

\author{}
\date{\vspace{-2.5em}}

\begin{document}

\subsection{27.}\label{section}

Doch in plötzlichen Zusammenhängen denken ist eine Stärke der Mitglieder
der congrégation; spontane Zustände in ihre Wirkungen zu zerlegen
(Vektoren) lernte man schon durch einfache Gemeinsamkeiten untereinander
die jeder von uns aufwies. Ich werde nicht auseinandersetzen wie wir uns
darin von den anderen Gemeinschaften unterschieden die n.~nach eigenen
Gesetzen handelten. Tatsächlich waren die Unterschiede zu ihnen nicht
sehr groß auszumachen, aber im Wesen der Potenz (jener Gradation) lag
schon begriffen was nur n.~ausgesprochen werden mußte: wir haben weil
die Standardschrift hier mehr wog als eine der alten Hochsprachen bald
darauf verzichtet wirklich miteinander zu sprechen. Das wichtige hat
überlebt das war die Kommunikation und wie sich sich gestaltete in
diesen Tagen wenn ihr herangewachsen seid (die drei Kinder und J.) ist
für heute unerheblich, worum es geht ist der Satzbau, \emph{der ist das
Primäre.} (bis zu meinen eigenen Wortschatzitaten\ldots, Benn) Wie komm
wir da nochmal raus, fragt jemand. Es gibt immerh. die Möglichkeit einer
Verneinung, auch einer Verneinung von Wahrheiten die längst anerkannt
sind, auch jener die man selbst anerkennt. Dazu genügt es bereits sich
in die eigene Geschichte zu begeben. Wie oft haben wir versucht aus
Fehlern etwas zu lernen und wie sehr sind wir doch darauf zurückgestoßen
worden daß ja das Lernen an sich keine grundgelegene dem Verstand schon
beigegebene Eigenschaft ist sondern daß es selbst zu ihm schon einer
Vorraussetzung bedarf: einer nicht selbstverständlichen Offenheit den
Dingen gegenüber um die es geht. Denn solange wir in uns die Starre
halten und die Trägheit kultivieren (was ein natürlicher Willensakt ist
- über den wir also hinweggehn müssen) wird sich kein Lernerfolg
einstellen können, die Dinge verweigern sich unserer Anschauung. Ich
habe einmal vom Interesse gesprochen das zb auch zwischen den
Elementarteilchen waltet als eine fünfte (Kraft) neben den klassischen
Wechselwirkungen. Solcherart Interesse zum Gegenstand der Betrachtung
hin zwingt ihn danach seinerseits gleichermaßen an uns heran und zieht
uns aus seiner Umwelt zu ihm hin. Wenn wir uns auf diesem Weg ihm nähern
und traun i. unsere eigenen Bemühungen \emph{ihn} zu erkennen trotz dem
immerverhüllenden Unsichtbarschleier (den wir uns gegen die Neugierde
vorhängten) zu offenbaren so fällt ein wenig Lichts wie als unserer
Monde der gemeinsamen Sonne auf das nächtlich Ersehnte seines Wesens und
wird so sichtbar für die von ihrem Spektrum verwöhnten Augen.\\
Frére Th.G. jedoch den wir zur Erforschung Unserer Beschaffenheit in die
Orakelstätte gesandt haben kehrte nicht mehr zurück sondern benutzte die
gewonnenen Erkenntnisse über den Orden zur Flucht \emph{innerhalb seines
Gesetzes,} was eine Katastrophe hieß sollte er mit seinem Wissen zu
Einfluß gelangen. N. ist nichts bekannt geworden das auf einen Versuch
es zu mißbrauchen hindeutete. In der Zeit jedoch wo darüber Unsicherheit
besteht ist keine weitere Schulung der Eleven möglich und dieses hat
schlimme Folgen in der Zukunft der pädagogischen Provinz - wenn es denn
eine solche überhaupt n.~geben konnte. Daß die Schüler nicht mehr
ausgebildet würden allein heißt n.~nicht daß der Wissenszuwachs einhält;
aber der Zerfall des Lehrkörpers in erodierende und restaurative Kräfte
dessen die einen davon fasziniert sind freies Wissen über der Tradition
stehn zu sehen und deren andere sich dagegen im Kampf gegen jene
jegliche Form aufgebende, die Wahrheit zu bloßer \emph{Information}
reduzierende Kommunikation in das Schweigen als letzter Bastion
(Kastaliens zB.) zu retten versuchen - dieses wird wenn der
Ausgebrochene nicht bald gefaßt wird zur Stillegung der Privilegien
führen müssen die das Unterrichten bisher garantierte.30\\
Es fielen ganze Seiten weg des Gedächtnisses und vielleicht werden wir
uns nicht daran erinnern wie es war zur Zeit des Orakels. Übrig blieb
nur eine Ahnung der leicht zu mißdeutenden Formalie: ob man den Kopf
bedeckt oder genau nicht beim Durchtritt der Sphären? Intuitiv möchte
man sich dem Alleswisser verbergen und genauso intuitiv seine Hut
abnehmen. Cassandra spricht in ihrem Schlaf über die anderen Götter die
es wohl auch einmal gab. Jedoch wie die uns lebendig werden sollen
bleibt ihr Geheimnis, ein weniges vor den anderen neben den Klagen und
allem Haß der aus ihr immer hervorbricht.

\end{document}
