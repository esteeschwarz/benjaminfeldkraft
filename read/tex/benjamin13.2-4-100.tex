% Options for packages loaded elsewhere
\PassOptionsToPackage{unicode}{hyperref}
\PassOptionsToPackage{hyphens}{url}
%
\documentclass[
]{article}
\usepackage{amsmath,amssymb}
\usepackage{iftex}
\ifPDFTeX
  \usepackage[T1]{fontenc}
  \usepackage[utf8]{inputenc}
  \usepackage{textcomp} % provide euro and other symbols
\else % if luatex or xetex
  \usepackage{unicode-math} % this also loads fontspec
  \defaultfontfeatures{Scale=MatchLowercase}
  \defaultfontfeatures[\rmfamily]{Ligatures=TeX,Scale=1}
\fi
\usepackage{lmodern}
\ifPDFTeX\else
  % xetex/luatex font selection
\fi
% Use upquote if available, for straight quotes in verbatim environments
\IfFileExists{upquote.sty}{\usepackage{upquote}}{}
\IfFileExists{microtype.sty}{% use microtype if available
  \usepackage[]{microtype}
  \UseMicrotypeSet[protrusion]{basicmath} % disable protrusion for tt fonts
}{}
\makeatletter
\@ifundefined{KOMAClassName}{% if non-KOMA class
  \IfFileExists{parskip.sty}{%
    \usepackage{parskip}
  }{% else
    \setlength{\parindent}{0pt}
    \setlength{\parskip}{6pt plus 2pt minus 1pt}}
}{% if KOMA class
  \KOMAoptions{parskip=half}}
\makeatother
\usepackage{xcolor}
\usepackage[margin=1in]{geometry}
\usepackage{graphicx}
\makeatletter
\def\maxwidth{\ifdim\Gin@nat@width>\linewidth\linewidth\else\Gin@nat@width\fi}
\def\maxheight{\ifdim\Gin@nat@height>\textheight\textheight\else\Gin@nat@height\fi}
\makeatother
% Scale images if necessary, so that they will not overflow the page
% margins by default, and it is still possible to overwrite the defaults
% using explicit options in \includegraphics[width, height, ...]{}
\setkeys{Gin}{width=\maxwidth,height=\maxheight,keepaspectratio}
% Set default figure placement to htbp
\makeatletter
\def\fps@figure{htbp}
\makeatother
\setlength{\emergencystretch}{3em} % prevent overfull lines
\providecommand{\tightlist}{%
  \setlength{\itemsep}{0pt}\setlength{\parskip}{0pt}}
\setcounter{secnumdepth}{-\maxdimen} % remove section numbering
\ifLuaTeX
  \usepackage{selnolig}  % disable illegal ligatures
\fi
\usepackage{bookmark}
\IfFileExists{xurl.sty}{\usepackage{xurl}}{} % add URL line breaks if available
\urlstyle{same}
\hypersetup{
  hidelinks,
  pdfcreator={LaTeX via pandoc}}

\author{}
\date{\vspace{-2.5em}}

\begin{document}

\subsection{2. Anto''ne}\label{antone}

Handlung des Thomaszwillings erfolgen und zwar: Sehn wie der Mond halb
eine weitere Schleife bis zum nächsten Schornstein zu drehen hat, das
sind dann vielleicht n.~zwei Stunden und er dürfte sie lange
wachbleiben. Vielleicht vermeidet trotzdem die Infinitive wenigstens
dort wo er schon gelernt hat zu beugen, da war er mir schon voraus.\\
\emph{Zurückstauchen} war nun das Wort das einmal einer gebraucht hatte
als er von Kafka erzählte und es war mir auch vorgekommen wie von W.
Borchert, der nämliches schrieb in späterer Zeit. Ich selbst stellte mir
meine Belange nicht so gegenwärtig vor daß etwa die Eltern genannt oder
genauer der Vater der die Kinder in den Boden zurücktritt wenn sie sich
etwas vorgewagte hätten. Also lernte ich wohl, in der Passage der
Notwendigkeit meiner Geschichten zur Schwere hin mir das immer zu
erarbeiten was sie schließlich nach mir in die Welt kommen ließ: sie
wurden wesentlich, so wie es hieß (Angelus Silesius): \emph{Mensch werde
wesentlich. Denn wenn der Mensch vergeht, so fällt der Zufall weg, das
Wesen, das besteht}. Mir ist es wichtig geworden als so zu
wiederholendes Motto über den Schriften, denn die nahm ich sehr ernst. -
Ich wollte aber gegen das Zurückstauchen etwas schreiben.\\
Viele Male schon war ich mit meinen Worten zu ihm gelangt (HB) und wußte
nicht mehr mir sein dazu Schweigen anders zu erklären als daß ich ihn
nicht erreichte. Etwas sagte aber mir ganz sicher, daß er mich vernahm
und selbst er wäre dort drüben taub für jegliche Dichtung geworden so
waren wir ja einmal so übereingekommen daß die Dichtung Wahrheit sei und
auch wenn er für alle anderen Wahrheiten zugeschlossen worden wäre,
meine müßte ihn immer n.~ankommen - denn ich vernahm ja seine ebenso:
die längst aus dem Leben gezogenen Schriften. Und vielleicht nur, weil
wir dieses beide wußten konnte überhaupt in der Vergangenheit das
entstehn, was schließlich seine Entdeckung wurde; die er mir nie zugab,
von der mir nie auch nur eine Idee kam bevor ich nach seinem Tod das
Archiv übereignet bekam. Ich weiß n.~sehr genau wann ich zum ersten Mal
mit der Theorie in Berührung kam. Es war eine schon sehr dunkle
Spätsommernacht am Uurainen in Finnland wohin ich die Jahre über Berlin
zu dieser Zeit regelmäßig verließ und auch dann wären die Fische
geköpft, die Pilze geputzt und das Wasser getrunken worden drei Monate
lang, und es wurde allg. dunkler und man sah also Sterne. Sitzend da und
hinausroch auf den See vor der Hütte und ich werde nicht mehr müde weil
sein Wasser einen wachbleiben läßt gegen alle körperlichen Stimmungen
höre ich eine Cellomusik auf dem Kopfhörer (Atterberg) vielleicht nur
weil damit gerade er wachgerufen ist, der das Cello so liebt und ich an
ihn denken muß der in Berlin krank ist; denke jedenfalls über einen
kryptischen Satz nach den er mir später vonjemandem hinterlassen wird
und den ich über seine ersten Schriften setzen werde: Was wir haben
haben wir nicht aus uns, was wir sind, sind wir nicht aus eigenem; wenn
wir dies, nur dies endlich lernen würden. Da tritt ein Funke in mich ein
den ich irgendwie erahnte als ich in die Sterne hinaufschaute aber ich
habe ihn trotzdem nicht kommen sehn. Aber der Satz, der sich da
plötzlich aufbaute gegen diesen ungeheuren Himmel, gegen diesen
Weltraum, der machte ihn ganz klein. Denn ich stand auf und konnte mich
erheben in ihn und ca. denken: da ist ja nichts über mir, das mich vor
ihm schützt, das mich von ihm abhält, das mich von ihm zurückweisen
kann. Ich stehe ja mitten darin mit meinen 1,78m über dem Erdboden und
nur was mich davor schützt in ihn hinauszutreiben, ist meine innerste
treibende Kraft die in jeder Zelle so tätig sei wie in jenem Raum der
sich da an meinem Gesicht schon mir entgegenbreitet. Ich kann also hoch
und meine Hände ausstrecken auf einen der Sterne zu und warte nur eine
Weile dann bin ich bei ihm. Aber wie will man dieses denn verkraften daß
da nichts ist zwischen uns, zwischen mir und dem Stern und zwischen mir
und dem gewaltig sich ausdehnenden Schwarz in alle Richtungen\ldots--
Das war der Moment wo mir aus seinen Worten zum ersten Mal eine Ahnung
von der kleinen Kraft zuteil wurde die unser eigen ist um mit dem was
uns umgibt irgendwie zurechtzukommen. Er hatte ihr dann auch irgendwann
einen Namen zugeteilt unter seinen Entdeckungen und weil es seine letzte
sein sollte und die kleinste unter allen aber wesentlich: eine unter
Schmerzen errungene und denn. nur die eine zurecht genannte nach ihm,
dem großen unserem schließlich gemeinsam gewonnenen Vordenker den sie
alle jetzt ruhig erfahren können - sie würde tatsächlich auf W.B.
recurrieren; nur, daß \emph{ich} es jetzt selbst n.~gar nicht wußte,
weil ich sie n.~nicht zum Ende erkannt habe. Ich werde mich dann
weiterhin davor hüten Gemeinsamkeiten zu suchen und über B. nur
sekundäres mir zu erfahren gestatten (Sholem) bis es vielleicht
unvermeidlich ist doch einen Blick in sein Werk zu tun um nicht allen
Vermutungen der anderen die mit mir dieses schreiben völlig schutzlos
ausgeliefert zu sein, ahnungslos. Aber dieses später, denn morgen sind
die Toten und das sind Geschichten von morgen und das ist der nächste
Tag. Morgen war immer der nächste Tag. Das kann ein ganz schön stauchen.

\end{document}
