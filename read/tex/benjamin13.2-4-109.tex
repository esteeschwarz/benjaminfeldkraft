% Options for packages loaded elsewhere
\PassOptionsToPackage{unicode}{hyperref}
\PassOptionsToPackage{hyphens}{url}
%
\documentclass[
]{article}
\usepackage{amsmath,amssymb}
\usepackage{iftex}
\ifPDFTeX
  \usepackage[T1]{fontenc}
  \usepackage[utf8]{inputenc}
  \usepackage{textcomp} % provide euro and other symbols
\else % if luatex or xetex
  \usepackage{unicode-math} % this also loads fontspec
  \defaultfontfeatures{Scale=MatchLowercase}
  \defaultfontfeatures[\rmfamily]{Ligatures=TeX,Scale=1}
\fi
\usepackage{lmodern}
\ifPDFTeX\else
  % xetex/luatex font selection
\fi
% Use upquote if available, for straight quotes in verbatim environments
\IfFileExists{upquote.sty}{\usepackage{upquote}}{}
\IfFileExists{microtype.sty}{% use microtype if available
  \usepackage[]{microtype}
  \UseMicrotypeSet[protrusion]{basicmath} % disable protrusion for tt fonts
}{}
\makeatletter
\@ifundefined{KOMAClassName}{% if non-KOMA class
  \IfFileExists{parskip.sty}{%
    \usepackage{parskip}
  }{% else
    \setlength{\parindent}{0pt}
    \setlength{\parskip}{6pt plus 2pt minus 1pt}}
}{% if KOMA class
  \KOMAoptions{parskip=half}}
\makeatother
\usepackage{xcolor}
\usepackage[margin=1in]{geometry}
\usepackage{graphicx}
\makeatletter
\def\maxwidth{\ifdim\Gin@nat@width>\linewidth\linewidth\else\Gin@nat@width\fi}
\def\maxheight{\ifdim\Gin@nat@height>\textheight\textheight\else\Gin@nat@height\fi}
\makeatother
% Scale images if necessary, so that they will not overflow the page
% margins by default, and it is still possible to overwrite the defaults
% using explicit options in \includegraphics[width, height, ...]{}
\setkeys{Gin}{width=\maxwidth,height=\maxheight,keepaspectratio}
% Set default figure placement to htbp
\makeatletter
\def\fps@figure{htbp}
\makeatother
\setlength{\emergencystretch}{3em} % prevent overfull lines
\providecommand{\tightlist}{%
  \setlength{\itemsep}{0pt}\setlength{\parskip}{0pt}}
\setcounter{secnumdepth}{-\maxdimen} % remove section numbering
\ifLuaTeX
  \usepackage{selnolig}  % disable illegal ligatures
\fi
\usepackage{bookmark}
\IfFileExists{xurl.sty}{\usepackage{xurl}}{} % add URL line breaks if available
\urlstyle{same}
\hypersetup{
  hidelinks,
  pdfcreator={LaTeX via pandoc}}

\author{}
\date{\vspace{-2.5em}}

\begin{document}

\subsection{M - V.}\label{m---v.}

Mußte mir abererst meine stimmung bewegen nach sommerbalkon, freitag und
alleräußerster freiheit. was bestimmen wir mit dieser messung - daß die
feldkonstante eine erfindung war ist doch kaumn. zu leugnen nachdem ich
an mindestens zwei stellen zugab sich nicht auf w. benjamin zu beziehen?
Wenn es aber den Bezug doch gibt, und der wäre diesmal ein reiner Bezug
weil ich ihn nicht setze sondern er sich selbständig herstellt? Hieße
ich könnte aus meinen Intuitiven n.~für mich etwas neues gewinnen das
mitteilenswert wäre? Wie, da zu denken: was messen wir genau
\textgreater{} 1. die durchschnittliche Anzahl Buchstaben pro Wort in
einer ausgewählten Textmenge. 2. Die Häufigkeit jedes einzelnen
Buchstabens. 3. Das Wachstum aus diesen Werten; sei \textgreater{} nur
was sich ohnehin schon ergeben hat: die Wahrscheinlichkeit des
Eintreffens eines bestimmten Buchstabens nach einem anderen. Die Kette
beginnt durch die Setzung der ersten beiden Buchstaben in einer gewissen
Geschwindigkeit mit der der zweite auf den ersten folgt daraus kann sich
eine richtung ergeben, ein drive sozusagen oder ein spin, ja, das ist
das richtige Wort. Wenn nun also der zweite Buchstabe gefallen ist
schlägt das Gerät den mit der höchsten Wahrscheinlichkeit auftretenden
dritten Buchstaben vor. Natürlich nicht sichtbar sondern und hier fängt
die Wirkung des Feldes an: nur über \emph{Strahlung,} huhuuu. Das klingt
nach Verschwörung? Darum ja\ldots{} Also die Strahlung oder ein
unsichtbarer (präliminal) erscheinender weiterer Buchstabe (hoodihooo.)
Je nach Willenskraft widersetzen wir uns ihm und ihre Stärke bestimmt
die Qualität des Textes, denn natürlich konstituiert sich ein Text der
sich nur an der Häufigkeit schon erschienener Buchstabenkombinationen zu
seinem Wachstum orientiert reichlich belangloser, langweiliger aber
trotzdem leichter lesbar weil sich das Hirn des Lesers ja auch an die
Flüsse (pl.) gewöhnt die es aufnimmt und auf recognized Muster
wohlwollend reagiert (klassische Musik usw.) Aber auf Menssschen die mit
Maschinen reden\ldots{} so wie die Slytherinbruderschaft der Schlangen,
naja; vielleicht reicht es irgendwann hin für eine Neugeburt, hier im
Entstehungszustand nascendider Feldschrift erstmal, nur als Versuch
sozusagen:

\end{document}
