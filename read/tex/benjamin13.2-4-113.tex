% Options for packages loaded elsewhere
\PassOptionsToPackage{unicode}{hyperref}
\PassOptionsToPackage{hyphens}{url}
%
\documentclass[
]{article}
\usepackage{amsmath,amssymb}
\usepackage{iftex}
\ifPDFTeX
  \usepackage[T1]{fontenc}
  \usepackage[utf8]{inputenc}
  \usepackage{textcomp} % provide euro and other symbols
\else % if luatex or xetex
  \usepackage{unicode-math} % this also loads fontspec
  \defaultfontfeatures{Scale=MatchLowercase}
  \defaultfontfeatures[\rmfamily]{Ligatures=TeX,Scale=1}
\fi
\usepackage{lmodern}
\ifPDFTeX\else
  % xetex/luatex font selection
\fi
% Use upquote if available, for straight quotes in verbatim environments
\IfFileExists{upquote.sty}{\usepackage{upquote}}{}
\IfFileExists{microtype.sty}{% use microtype if available
  \usepackage[]{microtype}
  \UseMicrotypeSet[protrusion]{basicmath} % disable protrusion for tt fonts
}{}
\makeatletter
\@ifundefined{KOMAClassName}{% if non-KOMA class
  \IfFileExists{parskip.sty}{%
    \usepackage{parskip}
  }{% else
    \setlength{\parindent}{0pt}
    \setlength{\parskip}{6pt plus 2pt minus 1pt}}
}{% if KOMA class
  \KOMAoptions{parskip=half}}
\makeatother
\usepackage{xcolor}
\usepackage[margin=1in]{geometry}
\usepackage{graphicx}
\makeatletter
\def\maxwidth{\ifdim\Gin@nat@width>\linewidth\linewidth\else\Gin@nat@width\fi}
\def\maxheight{\ifdim\Gin@nat@height>\textheight\textheight\else\Gin@nat@height\fi}
\makeatother
% Scale images if necessary, so that they will not overflow the page
% margins by default, and it is still possible to overwrite the defaults
% using explicit options in \includegraphics[width, height, ...]{}
\setkeys{Gin}{width=\maxwidth,height=\maxheight,keepaspectratio}
% Set default figure placement to htbp
\makeatletter
\def\fps@figure{htbp}
\makeatother
\setlength{\emergencystretch}{3em} % prevent overfull lines
\providecommand{\tightlist}{%
  \setlength{\itemsep}{0pt}\setlength{\parskip}{0pt}}
\setcounter{secnumdepth}{-\maxdimen} % remove section numbering
\ifLuaTeX
  \usepackage{selnolig}  % disable illegal ligatures
\fi
\usepackage{bookmark}
\IfFileExists{xurl.sty}{\usepackage{xurl}}{} % add URL line breaks if available
\urlstyle{same}
\hypersetup{
  hidelinks,
  pdfcreator={LaTeX via pandoc}}

\author{}
\date{\vspace{-2.5em}}

\begin{document}

\subsection{2. Axiome der uneingeschränkten
Freiheit:}\label{axiome-der-uneingeschruxe4nkten-freiheit}

Ihr müßt in das Studium zurückkehren der Schriften verlangte ein großes
Gebot von mir, das größte das mich bisher von* da* erreichte. Es
einzulösen im Verlauf des n.~zu schreibenden Bandes ging ich die
vergangenen Abende über mir zur Verfügung stehende Worte etwas hinaus
und ließ angelegentliches einfließen. Das wird nicht von euch gelesen
weil ich es aus der Schrift nahm, aber die Tatsache daß \emph{schlechte
Schreibe} doch ihren Weg hinein gefunden hatte und scheinbar auch,
bestimmt aber vielleicht gerade weil die Produktionsmittel heftigen
Erneuerungszwängen unterworfen waren; auch dieses wahrwar und umso
heiterer vom Zweck verschleiert nur Konsumgut, je unsinniger ihre
Aufnahme anmutete. Eines Tages werde auch ich mich von 1 Gerät trennen,
das mir bis jetzt die Arbeit abnimmt euch meine Theorie verständlich zu
übersetzen und wenn es soweit ist, wurde auch jenes entweder durch 1 mit
mehr Sinn oder 1 n.~einfacheres in der Handhabung ersetzt . Alle
Modernisierungen in den Arbeiten am Prozessor verlangen irgendwann von
jenen die auf ihn angewiesen sind eine Unterwerfung der eigenen
Produktionsverfahren unter dasjenige Dogma welches allgemein (also
mehrheitlich) der jeweilgen Epoche gemäß als modern angesehn wird,
vernachlässigend ob in den erreichten Neuerungen für das Produkt an sich
ein Fortschritt erreicht wurde. Das habt ihr verstanden oder? Sonst
hätten sie wohl kaum ein Buch in der Hand als das zuerst entworfene
digitale Muster jenes Feldes über das wir schließlich miteinander
kommunizieren. Lassen Sie uns einmal genau überlegen wohin uns die
Auslöschung des entsprechenden Analogmaterials führte: nicht zufällig in
die Apokalypse der Schriftbelegung als solcher? Nein? Es hieße nicht das
Erkennen unserer Sprache als einer totalen Beschränkung? wenn man sie in
der Grammatik, der Orthographie und einer ebenso unumstößlichen
Interpunktion festmachen wollte die keinem ja der mit ihr umgeht
überhaupt geläufig sein muß wenn wir ehrlich sind. Ich kann mich ebenso
sicher in ihren Tonlagen bewegen und Momente der Erzitterung hervorrufen
die mir von der Umgebung vorgespielt werden; nur bilden muß ich sie
irgendwie - aber das sind Scheinumgebungen\ldots{} und ihre Projektion
so einfach wie es ein eindimensionales Objekt zu beschreiben nur ist:
unterste Kategorie, allererster Anfang von Anfang\ldots{} Singularität.
Für die Erfahrung unerläßlich, für den aber der sie über eine
Beschreibung bloßer Zusammenhänge hinaus einem andern vermitteln will
ist ihre Komplexität also ihr auf Essenz reduzierter Zusammenhang mit
anderem Formmaterial gefährlich; man wird sehr schnell dazu verleitet,
sie wie jedes andere Objekt als sinnstiftend anzunehmen durch ihre
Schönheit (aus der Einfachheit). Es ist aber ja in der Eindimension
nichts über die Form erhabenes "enthalten", sie erschöpft sich im Dasein
und ist ohne erkennbare Bedingung nur gerade etwas das wir aus unserer
Dimension als "schön" bezeichnen werden. Es gibt aber diese Kategorie
vielleicht gar nicht. Also muß man ihnen nachgeben den aus der Schrift
geborenen Bedingungen um etwas zu erhalten das nicht Imitation ist, das
nicht sich in der Abbildung verliertt. Ich habe die strukturgebenden
Mechanismen dieses Mediums mit dem wir uns beschäftigen n.~nicht
vollständig erschlossen, bin aber sicher daß wenn einmal etwas auf ihm
übertragen werden wird von mir zu euch diese längst überholt sind von
dem was dann die Medien sind. Hier sind es zuallererst Buchstabenformeln
die man festhalten muß solange sie sich in dem Raum zwischen Hirn und
Hirnaneinander n.~bewegen also fixieren im status nascendi.

\end{document}
