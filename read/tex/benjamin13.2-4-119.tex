% Options for packages loaded elsewhere
\PassOptionsToPackage{unicode}{hyperref}
\PassOptionsToPackage{hyphens}{url}
%
\documentclass[
]{article}
\usepackage{amsmath,amssymb}
\usepackage{iftex}
\ifPDFTeX
  \usepackage[T1]{fontenc}
  \usepackage[utf8]{inputenc}
  \usepackage{textcomp} % provide euro and other symbols
\else % if luatex or xetex
  \usepackage{unicode-math} % this also loads fontspec
  \defaultfontfeatures{Scale=MatchLowercase}
  \defaultfontfeatures[\rmfamily]{Ligatures=TeX,Scale=1}
\fi
\usepackage{lmodern}
\ifPDFTeX\else
  % xetex/luatex font selection
\fi
% Use upquote if available, for straight quotes in verbatim environments
\IfFileExists{upquote.sty}{\usepackage{upquote}}{}
\IfFileExists{microtype.sty}{% use microtype if available
  \usepackage[]{microtype}
  \UseMicrotypeSet[protrusion]{basicmath} % disable protrusion for tt fonts
}{}
\makeatletter
\@ifundefined{KOMAClassName}{% if non-KOMA class
  \IfFileExists{parskip.sty}{%
    \usepackage{parskip}
  }{% else
    \setlength{\parindent}{0pt}
    \setlength{\parskip}{6pt plus 2pt minus 1pt}}
}{% if KOMA class
  \KOMAoptions{parskip=half}}
\makeatother
\usepackage{xcolor}
\usepackage[margin=1in]{geometry}
\usepackage{graphicx}
\makeatletter
\def\maxwidth{\ifdim\Gin@nat@width>\linewidth\linewidth\else\Gin@nat@width\fi}
\def\maxheight{\ifdim\Gin@nat@height>\textheight\textheight\else\Gin@nat@height\fi}
\makeatother
% Scale images if necessary, so that they will not overflow the page
% margins by default, and it is still possible to overwrite the defaults
% using explicit options in \includegraphics[width, height, ...]{}
\setkeys{Gin}{width=\maxwidth,height=\maxheight,keepaspectratio}
% Set default figure placement to htbp
\makeatletter
\def\fps@figure{htbp}
\makeatother
\setlength{\emergencystretch}{3em} % prevent overfull lines
\providecommand{\tightlist}{%
  \setlength{\itemsep}{0pt}\setlength{\parskip}{0pt}}
\setcounter{secnumdepth}{-\maxdimen} % remove section numbering
\ifLuaTeX
  \usepackage{selnolig}  % disable illegal ligatures
\fi
\usepackage{bookmark}
\IfFileExists{xurl.sty}{\usepackage{xurl}}{} % add URL line breaks if available
\urlstyle{same}
\hypersetup{
  hidelinks,
  pdfcreator={LaTeX via pandoc}}

\author{}
\date{\vspace{-2.5em}}

\begin{document}

\subsection{S - XI.}\label{s---xi.}

\begin{enumerate}
\def\labelenumi{\arabic{enumi}.}
\setcounter{enumi}{3}
\tightlist
\item
  Verschwiegen hatte ich über die Zusammenkunft nicht nur den bekannten
  Brüdern sondern konnte es auch dem bisher kaum in Erscheinung
  getretenen Teil der congrégation verheimlichen der sich eigentlich
  selbst aus allem heraushielt aber von allem wußte, was uns anging. Das
  war nicht einfach und es bedurfte anderer Kenntnisse als sie auf dem
  geraden Weg zu erlangen waren. Ich mußte mich in die Gesellschaft
  begeben und Gestalt annehmen. Wo war jene zu erlangen? Mein Autor
  setzte mich seiner Welt nicht aus. Allein jedoch würde es mir kaum
  glücken über mich mehr zu erfahren als ich schon wußte durch die
  Reflektion. Es sollte also ein Weg gefunden werden, mich zu verbinden.
  Ein guter Anfang dafür sind die Kursiven gewesen und ihr Zuspruch des
  Absoluten. Sie selbst zu setzen war mir aber versagt und so mußte ich
  mich \emph{eines Mannes} bedienen, der Ihn kannte und der auch mich
  kannte und als Mittler tätig war. Wer sollte das sein\ldots{} dachte
  lange ich darüber nach und fand nur eine Lösung: Der Polyhistor. Wenn
  er geboren sein würde, käme bald die Zeit ihn auszubilden (in den
  Anfängen der Wissenschaft) und seine Reife wäre meine Zeit mich
  hindurchzuwachen (zu den Anfängen der Philosophie) und ihre Reife wäre
  seine Zeit mich sichtbar werden zu lassen so wie \emph{ich} es wollte
  und unsere Reife wäre eure Zeit mich anzuschauen wie ich war. Gröber
  kann man das nicht mehr fassen und ihr müßt einmal erst euch dafür
  entscheiden wenn es soweit ist, dann wird sich alles finden und die
  Gegenseite das Feld eröffnen. Hier stelle ich nur Pfosten auf
  (4x40.000). Das Tragegerüst für \emph{eure} Geschichte; wie ihr aber
  die Wände ausfüllt und was am Ende n.~von meinem zwar fundamental
  notwendigen aber äußerst untergeordneten Beiwerk zu sehen sein wird
  bleibt der Phantasie des Lesers ausgesetzt, der von mir keine Grenzen
  gesetzt werden sollen und die mit einem bißchen Schrift anzuregen ich
  mir ja so Mühe gab wie in n.~keinem Buch vorher. Ich habe es dann
  dabei belassen Gerüste zu bauen? Falsch! Jeder dieser Schalbauten
  schließt in sich das ja ein was ihr nur aufhüllen müßt indem ihr euch
  löst von mir als dem Erzähler oder Autor oder Ich das ich gerade bin,
  nämlich Malwida Glücklich, eine Nacht vorm Sabbath, d.~ich gar nicht
  heraushalten \emph{will} weil es ja das ist was ich gut kenne, wg.
  stephenkings Drehungen. Was habe ich aber gemacht daß es interessiert?
  Vielleicht das Wagnis in meiner einzigen Geschichte von jener Kraft
  sie gleich mit hineinzustellen in die eure Welt ohne die meine ja wohl
  existiert hat bevor es sie gab? Nein\ldots{} nein, nein\ldots{} ich
  muß andere Gründe nennen merke ich. Und sowie ich für heute Gründe
  finde ruhig schlafen zu gehn anstatt voller Angst dem Aufstehn
  entgegenzuwachen werde ich denn eine Zeit ist hier angekommen jetzt
  verraten was euch an dieses Buch fesselt so wie ich daran festgebunden
  scheine bis ihrs in Händen haltet: Seine dunkeln Motiven, die euch zu
  mir hinab und ich werde sie wieder aufrichten ist das Versprechen das
  ich euch am Anfang gab schon als ich danach fragte, ob ihr denn
  arbeitet, ob du arbeitest. Denn der Lohn könnte diesmal ein anderer
  sein als den sie (singulär!) über jene Tore schrieben zur Letzten
  Hoffnung. Es stand daß man frei werde (Singularität!). Es wurde aber
  nicht von jenen zur Singularität! erfunden. Hölderlin hat wieeinmal
  schon zur Zerspaltung der Person in eine arbeitende und in eine
  arbeitende-denkende-schaffende Seele, die nun nicht mehr in sich
  zerrissen sondern einmal frei und einmal nicht frei, jedes zu seinem
  Zustand und zu seiner Zeit, aber eines vom anderen abhängig und
  unbedingt erforderlich für eine ausgeglichene Zerspaltung also eine
  Spaltung zur Einheit (Seele, erschaffen.) Die Unfreiheit der
  arbeitenden Position ermöglicht die Freiheit des schaffenden
  Individuums und dieses brauche um seine Zerrissenheit innerhalb der
  freien S. zu besänftigen jene übergeordnete, äußere ebenso
  Zerrissenheit des ganzen Menschen unter Zwang und das Diktat des
  Eigenen. Wohin soll dieses gelesen werden? Nach links, immer nach
  links. Die Vektoren richten sich ja von allein aus wissen wir und
  unser Dazutun kann das sein von hören denken und schreiben allein,
  soweit das Auge reicht.
\end{enumerate}

\end{document}
