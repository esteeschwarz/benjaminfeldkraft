% Options for packages loaded elsewhere
\PassOptionsToPackage{unicode}{hyperref}
\PassOptionsToPackage{hyphens}{url}
%
\documentclass[
]{article}
\usepackage{amsmath,amssymb}
\usepackage{iftex}
\ifPDFTeX
  \usepackage[T1]{fontenc}
  \usepackage[utf8]{inputenc}
  \usepackage{textcomp} % provide euro and other symbols
\else % if luatex or xetex
  \usepackage{unicode-math} % this also loads fontspec
  \defaultfontfeatures{Scale=MatchLowercase}
  \defaultfontfeatures[\rmfamily]{Ligatures=TeX,Scale=1}
\fi
\usepackage{lmodern}
\ifPDFTeX\else
  % xetex/luatex font selection
\fi
% Use upquote if available, for straight quotes in verbatim environments
\IfFileExists{upquote.sty}{\usepackage{upquote}}{}
\IfFileExists{microtype.sty}{% use microtype if available
  \usepackage[]{microtype}
  \UseMicrotypeSet[protrusion]{basicmath} % disable protrusion for tt fonts
}{}
\makeatletter
\@ifundefined{KOMAClassName}{% if non-KOMA class
  \IfFileExists{parskip.sty}{%
    \usepackage{parskip}
  }{% else
    \setlength{\parindent}{0pt}
    \setlength{\parskip}{6pt plus 2pt minus 1pt}}
}{% if KOMA class
  \KOMAoptions{parskip=half}}
\makeatother
\usepackage{xcolor}
\usepackage[margin=1in]{geometry}
\usepackage{graphicx}
\makeatletter
\def\maxwidth{\ifdim\Gin@nat@width>\linewidth\linewidth\else\Gin@nat@width\fi}
\def\maxheight{\ifdim\Gin@nat@height>\textheight\textheight\else\Gin@nat@height\fi}
\makeatother
% Scale images if necessary, so that they will not overflow the page
% margins by default, and it is still possible to overwrite the defaults
% using explicit options in \includegraphics[width, height, ...]{}
\setkeys{Gin}{width=\maxwidth,height=\maxheight,keepaspectratio}
% Set default figure placement to htbp
\makeatletter
\def\fps@figure{htbp}
\makeatother
\setlength{\emergencystretch}{3em} % prevent overfull lines
\providecommand{\tightlist}{%
  \setlength{\itemsep}{0pt}\setlength{\parskip}{0pt}}
\setcounter{secnumdepth}{-\maxdimen} % remove section numbering
\ifLuaTeX
  \usepackage{selnolig}  % disable illegal ligatures
\fi
\usepackage{bookmark}
\IfFileExists{xurl.sty}{\usepackage{xurl}}{} % add URL line breaks if available
\urlstyle{same}
\hypersetup{
  hidelinks,
  pdfcreator={LaTeX via pandoc}}

\author{}
\date{\vspace{-2.5em}}

\begin{document}

\subsection{EPILOG}\label{epilog}

Ich hätte gern diese Sehnsucht. Da war ein Mann, alt aber aufrecht
sitzend in eim Fenstermorgen eines Erkers, der in die Nacht vorm
Parterre hinaussprang, inner Gegend wo kleine Gärten dieses vom Gehweg
trennen. Jeden Morgen ginge ich drei Jahre lang an dem Haus vorbei um
Gartenarbeit und wußte nicht mehr als, daß dort wie er am Tisch saß,
erleuchtet, Zeitung und eine Tasse neben sich, ich eines Tages werde
aufstehn wollen wenn nur die Zukunft näher mich an sich herangelassen u.
das Grauen der Morgen in die ich mich aus dem dickflüssigen Bettwachs
hin auslöste der gleichen Vergangenheit angehörten wie Schulwege,
Universitätsfluren und Bettlerarbeit. J\textquotesingle aimais de lui
voir de temps en temps wenn ich auferksam genug war, müde
geradausstolpernd. Aber ich lernte den Alten kennen -- eines Tages stand
er mir mitten auf meim Rückweg im Weg und sagte \emph{Guten Heimweg,
dreimalwegjunger Freund\ldots{}} Ich, knowing the readiness of my wit to
be slow comme un sang froid tiring a ploughshare sagte wohl etwa Danke,
Ja oder ähnlich, mußte aber irgendwie eine Geste aus meinem autistischen
Gesicht gelesen haben die er als wohlwollend annahm und eh
ich\textquotesingle s mich versah bin ich plötzlich stehngeblieben was
ich nie tat an Menschen. Er sagte, daß er mich die Morgende vor seinem
Fenster entlanglaufen und mich fragend in welchem Lehrjahr ich denn
jetzt sei so alt wie ich doch schon bin entfachte er in mir durch sein
Interesse I Stolz auf das was ich da tat; und wie alt ich trotzdem sei
antwortete ich weder verlegen n.~auch andersherum mit Ziffern zwischen
28 und 35 die ich für angemessen hielt. Daß es sich dabei auch um ein
tatsächliches Alter in Jahren gehandelt haben könnte geht mir erst jetzt
auf wo ich das der anderen mit dem meiner vergleiche die vor mir
sterben. Das wär jedenfalls die Geschichte um HB aus 29 und Teil des
Alten Namens Helmut Bröker gewesen und ich habe mir sie natürlich nicht
ausgedacht, er war der mich eines Sommertages für den Abend zu sich
einlud zu Lesungen, die mich heute so reich wie nur jemand daran werden
kann mit der Geschichte beschenktenund sie hat n.~nicht mal angefangen
die Nacht; nur darum bin ich in einem weiten Sommer nach Finnland
gefahren und holte das Wasser von dem er dringend brauchte. Er war nicht
alt und seine Zeit obwohl sie statistisch gekommen n.~nicht ganz vorbei,
so wie ich ihn k.lernte in den letzten Jahren. Daß das irgendwann zur
Verantwortung über ein wildes Archiv aus Zetteln und verzettelten
Büchern und Büchern auf Zetteln, Blättersammlungen und Briefen und
halbgeschrieben halbbesprochenen Tonbändern mit Vortragsmaterial führte
wagte ich mit einer schwach gewußten Ahnung in uns beiden wenn nicht
letztlich begründen so doch zumindest eine Kausalität einfach
feststellen und dabei belassen. Ich versuchte oft, mich dahin zu
erfinden wo er selbst in seinem Fensterlicht zum Garten gesessen haben
mochte und von der Arbeit aufblickend mich dann nicht zufällig sondern
sehr gewiß, sehr genau zu immer derselben Zeit und in immer denselben
grünen Hosen wohl sah. Und indem er mich schließlich an dem einzigen
Tag, der hierdrin für seine Realität eine Rolle spielt wirklich
\emph{ansprach}, in der Weise ansprach wie ich n.~mein Leben erlernen
muß, darin jedenfalls finde ich mich zurecht und danke ihm mit dem
Fortgang des Buches, das seine Initialen getragen hat, die es tragen
werden. \emph{Lesen Sie wenn I. der Mut dazu gegeben ist was ihnen
folgt.}

\end{document}
