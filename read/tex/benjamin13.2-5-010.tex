% Options for packages loaded elsewhere
\PassOptionsToPackage{unicode}{hyperref}
\PassOptionsToPackage{hyphens}{url}
%
\documentclass[
]{article}
\usepackage{amsmath,amssymb}
\usepackage{iftex}
\ifPDFTeX
  \usepackage[T1]{fontenc}
  \usepackage[utf8]{inputenc}
  \usepackage{textcomp} % provide euro and other symbols
\else % if luatex or xetex
  \usepackage{unicode-math} % this also loads fontspec
  \defaultfontfeatures{Scale=MatchLowercase}
  \defaultfontfeatures[\rmfamily]{Ligatures=TeX,Scale=1}
\fi
\usepackage{lmodern}
\ifPDFTeX\else
  % xetex/luatex font selection
\fi
% Use upquote if available, for straight quotes in verbatim environments
\IfFileExists{upquote.sty}{\usepackage{upquote}}{}
\IfFileExists{microtype.sty}{% use microtype if available
  \usepackage[]{microtype}
  \UseMicrotypeSet[protrusion]{basicmath} % disable protrusion for tt fonts
}{}
\makeatletter
\@ifundefined{KOMAClassName}{% if non-KOMA class
  \IfFileExists{parskip.sty}{%
    \usepackage{parskip}
  }{% else
    \setlength{\parindent}{0pt}
    \setlength{\parskip}{6pt plus 2pt minus 1pt}}
}{% if KOMA class
  \KOMAoptions{parskip=half}}
\makeatother
\usepackage{xcolor}
\usepackage[margin=1in]{geometry}
\usepackage{graphicx}
\makeatletter
\def\maxwidth{\ifdim\Gin@nat@width>\linewidth\linewidth\else\Gin@nat@width\fi}
\def\maxheight{\ifdim\Gin@nat@height>\textheight\textheight\else\Gin@nat@height\fi}
\makeatother
% Scale images if necessary, so that they will not overflow the page
% margins by default, and it is still possible to overwrite the defaults
% using explicit options in \includegraphics[width, height, ...]{}
\setkeys{Gin}{width=\maxwidth,height=\maxheight,keepaspectratio}
% Set default figure placement to htbp
\makeatletter
\def\fps@figure{htbp}
\makeatother
\setlength{\emergencystretch}{3em} % prevent overfull lines
\providecommand{\tightlist}{%
  \setlength{\itemsep}{0pt}\setlength{\parskip}{0pt}}
\setcounter{secnumdepth}{-\maxdimen} % remove section numbering
\ifLuaTeX
  \usepackage{selnolig}  % disable illegal ligatures
\fi
\usepackage{bookmark}
\IfFileExists{xurl.sty}{\usepackage{xurl}}{} % add URL line breaks if available
\urlstyle{same}
\hypersetup{
  hidelinks,
  pdfcreator={LaTeX via pandoc}}

\author{}
\date{\vspace{-2.5em}}

\begin{document}

\subsection{31 Punkte über Farben}\label{punkte-uxfcber-farben}

\begin{enumerate}
\def\labelenumi{\arabic{enumi}.}
\setcounter{enumi}{3}
\tightlist
\item
  Ich kann nichts sehen, das nicht auch Farbe ist.\\
\item
  Ich kann das nicht sehen, das keine Farbe hat.\\
\item
  Etwas, das ich nicht sehen kann, hat keine Farbe.\\
\item
  Es gibt aber Dinge, die ich nicht sehen kann und von denen ich nicht
  weiß, ob sie Farbe haben.\\
\item
  Sie hängen den Dingen an, die ich anschaue, indem ich sie anschaue.\\
\item
  Seine Farbe ist nur ein Attribut des Dinges neben anderen.\\
\item
  Habe ich eines Dinges Farbe erkannt, habe ich eine Eigenschaft
  erkannt.\\
\item
  Es gibt im Gehirn gespeicherte Entsprechungen für jede gesehene Farbe,
  die abrufbar sind.\\
\item
  Durch Lichtbrechung können Farberscheinungen hervorgerufen werden,
  welche meßbar sind.\\
\item
  Vom Auge werden die Lichtstrahlen gebrochen und in oder auf einem
  Medium abgebildet.\\
\item
  Dieses Medium kann wie ein Nährboden verstanden werden, der gestreute
  Frequenzen verstärkt.\\
\item
  Unterschiedliche Frequenzen können gleichzeitig im Raum abgebildet
  werden.\\
\item
  Das Auflösungsvermögen des Mediums zur Differenzierung der Frequenzen
  ist begrenzt.\\
\item
  Das Auflösungsvermögen des Mediums muß nicht mit der tatsächlichen
  Informationsdichte übereinstimmen.\\
\item
  Es können feinere Abstufungen vom Auge gebrochen werden, als
  abgebildet werden können.\\
\item
  Es können feinere Abstufungen vom Gehirn gespeichert werden, als in
  einem Moment abgebildet werden können.\\
\item
  Es kann also auf dem Wege der Abbildung im Medium Verlusterscheinungen
  geben.\\
\item
  als durch langsameres zeitliches Aufeinanderfolgen gesehener
  Farbeindrücke. Dafür muß das Auflösungsvermögen der zeitlichen
  Wahrnehmung verlangsamt werden, damit in der kleinsten ausgedehnten
  Zeiteinheit, die vom Gehirn verarbeitet wird, mehr farblich
  differenzierte Eindrücke aufgenommen werden können.\\
\item
  Eine Masse, die bewegt wird, wird umso schwerer, je schneller sie sich
  bewegt.\\
\item
  Für eine Masse, die bewegt wird, vergeht die Zeit relativ langsamer.\\
\item
  Die gebrochenen Lichtstrahlen bewegen sich mit konstanter
  Geschwindigkeit im Auge des Betrachters.\\
\item
  Sich schneller bewegende Lichtstrahlen mit schneller
  aufeinanderfolgenden Änderungen der Frequenz würden größere Massen
  bewegter Partikel zeitigen.\\
\item
  Größere Massen der Partikel erforderten nur eine kleinere Dichte des
  abbildenden Mediums.\\
\item
  Eine kleinere Dichte des abbildenden Mediums verursacht einer
  kontrastreichere Abbildung.\\
\item
  Einer kleineren Dichte des Mediums entspricht eine geringere
  Ausdifferenzierung desselben.\\
\item
  Sich langsamer bewegende Lichtstrahlen haben dasselbe Ergebnis wie
  eine stärkere Ausdifferenzierung des Mediums.\\
  These: Die Geschwindigkeit der vom Auge gebrochenen Strahlen kann
  innerhalb desselben verlangsamt und dadurch mehr wahrgenommen werden.
\end{enumerate}

\end{document}
