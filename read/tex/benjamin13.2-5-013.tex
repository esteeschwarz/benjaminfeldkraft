% Options for packages loaded elsewhere
\PassOptionsToPackage{unicode}{hyperref}
\PassOptionsToPackage{hyphens}{url}
%
\documentclass[
]{article}
\usepackage{amsmath,amssymb}
\usepackage{iftex}
\ifPDFTeX
  \usepackage[T1]{fontenc}
  \usepackage[utf8]{inputenc}
  \usepackage{textcomp} % provide euro and other symbols
\else % if luatex or xetex
  \usepackage{unicode-math} % this also loads fontspec
  \defaultfontfeatures{Scale=MatchLowercase}
  \defaultfontfeatures[\rmfamily]{Ligatures=TeX,Scale=1}
\fi
\usepackage{lmodern}
\ifPDFTeX\else
  % xetex/luatex font selection
\fi
% Use upquote if available, for straight quotes in verbatim environments
\IfFileExists{upquote.sty}{\usepackage{upquote}}{}
\IfFileExists{microtype.sty}{% use microtype if available
  \usepackage[]{microtype}
  \UseMicrotypeSet[protrusion]{basicmath} % disable protrusion for tt fonts
}{}
\makeatletter
\@ifundefined{KOMAClassName}{% if non-KOMA class
  \IfFileExists{parskip.sty}{%
    \usepackage{parskip}
  }{% else
    \setlength{\parindent}{0pt}
    \setlength{\parskip}{6pt plus 2pt minus 1pt}}
}{% if KOMA class
  \KOMAoptions{parskip=half}}
\makeatother
\usepackage{xcolor}
\usepackage[margin=1in]{geometry}
\usepackage{graphicx}
\makeatletter
\def\maxwidth{\ifdim\Gin@nat@width>\linewidth\linewidth\else\Gin@nat@width\fi}
\def\maxheight{\ifdim\Gin@nat@height>\textheight\textheight\else\Gin@nat@height\fi}
\makeatother
% Scale images if necessary, so that they will not overflow the page
% margins by default, and it is still possible to overwrite the defaults
% using explicit options in \includegraphics[width, height, ...]{}
\setkeys{Gin}{width=\maxwidth,height=\maxheight,keepaspectratio}
% Set default figure placement to htbp
\makeatletter
\def\fps@figure{htbp}
\makeatother
\setlength{\emergencystretch}{3em} % prevent overfull lines
\providecommand{\tightlist}{%
  \setlength{\itemsep}{0pt}\setlength{\parskip}{0pt}}
\setcounter{secnumdepth}{-\maxdimen} % remove section numbering
\ifLuaTeX
  \usepackage{selnolig}  % disable illegal ligatures
\fi
\usepackage{bookmark}
\IfFileExists{xurl.sty}{\usepackage{xurl}}{} % add URL line breaks if available
\urlstyle{same}
\hypersetup{
  hidelinks,
  pdfcreator={LaTeX via pandoc}}

\author{}
\date{\vspace{-2.5em}}

\begin{document}

\subsection{11. Masse}\label{masse}

Aber zuerst mußte ich mich in die Farben begeben, um das Kind
kennenzulernen. Ich gehe die Treppen ein paar Stufen hinauf. Ich
versuche, das große Denken nicht mit hoch zu tragen. Sie soll Musik
daraus entstehen lassen, ihm diese schwächelnde Objektivität zu nehmen.
Sie: ist das Kind, von dem ich denke, daß es ein Mädchen war, aber das
sind auch Sie, die mir die Stimme geben, ein alter Mann, krank bis zum
Tode aufgehoben in diesem Klang. Ich höre genau hin. Das Rascheln der
Seiten und Kleiderfalten, die sich berühren, Haut, auf Haut streichend,
Gesichter, Hände, strebsames Haar. Und dann wird mir plötzlich klar, daß
Sie, der Professor, auch die Frau gewesen sein könnten, diech nur nicht
sah. Die also ihre Rolle so gut wahrgenommen hat mir gegenüber, daß sie
sich verschleierte mit jenem kleinen Mädchen. Es gab ja dies Ewa
wirklich und vielleicht nur \emph{die Junge Frau Mahler} konnte
erklären, was für eine Beziehung ich zu ihm unterhielt. Aber sie selbst,
wenn sie verschleiernd den alten Professor so mit mir umging -- welchen
Glauben durfte ich ihr schenken?\\
Einmal, später, merkte ich, daß Gedankenstücke zu bearbeiten waren. Dann
ging ich also hinunter und wollte sie mit mir nehmen. Es waren mehr
geworden in der Zeit, sie fransten aus an den Kanten und wucherten in
das blendende Stück Straße vor dem Gebäude. Ich stand ohne Übersicht
mitten dazwischen, wohin mich das Haus entlassen hatte, ein Kristall in
der gesättigten Lösung. Das Raster fing an zu wirken. Es war immer alles
schwarz, bevor das Lichtteilchen erschien. Sein Weg, den es beschrieb
erst, ließ mich die Ahnung verfolgen und oft wußte ich etwas wie aus
Denkstrukturen vorarchischer Zeit; ohne hineinlangen oder etwa benennen
zu können, da war n.~kein Wort für mich, nur Wellenberge und Täler.
Angst vermutlich, das Rohe könnte stärker sein und die geformte
Oberfläche durchdrücken, sie war ja nur Träger\ldots{} aber welche
Masse, welch unmöglich hohe Bereitschaft, zu kristallisieren, die Dichte
dieser Vergangenen \emph{liquid crystal blackbox. }Ein grauer Handstumpf
führt Bewegungen aus, bedient Knöpfe und Schalter, grau also, weil ich
ihn ja nicht sehen kann außerhalb der box, aber ihn mir vorstelle, grau
und getrennt vom Körper, und papiern ist er wie zu einer Puppe gehörend
aus Maché. Die Befehle des Puppenkopfes kommen über dünne Drähte an und
steuern das Handding. Doch wie erstellt der die Befehle oder auf welche
Zeichen hin? Das müssen doch die sein, die er an der \emph{blackbox}
abliest. Aber was liest er ab? Was ich gelernt habe? Es sind die
Antworten auf seine Fragen, von denen ich immer dachte, er wüßte sie
längst und alles sei nur rhetorisches Spiel. Und nun das ernsthafte
Fragen, das bedrohliche, mit der Strafe. Ich kenne die Antworten, kenne
sie. Sie: das sind die vielen kleinen Gedankenstücke, die ich jetzt mit
hinauf nehmen werde, damit sie eine Wohnung bekommen.\\
Ich stieg also n.~einmal bis in die oberste Etage, dort hatte ich meine
Wohnungstür offengelassen, weil es nur kurz dauern sollte, bis ich
w.gekommen wäre. Sie war zu. So mußte ich klingeln und ich nahm an, daß
sich jetzt jemand hinter der Tür befand, die \emph{sie} nach sich
geschlossen hatten. Ich drückte meinen Knopf. Immer wieder. Bis sie
aufspringen mußte, dachte ich. Es wurde ein Mädchen sichtbar, das
\emph{seinen} Kopf hinter dem vorstehenden Rahmen halb verbarg und
lächelte. Bis ich das erwidern konnte, hatte sie das Gesicht, ein
ruhiges und schönes Gesicht, einmal nach dem Innenraum gewendet und es
war ihre Mädchenstimme nur nicht zu hören, weil sie flüsterte. Nicht mir
zu\ldots, aus der Tür heraus raschelte sie mit den Lippen, als wenn es
kleine Füße wären, die gleich kommen wollten, wenn man sie rief. So
sprach ich vorwärts in den Wohnungsbereich, daß ich da wohnte usw.\\
Da ließ man mich hinein. Fremd war ich und ungeduldig, alles
anzuschauen, wenn man durch die Räume sich tastet und versucht,
nirgendwo anzustoßen\ldots{} hatte ich gelesen, doch da sind
unvorsichtige Menschen gewesen, die mit ungestümen Gesten
vorwärtsdrängten und ihre Hände nicht aus einem Gesicht lassen wollen.
Ein Mädchen war hier und eine schwache Mutter, die sich in ein seltsam
steifes Tuch getan hatte, das sie aufrecht hielt. Am Rücken der Kleinen
hingen zwei Zöpfe dunkel herunter und trennten das Haar bis in die
Stirn. Und diese Stirn war hier zweimal vorhanden, weiß, glatt und nach
oben wie abgehackt vom Scheitel, nach unten als Einleitung zu begreifen
zu einer Komposition aus zwei Jahren, die ich dem Hunger der beiden
Frauen abgelesen hatte. Ein Stadthunger, wie ihn nur \emph{Weltbürger}
zu fühlen imstande sind, aus knospenden Zweigen im Park oder dessen
Rasengerüchen leicht zusammengestellt und ihn dann so lange befrieden,
bis es sie zurück nach dem Winter, nach Dunkelheit und Kohlengeruch
zieht. Dann singen sie \emph{Wohl dem, der jetzt }n.* Heimat hat* und
meinen gut für sich gesorgt zu haben. Öfchen bullert.\\
\emph{Ich} trete auch in mein novemberwarmes Schlafzimmer, wo n.~einmal
das steife Gespenst wartet, das mich durch die Tür gelassen hatte.\\
das kind habe ich ins bett geschickt. komm zu mir.\\
Aber das andere Kind steht n.~in der Tür und lauscht, siehst du es
nicht?\\
Ich konnte es nicht sehen, mein Schatten nahm ihr das Licht, glaube ich.
Ich versuchte, den Schatten zu verdrängen. In die Zeit fällt das zurück,
aber nicht ohne einen Grund zu hinterlassen, den ich dir nannte.\\
Sieh doch manchmal hin, wenn du die Augen wandern läßt. Das ist es.\\
Sie sah es, wenn es sich in weichen Bewegungen von Tür zu Tür, von
Fenster zu Fenster, Blick zu Blick verdichtete, bis es einen Raum ganz
eingenommen hatte. Auch sie wollte jetzt flüstern, als ich anfing, ihr
Glauben zu schenken.\\
Soll ich mit ihm sprechen? fragte ich, die Gewohnheit.\\
ich habe es versucht. ich lag aufgewacht und hörte meinen atem nicht
mehr. zu dunkel war es, um etwas zu sehen. zu kalt, um aus dem bett zu
wollen, weil ich angst hatte. da sprach ich es an, dorthin, wo ich es
vermutete, zwischen tisch und fenster. da ist was, das nicht dunkelheit
sein kann, dachte ich.\\
es ist farbe geworden, ich konnte etwas von ihm erkennen.\\
Aber dein Kind war n.~farblos und erwidert die Umstände, in die ich es
entlasse. Es bleibt ohne Gewicht, ohne das Lebendgewicht. Ich habe den
Handel trotzdem vollzogen, auch mit einem nur hypothetischen Wissen von
seiner Ankunft und sollte belohnt werden für das Vertrauen. Ein Engel,
wies dort durch das Zimmer geht, unsichtbar schon vorbei, wenn ich mich
entsinne, daß hier einmal etwas stand, wohl ein Arbeitsstuhl mit
Velourpolster und Lehnen vor dem Schreibtisch, und jetzt Uhren darauf,
die von i. ans Ohr gehalten werden und aufgezogen. Doch scheint alles
seltsam wie unberührt. Was an alten Büchern in i. Hände kommt, ist
schnell beantwortet, in wenigen Minuten liest s. dir aus der
Familienbibel Stellen, die deinen Tag kennzeichnen und du wußtest
irgendwie, daß sie in dieser Zeit zu dir kommen sollten. Aber verwundert
schaust du ihr Gesicht an und blickst gegens Fenster. Eine Melodie:
vielleicht regnets gerade, `s hat viel geregnet in dem Sommer, ich
erinnere mich. Auch an Hofluft und Kochgerüche in dem alten Treppenhaus.
Und erst der \emph{Blauregen}, eine Kastanie völlig verschlingend, und
der so mitten im Hof in die Höhe wuchs, um ein mächtiges Dach zu bilden;
das dampfte nach dem Regen n.~eine Weile. Handgriffe also, die Elemente
zu bedienen\ldots{} brachtich ihr jeden Tag einige bei, die sie gut
behielt. Darüber hinaus nichts. Und sich wirklich erinnern hießja, allen
Stoff n.~mit ihr durchzugehen, der sie später bände. Ihr Unterricht
statt dessen: Ahnungen, sie in diese Richtung ziehen zu müssen; im
Abitur n.~Versuche auf Deutsch. Aber ich bin immer i. Lehrer gewesen,
Musiklektionen: Verbindungen wie zu vielem, das sie lernte und das Übung
war. Warum\ldots: Die Unschärferelation war ein Wort in einer Klausur
und Wort für Jahre, ebenso was sollten Kohlenstoffverbindungen,
Mendelsche Regeln, was soll eine Evolutionstheorie dir sagen, und wozu
die Anatomie. Und jedes Fach und jedes weitere Wissen und alle Sprachen
wie jeder Funken Mathematik, Kind! Erst wenn du anfängst, mathematisch
denken zu wollen, hat die Schule ihr Ziel erreicht\ldots, dann sei auf
der Hut!

\end{document}
