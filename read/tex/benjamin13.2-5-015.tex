% Options for packages loaded elsewhere
\PassOptionsToPackage{unicode}{hyperref}
\PassOptionsToPackage{hyphens}{url}
%
\documentclass[
]{article}
\usepackage{amsmath,amssymb}
\usepackage{iftex}
\ifPDFTeX
  \usepackage[T1]{fontenc}
  \usepackage[utf8]{inputenc}
  \usepackage{textcomp} % provide euro and other symbols
\else % if luatex or xetex
  \usepackage{unicode-math} % this also loads fontspec
  \defaultfontfeatures{Scale=MatchLowercase}
  \defaultfontfeatures[\rmfamily]{Ligatures=TeX,Scale=1}
\fi
\usepackage{lmodern}
\ifPDFTeX\else
  % xetex/luatex font selection
\fi
% Use upquote if available, for straight quotes in verbatim environments
\IfFileExists{upquote.sty}{\usepackage{upquote}}{}
\IfFileExists{microtype.sty}{% use microtype if available
  \usepackage[]{microtype}
  \UseMicrotypeSet[protrusion]{basicmath} % disable protrusion for tt fonts
}{}
\makeatletter
\@ifundefined{KOMAClassName}{% if non-KOMA class
  \IfFileExists{parskip.sty}{%
    \usepackage{parskip}
  }{% else
    \setlength{\parindent}{0pt}
    \setlength{\parskip}{6pt plus 2pt minus 1pt}}
}{% if KOMA class
  \KOMAoptions{parskip=half}}
\makeatother
\usepackage{xcolor}
\usepackage[margin=1in]{geometry}
\usepackage{graphicx}
\makeatletter
\def\maxwidth{\ifdim\Gin@nat@width>\linewidth\linewidth\else\Gin@nat@width\fi}
\def\maxheight{\ifdim\Gin@nat@height>\textheight\textheight\else\Gin@nat@height\fi}
\makeatother
% Scale images if necessary, so that they will not overflow the page
% margins by default, and it is still possible to overwrite the defaults
% using explicit options in \includegraphics[width, height, ...]{}
\setkeys{Gin}{width=\maxwidth,height=\maxheight,keepaspectratio}
% Set default figure placement to htbp
\makeatletter
\def\fps@figure{htbp}
\makeatother
\setlength{\emergencystretch}{3em} % prevent overfull lines
\providecommand{\tightlist}{%
  \setlength{\itemsep}{0pt}\setlength{\parskip}{0pt}}
\setcounter{secnumdepth}{-\maxdimen} % remove section numbering
\ifLuaTeX
  \usepackage{selnolig}  % disable illegal ligatures
\fi
\usepackage{bookmark}
\IfFileExists{xurl.sty}{\usepackage{xurl}}{} % add URL line breaks if available
\urlstyle{same}
\hypersetup{
  hidelinks,
  pdfcreator={LaTeX via pandoc}}

\author{}
\date{\vspace{-2.5em}}

\begin{document}

\subsection{Das Neue Buch}\label{das-neue-buch}

Dann habe ich ihn endlich in Wien aufgesucht, den Professor. Das war es,
wozu mir die Logik riet, statt daß ich im Archiv von Jyvväskyla Zeit
vertat. Ich hätte dort auch eine Antwort gefunden. Die mich von uns
\emph{über das Bauhaus direkt zu Schönberg und Mahler} geführt hätte.
Aber logisch wäre es nicht, so vorzugehen. Man kann angeblich die Zeit
vorausbestimmen, indem man an der Sprache abliest und ihrer Entwicklung,
wie sich die Gedankenwelt der Menschen verändern wird; was also die
Sprache ermöglichen wird, zu sein. Dann gehen wir zurück vielleicht bis
zu Beethoven, das reicht. Suchen Gemeinsamkeiten: ein mathematischer
Satzbau, Endreim, Katharsis, fünf Abteilungen in dramatischer Form. Und
hier? Das kann sich nur bewegen, wenn die Worte klassisch gewählt werden
statt nach dem Zufallsprinzip oder hörig einem kleinsten gemeinsamen
Nenner, der nur \emph{ich} heißt. Es muß jemand anders die Fügung
übernehmen, \emph{der nicht ich heißt}. Aber das kann nur das Kind sein,
wie es heranwächst, sich vervollkommnet. Wir wollen die Symbiose in
Frage stellen? \emph{Was, }es gab kein Gleichgewicht? Ich habe \emph{die
Junge Frau Mahler} erst kennengelernt, als sie schon ein altes Weib
geworden wäre mit allem, was menschlich dazugehört. Sie wohnte da in
einer Obhut der Stadt und ging nicht mehr hinaus, nicht in den Park, der
so nah vor den Anstaltstoren stattfand, aber auch nicht im Gehege mit
Hirschen auf und ab, nur in den langen Fluren, Jahr für Jahr, solange
sie laufen konnte. Was blieb jetzt n.~ihr Professor, den ich an ihr so
hoch schätzte? Es war, daß sie über Vergangenes so genau und deutlich zu
lesen schrieb, daß mir jenes Kind, das ich ihr aus Finnland mitgebracht
hatte, kein Geheimnis mehr verbergen konnte, sondern so vertraut wie
dies eigene Buch vor mir war und las in ihm, als hätte ich selbst seine
Seiten beschrieben. Aber es ist an ihr erwachsen geworden.\\
Sie tuts ja auch, schon immer, wie vorher auch jetzt. Doch sie
beschreibt nicht. Sie gibt Fügungen zu, die sie mich ahnen läßt. So
werden Gegenstände ohne Schatten ohne Gewicht ohne Ausdehnung in einem
wie unumgrenzten Raum geschaffen. Aber wenn man von den Grenzen wüßte?
Dann n.~weiter versuchen, etwas zu erkennen? Indem ich ihre Lesart
annehme, mich vielleicht verallgemeinere. Sie sieht darauf mit
geschulter Psychologie wie die Hexe Gertraud/M.U. ihren Traumgörge
umspinnt. Da sind die Fabeln, die ich meine, in einen mhd. Dialekt
verbannt, der immer unzugänglich bleiben wird, weil -- nein, dazu gibt
es \emph{hier} keine Antwort. Es ist nur in den Liedern aufgehoben, was
wir n.~verstehen dürfen. Aber so bleibt sie ganz erhalten und
doch\ldots{} einmal hielt ich sie gegen den Zugwind hoch über die Reling
und fast, fast wäre es so gewesen und sie verschwunden! Doch es warf
mich zurück, etwas n.~Unbekanntes, das ich besser kennen lernen wollte.
Ich wußte es vertraut wie einatmen-ausatmen, aber ohne meine
Notwendigkeit, nur, um sich selbst aufrechtzuerhalten. Es gab diese
Strukturen da, wenn ich hinaussah und dann in die Blätter, daß sich die
Wellenbewegungen auf das Notenpapier übertrugen und wenn ich etwas
nachgab, wurden die Töne sichtbar. Sie hatten ja selbst Angst! Wozu also
die Atemnot, unsere. Und der plötzliche Aufschlag der Augen. Was heben
wir ihn n.~und senken aber unseren Brustkorb dabei, als wenn wir selbst
lebten. Doch das Kind, nur das Kind tut es und es lauscht von den Türen
her den offen gelassenen, steht angelehnt draußen, das weiß ich.
Irgendwann wird es hier sein.\\
aber es gibt doch gar kein kind, weißt du das nicht?\\
Wie alt würde es sein, darf ich nicht denken. Wann läßt es das andere,
wirkliche zu? \emph{Ich }gebe dem echten jetzt Stunden in Rechnen und
Philosophie. Wir siezen uns erneut, nachdem wir für eine Weile ein Du
versucht hatten. Wir lachen auch. Nie aus Verlegenheit. Ich denke, das
Kind denkt physikalisch. Und weiß es gottseidank n.~nicht. Ich nenne ihm
Zentrum und Peripherie seines Systems, es kann etwas damit anfangen und
füllt die fehlenden Koordinaten aus. Ahnt aber nicht, wohin es geführt
wird. Ich selbst schon. Seit einiger Zeit versucht es endlich, Noten zu
lesen, versucht es, Stimmen auseinanderzuhalten. Ich kaufte ihm eine
Grammatik des Deutschen, es verkroch sich damit und jetzt schrieb es
Gedichte und die ersten Lieder. Soll ich das erlauben? Dann habe ich ein
weinrotes Heft aus gesammelten A4 Blättern gesehen, worin schon lange
nicht mehr meine Sprache spricht, sondern die vergangener Jahre, über
die ich uns hinwegglaubte. Einsetzen in etwas wollte sie das, wo es
erklären könnte, wie wir hierhergekommen waren, durch die
Membranfenster, von außen nach dem großen Innenbereich. Um die zwei
Stimmen auseinanderzuhalten, des äußeren, erstgenannten, das sich
entwickelt und des Innern, das bleibt, wie es geschaffen ist: Engel,
fliehendes Element, und unfertiges, Stückwerk gebliebenes. Doch wie
immer soll man dort hinhören, dort doch hinhören, wo alle Töne
eigentlich herkommen, durch den Urspalt, den doppelt versuchten. Wenn
das Bild entsteht, erwartet, aber unerklärlicher jedesmal für uns, doch
innen sieht man es dann muß man etwas glauben, irgendetwas: und wenn es
das Motiv gibt? bleibt die Trope ohne Möglichkeit einer Antwort. Darum
hören wir nichts mehr vom Urknall, jener anderen sphärischen Erklärung.
Es dürften Atome übriggeblieben sein, deren Schwingung man hörbar machen
kann\ldots{} der See atmet ein, atmet aus\ldots{} ich trinke das Wasser,
von dem ich geschieden bin. Einmal möchte ich vielleicht n.~zurückkehren
hierher und lasse also etwas übrig in der Flasche. Nur nicht verlieren,
sie nur nicht aus den Augen geben. Verschließen mit dem Siegel Luft,
dicht gegen das Berlin gesichert, seine Verwirrungen; ich werde beten
müssen, um es aufbrechen zu können. Aber der See hat doch geatmet, sah
ich n.~einmal hin in meinen Gedanken bis zum Stein hinüber, flach um
darauf ein paar Jahre verbringen zu können im Gebet,* nur Milch und das,
die einer Mulde im Stein entsprang. *Luft? Milchatem: säuerlich im
mittleren Alter, Zähne knirschen über der dampfenden Teerlache. Das
Dogma mit dem Lungenkrebs, den man so gerne vermieden gewußt hätte. Ich
bleibe mir und das Land erhalten, dem zu dienen ist und man wird nicht
errettet, es geht jetzt in den Krieg\ldots{}\\
Aber es geriet ihm keine Entscheidung über seine Herkunft. Deshalb gab
ich ihm jetzt, ihr, den Namen. Ein Mädchenname hörte nicht auf, zu
klingen, hier mußte es sein, Ewa. Vielleicht der erste Name, der ihm
wirklich gehört hatte. Also fing ich an, sie aus Helsinki zu rufen. Und
meistens kam sie, weil ich doch dicht mit ihr lief, in dem aufwachenden
Hafen umher, klare Luft umgab uns, schon ein Geruch von dem frischen
Fisch des Marktes, Kaffee, Morgengebäck. Ich bin in Finnland angekommen,
meine Seele jauchzt, meine Seele. Du kamst mit mir hierher und jetzt
müssen wir los. Ich werde vorgehen und sie wird folgen, Ewa, die voller
Süden ist und immer von Gelingen getrieben. Es gab nicht den Rückzug
oder die, wenn sie nachdachte, abschweifenden Versuche, das Andere zu
fassen, vielleicht doch mich neben ihr, von dem sie absah, aber sonst
nicht viel. Wenn ich sie doch retten könnte und etwas gelänge, wie, sie
irgendwann daran zu hindern, nachts vom Schiff zu springen; sie wäre mir
nicht näher geblieben oder offener, aber still wie ein kleines Meer.

\end{document}
