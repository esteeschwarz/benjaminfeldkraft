% Options for packages loaded elsewhere
\PassOptionsToPackage{unicode}{hyperref}
\PassOptionsToPackage{hyphens}{url}
%
\documentclass[
]{article}
\usepackage{amsmath,amssymb}
\usepackage{iftex}
\ifPDFTeX
  \usepackage[T1]{fontenc}
  \usepackage[utf8]{inputenc}
  \usepackage{textcomp} % provide euro and other symbols
\else % if luatex or xetex
  \usepackage{unicode-math} % this also loads fontspec
  \defaultfontfeatures{Scale=MatchLowercase}
  \defaultfontfeatures[\rmfamily]{Ligatures=TeX,Scale=1}
\fi
\usepackage{lmodern}
\ifPDFTeX\else
  % xetex/luatex font selection
\fi
% Use upquote if available, for straight quotes in verbatim environments
\IfFileExists{upquote.sty}{\usepackage{upquote}}{}
\IfFileExists{microtype.sty}{% use microtype if available
  \usepackage[]{microtype}
  \UseMicrotypeSet[protrusion]{basicmath} % disable protrusion for tt fonts
}{}
\makeatletter
\@ifundefined{KOMAClassName}{% if non-KOMA class
  \IfFileExists{parskip.sty}{%
    \usepackage{parskip}
  }{% else
    \setlength{\parindent}{0pt}
    \setlength{\parskip}{6pt plus 2pt minus 1pt}}
}{% if KOMA class
  \KOMAoptions{parskip=half}}
\makeatother
\usepackage{xcolor}
\usepackage[margin=1in]{geometry}
\usepackage{graphicx}
\makeatletter
\def\maxwidth{\ifdim\Gin@nat@width>\linewidth\linewidth\else\Gin@nat@width\fi}
\def\maxheight{\ifdim\Gin@nat@height>\textheight\textheight\else\Gin@nat@height\fi}
\makeatother
% Scale images if necessary, so that they will not overflow the page
% margins by default, and it is still possible to overwrite the defaults
% using explicit options in \includegraphics[width, height, ...]{}
\setkeys{Gin}{width=\maxwidth,height=\maxheight,keepaspectratio}
% Set default figure placement to htbp
\makeatletter
\def\fps@figure{htbp}
\makeatother
\setlength{\emergencystretch}{3em} % prevent overfull lines
\providecommand{\tightlist}{%
  \setlength{\itemsep}{0pt}\setlength{\parskip}{0pt}}
\setcounter{secnumdepth}{-\maxdimen} % remove section numbering
\ifLuaTeX
  \usepackage{selnolig}  % disable illegal ligatures
\fi
\usepackage{bookmark}
\IfFileExists{xurl.sty}{\usepackage{xurl}}{} % add URL line breaks if available
\urlstyle{same}
\hypersetup{
  hidelinks,
  pdfcreator={LaTeX via pandoc}}

\author{}
\date{\vspace{-2.5em}}

\begin{document}

\subsection{Teil A: Der Anfang der
Philosophen}\label{teil-a-der-anfang-der-philosophen}

So kommen wir aufs Land. Ich übte mit ihr das Farbenlesen. Zuerst der
leise Regen, der sich von selbst erklärte. Über dem See stand eine
Hälfte des Himmels grau und dicht, nach rechts hin hell, luftig. Von
ungefähr vorn her das Gewitter und gegen die Bäume unter dem lichten
Westhimmel sich abhebend die fast schwebenden Tropfen Wehendes, einfach
nur heruntergewehtes Wasser. Zum Grund hin, zur leicht bewegten Fläche
des Sees. Dort entstand die größte Farbe: \emph{paakkuunylla. nach dem
regen, fast glattes wasser, die sonne eine baumhöhe }n.* über der
waldkante. \emph{Der nächste Ton ist schon schwieriger, dafür
deutlicher, klarer: }pakasitta.* \emph{nebel geht über den see unter der
sonne hindurch. }Dann, nur kurz zu hören, weil jetzt die Sonne hinter
den Bäumen versinkt: \emph{tesantti. blutgefärbte wolken. }Irgendwann
Nacht, bald weite Nacht, durch die ich mit ihr verbunden war. Helle
Flächen darin vom restlichen Licht, die sich von Tag zu Tag
verkleinerten, je länger \emph{juhannus} zurück lag. Und dann ein
Rauchfeuer, das vorbereitet war, an der richtigen Stelle, Tabak rauchen
zu lernen. Neu zu lernen, darauf käme es an, erklärte ich ihr. Als
Mädchen sagte sie mir, daß wir heiraten müßten, weil sie mich so gerne
mochte. Es lag in der Stimme, daß das ihre einzige Möglichkeit war, mich
an diesen See zu binden. Auch wenn wir jetzt gemeinsam hergekommen waren
und alle Wege gemeinsam liefen, bin ich es, wenn sie schläft, der
n.~hier sitzt und sie anschaut. Das bleibt als Blick in den schwarzen
Waldstreifen erhalten, nur dort. Und wenn sie am Morgen aufwacht, bin
ich es immer n., der dorthin kuckt. Sie läßt die Ahnungen gelten, die
sich ihr von da ergeben. Ich möchte nichts mehr gezeigt bekommen, bittet
sie mich, und: du vergißt, daß alles Kinder sind, die nach uns kommen.
Also laß doch mich zuletzt sein. Wenn hier die Zeit vorüber ist und wir
uns in der Stadt in Berlin wiedertreffen, wo ich wohne, werde ich bei
dir zu Hause sein und du mein Gast. Doch sie kann mich schon nicht mehr
sehen, alle Augenfehler ihres Drittel Lebens verstärken sich gegenseitig
in dem Moment, als sie vom Feuer aufblickte mir zu. Ich saß aufrecht an
einen Baum gelehnt und konnte hören, wie es nach ihr schnappte. Ich
trank aus einem Becher Tee. \emph{Sie} war das Leuchten und sie war auch
die Insel, doch vor allem war sie Wald. Hier nicht, jetzt n.~nicht, wo
sie mich nicht sehen kann. Doch ich hatte mich schon nach ihr umgedreht.
Ein Streifen Hell lag n.~im Ausschnitt, dann bald nicht mehr, und es
brach an die Nacht. Finnische Nacht, vielleicht, daß es dann nicht mehr
hell werden würde, wenn ich aufwachte, damit mußte ich rechnen. Doch
hier lebte ich. \emph{Hier ist des Säglichen Zeit.}\\
Die Stelle hatte ich markiert, wo sie in den Wald verschwunden ist; ein
kleiner Gingkobaum, der mir jetzt immer n.~den Eingang zeigt. Da mußte
ich hinein, wenn ich zu ihr wollte, versuchte es mehrmals. Aber ich bin
lange nicht dort gewesen. Es gibt keinen Weg, keine Richtung, die ich
einschlagen kann. Nur, daß der Schritt von draußen nach dem Dunklen
getan wird, von einer Seite in die andere, dann langsames wiedersehen
mit den Nachtfalteraugen und etwas öffnet sich, ich habe es schon
erwartet. Ich kann ihr Raucharoma riechen im Haar, das sie mir wie eine
Glut entgegenhält, wo bist du gewesen. Im Wald, nur im Wald und in die
Bäume geklettert, wenn ich Angst hatte, oben sah ich über die Seen und
jeder See hatte seine eigenen Geister, die nachts darauf entlanggingen
von Stein zu Stein, \emph{von Seele zu Seele. }Dann raucht es, weißt du,
das rieche ich. Es ist, als wenn der See atmen würde und tief unten ist
Feuer. Vielleicht ist es wirklich so. Es gibt eine Legende: Der Fürst
verbannte seine jüngeren Brüder in die sechs Weltgegenden der Alten
Zeit. Einen in den Norden, einen nach Westen, einen nach Süden, einen
nach Osten, einen in den Himmel und einen unter die Erde. Weil aber die
Schwester der Sieben sein Wohlwollen hatte, erlaubte er ihr, Zuflucht in
den Wassern zu suchen, die die Reiche miteinander verbanden. So geht
jetzt die Sage, daß in jenen Nächten des Landes, wenn die Sonne nicht
mehr untergeht, die vier Himmelsrichtungen der oberen Welt mit der
Unterwelt vereint sind, weil die geliebte Schwester vom Bruderfürsten
das Versprechen erhalten hat, so lange nicht die Wasser hüten zu müssen,
wie sie nicht in sie eintaucht. Dann verlassen überall ihre Gesandten
die Seen und das Meer, um auszuströmen und die Nachrichten zu
überbringen von einem Bruder zum anderen, die sie gesammelt hat. Und das
Feuer steigt der Erde durch das Wasser herauf und verteilt sich mit dem
Wind in alle Richtungen an den Himmel, bis die Sonne schließlich unter
dem Horizont verschwindet. Deshalb hören wir in dieser Zeit alle
möglichen Stimmen und sehen manche Geister über den Wassern aufschweben.
Deshalb ist Gesang überall im Wald und in der Nacht, weil \emph{sie}
nicht schweigen muß.

\end{document}
