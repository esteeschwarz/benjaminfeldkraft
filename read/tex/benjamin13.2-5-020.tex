% Options for packages loaded elsewhere
\PassOptionsToPackage{unicode}{hyperref}
\PassOptionsToPackage{hyphens}{url}
%
\documentclass[
]{article}
\usepackage{amsmath,amssymb}
\usepackage{iftex}
\ifPDFTeX
  \usepackage[T1]{fontenc}
  \usepackage[utf8]{inputenc}
  \usepackage{textcomp} % provide euro and other symbols
\else % if luatex or xetex
  \usepackage{unicode-math} % this also loads fontspec
  \defaultfontfeatures{Scale=MatchLowercase}
  \defaultfontfeatures[\rmfamily]{Ligatures=TeX,Scale=1}
\fi
\usepackage{lmodern}
\ifPDFTeX\else
  % xetex/luatex font selection
\fi
% Use upquote if available, for straight quotes in verbatim environments
\IfFileExists{upquote.sty}{\usepackage{upquote}}{}
\IfFileExists{microtype.sty}{% use microtype if available
  \usepackage[]{microtype}
  \UseMicrotypeSet[protrusion]{basicmath} % disable protrusion for tt fonts
}{}
\makeatletter
\@ifundefined{KOMAClassName}{% if non-KOMA class
  \IfFileExists{parskip.sty}{%
    \usepackage{parskip}
  }{% else
    \setlength{\parindent}{0pt}
    \setlength{\parskip}{6pt plus 2pt minus 1pt}}
}{% if KOMA class
  \KOMAoptions{parskip=half}}
\makeatother
\usepackage{xcolor}
\usepackage[margin=1in]{geometry}
\usepackage{graphicx}
\makeatletter
\def\maxwidth{\ifdim\Gin@nat@width>\linewidth\linewidth\else\Gin@nat@width\fi}
\def\maxheight{\ifdim\Gin@nat@height>\textheight\textheight\else\Gin@nat@height\fi}
\makeatother
% Scale images if necessary, so that they will not overflow the page
% margins by default, and it is still possible to overwrite the defaults
% using explicit options in \includegraphics[width, height, ...]{}
\setkeys{Gin}{width=\maxwidth,height=\maxheight,keepaspectratio}
% Set default figure placement to htbp
\makeatletter
\def\fps@figure{htbp}
\makeatother
\setlength{\emergencystretch}{3em} % prevent overfull lines
\providecommand{\tightlist}{%
  \setlength{\itemsep}{0pt}\setlength{\parskip}{0pt}}
\setcounter{secnumdepth}{-\maxdimen} % remove section numbering
\ifLuaTeX
  \usepackage{selnolig}  % disable illegal ligatures
\fi
\usepackage{bookmark}
\IfFileExists{xurl.sty}{\usepackage{xurl}}{} % add URL line breaks if available
\urlstyle{same}
\hypersetup{
  hidelinks,
  pdfcreator={LaTeX via pandoc}}

\author{}
\date{\vspace{-2.5em}}

\begin{document}

\subsection{III. Ton}\label{iii.-ton}

Wann habe ich angefangen, über den See hinauszugehen? Ich begann, Fragen
zu stellen: Arbeitest du wieder? Du mußtest es verneinen, weil nur
passive Strukturen geschaffen worden waren. Ich redete dir vom See, ich
sprach deine Sprache und dachte über deinen Wald nach, so sehr, wie ich
n.~nie über etwas anderes nachgedacht hatte. Ich dachte so sehr tiefes
darüber, daß mir das selbst wie ein Element erschien, aus dem ich
schöpfen konnte. Der Wald, der echte Wald ich hatte ihn ja kaum gesehen.
Ich hatte mich sofort verlaufen darin, das stimmt, ich hatte richtige
Angst und bin panisch umhergestolpert, bis endlich ein Weg sichtbar
wurde, der mich hinausführte. Aber bei dir sollte es sich ja wirklich um
Noten handeln, Blätter aneinandergeheftete ohne Überschriften, nur aus
einer Stimme ablesbar das Dickicht dieser Sinfonie, der hohe Herr und
sein Manuskript erster Hand, das wir hier vor uns haben. Versuch doch,
was damit anzufangen, das ihm gerecht wird. Du weißt, daß du sie
fertigschreiben sollst. Du hast das Material zur Verfügung, das nicht
altern kann, nur älter wird. Das sollen zwei Jahre sein, die sich
abheben. Wenn es größer wird als du selbst, halte es fest, so lange du
kannst, bevor du es weggibst und weggeben mußt du es, aber so lange nur,
wie du es zu finden brauchst\ldots{} weggeben mußt du etwas, weil der
Aufbau in Stufen verlangt, daß wir uns trennen. Es ist soweit..- Es
werden Namen fallen. Über dem Wasser werden jetzt Namen ausgesprochen.\\
Aber n., ich: denke n.~immer über den Wald nach; ich habe Angst. Es war
ein Märchenwald glaube ich, der selbst durchlebt werden wollte, seine
Geschichten sind nicht anders als mit dem Körper erlernbar. Die Psyche
allein, die immer neue ewige Bettlerin, wenn ich auf den Weggang schaue,
steht sie später an der selben Stelle fest, als hätte ich ihr nicht
längst Tribut gezahlt. Sie kann mir nichts mehr beibringen über die Welt
und den Wald soll ich ertragen, niemand kann ihr das abnehmen. Aber
langsam kenne ich mich aus. Wenn es über Abend geht, haben sich in den
Fußspuren genug Tropfen des Nieselregens gesammelt, daß ich mein Gesicht
waschen kann. Auf dem See: Hier ist jetzt Wind, ein großer, starker
Wind, der über das Wasserdunkel hinzieht und die Birken rauschen, die
dünnen, und die Pappeln, ein paar Kiefern stehen steif und
widerspenstig. Am Ufer schlagen Wellen an die Findlingssteine, das Boot
schaukelt, halb an Land liegend, man glaubt: wie an Worte. Das kann man
hören; was es heißen könnte, hier wird es angeschwemmt. Auch kleine Seen
haben ihr Treibgut. Ich warte also, immer sitzend, immer mit dem
Ausblick zwischen den Birken hindurch, wo das alles herkommt. Es gibt
natürlich diesen Ursprung auf irgendeiner Seite oder die Insel, deren
Bäume ein paar Schatten ins Wasser werfen. Aber was davon übrig bleibt
zu hören, wenn ich das wacklige Ruderboot betrachte, das sind die
quietschenden Riemen und ein glucksender Hohlraum, ganz bestimmte Töne.
Vielleicht habe ich ja etwas davon aufgenommen? Ich kann mich nur an den
Regen erinnern, immer stärker und barfuß auf dem nassen Steg. Und an
Ewa, schlafend. Sie bewegt sich. Ich möchte sie weiter anschauen, doch
dreht sie sich gerade um. Ein kleines Ohr, das in die Nacht hinaussteht
und alles wahrnehmen kann, längst mehr, als ich n.~jemals vorstellen
möchte. Darum bleibe ich ruhig und ich weiß, daß sie mein Herz trotzdem
schlagen hört und, wenn es schneller geht, unruhig wird. Doch das ist
nur der schwarze Tee. Ich will einen Blick auf sie tun, stehe von der
Bank auf und setze mich neben dem Kopfende ihres Bettes auf einen
Klappstuhl. Man blickt herunter, ein wenig Mond schien herein? Nein, es
war die umsonst hell gebliebene Nacht des Sees, mit der er mich dazu
bringen wollte, hier zu bleiben. Doch das ist her und vielleicht wird es
einmal doch Mond gewesen sein, der ins Fenster schien und ein paar
Schatten machte. Ich konnte das kleine Ohr kaum sehen zwischen Kissen
und Decke, wie es lauschte aus dem Schlaf in mein Herz hinein. Aber als
es hörte, wie das schlug und daß es ihm gut erging, war es ruhig und
schlief bis morgens. Dann war ein neuer Tag und die Geschichten dieses
Tages waren die von morgen, weil morgen immer der nächste Tag gewesen
ist.\\
Dann ist ja dein Morgen. Und kein Tag durfte neu entstehen und sagen,
daß er es in Wirklichkeit, der wahre Tag und einzige sei. Warum also
nicht den Moment doch in den Tag danach legen und hier, denn \emph{Das}
\emph{Zeitschloß hier ist des Säglichen Zeit}, die Nacht meinen nur sie.
Bis ich aufgewacht war und sie im Hemd über Rühreiern hantieren sah. Es
gab wenig zu essen, aber Gutes, und Kaffeee und abends Fische oder
Pilze. Den Tag über, wenn ich für sie schrieb, nachdem die Fische
gefangen und die Pilze gesammelt waren, saß ich draußen auf der Veranda
des kleinen Holzhäuschens. Den Tag und dann die Nacht, aber den Tag
eben, so wichtig, wie er war. Sie benutzte ihre Geige sonst nur für
Bruch. Ich blieb fast lautlos außer dem Rascheln des Bleistifts,
schnell, langsam, Pausen im Gehirn. Fallender Regen. Das Waldstück, wie
es da rauscht. Wenn eine Tür zugeschlagen wurde vom Wind, erschrak ich.
Immer arbeiten, die Schriften sortieren, überall etwas finden wollen für
sie, weil sie die Noten ja gesehen haben mußte, bevor ich sie
zurückerhielt. Also suchte ich fortwährend zwischen allen Briefen und
Manuskripten das Matrial, das wichtig gewesen war und das man nicht
verändert hatte, so, daß sie weiter lernen konnte. Es gab jetzt ein
zweites Heft, in dem sie die erste Notation umging. Sie hatte meine
Handschrift schnell gelernt und so konnte ich fast alles sofort hören,
was ich geschrieben hatte. Es war der Anfang der Übungen und der Rest
n.~nicht auf Papier. Sie war die einzige, die es jemals so gespielt hat.
Daß sie irgendwann taub geworden war halte ich für den wirklichen Grund
ihres Falls, doch bis dahin waren es die Tage mit jener ersten Musik,
die sie mir schuf. So, daß mir ihr kleines Leben, wenn ich von hier aus
darauf blicke, wie alles erschien, das mir jemals wichtig sein würde,
wichtig gewesen sein konnte, weil nichts anderes mehr Zeit bedeutete,
die auf einmal verloren wäre, wenn ich sie verlor. Und nur die Zeit dann
war, was zählte. Aber ich erinnere mich an den Codex und breche statt
dessen meine Nachtstunden an, wir frieren kaum mehr an der Nordgrenze
des Lichtes. Das Geigenspiel hält sie warm, glaube ich und mich meine
Erwartung eben des Morgen, in das ich alles stelle. Vielleicht zu viel,
ich könnte einsam daran werden, nur dem Morgen zu leben, wartend auf ein
Wort, das die Musik preisgibt. Ein Wort wie: Holz. Weil es nicht klingt
und selbst nur als Körper zu gebrauchen ist für ein anderes: Das könnte
damals das Kupfer gewesen sein, aber auch dieses nur Mantel für einen
Eisenkern. Wo ist das Wort? Ich sehe etwas von Modulation und
Amplituden, man rechnet eine Spannung zusammen, \emph{doch wo ist das
Wort?} Wild jage ich hin und habe vielleicht Gold gefunden, ich rechne
es mir aus aber woraus soll das Wort entstehen? Muß ich lesen? Doch die
Buchstabenbücher sind so schwer mit ihrem schwarzen Geschwafel, daß ich
anfange, in den Notizen zu schmökern. Dann gelingt es, eine Zeile zu
entziffern des Blaustifts, ihrer dünnen Schrift, die Seite für Seite
versucht, ein Rätsel zu lösen, das ich n.~gar nicht kenne. Ich kann so
auch nicht sagen, ob die Versuche ihrer Hebungen richtig sind, spüre
nur,\\
nach einer halben Seite von verloren geglaubten Wörtern, die irgendwo
sich in dieses Gestrüpp dort der Scham geflüchtet haben weil sie nicht
verraten werden wollten jene hierher verschlagenen Gedanken ihnen
nachschicken: macht auch sie nicht lebendig; genausowenig das Geheime,
das sie mitnahmen. In ihre Abwesenheit, in ihrer unendlichen Scham. Das
ist kein Versteck, das ist versunken gewesen, was wir dort raufholen. Es
bleiben ein paar Zeilen n., bis die vergangene Seite entdeckt ist. Was
tue ich, wenn ich das wiedersehe, was auf die Rückseite verschwommener
Tafeln geschrieben wurde. Was immer nur für den Lehrer und niemals für
den Schüler bestimmt war, gesehen zu werden? Die Arbeit besteht darin,
die Mauer im Kopf nicht mehr zu denken. Aber nicht, um sie durchbrechen
zu können: sondern, um sie sich in der Welt manifestieren zu lassen, so
daß

\end{document}
