% Options for packages loaded elsewhere
\PassOptionsToPackage{unicode}{hyperref}
\PassOptionsToPackage{hyphens}{url}
%
\documentclass[
]{article}
\usepackage{amsmath,amssymb}
\usepackage{iftex}
\ifPDFTeX
  \usepackage[T1]{fontenc}
  \usepackage[utf8]{inputenc}
  \usepackage{textcomp} % provide euro and other symbols
\else % if luatex or xetex
  \usepackage{unicode-math} % this also loads fontspec
  \defaultfontfeatures{Scale=MatchLowercase}
  \defaultfontfeatures[\rmfamily]{Ligatures=TeX,Scale=1}
\fi
\usepackage{lmodern}
\ifPDFTeX\else
  % xetex/luatex font selection
\fi
% Use upquote if available, for straight quotes in verbatim environments
\IfFileExists{upquote.sty}{\usepackage{upquote}}{}
\IfFileExists{microtype.sty}{% use microtype if available
  \usepackage[]{microtype}
  \UseMicrotypeSet[protrusion]{basicmath} % disable protrusion for tt fonts
}{}
\makeatletter
\@ifundefined{KOMAClassName}{% if non-KOMA class
  \IfFileExists{parskip.sty}{%
    \usepackage{parskip}
  }{% else
    \setlength{\parindent}{0pt}
    \setlength{\parskip}{6pt plus 2pt minus 1pt}}
}{% if KOMA class
  \KOMAoptions{parskip=half}}
\makeatother
\usepackage{xcolor}
\usepackage[margin=1in]{geometry}
\usepackage{graphicx}
\makeatletter
\def\maxwidth{\ifdim\Gin@nat@width>\linewidth\linewidth\else\Gin@nat@width\fi}
\def\maxheight{\ifdim\Gin@nat@height>\textheight\textheight\else\Gin@nat@height\fi}
\makeatother
% Scale images if necessary, so that they will not overflow the page
% margins by default, and it is still possible to overwrite the defaults
% using explicit options in \includegraphics[width, height, ...]{}
\setkeys{Gin}{width=\maxwidth,height=\maxheight,keepaspectratio}
% Set default figure placement to htbp
\makeatletter
\def\fps@figure{htbp}
\makeatother
\setlength{\emergencystretch}{3em} % prevent overfull lines
\providecommand{\tightlist}{%
  \setlength{\itemsep}{0pt}\setlength{\parskip}{0pt}}
\setcounter{secnumdepth}{-\maxdimen} % remove section numbering
\ifLuaTeX
  \usepackage{selnolig}  % disable illegal ligatures
\fi
\usepackage{bookmark}
\IfFileExists{xurl.sty}{\usepackage{xurl}}{} % add URL line breaks if available
\urlstyle{same}
\hypersetup{
  hidelinks,
  pdfcreator={LaTeX via pandoc}}

\author{}
\date{\vspace{-2.5em}}

\begin{document}

\subsection{FUNKEN}\label{funken}

Ich habe Grund erreicht. Ich glaube, jetzt lebe ich. Kälte und Blitze.
Der Dachraum über dem Haus. Hier werde ich geboren, möchte ich geboren
worden sein, weil es der Ort war, der sich höher über dem Meeresspiegel
befand, als andere hätten liegen können. Es beginnt meine Geschichte mit
einem Blitzschlag. Vielleicht, daß es besser gewesen wäre, den See zu
verlassen als das Gewitter anfing. Doch wie soll ich mich daran
erinnern, gefangen immer n.~in dem Magnetfeld der Entladung? Dem Ich
etwas senden, das es zur Vernunft bringt und die mit Saughaltern an den
Schläfen befestigten Drähte nicht mehr für Elektroden hält, sondern für
das, was sie sind, Abnehmer zum Messen der Hirnströme. Aber glaube ich
ihm das? Ich war im Boot, das weiß ich n.. Und daß Strom floß auf
einmal, innen oder außen war nicht sicher, aber es war Bewegung da der
Elementarteilchen. Das entnehme ich den Schwingungen der
Seewassermoleküle. Es sind nicht mehr viele, aber jedes hinterließ eine
Spur, als es zerfiel. Und eine größere Spur zieht sich dahin, einmal
gemeinsam über das Wasser getragen aus dem Seenland, wo uns das Feuer
entzündet wurde. Warten auf den Funken jetzt in späterer Zeit, der hier
etwas anstecken könnte. Ein Stein nur, Stein, der an Stein schlägt, der
große Beweger. Warten. Dann: wie der Stein selbst wartet, der mir aus
der Hose gefallen ist vom Steg ins ja flache Wasser, oder wollte ich ihn
nicht fallen lassen und habe die Hand auch über den Stegrand
ausgestreckt? Grünlich eckig ein wenig Würfel mit Äderchen lebendigen,
den ich beim Sieben des Komposthaufens gefunden hatte. Hätte ich ihn
doch bewußt dort gelassen wie an einem guten Grab. Habe ich\ldots, habe
ich doch.\\
Statt zu warten die Fahrt: vollgeschriebene Blätter, die keine Note mehr
ertragen konnten; über das kleine Meer nur ein wenig Melodie n.~imstande
zu bewegen, das andere war ausgefüllt. Ich schaue zwischen den Streben
der Reling hindurch auf das Wasser, dann Aufblicken und die
weiterfließende Bewegung, die sich auf alles Betrachtete überträgt,
Schiffsboden, die eigene Hand, der Mensch neben mir alles flüssig, unter
den gespeicherten Wellen Elektronenstrahlen \emph{Partikelgestöber
}dringt ins Manuskript. Dagegen muß ich mich wehren und setze meinen
Blick neu an. Initiale. Das bleibt relativ, unbesehen läßt die Hand das
nicht geschehen. Wenn man die Wellenberg und Talfahrt beibehält, fangen
die Augen selbst zu schwimmen an, ich hatte das erlebt. Erst unter Druck
gleicht sich das Gehirn den tatsächlichen Bedingungen an. Innen
angelangt auf den Rücken der Augen gleicht das Wabern der äußeren
Zustände so sehr den zellimmanenten Strukturen, daß keine Verwandlung
mehr nötig ist in elektrische Plus und Minus, \emph{thoughts derive from
persistently incoming rays of moved matter, constructing and
deconstructing floating images on the adressed recipient.} Zwei Spalte,
begrenzt durch die drei Eisenstäbe, das ist ein großer Doppelspalt.10
Doch das Wasser, nein, es bedingt nicht durch seine Wellenbewegungen,
daß in meinen Augen Inteferenzmuster entstehen, die sich abbilden lassen
auf dem Schirm der eigenen Handfläche, und wer will das einsehen.*
\emph{Nur ein grünlichgrauer, eckig und nicht runder dabei weich und
nicht scharfkantiger Stein, der von dünnweißen Adern und n.~feineren
roten Äderchen durchzogen war, fiel mir aus der Hand, ausgestreckt am
Rand des Stegs über dem sehr klaren Wasser. Ich hatte dieses jeden Tag
getrunken in schwarzem Tee. Es hielt mich wach. Wenn ich einschlafen
wollte und mich hingelegt hatte zur Nacht, quoll das Blut durch den
Kopf, schwerer und dickflüssiger als sonst und voller Bilder, und
angefüllt mit Lebendigem, das es sich anverwandelt hatte im heiligen
Kreislauf finnischer Wasser. Es schwingt mit den nächtlichen Molekülen
da draußen im Einklang, anderes hob sich ab von der Oberfläche, um als
Nebel gesehen zu werden, Blicke jede Nacht dorthinüber, wo ich sie
zuerst gesehen hatte. Wen. Wen, ach, sehen konnte man viel nachts, Wehen
vielleicht hatte ich wahrgenommen, vom Wasserspiegel ab sich hebendes
gegen den Waldhintergrund, der blieb schwarzgrau bis zur endgültigen
Sonne. Da also Schlieren. Die Nachtabsenkung drückt sie gegen den n.~den
hellen Himmel spiegelnden glatten See herunter, es finden dort Kämpfe
statt zwischen auf und absteigenden Luftschichten, nichts will sich
richtig vermischen mit dem andern, das }Seewasser* ist erhitzt. Das geht
mir durch den Kopf, ich versuche dagegen einzuschlafen, dagegen meinen
Körper mehrmals zu entwässern in den Waldrand vor der Hütte. Es ist
kalt. Als es langsam wärmer wird, schlafe ich ein. Die Wassermoleküle
sind ruhig.

\end{document}
