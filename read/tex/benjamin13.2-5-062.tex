% Options for packages loaded elsewhere
\PassOptionsToPackage{unicode}{hyperref}
\PassOptionsToPackage{hyphens}{url}
%
\documentclass[
]{article}
\usepackage{amsmath,amssymb}
\usepackage{iftex}
\ifPDFTeX
  \usepackage[T1]{fontenc}
  \usepackage[utf8]{inputenc}
  \usepackage{textcomp} % provide euro and other symbols
\else % if luatex or xetex
  \usepackage{unicode-math} % this also loads fontspec
  \defaultfontfeatures{Scale=MatchLowercase}
  \defaultfontfeatures[\rmfamily]{Ligatures=TeX,Scale=1}
\fi
\usepackage{lmodern}
\ifPDFTeX\else
  % xetex/luatex font selection
\fi
% Use upquote if available, for straight quotes in verbatim environments
\IfFileExists{upquote.sty}{\usepackage{upquote}}{}
\IfFileExists{microtype.sty}{% use microtype if available
  \usepackage[]{microtype}
  \UseMicrotypeSet[protrusion]{basicmath} % disable protrusion for tt fonts
}{}
\makeatletter
\@ifundefined{KOMAClassName}{% if non-KOMA class
  \IfFileExists{parskip.sty}{%
    \usepackage{parskip}
  }{% else
    \setlength{\parindent}{0pt}
    \setlength{\parskip}{6pt plus 2pt minus 1pt}}
}{% if KOMA class
  \KOMAoptions{parskip=half}}
\makeatother
\usepackage{xcolor}
\usepackage[margin=1in]{geometry}
\usepackage{graphicx}
\makeatletter
\def\maxwidth{\ifdim\Gin@nat@width>\linewidth\linewidth\else\Gin@nat@width\fi}
\def\maxheight{\ifdim\Gin@nat@height>\textheight\textheight\else\Gin@nat@height\fi}
\makeatother
% Scale images if necessary, so that they will not overflow the page
% margins by default, and it is still possible to overwrite the defaults
% using explicit options in \includegraphics[width, height, ...]{}
\setkeys{Gin}{width=\maxwidth,height=\maxheight,keepaspectratio}
% Set default figure placement to htbp
\makeatletter
\def\fps@figure{htbp}
\makeatother
\setlength{\emergencystretch}{3em} % prevent overfull lines
\providecommand{\tightlist}{%
  \setlength{\itemsep}{0pt}\setlength{\parskip}{0pt}}
\setcounter{secnumdepth}{-\maxdimen} % remove section numbering
\ifLuaTeX
  \usepackage{selnolig}  % disable illegal ligatures
\fi
\usepackage{bookmark}
\IfFileExists{xurl.sty}{\usepackage{xurl}}{} % add URL line breaks if available
\urlstyle{same}
\hypersetup{
  hidelinks,
  pdfcreator={LaTeX via pandoc}}

\author{}
\date{\vspace{-2.5em}}

\begin{document}

\subsection{Das III. Buch}\label{das-iii.-buch}

Im Gang zu den höhergestellten Invaliden, den ich betrete, liegen Bücher
aus auf kleinen Tischen. Einmal nahm ich eines mit auf mein Zimmer, als
niemand mich sehen konnte. Es war ganz das falsche Buch, merkte ich
später. Aber ich hatte mich aus einem mir bis jetzt unbekannten Grund
dafür entschieden und versuche, herauszufinden, warum es mir so
schwerfiel, es zu lesen. Es war vielleicht ein Sachbuch gewesen oder
eine kleine Geschichte der Musik im 20. Jh.., jedenfalls kam darin vor
ein Bericht über eine unbekannt gebliebene Komponistin Ewa Laplace, die
nur deshalb namens erwähnt wurde, weil sie bei dem Versuch, die 10. von
Mahlers Sinfonien aus den erhaltenen Versatzstücken zu rekonstruieren
auf etwas gestoßen ist, das der Autor scheinbar für wichtig gehalten
hatte, uns mitzuteilen. Es handelte sich dabei um eine Bekenntnisschrift
in Notenform, die angab, warum Mahler selbst sich nicht mehr dazu in der
Lage sah, seine von ihm als erstes Echo seines lebendigen L.n.ms
vernommenene Musik zu vollenden: es hieß, er hätte die ihm verbleibende
Zeit vom Bewußtseines nahenden Endes einzig damit verbringen müssen,
alles Vorhergegangene zu vernichten, wenn er die Gestaltung dieser
symphonie absolut ausführen wollte. Darum: es war ganz das falsche Buch.
Man konnte nicht darin gelesen haben, ohne hernach zu bezweifeln, daß es
zu lesen in gewohnter Weise, nämlich der Erfassung von Zusammenhängen
durch die Rekombination von Wörtern im eigenen Geist, die der Autor
\emph{in seinem Geist} zusammengefügt hatte, die richtige sein soll, den
Sinn \emph{des} Buches aufzudecken, das man in der Hand hielt. Vielmehr
schlug er mit seinem Verweis auf die Laplacesche Idee von der einzigen
Möglichkeit einer Renaissance von Urmaterial durch die Vernichtung allen
Zuwachses und der resultanten Vermeidung künftiger Forschung ein Kapitel
der Restauration auf, das sich nicht darauf beschränken würde, "nichts
mehr dazu{[}zu{]}tun und in den Anfängen {[}zu{]} verharren", sondern
das den Golem, den es meint, nicht mehr untersuchen zu müssen, ständig
zu schaffen verdammt ist. Das Mädchen Laplace selbst, das Mahler
n.~kannte, verhielt sich zum Gegenstand ihrer Arbeit so konsequent, daß
sie die einzige Quelle, auf die man sich heute berufen könnte, wenn man
weiterarbeiten wollte ihre Interpretation der Auslassung in ihren Tod
mitnahm, auf hoher See. In den Archiven des Aaltomuseums in Jyvväskyla
findet man sie in sechs Karteikarten einer ausleihenden Position:
\emph{Sie schrieb nieder, was sie hörte, sie behielt, was sie wußte und
gab nur weiter, was wahr war.}

\end{document}
