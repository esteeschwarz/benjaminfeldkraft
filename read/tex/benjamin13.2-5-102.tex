% Options for packages loaded elsewhere
\PassOptionsToPackage{unicode}{hyperref}
\PassOptionsToPackage{hyphens}{url}
%
\documentclass[
]{article}
\usepackage{amsmath,amssymb}
\usepackage{iftex}
\ifPDFTeX
  \usepackage[T1]{fontenc}
  \usepackage[utf8]{inputenc}
  \usepackage{textcomp} % provide euro and other symbols
\else % if luatex or xetex
  \usepackage{unicode-math} % this also loads fontspec
  \defaultfontfeatures{Scale=MatchLowercase}
  \defaultfontfeatures[\rmfamily]{Ligatures=TeX,Scale=1}
\fi
\usepackage{lmodern}
\ifPDFTeX\else
  % xetex/luatex font selection
\fi
% Use upquote if available, for straight quotes in verbatim environments
\IfFileExists{upquote.sty}{\usepackage{upquote}}{}
\IfFileExists{microtype.sty}{% use microtype if available
  \usepackage[]{microtype}
  \UseMicrotypeSet[protrusion]{basicmath} % disable protrusion for tt fonts
}{}
\makeatletter
\@ifundefined{KOMAClassName}{% if non-KOMA class
  \IfFileExists{parskip.sty}{%
    \usepackage{parskip}
  }{% else
    \setlength{\parindent}{0pt}
    \setlength{\parskip}{6pt plus 2pt minus 1pt}}
}{% if KOMA class
  \KOMAoptions{parskip=half}}
\makeatother
\usepackage{xcolor}
\usepackage[margin=1in]{geometry}
\usepackage{graphicx}
\makeatletter
\def\maxwidth{\ifdim\Gin@nat@width>\linewidth\linewidth\else\Gin@nat@width\fi}
\def\maxheight{\ifdim\Gin@nat@height>\textheight\textheight\else\Gin@nat@height\fi}
\makeatother
% Scale images if necessary, so that they will not overflow the page
% margins by default, and it is still possible to overwrite the defaults
% using explicit options in \includegraphics[width, height, ...]{}
\setkeys{Gin}{width=\maxwidth,height=\maxheight,keepaspectratio}
% Set default figure placement to htbp
\makeatletter
\def\fps@figure{htbp}
\makeatother
\setlength{\emergencystretch}{3em} % prevent overfull lines
\providecommand{\tightlist}{%
  \setlength{\itemsep}{0pt}\setlength{\parskip}{0pt}}
\setcounter{secnumdepth}{-\maxdimen} % remove section numbering
\ifLuaTeX
  \usepackage{selnolig}  % disable illegal ligatures
\fi
\usepackage{bookmark}
\IfFileExists{xurl.sty}{\usepackage{xurl}}{} % add URL line breaks if available
\urlstyle{same}
\hypersetup{
  hidelinks,
  pdfcreator={LaTeX via pandoc}}

\author{}
\date{\vspace{-2.5em}}

\begin{document}

\subsection{b.}\label{b.}

\emph{Und} wenn sie immer neu die Gelegenheit bekäme, die Geschichte
anders zu schreiben sie würde es nicht tun. Denn sie hat sich so
zugetragen, wie sie euch hier erzählt wird, an diesem Abend, dem letzten
unserer alten Zeit. Was ihr daraus machen werdet, wenn ihr selbst
anfangt, euch die Enden zu gestalten nach eurer Zeit, kann ich mir schon
vorstellen; aber gefallen tut mir das nicht, was ich dann sehe. Doch ihr
seid die Jungen und ich bin Teil eurer Vergangenheit, wenn ihr das hier
lest; so sehr Vergangenheit, wie sie es mir einmal von sich
sagte\ldots{} wenn sich ihr altes Leben plötzlich in die Wiege legt, die
Wiege, wo immer man sich sein Vergangenes heranzog. Vielleicht
wechselten wir manchmal unser Gesicht, mit dem wir hineinschauen;
möglich auch, daß wir das letzte Spielzeug zurückfordern, weil es
danklos uns entrissen wurde und nun bereuen wir fast, es hineingelegt zu
haben, sehend, daß das Kind, dem wir also meinten, das alles zu
schenken, längst nicht mehr dasselbe Unschuldige ist und so ungefreit,
wie wir es zurückließen : da also sich beherrschen und erwachsen
bleiben, da sich über den Trieb erheben und der Aggression Herr werden
und der Reflexe; was möchte man nicht alles dafür geben, selbst schon
genauso weit fortgeschritten und reif zu sein, wie dieses plötzlich alte
Leben in der eigenen Wiege mit den allzu bekannten Augen, das uns mit
unseren jüngeren anschaut und fragt: bin ichs n.? Aber es bleibt nichts
mehr zu tun n.~zu fragen, als erneut sich selbst zu bestätigen, wenn die
erkennenden Gebärden getan sind und die gewohnte Sicherheit (in allem
Umgang) Einzug hält selbst da, wo das Geheimnis war die Chance für einen
Neuanfang. Der war also auch verspielt.\\
\emph{II. Aber sie, fängst wieder neu an} wenn ihr auch alles geheim
bleibt, was sie n.~tun wird in diesem Leben. Das ist wahre Zuversicht.
Es gibt einen Spiegel dafür: man hält ihn vor eines anderen Gesicht,
während der die Augen geschlossen hatte und er öffnete sie ganz
unschuldig. Aber was sieht er dann nur, so lange er n.~nichts von dem
Spiegel weiß, der vor seinen geschlossenen Augen wartet? Ein zerrissenes
Bild in seine Erwartungen und dem horror vacui? Was denn, wenn da nichts
wäre außen drauf auf dem, was ich innen sehe? Und ohne dieses überhaupt
ein Innen vorstellbar das mehr wäre, als ich jetzt wahrnehme, wo ich
n.~an die Maske glauben kann, die die Haut aufrechterhält. "Soweit es
sich um integrierte Körperempfindungen handelt, ist die Haut lediglich
die Umhüllung unseres wahren Selbst und dessen, was in uns ist. Aber in
den tiefen, infantilen Schichten unseres Denkens sind wir nicht ganz
sicher, ob in uns irgendetwas ist, das nicht von außen in uns
hineingesteckt wurde.\textquotesingle{} " Sagen ein paar Psychologen um
1927 in dem Buch der Jungen Frau Mahler, so, daß ich nicht mehr wissen
will, ist dieses Körperding, mit dem ich herumlaufe mein eigenes oder
Zerrbild des bedrohenden Spiegels, der da wartet, immer, wohin ich auch
die Augen öffne. Seht ihrs Freunde, seht ihrs nicht\ldots{}31\\
Aber sieh, fängst wieder Neues an auszuplaudern, was dich so
beschäftigt, ja. Ich geh sie begleiten, kurze Wege vor das Haus, über
die Straße, durch den Park zurück, ein langsames wie wenn man in der
Kirche laufen gelernt hätte, steif, unentspannt, voller Sehnen und
Muskelspürung. Dann will ich sie schubsen zum Aufwachen, eine spontane
Reaktion hervorrufen der Beine und Gelenke, um das Gleichgewicht
herauszufordern. Daß sie nicht verlernt sich aufrecht zu halten. Lasse
es bleiben und dann fallen wir vielleicht beide einmal zufällig, sich am
Knie zu begegnen. Das ist die alte Zeit, jenes die neue, den Schnitt
führe ich gerade durch die Hälfte; wenn die Arbeit getan ist, treffen
sie sich hier in der Mitte zwischen Gut und Böse. Nicht ich, du oder
dieses Kalb da mit dem goldenen Horn\ldots{} der erste Widder der
Menschheit. Das ist abgegolten.\\
Also fang ich an, ihre Geschichte zu erzählen, weil sie sich alle
schämen für sie: Sie hat dafür ihren Traum gehabt von langer Küste,
steil ansteigendem Gefelse: wenn man richtigherum geht, rechts! da
stehen wir dann oben, nur um hinab und weit zu schauen und nennen es Das
Ende der Welt. Wie in einem Bild sieht das da aus vor uns, ist aber ein
echter Weitblick, ohne Rand rechts oder links, n.~oben/unten nur das
blauweiße Himmelsband auf blaudunkles (September) Wassermeer abgelegt.
Wir haben eine Sanddornhecke durchbrochen, und schon wissen wir nicht
mehr, wie es davor war, keine Hecke natürlich, sondern wildes Gesträuch,
das in die Furt ragte. Aber wir kennen den Weg und irgendwann wurde er
gangbar, bevor man dort angelangt ist. In Berlin gab es einen
ebensolchen stillen Flecken, Das Paradies genannt. Wenn Milton ihn so
hätte anschauen müssen, wie ich es tat, in der sehnsüchtigen
Verlassenheit der Stadt dort, gäbe es keine Wiedergewinnung, es müßte
immer verlorenes heißen, für immer verlorenes\ldots{}\\
Und plötzlich ist da was ich höre: ein Erwachen? Schlief ich denn schon
mal? Vom kühlen Fenster ging mich die Nacht an. Ein Grund war da nicht.
Nur immer man wartet, vielleicht ergibt sich was aus der schlaflos
gebliebenen Dunkelheit da draußen. Doch sicher ist es nie, nicht ein
Wort, auf das man sich verlassen könnte. Bastardschemen, \emph{parental
gaps} zwischen meinem Hirnaneinander mit dem täglich erlogenen
Fortschritt; der ist gekauft, jeder neue Buchstabe will Anfang sein.
Will aller Anfang Anfang sein. La Parole: aufstehen, gegen das
sommermüde Blendwerk hoch, anheben ein Lid und das andere ist mir
n.~zugeklebt mit der Honigmilch aus paradise lost/regained und trotzdem
verlorenen Büchern über Mitternacht, wenn jede Minute zählt. Nur bleiben
dürfen für diese heiligen Brautnächte um die erste Erfahrung, auch über
ihren Tod hinaus. Aber auch im neuen Leben, das dann beginnt\ldots{} mit
unfertigen Erinnerungen, die wir uns nicht zu erzählen wagen, sind wir
immer nur \emph{allem Abschied voran.}\\
Die Erzählung: sie hat eine Sabbathnacht lang mit dem Atmen aufgehört,
jener schweren Arbeit, die jetzt eine Maschine für sie tat. So fühlte
sie sich besser, es war keine Anstrengung mehr da, die Maschine bediente
jemand anderes für sie und hatte ein waches Auge, schichtwechselnd. Als
ich am Sonntag wie jeden Sonntag nach der Kirche sie besuchen ging,
bemerkte ich zuerst keine Veränderung, aber ein wenig war die Anspannung
von ihr gewichen, dachte ich. Bis ich sie sprechen hörte: nicht ihre
Stimme, nicht mehr der gewohnte Duktus, kein Wort war wahr, wovon sie
redete. Das war von einem Erdbeerfeld, wo sie arbeiten wollte, wenn
\emph{alle Stricke reißen}. Das war doch längst vorbei, hörst du, du
mußt nicht mehr arbeiten gehen, die Hebammen der Lüfte\ldots{} wenn du
nur wieder gesund würdest. Ich habe ein paar Stunden für sie gebetet
aber ist das im Norden angekommen, wo der große Unfall geschah?\\
Ist es. Weil ich es erzählt habe, nur deshalb mußte sie mir vom Schiff
springen und da ja Hier ist des Säglichen Zeit ist, wo ich mich in aller
Ruhe daran erinnern kann, wie schön die regenquellenden Tagewälder grün
waren, möchte keine Unruhe aufkommen über sie. Sie schlafen nur.
\emph{Sie sind uns nur vorausgegangen.} Alles braucht man an diesen
Kräften und deshalb schone ich, was mich ernährt. Nicht in der ersten
Nacht alles schon zu Ende gesehen haben, was es zu entdecken geben
soll.\\
Dann ist\textquotesingle s auf einmal hell geworden unter uns. Als wenn
sie angefangen hätte zu leuchten und deshalb könnte ich nichts mehr
erkennen mit den nachtschwarz gewöhnten Augen auf den weißen Blättern,
die hier verstreut waren. Wahrh. war ich gewöhnt, also Dunkelheit immer
und Stimmen daraus, die sich ergaben. Doch plötzlich Helligkeit und
klare Umgebungsgeräusche, die man unterscheiden konnte also trennen vom
Hintergrund das wichtige, was gesprochen wird. Als wenn die Sinne
dazugelernt hätten durch das Buch. Erklärend, wie sich die Konsonanten
verhalten gegen das universale Vokalsystem. Es behielt Recht. Ich
zeichnete mir die Koordinaten auf: das ging in eim Feynmandiagramm, die
Ansicht war umzusetzen in die Sprache Mahlers, egal wie hundert die
Jahre vergangen sind an seinem L.nam.32 Er würde es gehört haben, was
hier auf ihn wartete zu entstehen. Ich habe die Sendung schon
abgeschickt in Gedanken, doch etwas hatte mich zurückgehalten, den
letzten Schnitt zu tun. Das war Glück und gelang nur so: weil es eine
künstliche Atmosphäre gab um den Gegenstand, daß er mir nicht aus den
Händen glitt, die nicht hafteten aneinander, füreinander, die linke für
die rechte und im Spiegelbild auch nicht, wo ich sie hineinlegte. Es
sollte eine Gefahr sein; und schon ist es dem Körper zuviel geworden und
schreckt alles zurück: die große unfaßbare Zeit Zukunft, die sich auf
einmal anfassen ließ (wie ein steifes Glied in der nicht geölten Kette.)
Das herumfiebert und das Blut anschwillt zu einer von innen gespürten
Größe, innen die Nerven reizend bis zur biomechanischen Kataklysme. Da
mußte etwas abspringen und zerklingen Stahl auf Stahl das sich reibt die
ganze Zeit, bis das Schwache nachgibt. \emph{composite III: }vermute ich
eine Strukturlosigkeit wie die innerhalb eines Hohlraumes im Kristall,
in dem es eine freie, endlose Schwingung geben kann, ohne Richtung, Ziel
oder Begrenzung und die nur sich selbst immer als Ursache hat; es
verliert keine Kraft, ergänzt sich aus der Umgebung.\\
Bis Johannes die Zeit gefunden hatte, seine Worte niederzuschreiben,
vergingen sagen wir 80 Jahre dem Jüngstem der Hinterbliebenen. Das war
ungefähr die Zeit, die es bräuchte, daß seine Enkel Enkel zeugten. Die
sind nötig gewesen, um ihn des Fortbestands seiner Erzählung zu
versichern. Es sollten drei Generationen darüber wachen, daß die
Zeugnisse nicht gefälscht würden. Bis heute hat sich daran nichts
geändert. Aber Kinder selbst hat er nicht gehabt, Johannes. Das sind
seine eigenen Worte, nur jene leben fort und zeugen. Ich habe eins
gehört aus seinem Mund, das war so Schall, daß man es gar nicht
wiederholen kann. Nur sich erinnern\ldots{} aber ohne die Möglichkeit
einer Fälschung in die Welt hinaus. Also diesmal nichts verfälschen:
\emph{Schreiben Sie nur von dem, was Sie kennen!}\\
"Ich lege mich in einen weißen Strand und bin selber dunkelndes
Fabelwesen, dem großen Buch der Geheimnisse entsprungen. Wer mich dort
gesehen hat am Strand, der denkt, das kennt er irgendwie. Aber dem ist
nicht so, er kann mich nicht; niemand jemals, zu dem ich nicht hin und
sage selbst: ,Ich kenne dich.` Dann kennt er auch mich."\\
Nur ich bleibe das Fabelwesen und es kam jemand hinzu, der erzählen
darf, was er gesehen hat, wenn er sich an die Worte n.~erinnert und die
Bilder. Denn voll war der Sand davon und es ist ein weiter Strand, der
sich einmal um die halbe Insel zog, vorbei am Ende der Welt hoch oben,
vorbei an einem steil abfallenden Hang aus grünem Tonmatrial, dann um
die Spitze im Nordwesten herum zwischen Felsgeröll, weiter die große
unbetretene Zone, die wir jetzt vor uns haben. Ich kenn Euch, sagt es
von dort, und so haben sie sie geöffnet: die Ephemeriden.33

\end{document}
