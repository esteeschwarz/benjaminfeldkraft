% Options for packages loaded elsewhere
\PassOptionsToPackage{unicode}{hyperref}
\PassOptionsToPackage{hyphens}{url}
%
\documentclass[
]{article}
\usepackage{amsmath,amssymb}
\usepackage{iftex}
\ifPDFTeX
  \usepackage[T1]{fontenc}
  \usepackage[utf8]{inputenc}
  \usepackage{textcomp} % provide euro and other symbols
\else % if luatex or xetex
  \usepackage{unicode-math} % this also loads fontspec
  \defaultfontfeatures{Scale=MatchLowercase}
  \defaultfontfeatures[\rmfamily]{Ligatures=TeX,Scale=1}
\fi
\usepackage{lmodern}
\ifPDFTeX\else
  % xetex/luatex font selection
\fi
% Use upquote if available, for straight quotes in verbatim environments
\IfFileExists{upquote.sty}{\usepackage{upquote}}{}
\IfFileExists{microtype.sty}{% use microtype if available
  \usepackage[]{microtype}
  \UseMicrotypeSet[protrusion]{basicmath} % disable protrusion for tt fonts
}{}
\makeatletter
\@ifundefined{KOMAClassName}{% if non-KOMA class
  \IfFileExists{parskip.sty}{%
    \usepackage{parskip}
  }{% else
    \setlength{\parindent}{0pt}
    \setlength{\parskip}{6pt plus 2pt minus 1pt}}
}{% if KOMA class
  \KOMAoptions{parskip=half}}
\makeatother
\usepackage{xcolor}
\usepackage[margin=1in]{geometry}
\usepackage{graphicx}
\makeatletter
\def\maxwidth{\ifdim\Gin@nat@width>\linewidth\linewidth\else\Gin@nat@width\fi}
\def\maxheight{\ifdim\Gin@nat@height>\textheight\textheight\else\Gin@nat@height\fi}
\makeatother
% Scale images if necessary, so that they will not overflow the page
% margins by default, and it is still possible to overwrite the defaults
% using explicit options in \includegraphics[width, height, ...]{}
\setkeys{Gin}{width=\maxwidth,height=\maxheight,keepaspectratio}
% Set default figure placement to htbp
\makeatletter
\def\fps@figure{htbp}
\makeatother
\setlength{\emergencystretch}{3em} % prevent overfull lines
\providecommand{\tightlist}{%
  \setlength{\itemsep}{0pt}\setlength{\parskip}{0pt}}
\setcounter{secnumdepth}{-\maxdimen} % remove section numbering
\ifLuaTeX
  \usepackage{selnolig}  % disable illegal ligatures
\fi
\usepackage{bookmark}
\IfFileExists{xurl.sty}{\usepackage{xurl}}{} % add URL line breaks if available
\urlstyle{same}
\hypersetup{
  hidelinks,
  pdfcreator={LaTeX via pandoc}}

\author{}
\date{\vspace{-2.5em}}

\begin{document}

\subsection{c.~Wahrnehmung der äußeren
Belange}\label{c.-wahrnehmung-der-uxe4uuxdferen-belange}

Auf einem traurigen Hügel ein wenig über der Stadt steht ein
unscheinbares Gebäude, einer Kapelle oder einem größeren Bildstock, wie
man sie aus Italien und Süddeutschland überall kennt ähnlich. Das ist
verschlossen mit einer Gittertür. Nur Katzen und alles was kleiner ist,
kann hineinschlüpfen. Ein mittlerer Pudel zum Beispiel würde da nicht
durchpassen. Es ist aber keine Kapelle, man kennt so etwas dort nicht.
Es \emph{ist} ein Altar und war zunächst nur ein Stein am Feldrand
zwischen Olivenbäumen. Den hatte einer aufgestellt, oben angehauen und
glattgeschliffen und immer, wenn er dort vorbeikam auf dem Weg brachte
er etwas dar, das er bei sich hatte und das teuer war. Aber oft führte
ihn sein Weg nicht dorthin, er war Nomade im ganzen Land. Und doch baute
er von mal zu mal langsam eine Stätte um den Stein herum, die diesen
schützen sollte und kenntlich machen. Kenntlich machen als was? Als
Eigentum? Den Findling am Weg jedenfalls umgab bald eine künstliche
Hüllform, ausgerichtet mit drei Fensterchen und der Türe in die vier
Himmelsrichtungen, dieselbe nach Osten. Von dorther betete der fromme
Mann. Kniete vor dem Stein, hatte ein Rauchfeuer entzündet, das durch
ein geschirmtes Loch in der kuppelförmigen Decke liebliche Gerüche
entweichen ließ. Ich kaufte in der Stadt mir (die Höhe) eine Miniatur
und lege, wenn mir danach ist, auf die kleine, kleine Schale im Innern
ein glühndes Stück Kohle und einen winzigen Brocken Galbanum. Dann
raucht es aus der Gugelöffnung wie daselbst in Bethel, wo Jakob seinen
Namen erhielt und wo jetzt die Katzen aus und eingehen, weil immer
jemand Fleischstücke auf dem Stein liegenläßt. Aber damals; es gab
nichts heilgeres und selbst das später errichtete A.heiligste konnte
nicht das Wasser, um den Stein zu netzen, Abrams Öl war immer feiner und
für ewig darauf vergossen, so daß daran abfiel und abperlte, was ihn
reinwaschen wollte. Ich hätt ihm gern dabei zugesehen und gelernt, wie
man ehrfürchtig handelt und treu. Doch soviel Zeiten liegen dazwischen,
wie man denken kann von einem Leben zum andern. Immer geschieht ein
neues Unheil und weitere Prüfungn stehn bevor, die man an diesem Stein
(la pierre) erfahren wird, wenn er uns losläßt, wenn wir ihn also
verlieren: diesen schwach uns erhaltenen, aber uns stark erhaltenden
Glauben.\\
Auf meinem Weg in die Gemeinde der Hüter und Hüterinnen bin \emph{ich}
nur selten jemam mehr als zwei, drei mal begegnet, und selbst jene paar
mal ist es meistens so gewesen, daß einer von uns den andern abstieß.
Ich bin nicht immer glücklich zurück im Leben gelandet, es gab Momente
die wie außerhalb zu stehn schienen und eine Rückkehr aus ihnen so lange
unmöglich machten bis die Zeit selber es einem erlaubte sich von ihnen
zurückzuziehen. Ob es für den anderen Hüter genauso war weiß ich
natürlich bis heute nicht, denn auch ich stieß ihn ab, so viel ist
sicher. Wenn sich zwei ganz nahekommen wollten bedurfte es einer
\emph{Kommunikation}. Damals wußte wir n.~nicht genau was das war nur
daß manche es taten und konnten und andere nicht. Wir nannten jene
\emph{Leute} Psychos, die über die Gabe verfügten einem etwas über sich
mitzuteilen. Lange Zeit bin ich zB. mittags aufgestanden und spät ins
Bett gegangen, was dazu führte daß auch ich einer wurde, der sich an
Dunkelheit gewöhnt warwas dazu führte sie zu schätzen. Aber jetzt, was
morgen um sechs beginnt sind eigentlich nur die Stunden um die Dämmerung
und dann wenn alles gedämpft auftritt wegen der fehlenden Verstärkung
durchs Licht eine Zeit, in der mir meine Gedanken weniger fremd sind.
Das Gehirn hat sich einmal in der für die Reflexion wichtigen Phase des
Lebens die bei mir zwischen 19 u. 27. zu liegen kam darauf eingestellt
nur von midijour an aktiv an seiner Verwirklichung teilzunehmen und den
Rest seiner Aktivitäten die Geometrie des Traums. Es war auch mal
anders. Nur jetzt haben wir den Gehirnmittelpunkt erreicht mit unseren
Überlegungen zur Hybridtheorie die uns n.~weiter beschäftigen wird.
Soviel sei gesagt: daß Schlaf an offenen Gräben ein gefährlicher Schlaf
ist. Ich werde mir Mittel ausdenken die mir den Tag zurückgeben auch
ohne lange Zeit früh schlafen zu gehn. Und so albern es klingt werdich
an meim Wasserhaushalt etwas verändern glaub ich. Hier muß ein Wort
stehn das soviel bedeuten soll wie weit dürfen wir uns guten Gewissens
im Bereich dieses Traums aufhalten die wir doch mit dem unerbittlyken
Wake endigen sollten? Es wird darauf keine Antwort geben denn auch hier
die Grenze, klamm und kältlich, die uns daran erinnert wie früh es
morgen früh sein wird wenn ich die Nacht gegen Mitternacht dehne. Eine
einfache Frage zwischendurch die schnell und spontan beantwortet werden
muß um weiterlesen zu dürfen: Wie oft ist der Gläserklang zu hören wenn
auf dem Nachbarbalkon im Hof drei Leute mit dem Wein anstoßen? Das ist
die Wachheit in der ich mich befinde unds würd nicht wundern wenn n.~ein
Kapitel Konstruktion gelesen ward bis ich endlich rechtzeitig müde
werde. Doch wer will das schon, müde sein. Müde sein ist wie, müde sein
klingt auch nach Körperlichem, nach Bedürfnissen die wir uns ja
abschlagen wie nur dem wirklichen Feind. Vielleicht aber das Wort
dagegen: misomatisch? Ich werd sie mir geben diese Eigenschaft und den
Feind benennen. \emph{Konklusion: eine Befreiung von der Wahrnehmung ist
eine Befreiung vom Feld.} Das war so sandgelb und es gilt hier in seiner
nächsten Stufe, der Befreiung vom Körper als der Befreiung von der
Wahrnehmung und der gelungenen Passage durchs Reich der Notwendigkeit in
das der Freiheit. Damit wär auch diesem Boden endlich\ldots{} wo wir ihn
zu euch verließen\ldots{} das Schweigbrot eingesät.

\end{document}
