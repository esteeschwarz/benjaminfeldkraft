% Options for packages loaded elsewhere
\PassOptionsToPackage{unicode}{hyperref}
\PassOptionsToPackage{hyphens}{url}
%
\documentclass[
]{article}
\usepackage{amsmath,amssymb}
\usepackage{iftex}
\ifPDFTeX
  \usepackage[T1]{fontenc}
  \usepackage[utf8]{inputenc}
  \usepackage{textcomp} % provide euro and other symbols
\else % if luatex or xetex
  \usepackage{unicode-math} % this also loads fontspec
  \defaultfontfeatures{Scale=MatchLowercase}
  \defaultfontfeatures[\rmfamily]{Ligatures=TeX,Scale=1}
\fi
\usepackage{lmodern}
\ifPDFTeX\else
  % xetex/luatex font selection
\fi
% Use upquote if available, for straight quotes in verbatim environments
\IfFileExists{upquote.sty}{\usepackage{upquote}}{}
\IfFileExists{microtype.sty}{% use microtype if available
  \usepackage[]{microtype}
  \UseMicrotypeSet[protrusion]{basicmath} % disable protrusion for tt fonts
}{}
\makeatletter
\@ifundefined{KOMAClassName}{% if non-KOMA class
  \IfFileExists{parskip.sty}{%
    \usepackage{parskip}
  }{% else
    \setlength{\parindent}{0pt}
    \setlength{\parskip}{6pt plus 2pt minus 1pt}}
}{% if KOMA class
  \KOMAoptions{parskip=half}}
\makeatother
\usepackage{xcolor}
\usepackage[margin=1in]{geometry}
\usepackage{graphicx}
\makeatletter
\def\maxwidth{\ifdim\Gin@nat@width>\linewidth\linewidth\else\Gin@nat@width\fi}
\def\maxheight{\ifdim\Gin@nat@height>\textheight\textheight\else\Gin@nat@height\fi}
\makeatother
% Scale images if necessary, so that they will not overflow the page
% margins by default, and it is still possible to overwrite the defaults
% using explicit options in \includegraphics[width, height, ...]{}
\setkeys{Gin}{width=\maxwidth,height=\maxheight,keepaspectratio}
% Set default figure placement to htbp
\makeatletter
\def\fps@figure{htbp}
\makeatother
\setlength{\emergencystretch}{3em} % prevent overfull lines
\providecommand{\tightlist}{%
  \setlength{\itemsep}{0pt}\setlength{\parskip}{0pt}}
\setcounter{secnumdepth}{-\maxdimen} % remove section numbering
\ifLuaTeX
  \usepackage{selnolig}  % disable illegal ligatures
\fi
\usepackage{bookmark}
\IfFileExists{xurl.sty}{\usepackage{xurl}}{} % add URL line breaks if available
\urlstyle{same}
\hypersetup{
  hidelinks,
  pdfcreator={LaTeX via pandoc}}

\author{}
\date{\vspace{-2.5em}}

\begin{document}

\subsection{II. Mignon (E.)}\label{ii.-mignon-e.}

\emph{VIII. Vom Satz des ersten Grundes }an gerechnet blieb mir nicht
mehr viel Zeit, von Jean etwas zu erfahren, das mich der Wirklichkeit
des Schlüssels näher bringen wollte. Ich bin n.~weit von Ahnungen
entfernt, wohin es mich tragen wird, wenn ich erst in seinem Besitz bin.
Ich hatte richtig vermutet, daß sie zum gegebenem Punkt zusammenkamen
und die Zeugnisse an sich gebracht hatten. Das hieß: alle Zeugnisse; vom
ersten unsicher vermerkten Wort über den Anfang von allem und jedem bis
zum Davidschild der letzten Wache. Doch es waren ja jetzt ihre eigenen
Noten, an die ich gelangen mußte, nicht mehr die freien Kenntnisse um
jenen von der Zeit abgeschiedenen, die er willig vermehrt hatte. Jean
mußte gewonnen werden, und ihr Vorwissen beseitigt. Es gab diese
Jungfrau in ihrem Leben eine angstlos gestaltete Kindheit war beendet
worden, weil sie ihr erschienen war und Fragen stellte über ihr Erbe.
Vielleicht schon damals hörte sie also auf, alles nur auszudenken, und
setzte ihr Wollen um. Aber nicht wie das Mädchen von Orleans mit einem
prophetischen Gesicht begabt mußte sie ihr Gehör aus sich wenden und
schwer erlernen, die Zungenrede in Mission zu verwandeln. Doch sie hatte
\emph{Zuhörer, }man war ihr gefolgt, hatte ihre Schritte beobachtet,
ihre Worte mitgeschrieben, ihre leisen Zitate geisterten kaum merklich
in der Welt umher, unaufdringlich aber präsent. Nur verstand sie es
nicht, die Lösung der Sendung aufzugeben; und paradoxerweise und
folgerichtig wuchs die Zahl der Jünger mit jedem Mysterium, das seine
Aufklärung verweigerte. Jemand schrieb sie alle auf, so wie er sie
erfuhr und hätte ich endlich den Schlüssel, der das über der Schrift
liegende Antwortrauschen entzerren kann ich weiß jetzt, daß die 10.
Symphonie nie hätte geschrieben werden müssen zum Tod, sie wäre ein Werk
des Lebens geworden zu seiner Macht, die er nur auszuteilen brauchte, um
jeden neuen Schöpfungsakt vollständig zu inspirieren. Es schafft sich
auch hier, während ich das notiere, ein selbständiger Anhang von
Erklärungen, die ihre Fußnoten im Buch hinterlassen. Nur es soll keine
wissenschaftliche Arbeit sein, sondern reiner Quelltext über den
fragwürdigen Zustand der Musik um tausendneunhundertelf. Was war
erreicht? Nicht das Wien seiner Meister aber die Gloriette als Cafehaus
mit Blick über Österreich-Ungarn\ldots{}

\end{document}
