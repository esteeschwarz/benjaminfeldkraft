% Options for packages loaded elsewhere
\PassOptionsToPackage{unicode}{hyperref}
\PassOptionsToPackage{hyphens}{url}
%
\documentclass[
]{article}
\usepackage{amsmath,amssymb}
\usepackage{iftex}
\ifPDFTeX
  \usepackage[T1]{fontenc}
  \usepackage[utf8]{inputenc}
  \usepackage{textcomp} % provide euro and other symbols
\else % if luatex or xetex
  \usepackage{unicode-math} % this also loads fontspec
  \defaultfontfeatures{Scale=MatchLowercase}
  \defaultfontfeatures[\rmfamily]{Ligatures=TeX,Scale=1}
\fi
\usepackage{lmodern}
\ifPDFTeX\else
  % xetex/luatex font selection
\fi
% Use upquote if available, for straight quotes in verbatim environments
\IfFileExists{upquote.sty}{\usepackage{upquote}}{}
\IfFileExists{microtype.sty}{% use microtype if available
  \usepackage[]{microtype}
  \UseMicrotypeSet[protrusion]{basicmath} % disable protrusion for tt fonts
}{}
\makeatletter
\@ifundefined{KOMAClassName}{% if non-KOMA class
  \IfFileExists{parskip.sty}{%
    \usepackage{parskip}
  }{% else
    \setlength{\parindent}{0pt}
    \setlength{\parskip}{6pt plus 2pt minus 1pt}}
}{% if KOMA class
  \KOMAoptions{parskip=half}}
\makeatother
\usepackage{xcolor}
\usepackage[margin=1in]{geometry}
\usepackage{graphicx}
\makeatletter
\def\maxwidth{\ifdim\Gin@nat@width>\linewidth\linewidth\else\Gin@nat@width\fi}
\def\maxheight{\ifdim\Gin@nat@height>\textheight\textheight\else\Gin@nat@height\fi}
\makeatother
% Scale images if necessary, so that they will not overflow the page
% margins by default, and it is still possible to overwrite the defaults
% using explicit options in \includegraphics[width, height, ...]{}
\setkeys{Gin}{width=\maxwidth,height=\maxheight,keepaspectratio}
% Set default figure placement to htbp
\makeatletter
\def\fps@figure{htbp}
\makeatother
\setlength{\emergencystretch}{3em} % prevent overfull lines
\providecommand{\tightlist}{%
  \setlength{\itemsep}{0pt}\setlength{\parskip}{0pt}}
\setcounter{secnumdepth}{-\maxdimen} % remove section numbering
\ifLuaTeX
  \usepackage{selnolig}  % disable illegal ligatures
\fi
\usepackage{bookmark}
\IfFileExists{xurl.sty}{\usepackage{xurl}}{} % add URL line breaks if available
\urlstyle{same}
\hypersetup{
  hidelinks,
  pdfcreator={LaTeX via pandoc}}

\author{}
\date{\vspace{-2.5em}}

\begin{document}

\subsection{P - 6 (statt N)}\label{p---6-statt-n}

Der mögliche Beginn wurde mir aufgezeigt, darin ich mich nach Wien
begeben hätte, die Associate Music aufzusuchen. Aus deren Manuskript
hatte ich alles lernen können, selbst das schwache Französisch Mahlers.
Aber es war ein neuerliches Studium nötig im Archiv: bis jetzt war es
immer nur um die Musik gegangen Mahler jedoch wollte etwas anderes. Wenn
ich über seine Architektur etwas finden wollte, sind andere Quellen
nötig als die primären und sekundären Schriften. Ich sollte weiter hinab
gehen müssen, in die Krypta sozusagen seiner Kathedrale. Da ist die
Triangulation, geschrieben nur in der Maschinensprache seiner Zeit, das
sind Noten statt Buchstaben, die eine feste Struktur ausbildeten, also
chaotische Zustände in der Kristallisation begriffen. Denn der
Transistor war n.~nicht erfunden bis 1947, der aber mir erlaubt, die
Komposition im Entstehungszustand \emph{nascendi} unentwegt zu
beobachten und abzuhören, wo sich jeder falsche Ton verbergen könnte,
der das Gebäude zum Einsturz bringen würde. M. mußte ja auslöschen, um
die Versionen voneinander trennen zu können. Jedoch hier, im Geburtsort
seiner Vollendung löste sich jedes Zittern um die Bögen der ersten
Streicher von selbst in Musik auf; ohne Kraftaufwand, nur die 75W-Stufe
muß betrieben werden, um etwas auf dem Kopfhörer zu haben. Und so gehe
ich in das Feld, immer genauer höre ich hin. Ich habe seine Stimme
entdeckt, wie sie mitsummt; versteckt von seinen geschlossenen Augen,
aber deutlich höre ich sie, es ist unvermeidlich. Ein Rockärmel schleift
vielleicht irgendwo, das ist nicht sicher, aber Seitenblättern. Und
n.~feiner: das Reiben im Sessel, wenn sich jemand bewegt, stärkeres
Rascheln Stroh darin, vielleicht auch Staub gerade, wie er auffährt.
Husten, so wie heute auch damals. Und er erträgt es, weil er hofft, doch
n.~an das zweite Scherzo heranzureichen, das sich so deutlich schon
abgebildet hat im visionären Particell. Nur n.~diese 163 Sekunden
aushalten, irgendwie hinbringen und dann darf er das Purgatorium
anfangen. Aber er weiß nicht, daß ihm dafür nur 28 Takte bleiben werden,
der Rest ist meine Arbeit gewesen. Das schreibe ich der kleinen Ewakatze
in ihre Maushände und sie lernt etwas daraus: man muß sich mit dem Buch
hineinsetzen und Seite für Seite lesen. Vor dem Sonnenaufgang über dem
Gebirge, in die Nebelsenke des Tales blickend; vor den Fenstern des
n.~schlafenden Zuges, der sie nach Finnland; Rostock, Helsinki mit dem
Schiff, rauchend am rauchenden Schornstein. Trinkend den starken
schwarzen Tee: das ist schon dort, wo sie mich trifft, gekocht mit dem
Seewasser und der Stein, jadegrün zählt jede Kanne und merkt sie sich in
der Uferschleie. Da fangen die Hechte, legen Aalschnüre mit Fischleichen
aus und meistens gibt es Pilze. Man muß das vorstellen: Spätherbst und
wir nähern uns der Zeit, wo die Sonne nicht mehr aufgehen wird. Hier
selbst bin ich froh von der geschenkten ruhnden Dunkelheit; aber dort?
Wenn nichts wirklich erwacht die zwei Monate Winterschlaf im Horizont
Europens\ldots{} lege ich dorthinein eine Sehnsucht, darf ich das? Und
ihr das neiden, dieses somnambule Hindämmern in ihrem \emph{Traum kurz
vor manuellen Hirnstrukturen? }Ja, er darf das, denn auch er ist ein
Faust auf der Wache, \emph{dies patrie, }und wenn sich auch grundlegend
die Wissenschaften dieser beiden Männer unterschieden, von dem einen
Wort und vom andern Musik ausgeströmt waren nach der Unsterblichkeit, so
sehen wir genau hin und* wer Ohren hat, der höre, \emph{wie das aus den
von den Bögen gestrichenen Saiten sich ablesen läßt, die }Partitur
tenebrae, \emph{die Sie uns vorenthalten wollten, lieber M. War sie so
unvollendet denn? Daß Sie alles in das Schweigen zurückzudrängen
gedachten, aus dem es Ihnen entsprang? Ich selbst bin nur der Anfang
ihres Fortschreitens und dieses hier das Ende der zweiten Dekade oder
wie wir sie immer nennen wollen, jene Zusammenkunft von Schrift und
Wort, Dekade, weil in der Tradition stehend, die 9 überwunden zu haben,
n.~ohne reif zu sein für das Dezemvirum ausgebildeter Tonsetzer. Also
sich ebenso geworfen zu fühlen, wie jeder andere, der in ihren Kreis
geraten wäre. Ich ziehe mich zurück und stehe zur Verfügung; der dritte
Satz wird neu entworfen. Und irgendwann merken wir, was fehlte: Es sind
bei angenommen einer Minute Lesezeit pro Seite um 3 Stunden zu lesen;
nicht viel, bemerken wir erleichtert. Aber es hielt uns etwas davon ab,
leichtfertig weiterzuschreiben. Ich ging statt blind weiter vorwärts
also lieber in diesen hier geöffneten zweiten Satz }(Die Zweite Dekade:)
\emph{zurück, um ihn mit der Partitur abzustimmen. Diese verlangte 12
min 15 Spielzeit, }intime\emph{. Eigentlich ein Scherz, aber
aufgerechnet ergab sich eine halbe Stunde Text mit höchsten
Anforderungen. Der mußte erfunden werden. Später wußte ich, ging das
Geschriebene }in die Schöpfung ein\emph{, aber jetzt würden es nur erst
Worte sein, die ich brauchte. Es fehlten 128 Stück, um die Proportionen
zu wahren, die das Werk vorschrieb. Wie leicht gehen diese von den
Lippen, wenn man an einem sonnigen Tag mit Freunden zuviel preisgibt.Und
wie unendlich schwer, wenn es gilt, mit dem Geliebten zu sprechen. Der
Garten läßt mich mitten 175 glücklich sein, ich habe die einmal fast
gezählt in 9h. Ein paar haben Bestand, weil es die letzten Worte sind,
bevor er }für immer in Ruhe *gebettet wird. Das zumindest glauben wir
weil es keiner wirklich weiß. Und so lange niemand etwas anderes sagt,
glauben es auch andere. Der Glaube ist stark in ihnen und die Macht
auch, ihn zu gebrauchen.

\end{document}
