% Options for packages loaded elsewhere
\PassOptionsToPackage{unicode}{hyperref}
\PassOptionsToPackage{hyphens}{url}
%
\documentclass[
]{article}
\usepackage{amsmath,amssymb}
\usepackage{iftex}
\ifPDFTeX
  \usepackage[T1]{fontenc}
  \usepackage[utf8]{inputenc}
  \usepackage{textcomp} % provide euro and other symbols
\else % if luatex or xetex
  \usepackage{unicode-math} % this also loads fontspec
  \defaultfontfeatures{Scale=MatchLowercase}
  \defaultfontfeatures[\rmfamily]{Ligatures=TeX,Scale=1}
\fi
\usepackage{lmodern}
\ifPDFTeX\else
  % xetex/luatex font selection
\fi
% Use upquote if available, for straight quotes in verbatim environments
\IfFileExists{upquote.sty}{\usepackage{upquote}}{}
\IfFileExists{microtype.sty}{% use microtype if available
  \usepackage[]{microtype}
  \UseMicrotypeSet[protrusion]{basicmath} % disable protrusion for tt fonts
}{}
\makeatletter
\@ifundefined{KOMAClassName}{% if non-KOMA class
  \IfFileExists{parskip.sty}{%
    \usepackage{parskip}
  }{% else
    \setlength{\parindent}{0pt}
    \setlength{\parskip}{6pt plus 2pt minus 1pt}}
}{% if KOMA class
  \KOMAoptions{parskip=half}}
\makeatother
\usepackage{xcolor}
\usepackage[margin=1in]{geometry}
\usepackage{graphicx}
\makeatletter
\def\maxwidth{\ifdim\Gin@nat@width>\linewidth\linewidth\else\Gin@nat@width\fi}
\def\maxheight{\ifdim\Gin@nat@height>\textheight\textheight\else\Gin@nat@height\fi}
\makeatother
% Scale images if necessary, so that they will not overflow the page
% margins by default, and it is still possible to overwrite the defaults
% using explicit options in \includegraphics[width, height, ...]{}
\setkeys{Gin}{width=\maxwidth,height=\maxheight,keepaspectratio}
% Set default figure placement to htbp
\makeatletter
\def\fps@figure{htbp}
\makeatother
\setlength{\emergencystretch}{3em} % prevent overfull lines
\providecommand{\tightlist}{%
  \setlength{\itemsep}{0pt}\setlength{\parskip}{0pt}}
\setcounter{secnumdepth}{-\maxdimen} % remove section numbering
\ifLuaTeX
  \usepackage{selnolig}  % disable illegal ligatures
\fi
\usepackage{bookmark}
\IfFileExists{xurl.sty}{\usepackage{xurl}}{} % add URL line breaks if available
\urlstyle{same}
\hypersetup{
  hidelinks,
  pdfcreator={LaTeX via pandoc}}

\author{}
\date{\vspace{-2.5em}}

\begin{document}

\subsection{Letztes Kapitel}\label{letztes-kapitel}

Wir stolperten also hinaus. Licht usw., stark lyrische Momente von
untergehenden Sonnen und Heidewiesen im Mondlicht. Es rauschen
Brandungstöne und Möwen schreien nachts. Aber wir sind hier gelandet und
warum weiß ich nicht. Ich kroch aus meiner kalkausgestrichenen Erdgrube,
ich rieselte der Sand aus meiner Urne ins M.r. Wir sind zu zweit, das
ist Ewa (Alwa/Alma/mater/Mahler-Laplace) und Joannes (kaanan/Jean/Thomas
aber, der Zwilling genannt wird) sind also ungefähr sechs bis sieben
\emph{personae}, die ich mit einführen muß neben mir selbst, Zwilling,
der andere \emph{von Weinen. }Wir stolperten also hinaus: 1. Orpheus,
2.-7. vorgenannte, 8. Hermes (trismegistos) und 9. Eurydike, von ihm
geleitet an der Hand, mit der schon wir wissen zu vielen Berührung für
ihr n.~jungfräuliches Schattendasein. Und da geschieht die Wendung: Er
dreht sich (O.) \emph{nicht!} um diesmal, weil ja die Schlange von
Leuten ihm folgt und das Rauschen der Füße ihm Gewißheit gibt (der
folgenden.) Aber was dann? Warum stockt das Bild, warum wartet die
Prozession? Was bedeutet die \emph{Wendung} in diesem Fall? Ist es
Eurydike, die sich abgewendet und von selbst umgekehrt ist, als kennte
sie ihr unnachgiebiges Schicksal? Aber sie wäre ja errettet worden, ahnt
auch sie, ohne vom ihr voran Fliehenden gewußt zu haben; das Saitenspiel
hat \emph{sie} nicht gehört\ldots{} doch wir, und unser eines Ohr (eines
unserer) erinnert sich, vielmehr das Wasser in seinem Labyrinth, ein
paar seiner Atome. Das schicken wir zu ihr durch die Zeit. Sie hält
inne, weil es in ihr zähflüssig ist und stolpert, als sie durch die Tür
will und alles zurückdrängt. Wir jetzt empfinden das als unendlich
langes Zögern, weil wir den Fortgang der Geschichte ja längst kennen.
Für E. konzentrierten sich aber ihre Handlungen in momentanen Spitzen,
die aus unserer Erzählung herausragten. Nur dann war sie selbst
aufmerksam genug und geduldig, dem Geschehen zu folgen. Schon verlor sie
H.s Hand, bekam große Angst wo sie plötzlich stehe, denn tot war sie
nicht aber lebte nicht n.~nicht mehr, war also auf irgendeinem Weg
vorwärts, aufwärts aus Erinnerungsdunkel von Vorhöllengesichtern, die
sie gesehen hatte, wann, wohin? Wir merkten das in der Mitte zwischen
den beiden (drei) Gefährten, daß die Spannung vom vordern Ende über
H.mes nach hinten abfiel. Jener zog an, vermittelte nach vorn zu uns und
wir gaben das in Worten weiter an die Spitze: \emph{Orpheus, Du
bist\textquotesingle s, der uns Jenseitige zieht\ldots{} ein Rauschen
fährt über das Neunte Gebiet! }Und dieses hat er vernommen wissen wir
jetzt, denn er spielte weiter und laut seine Leier über uns hinweg nach
\emph{E.}\\
Ich beharre auf der letzten von sieben verbliebenen Seite des antiken
Instruments (Schildkrott), um Ewa die Komposition doch n.~spielen lassen
zu können, obwohl ich im Laufe ihrer Rekonstruktion lernte, nicht an die
Spielbarkeit zu glauben, jedenfalls nicht auf sie zu hoffen. Aber sie
erzeugt tatsächlich Töne aus den Noten. Ich habe mich lange gefragt, ob
auch etwas zu hören gewesen wäre, wenn sie das andere alte Instrument
(α/β/γ--ton/Kupferrohr) zur Verfügung gehabt hätten damals, zur
Entstehungszeit. Ob der alte Mahler etwas geahnt hat von diesem
entsetzlichen Fund, den er glücklich nicht mehr miterleben mußte? Aber
irgendwie wußte, daß zu jenem prähistorischen Orpheus auch n.~ein
Gegenstück epischer Art existieren mußte, das nur n.~nicht entdeckt war
bzw. das n.~bevorstand, sich zu verwirklichen? Wohin würde das seine
Dichtung werfen, wenn plötzlich doch Worte dafür in Frage kämen, sie zu
übersetzen? Diese Musik??! \emph{Keine Engelszungen, }kein Evoi, kein
Requiem. Nur Schlagschatten nachtwärts gewandt, die das Kernniveau auf
meine Fähigkeiten herunterbrechen. Wenn es auch nur einen mutigen
Dirigenten gäbe, \emph{einen Leser, der das Werk mit seinen eigenen
Augen läse\ldots{}}

\end{document}
