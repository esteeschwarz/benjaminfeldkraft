% Options for packages loaded elsewhere
\PassOptionsToPackage{unicode}{hyperref}
\PassOptionsToPackage{hyphens}{url}
%
\documentclass[
]{article}
\usepackage{amsmath,amssymb}
\usepackage{iftex}
\ifPDFTeX
  \usepackage[T1]{fontenc}
  \usepackage[utf8]{inputenc}
  \usepackage{textcomp} % provide euro and other symbols
\else % if luatex or xetex
  \usepackage{unicode-math} % this also loads fontspec
  \defaultfontfeatures{Scale=MatchLowercase}
  \defaultfontfeatures[\rmfamily]{Ligatures=TeX,Scale=1}
\fi
\usepackage{lmodern}
\ifPDFTeX\else
  % xetex/luatex font selection
\fi
% Use upquote if available, for straight quotes in verbatim environments
\IfFileExists{upquote.sty}{\usepackage{upquote}}{}
\IfFileExists{microtype.sty}{% use microtype if available
  \usepackage[]{microtype}
  \UseMicrotypeSet[protrusion]{basicmath} % disable protrusion for tt fonts
}{}
\makeatletter
\@ifundefined{KOMAClassName}{% if non-KOMA class
  \IfFileExists{parskip.sty}{%
    \usepackage{parskip}
  }{% else
    \setlength{\parindent}{0pt}
    \setlength{\parskip}{6pt plus 2pt minus 1pt}}
}{% if KOMA class
  \KOMAoptions{parskip=half}}
\makeatother
\usepackage{xcolor}
\usepackage[margin=1in]{geometry}
\usepackage{graphicx}
\makeatletter
\def\maxwidth{\ifdim\Gin@nat@width>\linewidth\linewidth\else\Gin@nat@width\fi}
\def\maxheight{\ifdim\Gin@nat@height>\textheight\textheight\else\Gin@nat@height\fi}
\makeatother
% Scale images if necessary, so that they will not overflow the page
% margins by default, and it is still possible to overwrite the defaults
% using explicit options in \includegraphics[width, height, ...]{}
\setkeys{Gin}{width=\maxwidth,height=\maxheight,keepaspectratio}
% Set default figure placement to htbp
\makeatletter
\def\fps@figure{htbp}
\makeatother
\setlength{\emergencystretch}{3em} % prevent overfull lines
\providecommand{\tightlist}{%
  \setlength{\itemsep}{0pt}\setlength{\parskip}{0pt}}
\setcounter{secnumdepth}{-\maxdimen} % remove section numbering
\ifLuaTeX
  \usepackage{selnolig}  % disable illegal ligatures
\fi
\usepackage{bookmark}
\IfFileExists{xurl.sty}{\usepackage{xurl}}{} % add URL line breaks if available
\urlstyle{same}
\hypersetup{
  hidelinks,
  pdfcreator={LaTeX via pandoc}}

\author{}
\date{\vspace{-2.5em}}

\begin{document}

\subsection{Huit}\label{huit}

\emph{" \ldots{} }e-Moll ist nicht die richtige Tonart gewesen", sagte
Jean. Langsam, Liebling, ich konnte ihn verstehen. Oder auch: sie. Jean
war sich nie sicher, ob es nicht ein Mädchen gewesen ist seit den Tagen
der ersten Spaltung. Ich habe ihm deshalb das Zwitterwesen, den
Thomaszwilling dazuerfunden, den wir beide schon aus den Evangelien
kannten, immer doubtful, immer en train d\textquotesingle inventer une
nouvelle vague. Daraus ist ihm etwas erwachsen wie: eine feste
Sicherheit im Glauben, dessen Fundament es selbst sich goß, stets der
Probe des Zwillings ausgesetzt.\\
„E-moll kann gar nicht richtig sein``, antwortete ich. "Dafür sind wir
n.~viel zu weit entfernt vom Finale, als daß sich e-moll bis dahin
aushalten ließe.`` Ein Zwillingskommentar, wie ich ihn selbst nie von
mir gegeben hätte, ihn mir nicht zugetraut. Doch weil M. sein Gegenstück
war, konnte er sich diese Ehrlichkeit erlauben, ohne sich wirklich in
Gefahr zu begeben. Sie waren verbunden. Sie waren Pech und Schwefel,
Herz und Seele. Ob sich etwas zugetragen hat zwischen ihnen, das über
unser Verständnis einer Meister-Schüler-Beziehung hinausging, weiß ich
allein nicht, jemand müßte meine Erinnerungslücken auffüllen mit dem
verschütteten Material dieser drei Jahre. Etwas lag hinter mir, wußte
ich, das nicht nur der stenographischen Verkürzung zum Opfer gefallen
ist; das mehr als die Leerzeichen sein mußte und der notwendige Abstand
zwischen den Zeilen.\\
"Man sollte danach suchen können, meinst du nicht? Und suchen woanders
als an den Blatträndern nur und auf den notenleeren Rückseiten." "Wo
aber dann?" frage ich zurück. Es bleibt uns ja nur die Mappe mit den
Manuskriptblättern und ein paar der Briefe, die Ewa für wichtig genug
hielt, um Transkriptionen anzufertigen. "Was wir auch finden es wird uns
nicht gesucht haben. Wir selbst müssen die Energie erfinden, darauf zu
stoßen. Vielleicht zufällig, aber ganz sicher nicht ohne Anstrengung.
Der Feind sitzt nicht im Papier, sondern in dem schwarzen Fluß, der
darüber hinströmt. Man kann ihn nicht auslöschen, nur immer besser
kennenlernen und versuchen, ihn zu begreifen: damit man ihn eines Tages
besiegen lernt." Also lesen, statt zu kämpfen. Die Revolution hat es nie
gegeben, für die es sich lohnte, zu sterben und das Werk lebt, war ich
mir bewußt. Viel mehr mußte ich nicht wissen. Ich dachte, wir hätten
Grund erreicht: aber der feste Boden war Illusion. Man sollte danach
suchen können nach einem Hinweis am Rand der Geschichte. Vielleicht
erschloß sie sich ihnen. Etwas habe ich schon gefunden: es gab
elektronische Teilpattern im Zwischenraum der Notenschrift, wenn man die
akustische über der optischen Abbildung laufen ließ. Das ergab nicht
immer sofort sinnvolle Informationen, aber es genügte oft ein wenig
Phantasie, um sich seine Idee vorstellen zu können. Mahler wußte selbst
nicht in jedem Augenblick der Komposition um die Nachrichten, die er
verschlüsselte. Nur, daß er es fortwährend tat in jeder Intervallsetzung
war ihm klar und daß er nichts dagegen tun konnte. Was hier davon alles
ankommt, wenn ich die Bruchstücke ausführe, kann unmöglich nur seiner
Bewußt entsprungen sein, da ist mehr im Spiel gewesen
\emph{Archetypenlyrik.} Oder auch Gleichklang der sphärischen
Bedingungen jetzt\ldots{} und damals, wie immer der auch zustande
gekommen sein mag; als wenn sich etwas außer uns für die menschliche
Zeitrechnung interessiert hätte, die mich von ihm trennt über diesen
willkürlichen Raum von hundert Jahren. (Die Wasserorgel kennt auch keine
Tonabstände, die sich in Verschiedenheit der Schwingungszahlen messen
ließen, nur im Labyrinth des Innenohrs endlich erkennt sich die Frequenz
wieder.) Was also ließ ihn diese Verhältnisse wählen, um ins Leben
zurückzukehren? \ldots der Hauch ging von den Himmlischen aus und bleibt
ihr Werk auf weichem Boden, \emph{selige Genien.} Man ist nur im Traum
dagegen immun und der Traum bleibt zu bestehen, in dem man sich dem
Negativen zuwendet und es aushält, ohne es zu verneinen. In der
\emph{partitur tenebrae }ist es uns erlaubt, mit dem Bösen umzugehen als
einem Gegenüber, das nicht zu fürchten ist für sein Dasein, seine
Anwesenheit in unserem Leben: sondern nur für seine Macht über uns, die
wir ihm unerkanntermaßen zugestehen. Sie gibt uns die Freiheit von
Seiner Macht und das ist, was uns so gefährlich erscheint; als wenn wir
in der einen ohne die andere sein könnten\ldots{} gefährlich aber auch,
weil sie uns aus dem größer werdenden Abstand nur n.~mächtiger erscheint
als kritische Masse von Energieträgern, die nur n.~im Bann gehalten wird
durch das aufgespannte Feld die Benjaminfeldkraft. Also lassen Sie uns,
Thomas, jetzt einig werden, daß wir nur allein das Feld unbeschadet
durchschreiten können. Seine Strahlkraft ist immens, das wissen wir
beide und niemand anderes würde nur einen Bruchteil einer Sekunde darin
aushalten. Aber Sie und ich als Markpfeiler jenes Zusammentreffens von
1911 und 2011 sind dazu in der Lage. Wenn Sie gehen, gehe ich auch. Der
Kern befindet sich auf halbem Wege von mir zu Ihnen, wir haben also
beide dasselbe Recht, die Theorie für uns zu beanspruchen: wie sie
lauten wird, ist hier n.~nicht gewiß, so lange keiner den ersten Schritt
getan hat zu Mitte hin.\\
"It\textquotesingle s the gold, that drags you, right?", said the
cunning little vixen. "We love much more what is stone, what is crystal
and black. We love the \emph{liquid crystal blackbox}." Im Herzen der
Maschine steht ein mächtiges Verhältnis, und die Kraft es aufzulösen,
liegt bei Ihr. "Who ist the She, which lays on you such effort that no
one dares to speak to you but from the wily serpents jaw?" Kein Auge,
das nicht weint, wenn es ihr Kohlestift erreicht. \emph{"Nehmen Sie sich
zusammen, Woyzeck. Er ist kein Tier. Nur ein Tier hat Natur. Nur ein
Hundsfott hat courage. Ich bin nur in Krieg gegangen, um mich in meiner
Liebe zum Leben zu befestigen. Grotesk!"} Der Krieg hatte begonnen in
drei Jahren. Was also erwartete uns n.~in diesem Leben. Unsere Tochter
war herausgerissen; der wirkliche Krieg, wie er 1939 begonnen werden
würde neben uns und über uns hinweg bedeutete unser aller Ende im Sinne
meiner Kultur. Wir sind auch keine A.r gewesen, ein Begriff, der sich
mir erst n.~erschließen sollte, sowie ich Kenntnis über die Umstände
meiner Geburt erlangen würde, deren Relevanz mir vorher nicht bewußt
war. In zwei Kriegen war versucht worden, dieses Land auszulöschen. "Der
Dritte steht kurz bevor. Doch es geht nicht mehr um Grenzen, nicht um
realen Boden. Auch nicht mehr um Macht oder weltliches überhaupt. Es
wird darum gehen, der Kultur ihren \emph{geistigen }Nährboden zu
entziehen, es wird gehen um eine Vorherrschaft im Präsens. Und kein
Vergangenes wird helfen können, auch nicht die Zukunft. Denn heute ist
der Satzbau \emph{das Primäre.}" Aber wo finden wir uns n.~in den
Sätzen, den ausgestorbenen Gründen unserer Existenz? Wir müssen jetzt
einander helfen, im Feld. Sie müssen mir Rechenschaft ablegen über die
Vergangenheit und ich kann Ihnen über die Zukunft berichten.

\end{document}
