% Options for packages loaded elsewhere
\PassOptionsToPackage{unicode}{hyperref}
\PassOptionsToPackage{hyphens}{url}
%
\documentclass[
]{article}
\usepackage{amsmath,amssymb}
\usepackage{iftex}
\ifPDFTeX
  \usepackage[T1]{fontenc}
  \usepackage[utf8]{inputenc}
  \usepackage{textcomp} % provide euro and other symbols
\else % if luatex or xetex
  \usepackage{unicode-math} % this also loads fontspec
  \defaultfontfeatures{Scale=MatchLowercase}
  \defaultfontfeatures[\rmfamily]{Ligatures=TeX,Scale=1}
\fi
\usepackage{lmodern}
\ifPDFTeX\else
  % xetex/luatex font selection
\fi
% Use upquote if available, for straight quotes in verbatim environments
\IfFileExists{upquote.sty}{\usepackage{upquote}}{}
\IfFileExists{microtype.sty}{% use microtype if available
  \usepackage[]{microtype}
  \UseMicrotypeSet[protrusion]{basicmath} % disable protrusion for tt fonts
}{}
\makeatletter
\@ifundefined{KOMAClassName}{% if non-KOMA class
  \IfFileExists{parskip.sty}{%
    \usepackage{parskip}
  }{% else
    \setlength{\parindent}{0pt}
    \setlength{\parskip}{6pt plus 2pt minus 1pt}}
}{% if KOMA class
  \KOMAoptions{parskip=half}}
\makeatother
\usepackage{xcolor}
\usepackage[margin=1in]{geometry}
\usepackage{graphicx}
\makeatletter
\def\maxwidth{\ifdim\Gin@nat@width>\linewidth\linewidth\else\Gin@nat@width\fi}
\def\maxheight{\ifdim\Gin@nat@height>\textheight\textheight\else\Gin@nat@height\fi}
\makeatother
% Scale images if necessary, so that they will not overflow the page
% margins by default, and it is still possible to overwrite the defaults
% using explicit options in \includegraphics[width, height, ...]{}
\setkeys{Gin}{width=\maxwidth,height=\maxheight,keepaspectratio}
% Set default figure placement to htbp
\makeatletter
\def\fps@figure{htbp}
\makeatother
\setlength{\emergencystretch}{3em} % prevent overfull lines
\providecommand{\tightlist}{%
  \setlength{\itemsep}{0pt}\setlength{\parskip}{0pt}}
\setcounter{secnumdepth}{-\maxdimen} % remove section numbering
\ifLuaTeX
  \usepackage{selnolig}  % disable illegal ligatures
\fi
\usepackage{bookmark}
\IfFileExists{xurl.sty}{\usepackage{xurl}}{} % add URL line breaks if available
\urlstyle{same}
\hypersetup{
  hidelinks,
  pdfcreator={LaTeX via pandoc}}

\author{}
\date{\vspace{-2.5em}}

\begin{document}

\subsection{Cinque}\label{cinque}

\emph{Das Thomasich näherte sich dem goethischen Raum. Man sah, daß es
zwei Figuren warn. Je dichter die Nadel dem Scheitepunkt rückt, um so
heftiger zieht es sie an, sich neben ihn zu stellen und etwas laut zu
rufen. Wir spüren ihr Zögern, erkennen es am Händeflattern, das unruhig
von ihm aus. Man denkt, er kanns nicht erwarten, mit den Fingern nach
den Kugeln zu langen. Aber schleppend kommt der Guß voran, zögerlich die
Bewegungen des Faustischen wie von einer schwarz angehaltenen
Prozession.} Gedankenkraft führt mich heraus. Ich schaff es, sie nicht
mitzunehmen. Nicht, daß ich mich nicht trotzdem umgeschaut hätte oder
die Gebote anders nicht gehalten; schon der Wille, ihre Schriften übers
Ende zu retten\ldots{} sollte Anmaßung gewesen sein, genug, sie richtig
zu verlieren; welche Ironie. Nur ich saß an ihrm L.nam aufgebahrt wie
selbst zum letzen Schein verurteilt und was ich dann sah, war nicht mehr
ihr Gesicht oder eines, das ich überhaupt gekannt hätte. Es war wie von
einem Schleier, den jemand darüber geworfen hatte ein Schutz, als
Blendung derer, die versuchen sollten, sie zu entweihen. Und darum sind
wir schon längst Teil dieser Geschichte, ohne die \emph{Passarge}
bemerkt zu haben. Gehen wir ruhig n.mal in sie zurück\ldots{}\\
An den anderen Ort 25,3 Meter nur darüber, als wenn sie dort schon fast
aufgehoben wäre wie für ewig; die 25 Meter Überwasser lassen sich auf
Stufen erreichen. Es sind elf Intervalle anliegend den Decks
einschließlich des obersten. Die Mannschaftsräume befinden sich auf
Sieben. Wenn sie jemand vom Sprung abhalten will, dann ist jetzt der
richtige Zeitpunkt, kurz bevor sie sich vergewissert, daß niemand
zuschaut. Danach geht alles viel zu schnell, als daß ich keine Absicht
darin mehr vermutete, wie sie mich zu der Fahrt eingeladen über das
kleine Meer hatte und ich muß nicht Psychologe sein (Mahler-Laplace) um
in ihrem wiedererzählten Sprung von der Reling der Troja, die jetzt
mitten auf der Ostsee ungefähr so weit vom Zielhafen in Rostock
dahinfuhr, wie ich, zeitlich in umgekehrter Richtung von dem Entschluß
in Helsinki dazu entfernt, auch sah, sein Gelingen zu erkennen\ldots{}
ihre Kindheit machte mir Angst, besonders weil ich sie abgeschlossen
dachte und heimlich tut sie sich jetzt auf, während ich n.~glaube, daß
es das fremde Land und seine unbekannte Sprache waren, die sie
unvollendet erscheinen ließen\ldots{} Es war aber was ganz anderes; die
wenn auch nicht zu einem Abschluß gebrachte Ausbildung zum Tonsetzer
hatte den entscheidenden Schritt zur Erkenntnis dazugetan: Der Symphonie
hätte ein Auge gefehlt. Es war scheinbar durch einn Glaskörper ersetzt
worden, der jetzt fast hundert Jahre das hielt, was man von ihm
erwartete: das Stück gut aussehen zu lassen auf dem Papier. Niemand
zweifelte n.~an der Echtheit und sein Durchwärmtsein vom Geist des
Komponisten. Doch ein Geheimnis blieb, welches mit ihm unter den
Grasbüscheln begraben liegt: Wie können wir es hören lernen mit diesen
halben Augen nur? Das habich mich bemüht zu enträtseln in den
vergangenen Jahren. Ich ließ ein Licht, einen Blumenzweig, eine Bitte
dort zurück im Schneetreiben an meinem eigenen Geburtstag. Es sieht so
aus, als hätte Mahler die Bitte vernommen. Er würde Ewa springen und
mich sie auffangen lassen. Sie könnte etwas tun zu seinem Stück. Ihr
entstand eine Harmonie aus den Grundsätzen, den von ihm hinterlassenen
Pfeilern der partitur tenebrae. Meine Aufgabe war beendet in jenem
Moment, als wir uns über die Reling gebeugt von ihr verabschiedeten,
ihrem Wunsch gemäß ins nächste Meer gelassen. Welche Rolle der Wind und
ein Urvogel und ein mögliches Knie von den vielen gebeugten zu spielen
hatten, ahnte ich n.~nicht. Und daß Geburt immer nur wieder Geburt
zeugen würde, auch wenn ich selbst ihr Mann blieb, zeugungsunfähig bis
zur Umnachtung, war uns lange nicht klar obwohl wir einiges wußten über
ihre Jungfernschaft bis zum vollendeten Tode (\emph{clinici.)} Der war
lange vorbereitet gewesen und ging deshalb leicht von der Hand. Nur erst
die Entscheidung dazu ist schwer gefallen. Die betreffenden Figuren
waren verinnerlicht, es mußte lediglich der Stein voran ins Wasser
geworfen werden, alles nachfolgende stützte sich auf die geläufigen
Schemata heutiger Harmonien. Und da war Wachstum, "von überall her wuchs
es mir zu" hatte M. selbst gesagt und seine Stimmung erzeugte in den
Resonanzkörpern die unsterblichen Ideen schwingenden Materials, egal
welchen Ursprungs; selbst Papier konnte zum Medium werden unter einer
dünnen Rauchschicht, mit welcher man es belegt hatte und darauf jetzt
einen Diamanten laufen ließ (die erste Aufnahme aus den 60er Jahren
eines Franzosen ,Claire de la Lune\textquotesingle.) Wie weit kommen wir
Mahler entgegen mit unserem Werk? Ewa, du mußt dich erheben. Levez-toi!
Es geht nicht, n.~eine Rippe zu entbehren. Wo ist dein Atem, Gott, mit
dem du mich erwecktest, warum versagst du an ihr? Es sind nur Bäume um
mich\ldots{} nur Wald.\\
Dann gab es keinen Halt mehr an diesem weißen Strebengerüst, auf das ich
geklettert war. Wenn man einmal eine gewisse Höhe erreicht hatte, gab es
stehend einfach keinen Halt außer jenen durch die Füße. Aber diese Füße
sind ja nicht mehr meine eigenen. Sie haben mich die letzten Stufen zum
Mannschaftsdeck steigen lassen, als gäbe es gar keinen Boden unter
ihnen, den man berührt. Man schwebt fast, der Druck des Körpers gegen
den Untergrund ist gleich null. Obwohl ich hinaufstieg, brauchte ich
keine Kraft, es zog mich an meinem Schwerpunkt hoch, der irgendwoanders
lag jetzt, außerhalb des Schiffes schon. Darum alles so leicht, jetzt zu
fallen. Ich habe das Manuskript bei mir:\\
Mahlers X. Sinfonie als Auftrag im Diagramm an
2011-05-18-16:25:00/+58˚21\textquotesingle29", +20˚03\textquotesingle22"
ist fertiggestellt worden. Ich ziehe meine Kleidung fast ganz aus. Ich
habe meine Uhr abgelegt. Jemand anderes wird mich zurückholen, wenn ich
gesprungen bin weiß ich der nicht ich ist. Vielleicht du, Sie oder Ihr
wann immer es eine Gelegenheit gibt, den Tod fürwen zu überwinden, die
ihr gelernt habt. Meine Möglichkeit habe ich jetzt und hier dargelegt in
dieser Schrift, die mit 39.179 Wörtern beendet ist.\\
\emph{Epilog, Nachspiel, Geflüster; schließlich Applaus in den ersten
Reihen im Parkett, }weil man unruhig wurde und eine zu lange Pause
entstanden war nach dem letzten Ton. Dieses auszuhalten hatte sich Me.
B. (Bernstein/Boulez/Barenboim\ldots) nicht nehmen lassen. Kein
Sinkenlassen, keine Verbeugung, kein Umschauen, keinerlei Anzeichen
einer Beendigung. Das mußte sich jeder Protagonist selbst ausdenken im
Auditorium, dem sie immer n.~starr den Rücken zukehrten. Kein wirkliches
Ende könnte glaubhafter ausklingen als dieses schon vor dem doppelten
Taktstrich einsetzende Verlangen des Komponisten, sich in ein neues Werk
begeben zu dürfen, welches er heute n.~beginnen würde sobald die
vorgeschrie bene Zeit erreicht war. Nachdem die ersten Handflächen
aneinand errie ben und sich das zaghaft vermehrte, anfangs chaotisch und
dann die Menschen bemerkten, daß jetzt geklatscht werde; nachdem also
alle glaubten das Ende wirklich gehört zu haben: Da stand es wie in
Geisterlettern über dem Podium, da erhob er sich wie der Wind,
\emph{dessen Rauschen du wohl hörst und Sausen, aber nicht weißt, woher
er kommt und wohin er geht: das war }Der Gott aus der Maschine, den sie
vernahmen und nicht wissen, \emph{intime}, ist es n.~einmal teilbar
dieses Ende in seine Korpuskeln und strahlt es schließlich n.~etwas ab
über das gehörte Werk hinaus, das einem nur eingeblasen werden kann wie
vom eigenen G., dem einzigen, dem man glaubte und nun hat er sich einem
kurz gezeigt? In Mahlers Sinfonie? Sie waren hingegangen, erzählte man
sich später, weil ein Werk unaufführbar anstand und das sollten wir
erleben. Wollte man darüber reden müssen am Ende? Wir hatten keine
Ahnung, w. u. e.wartete; sicher, man kannte etwa die Geschichte um die
Sinfonie, wir warn Musikliebhaber und gelesen hatten wir vom Mysterium
der Zahl 9, das sich um die an ihrer Übersteigung scheiternden
Sinfoniker rankt. Also hier hatte jemand etwas versucht, das man zur
Kenntnis nehmen mußte, wollten wir uns in der Musik weiterbewegen. Wir
haben von Ewa Laplace niemals gehört und werden forschen, was es mit der
Jungen Frau Mahler auf sich hat. Sicher ist nur, daß ihr Stück gelang.
Wir sitzen auf unseren harten Stühlen im Rund um die be.ner
Philharmoniker, der Vorhang über ihnen scheint sich nicht zu bewegen und
das verstärkt die Pause ins Unerträgliche; wir können nicht mehr, wir
möchten aufstehen und schreien, M.B. ist nur eine zufällige Adresse für
unseren Jubel, er muß jetzt dahin und durch ihn durch zu Mahler, auch
durch Ewa, durch schwingende Legierungen und jeden Grashalm, der das
Kalbsfell v. Pauken hervorbrachte und jeden Violenholz ziehenden Tropfen
\emph{Wassers} dieser unserer heiligen Mutter Erde. \emph{fin, /x}

\end{document}
